\documentclass[12pt]{article}

\usepackage{setspace}

\usepackage{amsmath, amsfonts, amssymb, graphicx, color, fancyhdr, lipsum, scalerel, stackengine, mathrsfs, tikz-cd, mdframed, enumitem, framed, adjustbox, bm, upgreek, x	color}
\usepackage[framed,thmmarks]{ntheorem}
\usepackage[mathscr]{euscript}

%set up theorem/definition/etc envs
%Problems will be created using their own counter and style
\theoreminframepreskip{0pt}
\theoreminframepostskip{0pt}
\newframedtheorem{prob}{Problem}[section]
\newenvironment{hwprob}[1]
{\renewcommand{\theprob}{#1}%
 \addtocounter{thm}{-1}%
 \begin{prob}}
{\end{prob}}

%solution template
\theoremstyle{nonumberbreak}
\theoremindent0.5cm
\theorembodyfont{\upshape}
\theoremseparator{:}
\theoremsymbol{\ensuremath\spadesuit}
\newtheorem{sol}{Solution}

%Theorems, Lemmas, and Corollaries
\theoremstyle{changebreak}
\theoremseparator{}
\theoremsymbol{}
\theoremindent0.5cm
\theoremheaderfont{\color{violet}\bfseries} 

\newtheorem{thm}{Theorem}[subsection]
\theoremheaderfont{\bfseries}
\newtheorem{lem}[thm]{Lemma}
\newtheorem{cor}[thm]{Corollary}

%Create a new env that references a theorem and creates a 'primed' version
%Note this can be used recursively to get double, triple, etc primes
\newenvironment{thmprime}[1]
  {\renewcommand{\thethm}{\ref{#1}$'$}%
   \addtocounter{thm}{-1}%
   \begin{thm}}
  {\end{thm}}

\setlength\fboxsep{15pt}

%Shade definitions
\theoremindent0cm
\theoremheaderfont{\normalfont\bfseries} 
\def\theoremframecommand{\colorbox[rgb]{.9,.8,1}}
\newshadedtheorem{defn}[thm]{Definition}

%Shade definitions
\theoremindent0cm
\theoremheaderfont{\normalfont\bfseries} 
\def\theoremframecommand{\colorbox[rgb]{.9,.9,.9}}
\newshadedtheorem{ex}[thm]{Example}

%Man, that's really good! Let's use the same thing for definitons.
\newenvironment{defprime}[1]
  {\renewcommand{\thethm}{\ref{#1}$'$}%
   \addtocounter{thm}{-1}%
   \begin{defn}}
  {\end{defn}}

%proofs
\theoremstyle{nonumberbreak}
\theoremindent0.5cm
\theoremheaderfont{\sc}
\theoremseparator{}
\theoremsymbol{\ensuremath\spadesuit}
\newtheorem{prf}{Proof}
\newtheorem{conj}{Conjecture}

%remarks
\theoremstyle{change}
\theoremindent0.5cm
\theoremheaderfont{\sc}
\theoremseparator{:}
\theoremsymbol{}
\newtheorem{rmk}[thm]{Remark}

%Replacement for the old geometry package
\usepackage{fullpage}

%Put page breaks before each part
\let\oldpart\part%
\renewcommand{\part}{\clearpage\oldpart}%

%Center each figure by default
\makeatletter
\g@addto@macro\@floatboxreset{\centering}
\makeatother

%header stuff
\setlength{\headsep}{24pt}  % space between header and text
\pagestyle{fancy}     % set pagestyle for document
\lhead{Notes on Lie Algebras} % put text in header (left side)
\rhead{Nico Courts} % put text in header (right side)
\cfoot{\itshape p. \thepage}
\setlength{\headheight}{15pt}
\allowdisplaybreaks

%Set of Integers
\newcommand*{\Z}{
\mathbb{Z}
}
%Set of Natural Numbers
\newcommand*{\N}{
\mathbb{N}
}
%Set of Real Numbers
\newcommand*{\R}{
\mathbb{R}
}
%Set of Complex Numbers
\newcommand*{\C}{
\mathbb{C}
}
%Rationals
\newcommand*{\Q}{
\mathbb{Q}
}

%Section break
\newcommand*{\brk}{
\rule{2in}{.1pt}
}

\DeclareMathOperator{\Aut}{Aut}

%raise that Chi!
\DeclareRobustCommand{\Chi}{{\mathpalette\irchi\relax}}
\newcommand{\irchi}[2]{\raisebox{\depth}{$#1\chi$}} 

%Image
\DeclareMathOperator{\im}{im}
\DeclareMathOperator{\Hom}{Hom}
\DeclareMathOperator{\End}{End}

%Coker
\DeclareMathOperator{\coker}{coker}

%characteristic
%\let\char\relax
%\DeclareMathOperator{\char}{char}
\DeclareMathOperator{\ch}{ch}

%rank
\DeclareMathOperator{\rank}{rank}

%identity map
\DeclareMathOperator{\id}{id}

%some nice shortcuts
\DeclareMathOperator{\calB}{\mathcal{B}}

%Lie algebra stuff
\DeclareMathOperator{\gl}{\mathfrak{gl}}
\let\sl\relax
\DeclareMathOperator{\sl}{\mathfrak{sl}}
\DeclareMathOperator{\so}{\mathfrak{so}}
\DeclareMathOperator{\su}{\mathfrak{su}}
\let\sp\relax
\DeclareMathOperator{\sp}{\mathfrak{sp}}
\DeclareMathOperator{\tr}{tr}
\DeclareMathOperator{\ad}{ad}
\DeclareMathOperator{\Der}{Der}
\DeclareMathOperator{\Int}{Int}
\DeclareMathOperator{\Rad}{Rad}
\let\t\relax
\DeclareMathOperator{\t}{\mathfrak{t}}
\DeclareMathOperator{\n}{\mathfrak{n}}
\DeclareMathOperator{\h}{\mathfrak{h}}
\DeclareMathOperator{\g}{\mathfrak{g}}
\DeclareMathOperator{\diag}{diag}

%fix tilde
\let\tilde\relax
\newcommand*{\tilde}[1]{\widetilde{#1}}

% Enumerate will automatically use letters (e.g. part a,b,c,...)
\setenumerate[0]{label=(\alph*)}

\begin{document}
%make the title page
\title{Lie Algebras and Groups\vspace{-1ex}}
\author{A course by: Monty McGovern\\
Notes by: Nico Courts}
\date{Winter 2019}
\maketitle

\renewcommand{\abstractname}{Introduction}
\begin{abstract}
	These notes are my best attempt at following along with our \textit{Math 508 --
	Lie Algebras} course at UW. This is my first time trying to type my 
	notes on-the-fly in class so we'll see how well this goes. The course reference
	is Humphreys' \textit{Introduction to Lie Algebras and Representation Theory.}

	The course description follows:
	
	\brk

	This is the second course in the Algebraic Structures sequence. I will classify 
	finite-dimensional complex semisimple Lie algebras, also proving some structural 
	results on general Lie algebras along the way. Although one usually first 
	encounters Lie algebras in a manifolds course, the treatment (following the text) 
	will be entirely algebraic.
\end{abstract}

\section{January 7, 2019}
The homework is posted on Monty's website. :) 
\subsection{Lie algebras}

This course will be studying Lie algebras, but as opposed to their treatment in manifolds, 
we will be studying them from a purely algebraic point of view. The book (Humphreys)
actually never defines a Lie group.

\begin{defn}
	A \textbf{Lie Algebra} $L$ or $\mathfrak{g}$ over a field $k$ is a $k$-vector space (usually f.d.)
	along with a \textit{bracket operation} $[vw]:L\times L\to L$ such that $[\cdot\cdot]$ is 
	\begin{itemize}
		\item anticommutative,
		\item bilinear,
		\item $[x[yz]]=[[xy]z]+[y[xz]]$
	\end{itemize}
\end{defn}

\begin{rmk}
	The last principle above is actually equivalent to the \textit{Jacobi identity:}
	\[[x[yz]]+[y[xz]]+[z[xy]].\]
	This follows from bilinearity and anticommutativity of the bracket.
\end{rmk}
The most natural place for these to arise is as \textit{derivations} on an algebra!
\begin{defn}
	A \textbf{$k$-derivation} $d:A\to A$ on an algebra $A$ over $k$ is a $k$-linear map
	satisfying the Leibniz rule.
\end{defn}
\begin{rmk}
	Some key facts about derivations (for us):
	\begin{itemize}
		\item Given a fixed $a\in A$, the map $d_a$ sending $b\mapsto ab-ba$, the \textbf{commutator}
		$[ab]$ is a derivation.
		\item If $d,e$ are derivations, then so is $[de]=de-ed$, where $de$ is the \textit{composite}
		of $d$ and $e$ as opposed to the product.
	\end{itemize}
\end{rmk}

\subsection{Examples}

A main source of Lie algebras is (associative) algebras! \textit{Any associative $k$-algebra $A$}
becomes a Lie algebra over $k$, taking $[ab]=ab-ba.$ In particular, one obvious choice for $k$-algebra
is $M_n(k)=\gl_n(k)$, the (Lie) algebra of $n\times n$ matrices over $k$.

\textbf{Lie subalgebras} are what you'd expect (including closure under brackets). Notice
that if $L'\le L$, then they \textbf{must both be over the same field.}

If $L$ is a $k$-Lie algebra and $I\lhd L$ is an ideal of $L$, then the quotient space $L/I$ becomes a 
Lie algebra with $[x+I,y+I]=[xy]+I$ as the bracket.

A \textbf{Lie algebra homomorphism} is a map $\varphi:L\to L'$ such that $\varphi$ is $k$-linear
and $\varphi([xy])=[\varphi(x)\varphi(y)].$

We get the usual first isomorphism theorem $L/\ker\varphi\cong \varphi(L).$

\brk

Associative algebras are not the only source of Lie algebras, however! One example is 
$\sl(n,k)=\{n\times n\text{ matrices over } k\text{ with trace zero}\}$

Note that this is \textbf{not closed under product} since $\tr(AB)\ne\tr A\tr B$ but $\tr(AB)=\tr(BA)$
so $\tr(AB-BA)=\tr(AB)-\tr(BA)=0$.
\begin{defn}
	We call this algebra (or, in fact any subalgebra of $\gl(n,k)$) \textbf{linear}. Think 
	``Linear'' means ``of matrices.''
\end{defn}

We say that $\sl(n,k)$ has \textbf{type} $A_{n-1}$. Eventually we will see seven types
$A-G$ of semisimple Lie algebras. The shift in index will emerge later.

$\sl(n,k)$ is, in fact, a simple Lie algebra: for $k=\C$, $\sl(n,\C)$ has no ideals
apart from the trivial ones. 

\brk

Other non-associative examples include $k^n$ with a bilinear form $(\cdot,\cdot)$ which
is either symmetric or skew-symmetric and (in either case) is nondegenerate.
\begin{defn}
	$(\cdot,\cdot)$ is \textbf{nondegenerate} if the map $v\mapsto (v,\cdot)$ is injective. Equivalently
	there is no $v\in V$ such that $(v,w)=0$ for all $w\in V$.
\end{defn}

Given $V=k^n$ and a bilinear form on $V$, we can look at all $X\in \gl(n,k)=\gl(V)$ such that 
$(Xv,w)=(v,Xw)$. Then $X$ is \textbf{adjoint} with respect to the form. There is a similar definition for when
$X$ is \textbf{skew-adjoint.} One can check that 
$[XY]$ is skew-adjoint whenever both $X$ and $Y$ are.

\subsection{Generating (skew) symmetric forms}
It ends up that the dot product (which is a symmetric form) is misleadingly simple -- thus
we will look elsewhere.

If $M\in \gl(n,k)$ is symmetric, so that $M^t=M$, then $(v,w)=v^tMw$ is a symmetric. If 
instead $M$ is skew-symmetric, then the same definition yields a skew-symmetric form. 
This actually induces a one-to-one correspondence between matrices and forms.

In both cases, if $M$ is invertible, then the form will be nondegenerate. As a consequence, 
since skew-symmetric matrices are always singular in odd dimensions, we see that 
nondegenerate skew-symmetric forms (over $\ch k\ne 2$ where the two families of forms
coincide) exist only in even dimensions.

\subsection{A peek at classifications}
If we have a nondegenerate symmetric form where $n=2m$ is going to give us an algebra
of type $D_m$. If $n=2m+1$, then it is of type $B_m$. Both of these cases are called
\textbf{orthogonal.}

If instead we have a skew-symmetric form and $n=2m$, then this is of type $C_m$, and we 
call this algebra \textbf{symplectic.}

\brk

We will make a particular choice for our matrix $M$ and then study the resulting
Lie algebras in much more detail next time. The choices will be: 
\begin{itemize}
	\item For type $D_m$:
	\[\begin{pmatrix}
		0 & I_m\\
		I_m & 0
	\end{pmatrix}\]
	\item For type $C_m$:
	\[\begin{pmatrix}
		0 & -I_m\\
		I_m & 0
	\end{pmatrix}\]
	\item For type $B_m$:
	\[\begin{pmatrix}
		1 & 0 & 0\\
		0 & 0 & I_m\\
		0 & I_m & 0
	\end{pmatrix}\]
\end{itemize}

\section{January 9, 2019}
Today we will be looking deeply into the stucture of linear Lie algebras of types A-D.

\subsection{Linear Lie Algebras Revisited}
Recall that the \textbf{matrix unit} $e_{ij}$ is the matrix with 1 in the $(i,j)$ entry and 
zero elsewhere. And then $e_{ij}e_{kl}=\delta_{jk}e_{il}$ and furthermore
\[[e_{ij}e_{kl}]=\delta_{jk}e_{il}-\delta_{li}e_{kj}.\]
This is especially nice when $j=i$, called the \textbf{diagonal matrix unit}.

Then we look at type $A_{n-1}$ ($\sl(n,k)$). Let $D$ be the set of diagonal matrices in this algebra.
Notice the dimension is $n-1$ since then $n^{th}$ term on the diagonal is determined as the
negative of the sum of the other $n-1$ terms. Let $A=\operatorname{diag}(d_1,\dots,d_n)$. Then consider the eigenvalues
associated with $e_{ij}$: 
\[[Ae_{ij}]=(d_i-d_j)e_{ij}=(E_i-E_j)Ae_{ij}\]
where $E_i$ is the linear functional selecting the $i^{th}$ entry in $A$. Moreover, $D$ is abelian as a Lie algebra, so $D$
acts diagonally on $L=\sl(n,k)$ by commutation with eigenvalues $E_i-E_j$ and zero
for $1\le i,j\le n$ and $i\ne j$.

In the other classical cases B-D, there is always a matrix $M$ which defines the form $(v,w)=v^tMw$ 
as we saw yesterday. In all three cases, the Lie algebra exists consists of all skew-adjoint matrices
$X$ relative to the form. $B_m=\mathfrak{so}(2m+1,k)$ as well as $D_m=\mathfrak{so}(2m,k)$ and $C_m=\mathfrak{sp}(2m,k)$.

This condition translates to the form of matrices in the above Lie groups and the condition is always $Mx=-x^tM$ in all cases.

Type B:
\[\begin{pmatrix}
	0 & b_1 & b_2\\
	c_1 & m & n\\
	c_2 & p & q
\end{pmatrix}\]
where $c_1=-b_2^t$, $c_2=-b_1^t$, $q=-m^t$, $n^t=-n$ and $p^t=-p$.

Type C:
\[\begin{pmatrix}
	m& n\\ p & q
\end{pmatrix}\]
where $n^t=n, p^t=p,$ and $m^t=-q$.

Type D:
\[\begin{pmatrix}
	m& n\\ p & q
\end{pmatrix}\]
where $n^t=-n, p^t=-p,$ and $m^t=-q$.

Looking at the eigenvalues of elements of $D$ associated to vectors $e_{ij}-e_{m+i,m+j}$. Look at photos

Using a similar analysis, we can look at types $B$ and $D$. We define the functions $E_i$ similarly on the space of diagonalmatrices and gives a rise to the following collection of linear functions:
in $B_m:$ $\pm E_i$ and $\pm(E_i\pm E_j)$ and $D_m$ gives us $\pm(E_i\pm E_j)$.

This collection of functions in each case is called the \textbf{root system of the Lie algebra},
Then any complex simple finite-dimensional Lie algebra is classified by its root system.
The (perhaps surprising) fact is that this already encompasses all but finitely many of these
things up to ismorphism: the classical Lie algebras. Eventually we will learn more about the
\textbf{exceptional Lie groups.}

This section was a little hard to follow and the handling in Humphreys is easier to follow, 
but delays speaking about root systems and actually deriving the eigenfunctions (is that the right word?)
until significantly later. Monty seemed to think it was acceptable to delay the understanding of this a bit.

\subsection{Derivations and $\exp$}
Look an an arbitrary Lie algebra over a field $k$ where $\ch k=0$ (which we will mostly be assuming from here on)
. Let $\delta$ be a derivation of $L$, so that $[\delta x,y]+[x\delta y]=\delta[x,y]$. Assume that $\delta$ is nilpotent.

Then the ``power series'' (polynomial) is
\[\exp\delta=\sum_{i>0}\frac{\delta^i}{i!}\]
\begin{prob}
	This is a good exercise to go through: Check that
	\[[(\exp\delta)x,(\exp \delta)y)=[xy]\]
	for each $x,y\in L$
\end{prob}

\begin{rmk}
	This actually shows that $\exp\delta$ is an automorphism of $L$. Furthermore you'd find that
	\[(\exp\delta)(\exp(-\delta))=1.\]
\end{rmk}

What if $k=\R$ or $\C$? Then the power series (even when $\delta$ is not nilpotent!) always converges
and defines an automorphism as before.

\begin{lem}
	For all \textbf{complex} semisimple Lie algebras $L$ it turns out that the group
	generated by $\exp\ad x$ ($\ad x(y)=[xy]$) coincides with the group generated by all 
	nilpotent $\ad x$.
\end{lem}
\subsection{Adjoint group}
The last thing for today is to define the adjoint group:
\begin{defn}
	Let $L$ be a Lie algebra, then 
	\[\operatorname{Int}(L)=\exp\ad L\]
	is the \textbf{Adjoint group} of $L$. It is a subgroup (so we believe) of the Lie
	Group associated to $L$.
\end{defn}
\begin{rmk}
	Actually after talking to Monty, $\exp\ad L$ \textit{is} (essentially) the Lie group
	corresponding to $L$. Such a group is not unique, however.
\end{rmk}

Some examples of adjoint groups:
\begin{itemize}
	\item If $L=\sl(n,\C)$, then $\operatorname{Int}(L)=PSL(n,\C)=SL(n,\C)/\text{center}$
	\item If $L=\so(n,\C)$, then $\operatorname{Int}(L)=PSO(n,\C)$
	\item If $L=\sp(2n,\C)$ then $\operatorname{Int}(L)=PSp(2n,\C)$.
\end{itemize}

%%%% HW 1 - Due Jan 18 %%%%
\newpage

\section*{HW 1 -- Due January 18}
Do problems 1.9, 2.1, 3.8. 3.9, and 4.3.

\begin{hwprob}{1.9}
	When $\ch k=0$, show that each classical Lie algebra $A_l,B_l, C_l$ and $D_l$ satisfies $[LL]=L$. (This shows again that each algebra consists of trace zero matrices.)
\end{hwprob}
\begin{sol}
	\subsubsection*{Type $A_k$:}
	Recall that a basis for $A_l$ is the collection $\mathcal{B}=\{e_{ij}|i\ne j\}\cup\{e_{ii}-e_{i+1,i+1}|1\le i\le l\}$. Using linearity of the bracket,
	it then suffices to check that $\calB$ is contained in the algebra generated by $[\calB \calB]$. To see this, we use that
	\[[e_{ij}e_{kl}]=\delta_{jk}e_{il}-\delta_{li}e_{kj}.\]

	First, notice we have
	\[[e_{i,i+1}e_{i+1,i}]=e_{ii}-e_{i+1,i+1},\]
	so it suffices to show we also have the $e_{ij}$ for $i\ne j$.
	This is clear whenever $k\ge 2$ (so that we have three columns in our matrices). Let $i\ne j$ be arbitrary
	and $l$ be a (the) third index. Then 
	\[e_{il}e_{lj}]=e_{ij}.\]

	If $k=1,$ then 
	\[e_{21},e_{11}-e_{22}]=e_{21}+e_{21}=2e_{21}\]
	and so $e_{21}$ is in the space generated by $[\calB \calB]$ and an analagous proof
	gives us the rest of the basis.

	\subsubsection*{Type $B_k$:}
	This one is getting a bit long and is low on my priority list so might not be finished.
\end{sol}

\begin{hwprob}{2.1}
	Prove that the set of all inner derivations $\ad x$ for $x\in L$ is an ideal of $\Der L$.
\end{hwprob}
\begin{sol}
	Let $I=\ad L$. $I\subseteq \Der L$ since $\ad x$ is a derivation. This is a subspace since $\ad 0=0$ and since the bracket is bilinear
	\[\ad x(ay+bz)=[x,ay+bz]=[x,ay]+[x,bz]=a[xy]+b[xz]=a\ad x(y)+b\ad x(z).\]

	Take any $x\in I$ and let $f\in\Der L$ be arbitrary. But then for $y,z\in L$
	\begin{align*}
		[\ad x,f](yz)&=(\ad x f-f\ad x)(yz)\\
		&=[x,f(yz)]-f[x,yz]\\
		&=[x,f(y)z+yf(z)]-f([x,y]z+y[x,z])\\
		&=[x,f(y)z]+[x,yf(z)]-f([x,y]z)-f(y[x,z])\\
		&=[x,f(y)]z+f(y)[x,z]+[xy]f(z)+y[x,f(z)]-f([x,y])z\\
		&\hspace{2in} -[x,y]f(z)-f(y)[x,z]-yf[x,z]\\
		&=([x,f(y)]-f[x,y])z+y([x,f(z)]-f[x,z])\\
		&=([\ad x, f](y))z+y([\ad x,f](z))
	\end{align*}
	so $[IL]\subseteq I$ whence $I$ is an ideal of $L$.
\end{sol}

\begin{hwprob}{3.8}
	Let $L$ be nilpotent. Prove that $L$ has an ideal of codimension 1.
\end{hwprob}
\begin{sol}
	Assume that $L$ is nilpotent. That is, defining $L^0=L$, and by induction $L^i=[LL^{i-1}]$, this 
	means that $L^n=0$. 
	
	We proceed by induction on $\dim L$: for the base case, if $L=0$ the result doesn't really mean anything,
	and when $\dim L=1$, the trivial subspace has codimension 1.

	Assume now that $\dim L=i$ and that every Lie algebra of dimension less than $i$ has a codimension 1 ideal. 
	Then we have two subcases to consider. If $L$ is commutative then
	every vector subspace is an ideal, so take a codimension one subspace and we are finished.
	
	Finally assume $[LL]\ne 0$ and consider the quotient $Q=L/[LL]$. By 
	our assumptions (and since $L$ is nilpotent $L\ne[LL]$) we get that
	\[0<\dim Q<\dim L\]
	and that, in fact, by rank-nullity
	\[\dim Q = \dim L - \dim[LL].\]
	But then by the induction hypothesis there is an ideal $J\lhd Q$ with dimension $\dim Q - 1$
	and pulling this back through the quotient yields the ideal $\bar J\lhd L$ that has dimension $\dim Q-1+\dim[LL]=\dim L-1$. 
	This is our codimension one ideal and we complete the proof.
\end{sol}

\begin{hwprob}{3.9}
	Prove that every nilpotent Lie algebra L has an outer derivation (see 1.3) as follows: Write $L=K+Fx$
	for some ideal $K$ of codimension 1. Then $C_L(K)\ne 0$ (why?). Choose $n$ so that $C_L(K)\subseteq L^n$,
	$C_L(K)\not\subseteq L^{n+1}$ and let $z\in C_L(K)-L^{n+1}$. Then the linear map $\delta$ sending $K$ to zero
	and $x$ to $z$ is an outer derivation.
\end{hwprob}
\begin{sol}
	Using the result above, we can write every nilpotent $L$ as $K+Fx$ for some $K\lhd L$.
	Now since $K\le L$, and since $L$ is simple, for some $n$: $L^n=[L^{n-1}L^{n-1}]=0$,
	or equivalently every $n$-fold bracket in $L$ is zero.

	Assuming that $n$ is minimal, this means that one can construct a $(n-1)$-fold bracket
	\[x=[x_1[x_2[x_3\cdots[x_{n-2}x_{n-1}]\cdots]]]\]
	which is nonzero. But then $[xL]=0$ so in particular $[xK]=0$ whence $x\in C_L(K).$

	Therefore we can choose an $m$, as suggested, such that $C_L(K)\in L^m\setminus L^{m-1}$
	and let $\delta$ be the linear map sending $K$ to zero and $x$ to some $z\in C_L(K)\setminus L^{m+1}$.
	We will now establish this is an outer derivation.

	Let $a,b\in L$ be arbitrary with $a=k_a+l_ax$ and $b=k_b+l_bx$ and compute
	\begin{align*}
		\delta(ab) &= \delta(k_a+l_ax)(k_b+l_bx)\\
		&= \delta(k_ak_b+k_al_bx+l_axk_b+l_axl_bx)\\
		&=0+k_a\delta(l_bx)+\delta(l_ax)k_b+\delta(l_ax)l_bx+l_ax\delta(l_bx)\\
		&=(k_a+l_ax)\delta(l_bx)+\delta(l_ax)(k_b+l_bx)\\
		&=a\delta(k_b+l_bx)+\delta(k_a+l_ax)b\\
		&=a\delta(b)+ \delta(a)b
	\end{align*}
	so $\delta$ is a derivation. It remains to show it is not inner.

	Assume by contradiction that it were. Then $\delta =\ad w$ for some $w\in L$. But then notice
	that $[wK]=0$ so $w\in C_L(K)\subseteq L^m$. But then bracketing with one more thing puts us in $L^{m+1}$, so
	$[w,x]\in L^{m+1}$. But then
	\[[w,x]=\ad w(x)=\delta(x)=z\in L^{m+1}\]
	a contradiction. Thus $\delta$ is an outer derivation.
\end{sol}

\begin{hwprob}{4.3}
	This exercise illustrates the failure of Lie's theorem when $F$ is allowed to have positive characteristic.
	Consider the $p\times p$ matrices
	\[x=\begin{pmatrix} 
		0 & 1 & 0 & \cdots & 0\\
		0 & 0 & 1 & \cdots & 0\\
		\vdots & \vdots & \vdots & \ddots & \vdots\\
		0 & 0 & 0 & \cdots & 1\\
		1 & 0 & 0 & \cdots & 0
	\end{pmatrix}\quad y=\diag(0,1,\dots, p-1)\]
	Check that $[xy]=x$, hence that $x$ and $y$ span a two dimensional solvable subalgebra $L$ of $\gl(p,F)$. Verify that $x$ and $y$ have no common eigenvector.
\end{hwprob}
\begin{sol}
	Compute
	\[[xy]=\begin{pmatrix} 
		0 & 1 & 0 & \cdots & 0\\
		0 & 0 & 2 & \cdots & 0\\
		\vdots & \vdots & \vdots & \ddots & \vdots\\
		0 & 0 & 0 & \cdots & p-1\\
		0 & 0 & 0 & \cdots & 0
	\end{pmatrix}-\begin{pmatrix} 
		0 & 0 & 0 & \cdots & 0\\
		0 & 0 & 1 & \cdots & 0\\
		\vdots & \vdots & \vdots & \ddots & \vdots\\
		0 & 0 & 0 & \cdots & p-2\\
		p-1 & 0 & 0 & \cdots & 0
	\end{pmatrix}=\begin{pmatrix} 
		0 & 1 & 0 & \cdots & 0\\
		0 & 0 & 1 & \cdots & 0\\
		\vdots & \vdots & \vdots & \ddots & \vdots\\
		0 & 0 & 0 & \cdots & 1\\
		1 & 0 & 0 & \cdots & 0
	\end{pmatrix}=x\]
\end{sol}
Thus if $L$ is the vector space spanned by $x$ an $y$, since it is closed under bracket, 
it is a Lie subalgebra of $\gl(p,F)$. Now because of this computation it is easy to see that
$[LL]=Fx$, so $L^2=0$, whence $L$ is solvable.

Notice that multiplication by $x$ on the right acts by permuting the columns of a matrix so $x^p=\id$,
whence the eigenvalues of $x$ are roots of $x^p-1=(x-1)^p$ since $\ch F=p$. Let $v$ me any element of $L$ 
fixed by the action of $x$ via $\ad x$ on $L$. Then
\[bx=b[x,y]=[x,ax+by]=[x,v]=xv-vx=v\]
but then since the only eigenvalue can be 1, $v=x$. But $[y,x]=-x$, which cannot be a scalar multiple of $y$,
so $x$ and $y$ can have no common eigenvector.
\newpage
%%%% End HW 1 %%%%

\section{January 11, 2019}
A few more things about $\Int L$ in classical cases:

Begin with $n\times n$ matrices $M$ over $\R$ or $\C$. Given any such $M$, we have
\[e^M=\sum_{i>0}\frac{M^i}{i!}\]
always converges (that is, the series for each entry converges). Futhermore $e^Me^{-M}=I$.
\begin{rmk}
	Notice that over \textit{any} field over characteristic zero whenever $M$ is nilpotent 
	this still holds.
\end{rmk}

Now if $M$ is skew-adjoint with respect to a bilinear form $(\cdot,\cdot)$, then 
\[(M^iv,w)=(-1)^i(v,M^iw)\]
whence
\[(e^Mv, w)=(v,e^{-M}w)\]
and so 
\[(e^Mv,e^Mw)=(v,w)\]
for all $v,w\in\R^n$ or $\C^n$.

But notice that then $e^M$ is an \textbf{isometry of} or \textbf{preserves} the form. Moreover, 
if we set $A=\ad M$, then $A$ is given as the difference of two \textit{commuting} linear maps, 
left and right multiplication by $M$.

\brk

Using the binomial theorem: $e^AN=e^MNe^{-M}$. This, in a nutshell, is why the formulas for
$\Int L$ all involved modding out by the center, since conjugation by a scalar matrix is trivial! :)

\subsection{Combining Ideals}
Now we return to the case of general Lie algebras over arbitrary fields (no longer assuming $\ch k=0$). 
We have seen that a subspace $I\le L$ is an ideal if $[IL]\subseteq I$.

If $I,J\lhd L$, then $I+J$ is an ideal, as well as
\[[IJ]=\{\sum_1^n [x_iy_i]|x_i\in I, y_i\in J\}.\]

We also are sometimes interested in the direct sum of ideals $\oplus L_i$ where $[L_iL_j]=0$ when $i\ne j$.

\subsection{Examples}
$0$ and $L$ are ideals as well as the \textbf{center} $Z(L)$.

\subsection{Characterizing Lie Algebras as Linear}
We now want to realize many Lie algebras as (isomorphic to) linear ones. For now, note that the adjoint map $\ad$
is a Lie algebra homomorphism by the Jacobi identity. Then $\ker\ad=Z(L)$, so when $Z=0$, then $L$ is isomorphic to 
a linear Lie algebra. 

\begin{defn}
	If $Z(L)$, and the map $\ad:L\to L$ is an isomorphism, we say $L$ \textbf{acts faithfully}
	on itself (via $\ad$).
\end{defn}

\begin{defn}
	If we have a homomorphism $L\to\gl(n,k)$, then we call $k^n$ an $L$-\textbf{module}.
	We define the action of $L$ on $K^n$ through the homomorphism.
\end{defn}

Notice that $L$ is always an $L$ module over itself.

One can check, using the realization of $\sl(n,k)$ we saw last time that (when $\ch k\nmid n$)
that $\sl(n,k)$ is simple. This is actually the Lie algebra analogue of the fact that $M_n(K)$ is simple
as a ring over any field $K$.

\begin{rmk}
	Notice that if $\ch k |n$$,$ then \textit{any} scalar matrix $aI\in\sl(n,k)$
	so $Z(\sl(n,k))$ is nontrivial.
\end{rmk}

\brk

In any event you can look up the proof for $\sl(2,k)$ in the book. The thing to glean here is that
\[[xy]=h,\quad [hx]=2x,\quad [hy]=-2y\]
for $h=e_{11}-e_{22}$, $x=e_{12}$ and $y=e_{21}$. This will keep popping up throughout the course.

\subsection{Ideals of Linear Lie Algebras}
Many subalgebras of $\sl(n,k)$ or $\gl(n,k)$ have many ideals (which happen to be ring-theoretic ideals).

\begin{defn}
	Some notation:
	\[\t(n,k)=\{\text{upper-triangular matrices}\}\] 
	also 
	\[\n(n,k)=\{\text{strictly-upper-triangular matrices}\}\]
\end{defn}

\subsection{Derived Series}
For any Lie algebra, we can define
\begin{defn}
	The \textbf{derived series} of a Lie algebra $L$ is defined to be
	\[L^{(0)}=L,\quad L^{(1)}=[LL],\quad L^{(n+1)}=[L^{(n)}L^{(n)}]\]
	which form an ascending chain of ideals.
\end{defn}

\begin{defn}
	If $L^{(n)}=0$ for some $n$, and thus for all sufficiently large $n$, we call $L$ \textbf{solvable.}
\end{defn}
\begin{rmk}
	Note that actually there is a cooresponding definition for groups using commutators $[G,G]$.
	This was defined first and ported to Lie algebras. It actually happened in the reverse direction for:
\end{rmk}

\begin{defn}
	The \textbf{central series} of $L$ is 
	\[L^0=L,\quad L^1=[LL],\quad L^2=[L,L^1],\dots\]
	and if $L^n=0$ for some $n$, then we call $L$ \textbf{nilpotent.}
\end{defn}
\begin{rmk}
	Equivalently, we can say that $L$ is nilpotent if \textit{any $n$-fold bracket} in $L$ is zero.
\end{rmk}
\begin{rmk}
	Notice that every nilpotent Lie group is solvable, but the converse fails: $t(n)$ is solvable by not nilpotent.
\end{rmk}

\subsection{Examples of solvable algebras}
One can easily check that every ideal and homomorphic image of a solvable algebra is solvable. Therefore if $I$ and $J$ are solvable ideals of $L$, then 
so is $(I+J)/J\cong I/(I\cap J)$ is solvable, as is $I+J$. Furthermore if $I$ and $L/I$ are solvable, then $L$ is.

\begin{defn}
	$\Rad L$, the \textbf{radical of $L$}, is the unique largest solvable ideal of $L$.
\end{defn}
\begin{rmk}
	Any finite dimensional algebra has a radical (taken to be the sum of all solvable ideals in $L$).
\end{rmk}

Using this definition, we are able to give two different characterizations:
\begin{defn}[Semisimple Lie Algebra]\label{def-semisimple}
	$L$ is semisimple if $\Rad L=0$.
\end{defn}
\begin{defprime}{def-semisimple}[Semisimple Lie Algebra]
	$L$ is semisimple if it is a direct sum of simple Lie algebras.
\end{defprime}
This equivalence will not become clear until we learn about the Killing form, but at least it jives
with our experience with representation theory and semisimple modules.

\section{January 14, 2019}
I missed this day for an event. I will try to catch up if I have time.

\section{January 16, 2019}
We are continuing from last time when we were showing that as solvable Lie subalgebra of $\gl(n,k)$
(when $\ch k=0$ and $\bar k=k$) necessarily admits a vector $v\in k^n\setminus 0$ that is
a common eigenvector:
\[xv=\lambda(x)v\]
for some \textbf{linear} function $\lambda:L\to k$.

We argues (as in the parllel case of a Lie subalgebra of $\gl(n,k)$ consisting of nilpotent matrices
by induction on dimension and found the same result (except it is zero). In this case when $\dim L=1$,
since $k$ is algebraically closed, $L$ is spanned by asingle matrix $x$ and any eigenvector of $x$ does the trick.

Now assume that the result holds for solvable Lie algebras of dimension $<d$. Then since $L$ is sovable, we know
$[LL]\subsetneq L$. Let $I$ be any subspace containing $[LL]$ and properly contained in $L$ of codimension one.
Then $I$ is an ideal since $[LI]\subseteq[LL]\subseteq I$, so by the inductive hypothesis $I$ has a common eigenvector, 
so that there is a linear function $\lambda$ such that the \textbf{weight space}
\[V_\lambda=\{v\in k^n:xv=\lambda(x)v,\,\forall x\in I\}\]
is nonzero.

We know that $L=I+kx$ for some $x\in L$, so we must show that $x$ preserves $V_\lambda$: then any eigenvector
of $x$ on $V_\lambda$ will do the job. Given $v\in V_\lambda$, we must show that 
\[yxv=xyv+[yx]v=\lambda(y)xv+\lambda([yx])v\]
so we must show that $\lambda([xy])=0$.

Given $v\in V_\lambda$, look at hte powers $v,xv, x^2v,\dots$. Let $m$ be the least power such that the
$v,xv,\dots,x^mv$ are linearly independent and let $W_i$ be the span of the first $i$ of these.

Then one checks by induction for an $y\in I$ that $yx^iv=\lambda(y)x^iv\pmod{v,xv,\dots,x^iv}$
It follows that $y$ was in the subspace $W_m$ (oh no missed some stuff).

So the trave of $y$ on $W_m$ is $m\lambda(y)$. Now for any $y\in I$, the matrix $[xy]$ acts on $W_m$ ascendingthe commutator of two matrices having trace zero, whence $m\lambda([xy])=0$,
so since $\ch k=0$, $\lambda([xy])=0$, as desired.

\brk

As a consequence, given a solvable Lie subalgebra of $\gl(n,k)$ here is a chain of subspaces
\[0=V_0\le V_1\le\cdots\le V_n=k^n\]
or a \textbf{flag} such that $LV_i\subseteq V_i$ for all $i$: $L$ acts by upper triangular 
matrices with respect to a suitable basis.

Another way of putting this, using module language, is that given any \textbf{finite dimensional} module over
a solvable Lie algebra (over an algebraically closed characteristic zero field) has a one-dimensional submodules $N$.

Equivalently, the only finite-dimensional irreducible modules over a solvable Lie algebra are one-dimensional. We 
will use this later in the context of semisimple Lie algebras with large solvable subalgebra.

\begin{rmk}
	For the record, Lie's theorem \textit{definitely fails} when $k$ has positive characteristic.
	We see one in our homework, and another example comes from $\sl(2,k)$ where $\ch k=2.$

	We know that $\sl(2,k)$ has a basis: $(\begin{smallmatrix}
		1&0\\0&-1
	\end{smallmatrix})=I_2$, $x=e_{12}$, $y=e_{21}$. Here $[hx]=0=[hy]$ and $[xy]=h$. It is shown in the book
	and is easy to see that when $\ch k\ne 2$, $\sl(2,k)$ is simple, but here $\sl(2,k)$ is nilpotent and its
	bracket relations become $[xy]=z$, $[xz]=[yz]=0$.

	It ends up we can define (actually over any field) the \textbf{Heisenberg} Lie algebra, spanned
	by $x=e_{12},y=e_{23},z=e_{13}$ or more simply the $3\times 3$ strictly-upper-triangular matrices.

	The matrices in $\sl(2,k)$ for any $k$ have no common eigenvector, so our proof of Lie's theorem must
	fail somewhere? We indeed have an ideal $I$ of coimension one spanned by $h$ and $x$ and there is acommon eigenvector for $I$, 
	namely $e_1$. Then $I$ acts diagonally on $ke_1$ via the linear function sending $h$ to 1, $x$ to $1$. 
	Here $y$ plays the role of $x$ in the proof, but is outside the ideal. That is,
	\[\lambda[yx]=\lambda(h)\ne 0.\]
\end{rmk}

Returning to the good case of $k$, we can now apply our result on a linear solvable Lie algebras to arbitrary ones:
uGIven any solvable Lie algebra $L$, there is a flag $0=L_0\le L_1\le\cdots\le L_n=L$ of ideas of $L$
so that $[LL_i]\subseteq L_i$ Also $[LL]$ is nilpotent, since it acts on $L$ by strictly upper triangular matrices.

Both results fail for general solvable Lie algebras when the basefield is not nice. If a solvable Lie algebra $L$ 
\textbf{does} have a chain of ideals $0=L_0\le L_1\le\cdots\le L_n=L$ where $\dim L_i=i$, then we call $L$ \textbf{completely solvable.}

\subsection{Returning to Linear Algebra}
At this point we will take some time to return to an important idea from Linear algebra, namely the \textbf{Jordan canonical form}
of a square matrix. Recall Jordan blocks and similarity when eigenvalues lie in the field.

If $M$ is already in this form, then we can write $M=M_s+M_n$ where $M_s$ consists of the diagonal entries of each
block (zeroes elsewhere) and $M_n$ the off-diagonal entries. Then $M_s$ is diagonal (or for general $M$
diagonalizable), or \textbf{semisimple} and $M_n$ is nilpotent and $[M_sM_n]=0$ (they commute).

\begin{defn}
	$M=M_s+M_n$ is called the \textbf{Jordan decomposition of $M$.} We call $M_s$ and $M_n$ the \textbf{semisimple} 
	and \textbf{nilpotent} parts of $M$.
\end{defn}
\begin{rmk}
	Each are uniquely determined for $M$ and (and this is not generally proved in the first year 
	algebra sequence) each are equal to polynomials in $M$ with no constant term.x`
\end{rmk}

\section{January 18, 2019}
\begin{prob}
	The ``board problem'' for this week: Show that a semisimple Lie algebra over $\C$ is generated as a Lie
	algebra by two elements.
\end{prob}

\subsection{Returning to Jordan Decomposition}
Given an $n\times n$ over an algebraically closed field $k$ of any characteristic, let $T$ be the corresponding linear
transformation on $k^n$. We know from the proof of the Jordan canonical form that 
\[V=k^n=\oplus_{a\in k}\ker(T-aI)^n\]
the sum of the \textbf{generalized eigenspaces} of $T$ on $V$. We then use the CRT:
there is a polynomial $p(T)$ such that $p(T)=a\pmod (T-a)^n$ for all eigenvalues $a$ and
$p(T)=0\pmod{T}$.

This is clear since the eigenvalues are distinct, so that all moduli are pairwise relatively prime.
SO then $p(T)$ acts on $\ker(T-A)^n$ so it is \textbf{semisimple} (diagonal) and $T-p(T)$, acting by $T-a$ on
$\ker(T-a)^n$, acts nilpotently.

Hence $p(x)$ and $x-p(x)$ are polnomials in $x$ without constant term such that as matrices they commute
and $x$ is semisimple (diagonal) and $x-p(x)$ is nilpotent. Moreover, tere is only one way to write $x$ as $s+n$ where $s$ is semisimple, $n$ is nilpotent, and $[sn]=0$.
This is since given $x=s+n$ where $s,n$ both commute with $x$, so with $p(x)$ and $x-p(x)$, we get
\[p(x)-s+(p(x)-x)+n\]
(check this) with the left side the sum of \textbf{commuting} semisimple matrices is semisimple and
the right side is the sum of \textbf{commuting} nilpotent matrices whence nilpotent, but the only matrix
which is both semisimple and nilpotent is the zero matrix.
\begin{rmk}
	Notice that commutativity above is vital to the statement.
\end{rmk}

Why is this important? We want to study the representations of Lie algebras, even the linear ones, and we would like to know that the semisimple and nilpotent parts $x_s, x_n$ of
any $x$ lying in such an algebra \textbf{continue} to act semisimply and nilotently in the representation 
(and that the parts $x_s$ and $x_n$ lie in L if $x$ does, which is true when $L$ is semisimple).

We have already seen that if $d$ is diagonal, then 
<FILL THIS IN>

Now we want to head towards a criterion in terms of traces alone for a matrix to be nilpotent.
Eventually we want to determine when a Lie algebra is solvable.

\begin{lem}
	Suppose that $k$ is algebraically closed of characteristic zero and that $B\subseteq A$ are subspaces of $\gl(n,k)$ and that $x(\in M?)$ is a matrix such that 
	$\tr(xy)=0$ for all $y\in M=\{z\in\gl(n,k)|[zA]\subseteq B\}$. Then $x$ is nilpotent.
\end{lem}
\begin{prf}
	Write $x=s+n$, the Jordan decomposition. Since $x\in M, s,n\in M$ since $s$ and $n$ are polynomials in 
	$x$ without constant term. We must show that $s=0$ -- that all eigenvalues of $s$ are zero.

	Let $a_1,\dots,a_n$ be the eigenvalues of $s$. These all lie in $k$, so they span a finite-dimensonal $\Q$-subspace $E$
	of $k$. To see this, we will show that any $\Q$-linear function $f:E\to\Q$ must be zero. Given $f$, 
	choose a polynomial $r(s)$ such that $r(a_i-a_j)=f(a_i)-f(a_j)$ for all $i$ and $j$ (using the Lagrange interpolation theorem).

	This is consistent since $a_i-a)j=a_k-a_l$ (using the linearity of $f$) implies the image of these under $r$ are equal.
	Then looking at $r(\ad s)$, we find that its eigenvalues are $r(a_i-a_j)$ as well as $f(a_i)-f(a_j)$,
	whence $\sum f(a_i)a_i=0=\tr(sy)$ for asuitable $y$.

	Applying $f$, we get $\sum f(a_i)^2=0$ whence all $f(a_i)=0$ (since $f(a_i)\in\Q$) and $f=0$, as desired.
\end{prf}

\subsection{Cartan's Criterion}
How can we turn this into something useful for us in the Lie algebra world?
\begin{thm}[Cartan's Criterion]
	Let $L$ be a linear Lie algebra over (alg. closed and char zero) $k$. Suppose that $\tr(xy)=0$
	for all $x\in L$ and $y\in[LL]$. Then $L$ is solvable. 
	
	As a consequence, suppose that $L$ is a Lie algebra such that 
	$\tr((\ad x)(\ad y))=-$ for all $x$ and $y$ above. Then $L$ is solvable.
\end{thm}
\begin{rmk}
	Notice that the consequence really does follow from the theorem: if $\tr((\ad x)(\ad y))=0$
	then $\ad L\cong L/Z$ is solvable where $Z$ is the center of $L$ whence $L$ is also solvable
	since $Z$ is abelian.

	To prove that $L$ is solvable, under our hypothesis, it is enough to show that $[LL]$ is nilpotent, 
	but (due to Engel's theorem) it suffices to show that $[LL]$ consists of nilpotent matrices.
\end{rmk}
To prove this criterion, we need a fact about traces (the \textbf{associativity of traces}):
\[\tr([xy]z)=\tr(x[yz]).\]
\begin{prf}
	We know that $\tr(xy)=0$ for all $x\in L$, $y\in [LL]$ and any such $x$ maps $L$ to $[LL]$ under $\ad$.
	We would like to know that for any $x\in M$, that $\tr(xz)=0$ for all  $x\in[LL]$. We know this is true if $z\in L$
	so by the associativity of trace, this follows.

	Returning to the Jordan decomposition, let $A$ be any finite-dimensional algebra over $k$ (not necessarily associative)
	and let $d$ be a derivation of $A$. THen the semisimple parts $d_s$ and $d_n$ of $d$ are also derivations.

	To see this, we write $A$ as the direct sum of the generalized eigenspaces $A_a$ of itself under the action of $d$. Since $d$ is a derivation,
	it ends up becoming clear that $A_aA_b\subseteq A_{a+b}$, so that the transformation from $A$ to itself action by $a$ on $A_a$ is a derivation.
	But this is $d_s$. Then $d_n=d-d_s$ is also a derivation.
\end{prf}

\section{January 23, 2019}
Some clarifications from last time:
\begin{lem}
	Let $A\subseteq B$ be subspaces of $\gl(n,l)$ for some algebraically closed field of char zero.
	Let $M=\{x\in\gl(n,k)|[xB]\subseteq A\}$. Suppose that $x\in M$ such that $\tr(xy)=0$ for all $y\in M$.
	Then $x$ is nilpotent.
\end{lem}
\begin{rmk}
	This used the fact that if $x=x_s+x_n$ then $\ad x=\ad x_s+\ad x_n$ is the Jordan decomposition for $\ad x$.

	Furthermore, when applying this lemma to Cartan's criterion for sovability of Lie subalgebras of $\gl(n,k)$, 
	we claimed that, if $\tr(xy)=0$ for all $x\in L$ and $y\in [LL]$, then $L$ is solvable. This follows by our lemma
	since if $x\in M=\{z\in\gl(n,k)|[zL]\subseteq[LL]\}$ and if $y,z\in L$ then $\tr(x[yz])=\tr([xy]z)=0$ by hypothesis.

	So by the lemma $[LL]$ consists of nilpotents, so is nilpotent and thus solvable.
\end{rmk}

\begin{defn}
	If $L$ is any Lie algebra over $k$, then $\kappa(x,y)=\tr((\ad x)(\ad y))$, is a bilinear 
	form on $L$ and is called the \textbf{Killing form.} 

	It is symmetric and associative ($\kappa([xy],z)=\kappa(x,[yz])$).
\end{defn}
\begin{rmk}
	Now in the context of \textit{general (as opposed to linear) Lie algebras} we can restate
	Cartan to say:
\end{rmk}
\begin{thm}[Cartan Redux]
	If $\kappa(x,y)$ for all $x\in [LL]$ and $y\in L$, then $L$ is solvable.
\end{thm}

\subsection{Semisimple Lie Algebras}
What we are interested in, actually, is the polar opposite of solvable algebras: namely,
\textbf{semisimple} Lie algebras, having no nonzero solvable ideals.
\begin{thm}
	$L$ is semisimple if and only if the Killing form $\kappa$ is nondegenerate.
\end{thm}
\begin{rmk}
	This sort of naturally follows from our discussion above -- we want the opposite property for our algebra
	and we require that the form be as far from zero as possible.
\end{rmk}
\begin{rmk}
	If $I\lhd L$, we note that, if $\kappa_I$ is the Killing form for $I$ and $\kappa$ the Killing form for
	$L$, then $\kappa_I=\kappa|_{I\times I}$. This is because if $x,y\in I$ then $\ad x$ and $\ad y$ both map $L$ into $I$.
	Thus if $M$ is any $n\times n$ matrix over $k$ mappting $V=k^n$ to a proper subspace $W$, then $\tr M=\tr M|_W$, since 
	extending a basis from $W$ to $M$ is such that the added basis vectors make no contribution to the trace of $M$.
\end{rmk}

\begin{prf}
	If $\kappa$ is nondegenerate and $S$ is a solvable ideal of $L$, then $\kappa|X=\kappa S$ has
	\textbf{radical} $\{z\in S:\kappa(z,y)=0,\forall y\in S\}$ which is at least $[SS]$, and then $K$ would have a 
	radical as well.

	More precisely, if $S$ is a solvable ideal of $L$ then there is a nonzero abelian ideal $A$ of $L$
	(the last nonzero term in the derived series for $L$) and then if $x\in A$ and $y\in L$, then $\ad x\ad y$ maps $L$
	into $A$ and $(\ad x\ad y)^2=0$, so $\kappa$ has radical at least $A$.

	Conversely, if $L$ is semisimple, then the radical of $\kappa$ as defined aobve would be a solvable ideal of $L$,
	by associativity and Cartan's criterion, so $\operatorname{Rad}(\kappa)=0$.
\end{prf}

The nondegeneracy of the Killing form is \textbf{the key} result that makes the classification
of semisimple Lie algebras tractable. For instance, in the case of finite simple groups the classification required 
10,000 pages of journal papers and it boils down to not having any kind of simple criterion like this.

\subsection{Applications}
As a first step, we show 
\begin{thm}
	A Lie algebra $L$ is semisimple if and only if it is a finite direct sum of simple ideals $I$.
\end{thm}
\begin{prf}
	This is clear since if $I$ is an ideal then $I^\perp=\{y\in K:\kappa(y,x)=0\forall x\in I\}$
	is again an ideal of $L$ and $I\cap I^\perp=0$ by Cartan. Then we can keep doing this, pulling off minimal ideals.
\end{prf}

As a second application, we will show that ever derivation $d$ of a semisimple Lie algebra $L$ is inner.
\begin{prf}
	$L$ is semisimple, so $L=[LL]$ and $Z(L)=0$, so $\ad L\cong L$. Now if $D=\operatorname{Der} L$, then
	$L\lhd D$ (as in homework). If $L\ne D$, then take the orthogonal of $L$ under the killing form in $D$.

	Thus $L\cap L^\perp=0$ and $\dim L^\perp\ge\dim D-\dim L$, but $[LL^\perp]=0$, which says exactly that if $m\in L^\perp$, then $m(x)=0$
	for all $x\in L$, so $L^\perp=0$ and $L=D$.
\end{prf}

Next we can extend the Jordan decomposition to semisimple Lie algebras. We have already seen that if $L$ is semisimple then $\Der L$ is closed
under Jordan decomposition but $\Der L=L$, so any $x\in L$ can be written as $x_s+x_n$ \textbf{FOR $x_s,x_n\in L$ (important)} where $\ad x_s$ and $\ad x_n$
act semisimply and nilpotently on $L$ and that $[x_sx_n]=0.$

\begin{rmk}
But what if $L$ is a semisimple linear Lie algebra over $k$? Then if $x\in L$, we have  as above $x_s,x_n\in L$
but also $x=s+n$ for some $s,n\in\gl(n,k)$ where $s$ is semisimple and $n$ is nilpotent and $[sn]=0$. Then we can ask:
do $s$ and $n$ lie in $L$? The answer is yes, but requires more machinery Once we know this, there are immediate applications to 
representation theory of $L$ (maps $\varphi:L\to\gl(n,k)$):

Given any such representation $\varphi$ and $x\in L$ with $x=x_s+x_n$, we get that $\varphi(x_s)$ and $\varphi(x_n)$
\textbf{continue to act semisimply and nilpotently} on $k^n$. This is what allows us to get a real handle on the representation theory.
\end{rmk}

\begin{thm}
	Given any \textbf{finite dimensional} module $M$ over a finite dimensional $M$ over a semisimple Lie
	algebra $L$ is completely reducible.
\end{thm}
\begin{rmk}
	We will prove this next time, using also certain constructions on modules as well as the matrix
	representations of elements.

	This result ends up being extremely useful not only in the (more obvious) representation theory of $L$, 
	but also in the structure theory of $L$ by considering the adjoint action of $L$ on itself.
\end{rmk}

\section{January 25, 2019}
\subsection{Weyl's Theorem}
TOday we'll prove Weyl's theorem:
\begin{thm}[Weyl]
	Any finite dimensional module over a semisimple Lie algebra is complete reducible ( a finite direct sum of simple modules).
\end{thm}
To prove this, we need some preparation: namely a couple of methods of producing new moduels over any Lie algebra $L$ from old ones.

First of all, if $M,N$ are $L$-modules, then $M\otimes_k N$ is an $L$ module of dimension the product
of the dimensions of $M$ and $N$. Here we define the action
\[x\cdot(m\otimes n)=x\cdot m\otimes n+m\otimes x\cdot n\]
where we can quickly check that 
\[[xy]\cdot(m\otimes n)=x(y(m\otimes n))-y(x(m\otimes n)).\]

Similarly, $\Hom_k(M,N)$ is an $L$ module via
\[(xf)(v)=xf(v)-f(xv)\]
and again we can check this defines a module structure, noting that $xf=0$ for all $x\in L$ if and only if $f$ is an $L$ module homomorphism.

\begin{lem}[Shur's Lemma]
	If $M$ is an irreducible $L$ module, then the only $L$ module maps in $\operatorname{End}(M)$ are scalars.
\end{lem}

Last we need
\begin{defn}
	We have the \textbf{trace form} on $L$:
	\[(x,y)=(\tr\varphi(x))(\tr\varphi(y))\]
	and by associativity and Cartan's criterion, we know (assuming that $L$ acts faithfully) that $(-,-)$ is nondegenerate.
\end{defn}
Recall that given any basis $x_i$ of $L$, we can define the dual basis $y_i$ such that $(x_i,y_j)=\delta_{ij}$.
\begin{defn}
	The \textbf{Casimir element} attached to a homomorphism $\varphi:L\to\gl(n,k)$, for some $n$, to be the matrix
	\[c_\varphi=\sum_1^m\varphi(x_i)\varphi(y_i).\]
\end{defn}
\begin{rmk}
	This matrix commutes with $\varphi(x)$ for all $x\in L$. This can be done by showing $a_{ij}=-b_{ji}$

	Then in computing $[xc_\varphi]$, we find that the coefficients of $\varphi(x_i)\varphi(y_j)$
	is $a_{ij}+b_{ji}=0$. Thus $c_\varphi$ commutes with the matrices of $\varphi(L)$.

	Hence if $V=k^n$ ws te module being acted upon and if $V$ is irreducible, then $c_\varphi$ must be a scalar (Schur).
	What is this scalar? Well since $\tr c_\varphi=m=\dim L$, we get
	\[\frac{\dim L}{\dim V}I=\frac{m}{n}I.\]
\end{rmk}
\begin{prf}
	The proof here will basically follow Humphreys, but will include a comment that \textit{should have appeared} in the text
	(much to Monty's chagrin).

	Notice that we must show that every submodule $N$ of $M$ has a complement $N'$ so that $M$ is the direct sum of $N$ and $N'$ and $N'$ is a submodule.

	For the rest of it, peek at the book. :)
\end{prf}
\begin{rmk}
	This result fails for semisimple Lie algebras in positive characteristic, although we won't really be dealing with this much.
	This parallels the fact that Maschke's theorem fails when $\ch k$ divides the order of $G$.
\end{rmk}
\subsection{Structure}
We now consider the simples case of a semisimple Lie algebra, namely $\sl(2,k)$. Here we know that we have the basis
$h=e_{11}-e_{22}$, $x=e_{12}$ and $y=e_{21}$ where
\[[hx]=2x\quad [hy]=-2y\quad [xy]=h\]

Next time we will see that the diagonal matrix $hin L$ continues to act diagonally on \textit{any} finite dimensional representation of $L$.
Thus the first thing to look at here is how the representation $V$ decomposes as a sum of its eigenspaces
\[V=\oplus_{\lambda\in k}V_\lambda\]
but since $V$ is assumed to be finite dimensional, this is actually a finite sum. In particular, there is $\lambda$ such that 
$V_{\lambda+2}=0$.
\begin{defn}
	We call the $\lambda$ where $V_\lambda\ne 0$ \textbf{weights} of $V$.
\end{defn}
So there is a weight $\lambda$ such that $\lambda+2$ is not a weight. In general, due to the relations above on $L$,
\[xV_\mu\subseteq V_{\mu+2}\quad yV_\mu\subseteq V_{\mu-2}\]
and we call $x$ and $y$ (or their matrices) the \textbf{raising and lowering operators}.

Then we can pick $\lambda$ a weight such that $\lambda+2$ is not and pick $v_0\in V\setminus 0$. Then set
\[v_i=\frac{y^iv_0}{i!}\]
so 
\[hv_i=(\lambda-2i)v_i,\quad yv_i=(i+1)v_{i+1}\quad xv_i=(x-i+1)v_{i-1}\]
where this can be proved by induction.

But the nonzero vectors $v_i$ are independent, so there is a least $m$ with $v_m\ne 0$ and $v_{m+1}=0$. 
The last equation with $i=m+1$ then forces $\lambda-m=0$, so $\lambda=m$ is a nonnegative integer!

It follows that if $V$ is irreducible that $\dim V=m+1$ and $v_0,\dots,v_m$ are a basis for $V$.

This ends up capturing the situation with \textit{any} irreducible representation of $\sl(2,k)$. We get integer eigenvalues, all of the same parity, from $-m$ to $m$.

\section{January 28, 2019}
We'll start off with tying up some loose ends from last time: 
\begin{lem}
	Let $L$ be a semisimple subalgebra of $\gl(n,k)$ (once more, $k=\bar k$ and $\ch k=0$). Let $x\in L$. Then 
	the semisimple and nilpotent parts $x_s$ and $x_n$ of $x$ lie in $L$. Thus the Jordan decompositions of $x$ 
	(as a matrix and as an element in $L$) coincide.
\end{lem}
\begin{prf}
	To see this, recall that we know that $\ad x_x$ and $]ad x_n$ are polynomials with zero constant term in $\ad x$ and
	so bracket $L$ into $\gl(V)$. So if $x_s,x_n\in N=\{y\in\gl(n,k):[yL]\subseteq L\}$ and $L$ is an ideal of $N$.

	We want to show that $L-N$ but this is false! we need to get the scalar matrices out. So replace $N$ by those elements 
	such that if $y\in N$ then $y$ restricted to any irreducible submodule $k^n$ of $L$ has trace zero.

	We have $L\subseteq N$... (snip)

	So by Weyl's thoerem we can write $N=L\oplus M$ as an $L$ module. But $L$ is an ideal of $N$ and $[LM]=0$, so the matrices in $M$
	all commute with $L$. By Schur's lemma, they act by scalars on any irreducible submodule $W$ of $k^n$, which must be zero, by the trace condition.

	Since $k^n$ is the direct sum of irreducible modules, we get $M=$, so $L=N$.
\end{prf}

Hence we have fully justified our characterization of irreducible $(m+1)$-dimensional $\sl(2,k)$ modules $V$:
\begin{itemize}
	\item We always have a basis $v_0,\dots,v_m$ with $v_{-1}=v_{m+1}=0$ such that:
	\item $hv_i=(m-2i)v_i$
	\item $yv_i=(i+1)v_{i+1}$
	\item $xv_i=(m-i+1)v_{i-1}$
\end{itemize}
we can check that the bcracket relations are satisfied starting from this as a definition.
\begin{rmk}
	In fact, if you replace $m$ by any $\lambda\in k$ and let all $v_i$'s form a basis for all non-negative $i$ (that is easing
	the restriction $v_{m+1}=0$), you get an infinite dimensional $L$-module. Whenever $\lambda\in\N$, this module is \textbf{not}
	completely reducible.
\end{rmk}
\subsection{One more $\sl(2,k)$ Computation}
\[(\exp\ad ax)(\exp\ad hy)(\exp\ad ax)=\begin{pmatrix}
	1+ab & a(2+ab)\\
	b & 1+ab
\end{pmatrix}=\begin{pmatrix}
	0 & -b^{-1}\\
	b & 0
\end{pmatrix}\]
if $ab=-1$. Multiplying by its transpose, we get
\[\begin{pmatrix}
	b^{-2} & 0 \\
	0 & b^2
\end{pmatrix}\]
So whenever $K$ is algebraically closed, we get a set of generators of $\sl(2,k)$ as products of matrices in $\Int L$ which is therefore 
isomorphic to $\operatorname{PSL}(2,k)$.

The action of $L$ on any irreducible module $V$ always lifts to $\operatorname{SL}(2,k)$ and lifts to $\operatorname{PSL}(2,k)$ if and only if $\dim V$ is odd.

\brk

The moral to draw from our classification of $\sl(2,k)$ moduoes of finite dimension is that the weights ($h$-eigenvalues) always occur in strings
of integers. In particular the weight $i$ occurs overall with the same multiplicity (eigenspace dimension) as $-i$.

Also for an irreducible module either the weight zero or the weight 1 occurs with multiplicity one (but not both).

\subsection{Return to semisimple Lie algebras}
Consider decomposing such an $L$ as a module over itself via the adjoint representation. If $L$ had only nilpotent elements, it itself would be nilpotent (it isn't)
so $L$ has a \textbf{toral subalgebra} $T$ consisting of only semisimple elements.

Then, in fact, $T$ is abelian, since otherwise we would have $[xy]=ky$ for $x,y\in T$ and $k\ne 0$.
But on the other hand, since $\ad y$ is semisimple, but on the other hand $[yx]$ is a sum of eigenvectors of $\ad y$
each with nonzero eigenvalues, a contradiction.

\subsection{Decomposing $L$}
Now let's fix a \textbf{maximal toral subalgebra} $\h$ (that is, maximal among toral subalgebras). Then we have
\[L=\oplus_{\alpha\in\h^*}L_\alpha\]
where $L_\alpha=\{x\in L:[xh]=\alpha(h)x\}$, since any commuting set of semisimple (diagonalizable) matrices are simultaneously diagonalizable.

Now by Jacobi, $[L_\alpha L_\beta]\subseteq L_{\alpha+\beta}$ and bringing back in the Killing form,
\[\kappa(L_\alpha,L_\beta)=0,\quad\text{if}\quad \alpha+\beta\ne 0\]
so that $\kappa$ is nondegenerate on $L_0$:
\[\kappa(L_0,L_\alpha)=0\quad\forall \alpha\ne 0.\]

\begin{lem}
	$H=L_0$, or equivalently, $H$ is its own centralizer.
\end{lem}

The proof can be found in the book, but using this, we can rewrite our decomposition as
\[L=\h\oplus\bigoplus_{\alpha\in\h^*}L_\alpha\]
and this is called the \textbf{root space decomposition} of $L$ in terms of $\alpha\in\h^*$.
The terms $\alpha\in\h^*$  occurring above are called the \textbf{roots of $L$} (relative to $\h$). By definition \textbf{no root is zero.}

Notation: Write $\Phi$ to be the set of all (nonzero) roots in $\h^*$.

Since the restriction of $\kappa$ to $\h$ is nondegenerate, we get an identification of $\h^*$ with $\h$:
\[\alpha\in\h^*\Leftrightarrow t_\alpha\in \h\]
where $\alpha(h)=\kappa(t_\alpha,h)$. Moreover we can transfer $\kappa$ to $\h^*$:
\[(\alpha,\beta)=\kappa(t_\alpha, t_\beta)\]
which gives us a symmetric bilinear form on $\h^*$.

\brk

\textbf{Next Time: } we will place the form on a real vector space of dimension $\dim_k\h$. Best of all, 
$(-,-)$ will be positive definite, giving us a nice orthonormal basis.

\section{January 30, 2019}
Let $L$ be a semisimple Lie algebra, $\h$ a maximal toral subalgebra. Last time we saw the root space decomposition above. Note
we actually have $L=\h\oplus \bigoplus_{\alpha\in\Phi}L_{\alpha}$ where $\Phi\in\h^*$ are the values that give you nontrivial weight spaces,
called the \textbf{root system} of $L$ (relative to $\h$) with elements called \textbf{roots.} It ends up we can embed these root systems in
a Euclidean space, giving us a context in which to write some relatively simple axioms.

\subsection{Root Systems}
First,
\begin{lem}
	$\Phi$ spans $\h^*$.
\end{lem}
\begin{prf}
	Othersiwe, some nonzero $h\in \h$ would have $\alpha(h)=0$ for all $alpha\in\Phi$, but then $h$ would be in the center of $L$, a contradiction.
\end{prf}
\begin{lem}
	If $\alpha\in \Phi$, then $-\alpha\in\Phi$.
\end{lem}
\begin{prf}
	This is due to the fact that otherwise $L_\alpha$ would lie in the radical of the Killing form. Thus
	$[L_\alpha L_{-\alpha}]=\kappa t_\alpha$ where $\alpha(g)=\kappa(t_\alpha,h)$. Then you can follow the text.
\end{prf}
The takeaway from the discussion, which I admittedly slacked on typing, was that in considering $L_\alpha$ and $L_{-\alpha}$ we find that the structure
has to be precisely the structure of $\sl(2,k).$ 

\brk

Then $L$ becomes a module for $S_\alpha\cong\sl(2,k)$ under the adjoint action. Look at the submodule
\[M_\alpha=\h\oplus\bigoplus_{c\in k^*}L_{c\alpha}\]
Clearly $M_\alpha$ is stable under the $S_\alpha$ action. Weights of $h_\alpha$ are integers, so we know that 
$L_{c\alpha}=0$ if $c\notin\Z\cup\frac{1}{2}\Z$.

But also all occurrences of the weight zero come from $\h$, a codimension one space, namely $\ker\alpha$, dies under the action of $x_\alpha$ and $y_\alpha$.
The only other occurrence is $h_\alpha$ itself, which leads to $x_\alpha$ on the action $x_\alpha$ then dies (what)? Thus the weights occurring are $0,\cdots,0,2,-2$.

So the takeaway is that the weight 4 doesn't occur and $L_{2\alpha}=L_{\alpha/2}=0$, so the weight 1 never occurs.
Finally we get that $L_{c\alpha}=0$ if $c\ne \pm 1$ and $\dim L_{\alpha}=1$.

\brk

By the way, we also know that $[L_\alpha L_\beta]=L_{\alpha+\beta}$ by the way $x_\alpha$ and $y_\alpha$ act on any irreducible $S_\alpha$ module.

Finally we also know that if $\alpha$ and $\beta$ are roots, then 
\[\beta-\beta(h_\alpha)\alpha=\beta-\frac{2(\alpha,\beta)}{(\alpha,\alpha)}\alpha\]
is a root as well and $\frac{2(\alpha,\beta)}{(\alpha,\alpha)}\in\Z$. The equation above is just a reflection in Euclidean space.

\subsection{Rationality Properties}
We know that $\Phi$ spans $\h^*$: choose a basis $\alpha_1,\dots,\alpha_n$ of $\h^*$ consisting of roots. Then any root $\beta$ takes the form 
$\sum_i c_i\alpha_i$ for $\alpha_i\in k$. Apply $(-,\alpha_i)$ and divide by $(\alpha_i,\alpha_i)$, then we get
\[\frac{2(\beta,\alpha_i)}{(\alpha_i,\alpha_i)}=\sum_j c_j\frac{2(\alpha_j,\alpha_i)}{(\alpha_i,\alpha_i)}.\]

Finally we can do some row operations to find the unique solutions for the $c_j$ must lie entirely in $\Q$, so \textbf{all roots} lie in the $\Q$ subspace $E_\Q$
of $\h^*$ whose dimension over $\Q$ is $n=\dim_k\h^*=\dim_k\h$.

Moreover, for any $\alpha,\beta\in\h$,
\[(\beta,\alpha)=\kappa(t_\beta,t_\alpha)=\sum_{\gamma\in\Phi}\gamma(t_\alpha)\gamma(t_\beta)=\sum (\gamma,\alpha)(\gamma,\beta)\]
so in particular $(\beta,\beta)=\sum (\beta,\gamma)^2$ and dividing both sides by $(\beta,\beta)^2$ we get that $(\beta,\alpha)\in \Q$ for all roots and so more generally
for all $\alpha,\beta\in E_\Q$. Apply our formula one last time to see
that $(\lambda,\lambda)>0$ when $\lambda\ne 0$, so our form $(-,-)$ is symmetric, bilinear, and positive definite on $E_\Q$.

We know set up a formal vector space $E_\R$ as the $\R$ span of a basis of $\h^*$ fixed earlier consisting of roots. Then we know that
$\Phi\in\h^*$ satisfies the following properties:
\begin{enumerate}
	\item $\Phi$ spans $\R^n$
	\item If $\alpha\in\Phi$, the only multiples of $\alpha$ also in $\Phi$ are $\pm\alpha$.
	\item If $\alpha,\beta\in\Phi$, then $\beta-\frac{2(\beta,\alpha)}{)\alpha,\alpha)}\alpha\in\Phi$.
	\item $\frac{2(\beta,\alpha)}{(\alpha,\alpha)}\in\Z$ for all roots.
\end{enumerate}
\begin{defn}
	Any system $\Phi$ satisfying properties (a)-(c) above is an \textbf{(abstract) root system in $\R^n$}. If, furthermore,
	$\Phi$ satisfies $(d)$, we call it a \textbf{crystallographic root system.} In this case, the $\Z$-span of $\Phi$ is called the
	\textbf{root lattice} $\Lambda$, which is stable under all reflections $s_\alpha$ for $\alpha\in\Phi$.
\end{defn}
\begin{rmk}
	We note that for any root system $\Phi$ that the subgroup of $O_n(\R)$ generated by all $s_\alpha$ is a finite group $W$
	(finite since it acts on a finite set $\Phi$) called the \textbf{Weyl group.} $W$ consists of automorphisms of $\Phi$ -- that is,
	linear maps on $\R^n$ sending $\Phi$ to $\Phi$ and preserving the dot product.
\end{rmk}
We will classify (mostly) all finite groups $W$ arising in this way, both in the crystallographic and non-crystallographic cases, although we are primarily interested in the former.

\brk

We end with an example.

Look at $S_n$, the symmetry group on $n$ letters acting on $\R^n$ by coordinate permutations. Any coordinate flip, 
say the $i^{th}$ and $j^{th}$ coordinates, is the reflection $S_{e_i-e_j}$. These coordinate flips generate all of $S_n$, 
so $S_n$ is a Weyl group.

\newpage

\section*{Homework 2 -- Due Feb 1}
%4.5; 7.4,6,7; board problem: show that a semisimple Lie algebra is generated as a Lie algebra by two elements
\begin{hwprob}{4.5}
	If $x,y\in\End V$ commute, prove that $(x+y)_s=x_s+y_s$ and $(x+y)_n=x_n+y_n$.
	Show by example that this can fail if $x,y$ fail to commute. [Show first that $x,y$ semisimple (nilpotent) implies their sum is.]
\end{hwprob}
\begin{sol}
	Following the hint, we will first establish:
	\begin{lem}
		If $x$ and $y$ in $\End V$ commute and are semisimple (resp. nilpotent) then their sum $x+y$ is semisimple (resp. nilpotent).
	\end{lem}
	\begin{prf}
		Assume first that $x$ and $y$ are semisimple. Then $\ad x$ and $ad y$ can both be diagonalized. In particular, since they commute
		they can be diagonalized \textit{simultaneously} via some matrix $A$ to become diagonal matrices $D_x$ and $D_y$. But then their sum
		conjugated by $A$ is
		\[A[(\ad x)+(\ad y)]A^{-1}=A(\ad x)A^{-1}+A(\ad y)A^{-1}=D_x+D_y\]
		which is clearly diagonal, thus $x+y$ itself is semisimple.

		\brk

		Assume now that $x$ and $y$ are nilpotent, with $(\ad x)^\alpha=(\ad y)^\beta=0$. Then since
		\[[\ad x,\ad y]=\ad [xy]=\ad 0=0,\]
		we have that $\ad x=X$ and $\ad y=Y$ commute. So consider 
		\[(\ad(x+y))^{\alpha+\beta-1}=(X+Y)^{\alpha+\beta-1}=\sum_{i=0}^{\alpha+\beta-1}\binom{\alpha+\beta-1}{i}X^iY^{\alpha+\beta-1-i}=0\]
		where the last equality follows since the degree of each monomial is $\alpha+\beta-1$, which either forces the power of $X$ to be greater than $\alpha$
		or the power of $Y$ greater than $\beta$ (since if $i<\alpha,$ we get $\alpha+\beta-1-i>\beta-1$).

		Thus $x+y$ is nilpotent as well, completing the proof of the lemma.
	\end{prf}

	But then the statement itself follows simply by the uniqueness of the decomposition of $x+y$ into semisimple and nilpotent components: since 
	\[x+y=x_s+x_n+y_s+y_n=(x_s+y_s)+(x_n+y_n)\]
	is such a decomposition of $x+y$, these are \textit{the} semisimple and nilpotent components.
\end{sol}
\newpage
In the following three problems, let $L=\sl(2,k)$.
\begin{hwprob}{7.4}
	The irreducible representation of $L$ of highest weight $m$ can also be realized ``naturally''
	as follows. Let $X,Y$ be a basis for the two dimensional vector space $k^2$, on which $L$ acts as usual. Let $\mathscr{R}=k[X,Y]$
	be the polynomial algebra in two variables, and extend the action of $L$ to $\mathscr{R}$ by the derivation rule
	$z\cdot fg=(z\cdot f)g+f(z\cdot g)$. Show that this extension is well defined and that $\mathscr{R}$ becomes an $L$ module. Then show that
	the subspace of homogeneous polynomials of degree $m$, with basis $X^m,X^{m-1}Y,\dots,Y^m$ is invariant under $L$ and irreducible of highest weight $m$.
\end{hwprob}
\begin{sol}
	We simply extend the action linearly both for $\mathscr{R}$ and $L$ -- that is
	\[(\alpha x+\beta y) f=\alpha(x\cdot f)+\beta(y\cdot f)\]
	and
	\[x\cdot (\alpha f+ \beta g)=\alpha(x\cdot f)+\beta(x\cdot g)\]
	so by the linearity of derivations (and the bracket in $L$), all we have to check is that the bracket acts the right way under the derivation rule:
	\begin{align*}
		[xy]\cdot fg &= (xy-yx)\cdot fg\\
		&= xy \cdot fg - yx\cdot fg\\
		&= (xy\cdot f)g+f(xy\cdot g)-(yx\cdot f)g-f(yx\cdot g)\\
		&= (xy\cdot f - yx\cdot f)g + f(xy\cdot g - yx\cdot g)\\
		&= ([xy]\cdot f)g +f([xy]\cdot g)
	\end{align*}
	so this does in fact define an action on $L$. 

	Since the action is defined to be $k$-linear, we need only check that the $L$ action preserves the degree of a monomial in $X$ and $Y$.
	\[z\cdot X^\alpha Y^\beta = (z\cdot X^\alpha)Y^\beta+ X^\alpha(z\cdot Y^\beta)=\alpha(z\cdot X)X^{\alpha-1}Y^\beta+\beta(z\cdot Y)X^\alpha Y^{\beta-1}\]
	but since $z\in L$, $z$ takes $X$ and $Y$ to some linear combination of themselves (perhaps zero!), so the total degree of each of the
	above monomials is $\alpha+\beta$. Thus these elements span an $L$-module.

	Using $A=(\begin{smallmatrix}0&0\\1 & 0\end{smallmatrix})$ and $B=(\begin{smallmatrix}0&1\\0 & 0\end{smallmatrix})$ in $L$,
	we can get between any of these basis vectors since 
	\[A\cdot X^aY^b=aX^{a-1}Y^{b+1}\quad\text{and}\quad B\cdot X^aY^b=bX^{a+1}Y^{b-1}\]
	so this module is irreducible,

	Finally by considering the action of $C=(\begin{smallmatrix}1&0\\0 & -1\end{smallmatrix})$ on the basis, we see 
	\[C\cdot X^aY^b=aX^aY^b-bX^aY^b=(a-b)X^aY^b\]
	so the weights are $a+b=m,m-2, \dots, 2-m, -m$, so this is of highest weight $m$.
\end{sol}
\begin{hwprob}{7.6}
	Decompose the tensor product of the two $L$ modules $V(3)$ and $V(7)$ into the sum of irreducible submodules:
	$V(4)\oplus V(6)\oplus V(8)\oplus V(10)$. Try to develop a general formula for the decomposition of $V(m)\otimes V(n)$.
\end{hwprob}
\begin{sol}
	Using the fact that, if $v_i$ and $w_i$ are bases for $V(3)$ and $V(7)$, respectively, then $v_i\otimes w_j$ form a basis for $V(3)\otimes V(7)$ and that
	\begin{align*}
		h\cdot v_i\otimes v_j&=(h\cdot v_i)\otimes v_j+v_i\otimes(h\cdot v_j)\\
		&=(3-2i)v_i\otimes v_j+(7-2j)v_i\otimes v_j\\
		&=(10-2(i+j))v_i\otimes v_j
	\end{align*}
	Some quick analysis of the possible values $a_{ij}=10-2(i+j)$ can attain for $0\le i\le 3$ and $0\le j\le 7$
	tells us $-10\le a_{ij}\le 10$. Counting the number of $(i,j)$ pairs giving us each weight yields the list
	\begin{center}
		\begin{tabular}{c|c|c|c|c|c}
			0 & $\pm 2$ & $\pm 4$ & $\pm 6$ & $\pm 8$ & $\pm 10$\\\hline
			4 & 4 & 4 & 3 & 2 & 1
		\end{tabular}
	\end{center}

	Take $v_0\otimes w_0$, the basis element of highest weight in the tensor product. We have characterized irreducible $\sl_2$ modules,
	so we know that the $\sl_2$-span of this element must be a copy of $V(10)$, consisting of the $k$-span of a single vector in each weight space.
	
	I am certain one can find a vector $v$ in $L_8$ such that $y^9\cdot v=0$, but I had a difficult time finding one.
	So rather than demonstrating that each of these simple modules whose sum (whence direct sum) is all of $V(3)\otimes V(7)$, I appeal
	to theorem 5.2 in Humphreys (the Lie analogue of Maschke from group reps): since $V(10)$ is simple and we're over a characteristic zero field there is a submodules $W$ of $V(3)\otimes V(7)$ such that
	\[V(3)\otimes V(7)=V(10)\oplus W.\]
	
	Since $V(10)$ spans a one-dimensional subspace of each weight space, the dimension of each must decrease by one in $W$, giving us 
	dimensions (for weight spaces of $W$):
	\begin{center}
		\begin{tabular}{c|c|c|c|c|c}
			0 & $\pm 2$ & $\pm 4$ & $\pm 6$ & $\pm 8$ & $\pm 10$\\\hline
			3 & 3 & 3 & 2 & 1 & 0
		\end{tabular}
	\end{center}
	Then we can pick any nonzero vector in the highest-weight space $L_8$ and under $\sl_2$ we will generate a copy of $V(8)$.
	Iterating this process we get a copy of $V(6)$ and $V(4)$ until (by dimension considerations, if you like), the process terminates and
	\[V(3)\otimes V(7)=V(10)\oplus V(8)\oplus V(6)\oplus V(4).\]

	\brk

	After gaining some intuition for how these decompositions work in practice, I believe this general theory holds:
	\begin{conj}
		Let $m$ and $n$ be integers and let $V=V(m)\otimes V(n)$. Without loss of generality, assume that $m\le n$.
		Then 
		\[V\cong V(m+n)\oplus\cdots\oplus V(m+n-2\cdot k)\oplus\cdots\oplus V(n-m)\]
	\end{conj}
\end{sol}
\begin{hwprob}{7.7}
	In this exercise, we construct certain infinite dimensional $L$-modules. Let $\lambda\in k$ be
	an arbitrary scalar. Let $Z(\lambda)$ be the vector space over $k$ with countably-infinite basis $(v_0,v_1,\dots)$.
	\begin{enumerate}
		\item Prove that the formulas (a)-(c) of Lemma 7.2 define an $L$-module structure on $Z(\lambda)$, and that every nonzero $L$-submodule of $Z(\lambda)$
		contains at least one maximal vector.
		\item Suppose $\lambda+1=i$ is a nonnegative integer. Prove that $v_i$ is a maximal weight vector (e.g. $\lambda=-1$, $i=0$).
		This induces an $L$-module homomorphism $Z(\mu)\xrightarrow{\varphi}Z(\lambda)$ for $\mu=\lambda-2i$ sending $v_0\mapsto v_i$.
		Show that $\varphi$ is a monomorphism, and that $\im\varphi$ and $Z(\lambda)/\im\varphi$ are both irreducible $L$-modules
		(but $Z(\lambda)$ fails to be completely reducible when $i>0$).
		\item Suppose $\lambda+1$ is not a nonnegative integer. Prove that $Z(\lambda)$ is irreducible.
	\end{enumerate}
\end{hwprob}
\begin{sol}
	\textbf{Part (a):}
	
	To see that these relations define an $L$-module structure, we can rely on the fact that all vectors are
	\textit{finite} linear combinations of the $v_i$. This enables us to look ``locally'' at the $L$-module structure to confirm it 
	satisfies all required properties on a case-by-case basis.

	More specifically, for any $g\in L$ and any element $\sum_i c_i v_i\in Z(\lambda)$, take 
	\[S=\{i,g(i,k)|c_i,a^i_k\ne 0\}\] 
	where we define $g(i, k)$ and $a^i_k\in k$ to be such that 
	\[g\cdot v_i=\sum_k a^i_k v_{g(i,k)}\]

	Then $|S|<\infty$, so we can take $N=\max S$ and all $v_i$ that appear in the computation $g\cdot \sum c_i v_i$ are contained in 
	$V(N)$, so the $L$ structure follows from the $L$ structure there.

	\brk

	To see the last claim, notice that if $L'$ is a nonzero submodule of $Z(\lambda)$, then if $0\ne v_i\in L'$ then 
	since $x^{i+1}\cdot v_i=0$, there is some minimal $j$ such that $x^{j}\cdot v_i=0$. Then $v_{i-j+1}$ is a maximal
	vector.

	Furthermore examining the structure of $L'$, we get that we get $v_k\in L'$ for all $k\ge i$, which follows
	trivially from $y\cdot v_i=(i+1)v_{i+1}$.

	\textbf{Part (b):}

	Let $i=\lambda+1$ for some scalar $\lambda\in k$. Then by applying $x$, we get
	\[x\cdot v_i=(\lambda-i+1)v_{i-1}=(i-1-i+1)v_{i-1}=0\]
	so $v_i$ is a maximal weight vector in $Z(\lambda)$.

	Define $\varphi:Z(\lambda-2i)\to Z(\lambda)$ to be the map sending $v_0\to v_i$. Then we get
	\[\varphi(v_{1})=\varphi\left(y\cdot v_0\right)=y\cdot\varphi(v_i)=y\cdot v_i=(i+1)v_{i+1}\]
	and more generally 
	\begin{align*}
		\varphi(v_k)&=\varphi\left(\frac{1}{k!}y^k\cdot v_0\right)\\
		&=\frac{1}{k!}y^k\cdot v_i\\
		&=\frac{(i+1)(i+2)\cdots (i+k)}{k!} v_{i+k}\\
		&=\frac{(i+k)!}{i!k!}v_{i+k}\\
		&=\binom{i+k}{k}v_{i+k}
	\end{align*}

	But then the $\varphi(v_j)$ form a basis for the image (clearly) so if 
	\[\varphi\left(\sum c_k v_k\right)=\sum c_k\varphi(v_k)=0\]
	this implies that all $c_k=0$, so $\varphi$ in a monomorphism.

	Notice that $Z(\lambda)/\im\varphi\cong V(i-1)$, so is irreducible. Now since $\im\varphi\cong Z(\lambda-2i)$
	the problem reduces to showing the latter is irreducible. From (a) we know that any nonzero submodule of $Z(\mu)$ has a maximal vector $v_j$.
	Consider that in $Z(\mu)$
	\[x\cdot v_j=(\lambda-2i-j+1)v_{j-1}=0\]
	thus either $v_{j-1}=0$ (whence $j=0$ and this submodule is all over $Z(\mu)$) or else
	\[\lambda=2i+j-1=i-1\quad\Rightarrow\quad i+j=0\]
	but $v_{-i}=0$ for positive $i$, so this forces $i=j=0$, so the submodule is the entire module.

	\textbf{Part (c):}

	Assume now that $\lambda+1$ is not a nonnegative integer. Similar to above, any nonzero submodule has a maximum vector $v_j$.
	Then the equation we saw before
	\[2i+j-1=i-1\]
	has no solution, which forces $v_{j-1}=0$ and so $j=0$ and so the submodule is all of $Z(\lambda)$.
\end{sol}
\begin{hwprob}{(Board)}
	Show that a semisimple Lie algebra is generated (as a Lie algebra) by two elements.
\end{hwprob}
\begin{sol}
	Consider the decomposition (as a vector space) of any semisimple Lie algebra $\g$ with Cartan subalgebra $\h$ 
	into the sum of its root spaces
	\[\g = \h\oplus \bigoplus_{\alpha\in\Phi}\g_\alpha\]
	and chose $h\in\h$ and from each $\g_\alpha$ choose a nonzero $v_\alpha$. Define $v =h+\sum_{\alpha\in\Phi}v_\alpha$. I claim that 
	this vector, along with another vector generates everything. I'll keep thinking about this. :)
\end{sol}
\newpage

\section{February 1, 2019}
\begin{prob}
	This assignment's board problem: Show, using large tensor powers of the adjoint representation and $\sl_2$ theory,
	that any semisimple Lie algebra $L$ has irreducible modules of arbitrarily high degree.
\end{prob}

\subsection{Revisiting classical Lie Algebras}
In all cases the diagonal matrices form a maximal toral subalgebra with roots given by (relative the standard basis $e_i$ for $\R^n$):
\begin{itemize}
	\item $A_{n-1}$ is spanned by $e_i-e_j$ for $i\le i,j\le n$
	\item $B_{n}$ is spanned by $e_i-e_j$ and $e_i+e_j$ as well as $e_i$
	\item $C_{n}$ is spanned by $e_i-e_j$ and $e_i+e_j$ as well as $2ei$
	\item $D_{n}$ is spanned by (only) $e_i-e_j$ and $e_i+e_j$.
\end{itemize}

We can check in all cases that the axioms of a root system are satisfied: the differences, sums, and monomials act
on the roots by flipping, flipping and changing the sign, and just changing the sign, respectively.
In all cases, the root system is obtained from a lattice ($\Z$-span of an $\R$-basis) as follows:
\begin{itemize}
	\item $A_{n-1}$ the vector space is spanned by $e_1-e_2,\dots$ and the lattice is generated by the same vectors.
	Then the root system is realized as the set of such vectors of square length 2.
	\item $B_n$ we take the $\Z^n\subseteq\R^n$ vectors of square length 1 or 2.
	\item $D^n$ is the same, but just square length two.
	\item $C_n$ is a bit sketchy in terms of lengths -- we can't quite say all vectors of length 2 or 4. So instead
	we derive it from $B_n$ by replacing all $e_i$ by $2e_i$.
\end{itemize}
What are the Weyl groups of these sweet babies?
\begin{itemize}
	\item $A_{n-1}$: $S_n$ action on $\R^n$ by coordinate permutations. This is also the symmetry group of the unit simplex in $\R^n$.
	\item In $B_n$ and $C_n$ (reflections act the same way irrespective of magnitude) is the group of all permutations and sign changes 
	of the coordinates in $\R^n$. This can also be recognized as the hypercube or hyperoctahedron in $\R^n$ with vertices $\pm e_i$. This is 
	sometimes called the hyperoctahedral group.
	\item For $D_n$, this consists of all permutations \textit{even} sign changes of coordinates in $\R^n$. This has no 
	standard name. This can also be viewed as the group of symmetries of a polytope whose vertices are a subset of 
	those of a hypercube, but with \textit{evenly-many} negative. Again, there is no consensus about what to call these.
\end{itemize}
Later we will see the exceptional root systems (non-classical) and again they are taken from a lattice by taking all vectors of one or two specified square lengths.
\subsection{Some Non-Crystallographic Examples}
Here there are some $\alpha$ and $\beta$ where $\frac{2(\alpha,\beta)}{(\alpha,\alpha)}\notin\Z$.
\begin{ex}[$I_2(m)$]
	Start off with a regular $m$-gon in $\R^2$. The dihedral group of symmetries of this $m$-gon is not a Weyl group, 
	but it is a generalization: a \textbf{Coxeter group}: a finite group generated by reflections.

	To find the roots: if $m$ is odd, place the $m$-gon so its center is at $(0,0)$ and take $\pm v$ as $v$ runs over the vertices. If, instead,
	$m$ is even, center the $m$-gon at $(0,0)$ and take $\pm v$ for vertices $v$ along with $\pm 2m$, where the $m$ are 
	midpoints of edges.

	Why $2m$? Well in some special cases ($m=3,4,6$), this actually becomes a crystallographic system -- but not otherwise.

	There is also a degenerate case when $m=2$. Then the group is $S_2\times S_2$, generated by two orthogonal reflections.

	In all cases, the dihedral group is generated by just \textbf{two} reflections, corresponding to two roots which make as large an angle as possible.
\end{ex}

In general, the root system described in the above example is denoted $I_2(m)$ and the corresponding Coxeter group is
$D_m$, the dihedral group on an $m$-gon.

\subsection{Exceptional Root Systems}
\begin{ex}[$E_8$]
	The biggest such system is $E_8$. We begin with vector space $\R^8$ and a lattice spanned by all $e_i\pm e_j$
	as well as $(\frac{1}{2},\cdots,\frac{1}{2})$ and take all vectors in the lattice of square length 2.

	The lattice consists of all $(a_1,\dots,a_8$ where all $a_i\in\Z$ OR all $a_i\in \Z+\frac{1}{2}$ AND $\sum a_i\in 2\Z$.

	The roots are $e_i\pm e_j$ and vectors of the form $(\pm\frac{1}{2},\cdots,\pm\frac{1}{2})$ with \textbf{evenly many} $-\frac{1}{2}$. 
	There are 240 of them.
\end{ex}
Then we can define:
\begin{itemize}
	\item $E_7$ to be the set of vectors in $E_8$ orthogonal to $e_7+e_8$ (120 roots)
	\item $E_6$ to be the set of vectors in $E_8$ orthogonal to $e_7+e_8$ and $e_6+e_8$ (72 roots).
\end{itemize}

\begin{ex}[$F_4$]
	Vector space $\R^4$, with lattice spanned by $e_1,e_2,e_3,$ and $\frac{1}{2}(1,1,1,1)$.
	Then we get all $(a_i)\in\R^4$ with all $a_i\in\Z$ or all $a_i\in \Z+\frac{1}{2}$. Then we take all vectors of square length one or two.

	We get $\pm e_i$, $\pm(e_i\pm e_j)$ and $(\pm\frac{1}{2},\cdots,\pm\frac{1}{2})$ to get a total of 48 roots.
\end{ex}
\begin{ex}[$G_2=I_2(6)$]
	Vectors space is the one spanned by $e_1-e_2$ and $e_2-e_3$ in $\R^3$ and lattice is spanned by these and the roots 
	consist of these vectors as well as all vectors of square length 2 or six:
	\[(\pm (-2,1,1),\pm (1,-2,1),\pm (1,1,-2).\]
\end{ex}
\subsection{Positive and Negative Roots}
Given a root system $\Phi$, we can cut it in half by realizing $\R^n$ as an ordered space by saying $(v_1,\dots,v_n)$ is positive
if and only if the smallest index $i$ with $v_i\ne 0$ has $v_i>0$. Then the sum of two positive vectors is positive, as is a positive multiple.

Then define
\begin{defn}[Positive Roots]
	Set $\Phi^+=\Phi\cap\{\text{positive vectors}\}$. Call the roots in $\Phi^+$ and call $\Phi^+$ a \textbf{positive subsystem.}
\end{defn}

\section{February 6, 2019}
Recall from last itme that we were letting $\Phi$ be a root system with positive and negative roots $\Phi^+$ and $\Phi^-$ such that they partition $\Phi$,
are disjoint, and are closed under positive linear combinations.

Next we want ot exhibit a basis for $\R^n$, the ambient vector space, consising of positive roots such that
every root in $\Phi^+$ is a \textbf{positive} combination of basis vectors.

To do this, we already know that $\Phi^+$ has subsets $S$ (e.g. the whole thing) such that every root in $\Phi^+$ is a positive combination of things in $S$.
Let $\Delta$ be a minimal such subset under inclusion. I am having a slow morning -- take a look at the book. :)

\brk

The next claim is that $\Delta$ is independent. Otherwise, we have the relation $\sum_{\alpha\in\Delta}r_\alpha \alpha=0$ with
at least one nonzero $r_\alpha$. Moving all the negative coefficients over to the other side, we get the equality of two
(possibly empty) sums. If either side is trivial, this is a contradiction since $0\notin\Phi^+$.

If both are nonempty, then you can use the positivity of the inner product of one side with some $\beta$,
which then implies that the inner product of the other side with $\beta$ is negative -- again a contradiction.

\begin{defn}
	We call such a $\Delta$ a \textbf{simple subsystem} of $\Phi$.
\end{defn}

\begin{rmk}
	Notice that $\Delta$ is uniquely determined -- it consists exactly of those $\alpha\in\Phi^+$ that are \textit{not} positice combinations of two or more positive roots.

	Pick some $\alpha\in\Delta$, a simle root. The corresponding reflection $s_\alpha$ is also called simple. Then for any $\beta\in\Phi^+$, 
	$s_\alpha\beta$ is positive unless $\beta=\alpha$, as it must involve some simple root with a positive coefficent, which stays the same upon applying $s_\alpha$.

	Now let $\Phi^+$ and $(\Phi^+)'$ be two choies of positice roots, corresponding to two choices of positive vecors in $\R^n$. Let $\Delta,\Delta'$ be the simple subsystems
	corresponding two these two systems.

	If $\Phi^+\ne(\Phi^+)'$, there is some \textbf{simple} $\alpha\in\Delta$ which is \textbf{negative} in $(\Phi^+)'$. Applying $S_\alpha$ to $\Phi^+$, we get another positive
	system which contains the same elements of $(\Phi^+)'$ that $\Phi^+$ did, plus now $-\alpha$.

	Doing this repeatedly, we can bump up the size of the intersection $\Phi^+\cap(\Phi^+)'$ at each step, eventually yielding
	two equal positive sytems.

	So any two positive systems are conjugate under a product of simple reflections with respect to any system. That product is 
	an orthogonal transformation, so the angles between all pairs of simple roots don't depend on the choice of simple roots.

	Finally, if $\beta$ is a positive root, but not simple, then since it is a positive linear combination of simple roots, 
	$(\beta,\alpha)>0$ for some simple $\alpha$. Then $s_\alpha\beta=\beta-c\alpha$ for some $c>0$ is another positive root.
	Continuing this process must end with a simple root, so any positive root is a product of simple reflections applies to a simple root (in fact, any root)

	So the \textbf{entire root system} is completely determined by a knowledge of the square length of the simple roots and the angles between them.
\end{rmk}

\begin{rmk}
	In the crystallographic case, we can say more! Here
	\[s_\alpha\beta=\beta=\frac{2(\beta,\alpha)}{(\alpha,\alpha)}\alpha\]
	where the coefficient is in $\Z$! Here any positive root is a \textbf{positive integer combination} of simple roots.
\end{rmk}

\begin{rmk}
	Finally, what possible angles can occur between pairs of simple roots? If $\alpha,\beta$ are simple roots with simple reflections $s_\alpha,s_\beta$, then it must be that $s_\alpha s_\beta$
	must be an element of finite order lying in a finite group. Thus the angle must be a rational multiple of $2\pi$.

	In orderr to capture all positive roots that are combinations of $\alpha$ and $\beta$ that are \textbf{positive} combinations, the angle between $\alpha$ and $\beta$ msut be as wide as possible:
	$\pi-\frac{\pi}{m}$ for some $m$ (the order) if $\alpha\ne c\beta$.

	In the Crystallographic case, we get that $m=2,3,4,$ or $6$.
\end{rmk}
\subsection{Dynkin Diagrams}
Now we construct the \textbf{Dyinkin diagram} of our root system, if it is crystallographic, or its
\textbf{Coxeter graph} in general.

This is a graph in which the vertices correspond to simple roots $\alpha$. The vertices corresponding to $\alpha$ and $\beta$
are joined by an edge if $\alpha\not\perp\beta$. This splits into cases:
\begin{itemize}
	\item $m=3$, we take a single edge.
	\item $m=4$, take a double edge with arrow pointing toward the shorter root.
	\item $m=6$, we take a triple edge with the same condition as above.
\end{itemize}

In a Coxeter graph, we do the same except there is just one edge, labeled with $m$.

\begin{ex}
	Recall in the classical cases that a root is positive if its leftmost nonzero coordinate is positive. Then for $A_{n-1}$:
	\[\]
	I want to fill these all in later. For now, just including the notes:

	Notice that the diagrams for $B_2$ and $C_2$ are graph isomorphic (consisting of a single double edge!). This 
	tells us that, in fact, the Lie algebras are isomorphic. This relates one of the spin groups to one of the special linear groups although i forget which.
\end{ex}

\section{February 8, 2019}
Continuing from last time, we want to write down the simple roots that give rise to the Dynkin diagrams 
for the exceptional Lie algebras we saw (and I just realized I haven't copied in yet.)

\subsection{Exceptional Lie Algebra Dynkin Diagrams}
For $E_8$he roots are in the images I took. :)

For $E_7$ and $E_6$ you just remove the roots that we talked about last time.

For $H_3$ we take as the simple roots the edge midpoints of a regular dodecahedron or icosahedron.

For $H_4$, take the face centers of a 120-sided regular polytope in $\R^4$, called the hecatonicosahedroid. 

\subsection{Have we found them all?}
Now we want ot see that htis list of connected Dynking diagrams is complete! 
\begin{rmk}
	We say ``connected'' because if $\Phi$ and $\Phi'$ are two root systems in $\R^n$ and $\R^m$,
	then $\Phi\sqcup\Phi'$ is a root system in $\R^{n+m}$, the orthogonal direct sum. This gives two disjoint
	Dynkin diagrams (since they are orthogonal) so we restrict to the connected components.
\end{rmk}

We are actually going to classify the \textit{Coxeter graphs} corresponding to (crystallographic) Dynkin diagrams
starting with a simple subsystem of a crystallographic root system. We obtain a basis of $n$ unit vectors in $\R^n$ such that hte angle between any 
two of them is $\pi-\frac{\pi}{m}$ for $m\in\{2,3,4,6\}$, encoded by a coxeter graph where there is no edge if $m=2$, unlabelled if $m=3$ and unlabelled otherwise.

\subsection{Classifying Coxeter Graphs}
To classify these, we actually classify the graphs that are \textbf{not} of this type. That is, for each connected Coxeter graph
that we've seen so far, we are going to attach a slightly larger graph to it that \textit{doesn't occur} as a Coxeter graph of any root system
or, in fact, even a \textit{subgraph} of a Coxeter graph.

By section 10 of Humphreys, every connected Dynkin diagram and root system $\Phi$ corresponds to a unique lowest (highest) root.
Then every other root, if the lowest is subtracted, becomes a positive combination of simple roots and has nonpositive dot
product with the simple roots.

The resulting ``Dynkin diagram'' (it won't be) corresponding to the simple roots plus the lowest root 
comes from a \textbf{dependent} set of vectors -- in fact, one admitting a dependence relation with all
positive coefficients. It is not a Dynkin diagram since if it were, that combination of roots would have square length zero,
but this is impossible for any Dynkin diagram or subdiagram thereof. This is true \textit{even if some of the edge labels are increased.}

\subsubsection{What do we get when we do this?}
We call these the \textbf{extended} or \textbf{affine} Coxeter graphs.

For $\tilde A_{n-1}$ our lowest root is $e_n-e_1$, which attaches to both $e_1-e_2$ and $e_{n-1}-e_n$ and creates a cycle.

For $\tilde B_{n}$, we add $-e_1-e_2$ which links with the secondmost left root in $B_n$.

For $\tilde C_n$, we append $-2e_1$ to the leftmost root.

For $\tilde D_n$, we append $-e_1-e_2$ to the secondmost left root as in $\tilde B_n$.

\brk

For the exceptional algebras: $\tilde E_6$ gets $\frac{1}{2}(-1,\cdots,-1,1,1,-1)$. I got a picture of the rest.

The argument here can be found in Humphreys' \textit{Reflection Groups and Coxeter Groups} on page 37 that, once the forbidding configurations
above have been ruled out, the only connected diagrams left are the ones we've seen.

As a sketch, $B_n$ and $C_n$ are easy to see that the crystallographic property allows only two 
possibilities: that the rightmost simple root has square length \textbf{twice or half} the length of the other simple roots, but no other multiples.

In the type $E$ cases, we have one central vertex and three chains of lengths $p,q,$ and $r$ coming from it.
Let's consider the posibilities we get:
\begin{itemize}
	\item $\tilde E_6$: $(3,3,3)$
	\item $\tilde E_7$: $(2,4,4)$
	\item $\tilde E_8$: $(2,3,6)$
\end{itemize}
Notice that in all three cases the sum of the reciprocals of these values is 1! In fact, these are the only 
solutions to this this in the integers!

But then in the regular Dynkin diagrams:
\begin{itemize}
	\item $E_6$: $(3,3,3)$
	\item $E_7$: $(2,4,4)$
	\item $E_8$: $(2,3,6)$
\end{itemize}
Here we have the only solutions where the sum of the reciprocals are greater than 1 (with $p,q,r>1$). 
Well we actually have the case $(2,2,n)$, but these correspond to $D_{n+2}$.

In non-Euclidean geometry, this condition on $(p,q,r)$ arises all the time. The sum of reciprocals plays a role in the geometry of triangles.

\subsection{Finite subgroups of $SO(3,\R)$}
We actually played the $\frac{1}{p}+\frac{1}{q}+\frac{1}{r}>1$ game before in algebra!
\begin{itemize}
	\item $(2,2,n)$ is the group $D_n$
	\item $(2,3,3)$ is $T\cong A_4$
	\item $(2,3,4)$ is $O\cong S_4$
	\item $(2,3,5)$ is $I\cong A_5$, the icosahedral group.
\end{itemize}

The icosahedrom (or dodecahedron) pops up in the non-crystallographic Coxeter groups:
\begin{itemize}
	\item $H_3$ is all symmatries of an icosahedron or of a dodecahedron in $O(3,\R)$. This group is $A_5\times\Z_2$
	\item $H_4$ is a crazy picture, but the group is
	\[\frac{(A_5\times \Z_2)^2\ltimes \Z_2}{\langle \pm 1\rangle}\]
	which we call the Hecatonicosahedroidal group. :)
\end{itemize}

\section{February 13, 2019}
One final remark about the inequalities that came up: $\frac{1}{p}+\frac{1}{q}+\frac{1}{r}>1$ and $<1$ for $p,q,r\ge 2$.

\begin{rmk}
	if $\frac{1}{p}+\frac{1}{q}+\frac{1}{r}<1$, then the closest to 1 it can come is $\frac{1}{42}=\frac{1}{2}+\frac{1}{3}+\frac{1}{7}=1-\frac{1}{42}$.
	This actually comes up in enumerative algebraic geometry for exactly this reason. Of course, it is also the
	answer to the question of life, the universe, and everything as well. :)
\end{rmk}

We can also work out the coefficents of all simple roots in the highest root quite easily by induction:

Start with the extended Dynkin diagram: place the label 1 on the new vertex (highest root) and label
the others so that the corresponding combination of vectors has square length zero. This amounts to ensuring
that the sum of the labels of all vertices adjacent to agive one is twice the label of that vertex, where labels are counted doubly/triply
if they lie at the left end of a double/triple arrow.

In $\tilde A_{n-1}$, this forces labels all to be ones. In $\tilde B_n$, we get 1's at the ``forked'' end 
and twos elsewhere. $\tilde C_n$ has ones at the endpoints and twos elsewhere. $\tilde D_n$ has ones at the ends of both forks.
You can keep doing this for the exceptional ones but I will have to take a picture because they are hard to transcribe.

\brk

Now we go off ``in a completely different direction'': Let $\Phi$ e a crystallographic root system. Then there are two lattices attached to $\Phi\in\R^n$:
one is the \textbf{root lattice} spanned over $\Z$ by the roots -- denoted $R$. Another lattice is the \textbf{weight lattice} $P$
consisting of all $\lambda\in\R^n$ such that $\langle\lambda,\alpha\rangle=2(\lambda,\alpha)/(\alpha,\alpha)\in\Z$ for all $\alpha\in\Phi$.

Then $P$ and $R$ are free abelian groups of rank $n$. $P$ is generated by the \textbf{fundamental dominant weights} $\lambda_\alpha$
defined by $\langle\lambda_\alpha,\alpha\rangle = 1$ and $\langle\lambda_\alpha,\beta\rangle=0$
for $\alpha\ne\beta\in\Delta$.

Hence $P/R$ is a finite abeliang group, called the \textbf{fundamental group of $\Phi$}. Its importance is the following:
Complex semisimple Lie algebras are classified by their root systems. But what about complex semisimple Lie groups?
These are classified by their root system \textit{plus} a lattice lying between the root and weight lattices. Taking the root lattice 
leads to the adjoind group $\Int L$ (the smallest possible choice) if we take the weight lattice, then we get the simply-connected 
universal covering space of $\Int L$.

\begin{ex}
	For type $A_{n-1}$, $\Int \sl_n(\C)=PSL(n,\C)$. The simply-connected cover is $SL(n,\C)$ itself.
\end{ex}

\begin{rmk}
	When $k$ is not algebraically closed, we end up getting more forms to work with, but you can follow
	a similar technique to classifying the Lie groups. In positive characteristic things blow up pretty badly.
\end{rmk}

\brk

What did that have to do with the coefficients of the simple roots in the highest root? Well, in Humphreys exercise 13.13,
we see that every coset has a unique representative that is minimal in the sense that $\langle\lambda,\alpha\rangle=0$ or $1$ for all positive roots
$\alpha.$

It turns out that $\Phi^v$ (see Humphreys) is another root system called the \textbf{dual} to $\Phi$. Then if 
$\Delta$ is a simple subsystem for $\Phi$, then $\Delta v$ is a simple subsustem for $\Phi^v$. Then the condition
that $\langle\lambda,\alpha\rangle\in\Z$ is linear in $\alpha^v$.

\subsection{Takeaway}
We got a bit in the weeds for me, but the takeaway is that the fundamental groups are as follows:
\begin{itemize}
	\item $A_{n-1}$: $\Z/(n)$
	\item $B_n$: $\Z/(2)$
	\item $c_n$: $\Z/(2)$
	\item $D_n$: $\Z/(4)$ if $n$ is odd; $\Z_2\times\Z_2$ otherwise
	\item $E_6$: $\Z/(3)$
	\item $E_7$: $\Z/(2)$
	\item $E_8,F_4,G_2$: $\Z$
\end{itemize}

\begin{rmk}
	One more cool calculation (explained better in Humphreys' 1990 book): We can compute the order $|W|$:
	look at the orbit and stabilizer of the highest roots under the action of $W$: 

	The orbit consists of all roots with the same length as the heighest weight: the long roots.

	The stabilizer (discussed in Humphreys \S 10) is generated by all reflections fixing the highest root, corresponding to the roots \textbf{not}
	adjacent to the longest root in the extended diagram.
\end{rmk}

We have a final remark on root systems before returning to the land of Lie algebras:
\begin{rmk}
	Given a positive root $\beta$ in a root system $\Phi$ , we have seen that (if $\Delta$ is the corresponding simple subsystem)
	there is a product of simple reflections (attached to simple roots) sending $\beta$ to $\gamma$, a simple root.

	Thus there is a concatenated string
	\[\beta,\beta-\alpha,\cdots,\beta-\langle\beta,\alpha\rangle\alpha,\beta-\langle\beta,\alpha\rangle\alpha-\alpha',\cdots\]
	in which all terms of roots and the last root is simple. But then we can reverse this string to deduce that
	$\beta=\alpha_1+\alpha_2+\cdots+\alpha_m$ where all $\alpha_i$ are simple and every partial sum is a positive root.

	What we get from this, in terms of Lie algebras, is that one generator from each of the positive root spaces generates all the positive root spaces.
\end{rmk}

\newpage
\section*{Homework 3 -- Due Feb 15}
%10.10,13; 11.2; 13.4, board problem: using large tensor powers of the adjoint representation and sl(2) theory, show that any semisimple Lie algebra L admits irreducible representaitons of arbitrarily large dimension
\begin{hwprob}{10.10}
	Given $\Delta=\{\alpha_1,\dots,\alpha_l\}$ in $\Phi$, let $\lambda=\sum_1^lk_i\alpha_i$ with all $k_i\in\Z$ and all nonpositive or all nonnegative. Prove that
	either $\lambda$ is a multiple (possibly 0) of a root or else there exists $\sigma\in W$ such that $\sigma\lambda=\sum k_i'\alpha_i$ with 
	some $k_i'>0$ and some $k_i'<0$. 
	
	[Sketch: if $\lambda$ is not a multiple of any root, then the hyperplane
	$P_\lambda$ orthogonal to $\lambda$ is not included in $\cup_{\alpha\in\Phi}P_\alpha$. Take $\mu\in P_\lambda-\cup_\Phi P_\alpha$. Then find
	$\sigma\in W$ for which all $(\alpha_i,\sigma\mu)>0$. It follows that $0=(\lambda,\mu)=(\sigma\lambda,\sigma\mu)=\sum k_i(\alpha_i,\sigma\mu).$]
\end{hwprob}
\begin{sol}
	Assume that $\lambda$ is not a scalar multiple of any root in $\Phi$. As suggested above, this means that $P_\lambda$ 
	is not coincident with any $P_\alpha$ for $\alpha\in\Phi$. Let $\mu\in P_\lambda-\cup_\Phi P_\alpha$.

	Now by the theorem in section 10.3 (part a) gives us (since $\mu$ is regular) a $\sigma\in W$ such that $(\alpha_i,\sigma\mu)>0$ for all $i$.
	Then since $\lambda\perp\mu,$ (if $\sigma\lambda=\sum_i c_i\alpha$ where $c_i\in\Z$ since the integer lattice is fixed by $W$):
	\[0=(\lambda,\mu)=(\sigma\lambda,\sigma\mu)=\sum_i c_i(\alpha_i,\sigma\mu)\]
	which (since all the inner products are strictly positive) forces some $c_i>0$ and $c_j<0$, as desired.
\end{sol}

\begin{hwprob}{10.13}
	The only reflections in $W$ are those of the form $\sigma_\alpha$ $(\alpha\in\Phi)$.

	[A vector in the reflecting hyperplane would, if othogonal to no root, be fixed only by the identity in $W$.]
\end{hwprob}
\begin{sol}
	Let $\sigma\in W$ be an arbitrary reflection. Say that $\sigma$ is not a reflection through the hyperplane
	with normal vector $\lambda$ and let $\mu$ be a vector fixed by $\sigma$. If $\mu$ is not orthogonal to any root,
	it lies in the interior of one of the regions carved out by the hyperplanes $P_\alpha$ for $\alpha\in\Phi$. 
	Then by Lemma B in \S10.3, since $\mathfrak{C}(\Delta)$ is a fundamental domain for the $W$ action, let $\mu'=\sigma'\mu$
	be the (unique) representative of $\mu$ in $\mathfrak{C}(\Delta)$. 

	But then $\sigma'\sigma(\sigma')^{-1}(\sigma'\mu)=\sigma'\mu$, so by the above lemma $\sigma'\sigma(\sigma')^{-1}$
	is the identity map. But then 
	\[\sigma'\sigma(\sigma')^{-1}=\id\quad\Rightarrow\quad \sigma'\sigma=\sigma'\quad\Rightarrow\quad \sigma=\id.\]
\end{sol}

\begin{hwprob}{11.2}
	Calculate the determinants of the Cartan matrices (using induction on $l$ for types $A_l-D_l$).
\end{hwprob}
\begin{sol}
	In the following, let the caligraphic letters $\mathcal{A,B,C,D,E,F,G}$ correspond to the Cartan matrix 
	for type $A,B,$ etc.
\subsection*{($A_l$)}
	The claim is that $|\mathcal{A}_l|$ (determinant of the Cartan matrix) is $l+1$. For $A_1$ this is trivial.
	Assume now that this holds up to $A_{k}$. But notice that $\mathcal A_k$ is a minor of $\mathcal A_{k+1}.$ In particular,
	\[\begin{pmatrix}
		2 & -1 & \mathbf{0}^T\\
		-1 & \mathcal A_k &\\
		\mathbf{0} & &
	\end{pmatrix}\]
	where $\mathbf{0}\in\R^{k-1}$ and so
	\[|\mathcal A_{k+1}|=2\cdot |\mathcal A_k|+|\tilde{\mathcal A}_k|\]
	where $\tilde{\mathcal A}_k$ is the same as $\mathcal A_k$ with thr first row replaced with $(-1,0,\dots,0)$.
	But then subtracting the first row from the second (which leave the determinant unchanged):
	\[\tilde{\mathcal A}_k=\begin{pmatrix}
		-1 & \mathbf{0}^T\\
		-1 & \mathcal A_{k-1}\\
		\hat{\mathbf{0}} &
	\end{pmatrix}\rightsquigarrow\begin{pmatrix}
		-1 & \mathbf{0}^T\\
		\mathbf{0} & \mathcal A_{k-1}
	\end{pmatrix} \]
	($\hat{\mathbf{0}}\in\R^{k-2}$) and so by the inductive hypothesis $|\tilde{\mathcal A}_k|=-|\mathcal A_{k-1}|=-k$.

	But then we can conclude that
	\[|\mathcal A_{k+1}|=2\cdot |\mathcal A_k|+|\tilde{\mathcal A}_k|=2(k+1)-k=k+2\]
	as desired.

	\brk
	\subsection*{($B_l$)}
	Here the claim is $|\mathcal B_l|=2$. This can be seen directly (without induction) by using that 
	$\mathcal B_l$ differs in only one spot from $\mathcal A_l$: expanding the determinant from the 
	bottom right along the last column, we get 
	\[|\mathcal B_l|=2|\mathcal A_{l-1}|+2|\hat{\mathcal A}_{l-2}\]
	where $\hat{\mathcal A_{l-2}}$ is $\mathcal A_{l-2}$ with the last column replaced by $(0,\dots, 0,-1)$.
	Luckily, it can be seen by expanding upon the last column that
	\[|\hat{\mathcal A}_{l-1}|=|\tilde{\mathcal A}_{l-1}|=-(l-1)\]

	So finally:
	\[|\mathcal B_l|=2|\mathcal A_{l-1}|+2|\hat{\mathcal A}_{l-2}|=2(l)-2(l-1)=2.\]

	\brk
	\subsection*{($C_l$)}
	This isn't bad at all since $|\mathcal C_l|=|\mathcal B_l^T|=|\mathcal B_l|=2$.

	\brk
	\subsection*{($D_l$)}
	Here the claim is that $|\mathcal D_l|=4$ (for $l\ge 4$). Here we can again compute the determinant from the 
	bottom right, to see that 
	\[|\mathcal D_l|=2|\mathcal A_{l-1}|+|H|\]
	where $H$ is $\mathcal A_{l-1}$ with the second-to-last row replaced by $(0,\dots,0,-1,0)$.
	Doing row reductions, we can compute $|H|=-2|\mathcal A_{l-3}|$, and sol
	\[|\mathcal D_l|=2l-2(l-2)=4.\]

	\brk

	The matrices for the exceptional Lie algebras are just regular matrices and their determinants can 
	be computed (either by hand or your favorite CAS) :).
\end{sol}

\begin{hwprob}{13.4}
	Prove that the Weyl group of a root system $\Phi$ is isomorphic to the direct product of the
	respective Weyl groups of its irreducible components.
\end{hwprob}
\begin{sol}

\end{sol}

\begin{hwprob}{(Board)}
	Using large tensor powers of the adjoint representation and $\sl_2$ theory, show that any semisimple Lie algebra $L$ admits irreducible representaitons of arbitrarily large dimension.
\end{hwprob}
\begin{sol}

\end{sol}
\newpage

\section{February 15, 2019}
One last fact about root systems for now:  if we let $\Phi$ be a root system and $g$ an automorphism of $\Phi$
(orthogonal transformation of $\R^n$ permuting $\Phi$). If $\Phi^+$ is a positive system for $\Phi$, so is $g\Phi^+$ 
and we can find $w\in W$ sending $g\Phi^+$ to $\Phi^+$. Then $wg$ is an automorphism of $\Phi$ preserving $\Phi^+$,
thus also the corresponding simple subsystem $\Delta$, thus acting on the dynking diagram by an automorphism.

If $\Phi$ is such that this diagram is connected, then running through all the possibilities, there are just 3 of them 
where nondrivial diagram automorphisms occur: types $A_n, D_n,$ and $E_6.$

$A_n, E_6,$ and $D_n$ for $n>4$ all admit two automorphisms while $D_4$, most interestingly, admits a group $S_3$ of automorphisms.
The study of these kinds of things is called triality. It turns up that $S_3$ also acts by automorphisms on $\mathfrak{so}(8)$, the Lie algebra
of type $D_4$. If you compute the fixed point subalgebra of $D_4$, you end up finding out that it is, in fact, $G_2$. This can be seen in 
the diagram by ``folding''.

\subsection{(Officially) Returning to Lie Algebras}
FOr now, let $L$ be any Lie algebra over a field $k$ of characteristic zero admitting a vector space decomposition
$L=H\oplus_{\alpha\in\Phi\subseteq H^*}L_\alpha$
such that
\begin{enumerate}
	\item $H$ is abelian
	\item $L_\alpha=\{x\in L:[hx]=\alpha(h)x\}$
	\item $\Phi$ is a root system
	\item $\dim L_\alpha=1$; and
	\item $[L_\alpha,L_\beta]=L_{\alpha+\beta}$ when this makes sense
\end{enumerate}
Then $L$ is simple if the Dynkin diagram is connected and semisimple in general.

\begin{rmk}
	Monty mentioned today that he will not be following the approach in Humphreys -- showing that 
	any two Cartan subalgebras are conjugate. Instead, he will show this using an algebro-geometric 
	argument. This means we can skip reading chapters 15 and 16 of Humphreys.
\end{rmk}

\subsection{Uniqueness of Root Systems}
Now we want to see that any two semisimple Lie algebras with the same roots systems are isomorphic,

We have the following: let $L,L'$ be two semisimple Lie algebras with roots systems $\Phi, \Phi'$ and suppose that $\pi$ is an isomorphism from 
$\Phi$ to $\Phi'$ (thus an orthogonal isomorphism from a maximal toral subalgebra $H$ of $L$ to its
counterpart $H'$ in $L'$). Fix simple subsystems $\Delta$ and $\Delta'$ that correspond under $\pi$
and choose nonzero $x_\alpha\in L_alpha$ and $x'_{\alpha'}\in L_{\alpha'}'$ where $\alpha'=\pi(\alpha).$

There is a unique isomorphism from $L$ to $L'$ extending $\pi$ and sending $x_\alpha$ to $x_{\alpha'}'$. To
see this, assume first that $L,L'$ are simple. We know that there are $y_\alpha\in L_{-\alpha}$ and 
$y_{\alpha'}'\in L_{-\alpha'}'$ such that the $x_\alpha,y_\alpha,h_\alpha$ and the primed 
versions each span subalgebras $S_\alpha$ and $S_{\alpha'}$ isomoprhic to $\sl_2$.

Then you have this nice trick where you look at $L\oplus L'$ and end up proving they are, in fact, isomorphic. :)

\brk

\begin{thm}
	Let $g$ be any automorphism of the root system $\Phi$ of a semisiple Lie algebra $L$.
	Then there is a unique automorphism of $L$ extending $g$.
\end{thm}
\begin{rmk}
	To see this, we notice that $g$ is an automorphism of a maximal toral subalgebra 
	$H$ of $L$ sending roots to roots. So then we can pick a simple subsystem $\Delta$
	nonzero $x_\alpha\in L_\alpha$ and $x_{g\alpha}$ in $L_{g\alpha}$ and the above automorphism
	sends $x_\alpha\mapsto x_{g\alpha}$.
\end{rmk}

\section{February 20, 2019}
I missed the first section here but the definition of a Chevalley automorphism was discussed.

In the special case when our automorphism $\pi$ lies in $W$ (where $\pi$ is our simple system automorphism), 
there is a direct comstruction of $g_\pi$, which we have seen: since $W$ is generated by simple reflections,
it is enough to do this for $w=s_\alpha$, and then
\[g_{s_\alpha}=(\exp\ad x_\alpha)(\exp\ad -y_\alpha)(\exp\ad x_\alpha)\]

\subsection{The next logical quesetion}
Given a semisimple Lie algebra $L$ over an \textbf{algebraically closed} field $k$ of characteristic zero,
does its root system $\Phi$ depend on the choice of maximal toral subalgebra $H$?
\begin{thm}
	No.
\end{thm}
\begin{prf}
	We will prove this using some (very basic) algebraic geometry. There is an elementary (but much more ridiculous)
	argument in the text. Notice that in either case, the result \textbf{is not true} if $k$ is not algebraically closed.

	We will prove this for a general Lie algebra (not necessarily semisimple). First we will recall some facts from
	algebraic geometry: in particular the Zariski topology -- recall that here any two nonempty open sets intersect nontrivially.
	FUrthermore, we have a theory of morphisms from $k^n\to k^m$: in particular given a map $f:k^n\to k^m$ and a point $x\in k^n$,
	we get the differential $df_x:T_xk^n\to T_{f(x)}k^m$. Finally recall that if such an $f$ is surjective, then it maps
	an open subset containing $f(x)$.

	Let $L$ be any Lie algebra over $k$, and let $H$ be a maximal toral subalgebra. We have a root space decomposition
	\[L=L_0\bigoplus_{\alpha\in \Phi} L_\alpha\]
	and $H\subseteq L_0$. Fix a vector space complement $H'$ to $H$ in $L_0$ and look at the Zariski open subset $H''=\{h\in H':\alpha(h)\ne 0, \alpha\in\Phi\}$.
	Set $\mathcal H=H''+H_i$, which is Zariski open in $L_0$.

	Now let $b_1,\dots,b_m$ be a basis for $\oplus_\alpha L\alpha$ and we obtained by combining bases of each $L_\alpha$ 
	separately, and consider the map from $k^m\times L_0$  to $L$ sending
	\[(k_1,\dots,k_m,h)\mapsto (\exp\ad k b_1)\cdots(\exp\ad k_m b_m)(h)\]
	this map has a surjective differential at any $h\in\mathcal H$, by dimension counding, so it follows that given two maximal toral subalgebras
	$H_1,H_2$ of $L$, some automorphism of this type sends a regular element of $H_1$ to a regular element of $H_2$
	(for which no nonzero --- takes the value zer). But eh centralizer of the centralizer of such an element of $H_1$, if semisimple,
	lies in $H_2$, and similarly for $H_2$, so $H_1$ is sent to $H_2$.
\end{prf}

\subsection{Cartan Subalgebras}
Monty (and Humphreys) have actually be decieving us this whole time: in the literature what we really care about, 
instead of maximal toral subalgebras, are \textbf{Cartan subalgebras}, which are defined as a nilpotent subalgebra
which is normal in $L$: $N_L(H)=H$. This definition is actually equivalent to what we've been using
whenever $L$ is semisimple and in general over any algebraically closed field. FUrthermore any two Cartan subalgebras are
conjugate y $\Int L$.

\subsection{Classification continued}
We still must show that given any root system there is a semisimple Lie algebra (over $k=\bar k$)
which has $\Phi$ as its root system. We will only sketch the proof here -- see section 10 of Humphreys for more
information. We will instead give a defining set of relations for such a Lie algebra using only $\Phi$.

Notice that if we had $L$ with a root system $\Phi$ then we would have the basis $x_\alpha,y_\alpha,h_\alpha$ for $\alpha\in\Delta$.
Wow rough morning, huh?

\section{February 22, 2019}
Last time we learned that any two maximal toral (Cartan) subalgebras  of a semisimple Lie algebra over an algebraically closed field $k$ 
of characteristic 0 are conjugate under $\Int L$. In particular, $\dim H_1=\dim H_2=l$, which is called the \textbf{rank of $L$.}

We have anothe conjugacy result:
\begin{lem}
	Given a Cartan subalgebra $H$ with root space decomposition
	\[L=H\oplus\bigoplus_\alpha L_\alpha\]
	and a choice $\Phi^+$ of positive roots in $\Phi$, the subalgebra $B=H\oplus\bigoplus_{\alpha\in\Phi^+}L_\alpha$
	has a derived subalgebra $\oplus_{\alpha\in \Phi^+}L_\alpha$, which is nilpotent, so $B$ is solvable. In fact
	$B$ is maximal among solvable subalgebras of $L$.
\end{lem}

\begin{defn}
	We call any maximal solvable subalgebra of $L$ a \textbf{Borel subalgebra}.
\end{defn}
\begin{rmk}
	Any two Borel subalgebras are conjugate under $\Int L$. The quotient of $\Int L$ by the normalizer of such a subalgebra $B$ is called the \textbf{flag variety of $L$.}
\end{rmk}

\subsection{Today}
We are focusing today on how to construct a semisimple Lie algebra $L$ from aroot system $\Phi$, together with a basefield $k$ of characteristic zero via generators and relations. We need some definitions:
\begin{defn}
	The \textbf{free Lie algebra} on a set $x_1,\dots,x_n$ over $k$ is a subalgebra of the associative free tensor $k$-algebra
	generated by the $x_i$. In particular, it is the Lie subalgebra of this algebra generated by the $x_i$ -- spanned by all the $m$-fold brackets
	with the only dependence relations among the brackets following from the axioms of a Lie algebra (anticommutativity and the Jacobi 
	identity).
\end{defn}

We will follow this approach starting from $\Phi$: fix a simple subsystem $\Delta=(\alpha_i)_1^n$, a basis of $\Phi\subseteq\R^n$.
Then we take the 3n generators $x_\alpha,y_\alpha,h_\alpha$ of $L$ given the relations
\begin{align*}
	[h_i,h_j]&=0\\
	[h_i,x_j]&=\langle \alpha_i,\alpha_j\rangle x_j\\
	[h_i,y_j]&=-\langle\alpha_i,\alpha,j\rangle y_j\\
	[x_i,y_j]&=\delta_{ij}h_i
\end{align*}
with more relations to follow later.

\brk

First look at hte Lie algebra $L_0$ defined by these relations alone. THis is infinite-dimensional, but still has nice enough structure to lead to what we want.
The structure of $L_0$ is (as a cector space) $L_0=Y+H+X$, where $Y$ is genreated by the $y_i$, etc. Moreover, the $h_i,x_i,y_i$ are independent in $L_0$,
which follows from an explicit representation of $L_0$ on page 97 of Humphreys.

$H$ is easy: generators $h)i$ commute, so $H$ is jsut the polynomial ring over $k$ of the $h_i$, regarded as independent variables.

$Y$ and $X$ are harder, but it is easy enough to observe that these are both graded: an $m$-fold bracket of various
$x_i$ has weight the sum of the corresponding $\alpha_i$, which is an element of $H^*$. Similar for the $y_i$'s.

So we can write $X=\oplus X_\lambda$ where $X_\lambda=\{x\in X:[hx]=\lambda(h)x\}$ and similar for $Y$. Now each nonzero $X_\lambda$ corresponds to a $\lambda$ 
that is a nonnegative linear combination of $\alpha_i$ and the $Y_\eta$ correspond to $\eta$ which are 
nonpositive linear combinations. This gets us that
\begin{lem}
	\[\dim X_{\alpha_i}=1=\dim Y_{\alpha_i}\]
	and furthermore if $c\notin\{0,\pm 1\}$,
	\[\dim X_{c\alpha_i}=0=\dim Y_{c\alpha_i}.\]
\end{lem}

\brk

So how can we cut down $L_0$ to the size we want? We know that for all $i\ne j$ that $\alpha_i-\alpha_j$ is not a root,
whence $S_{\alpha_j}(\alpha_i-\alpha_j)=\alpha_i-(\langle\alpha_i,\alpha_j\rangle+1)\alpha j$
is also not a root. Call it $\alpha_i+k_{ij}\alpha_j$. We know the bracket of $x_i$ with $x_j$ $k_{ij}$ times has to be zero in our quotient, so
we must impose the further relations
\[(\ad x_j)^{k_{ij}}x_i=0.\]

If these relations are imposed (along with the correspinding ones for the $y_i$), then the ideal $I$ of $X$ generated by the $x_i$ relations are, in
fact, an ideal of $L_0$ (Humphreys p. 99) (same for $Y$). These ideals are generated by graded elements, so
\[(X+H+Y)/(I+J)\]
retains a graded structure for which the root spaces are still dimension one.

\brk

\begin{defn}
	An element $x$ where for each element $l\in L$ there exists a positive integer $a_l$ such that
	\[(\ad x)^{a_l}l=0\]
	is said to act \textbf{locally $\ad$-nilpotently} on $L$.
\end{defn}
So we have seen that $x_i$ and $y_i$ act locally $\ad$-nilpotently, so we get an automorephism of the quotient
\[(\exp\ad x_i)(\exp \ad -y_i)(\exp\ad x_i)\]
acts as the reflection $s_{\alpha_i}$ on the root spaces, preserving their dimensions.

It follows that the Weyl group $W$ acts on the root spaces of our quotient, preserving their dimensions.
Let $L$ be this quotient. We know that $\dim L_\beta=1$ for any $\beta\in\Phi$, since any such $\beta$ is 
conjugate to a simple root (where the root space has dimension 1).

Finall, if $\lambda\in H^*$, then some $W$ conjugate of $\beta$ is a linear combination of simple roots
with some positive and some negative coefficients. This doesn't occur in $L_0$.

The upshot here is that $L$ consists of $H$ plus a one-dimensional $L_\beta$ for all $\beta\in\Phi$ and 
nothing else. By $\sl_2$ theory, we get the bracket relations among root spaces. By a previous result, any Lie algebra with a root 
system is semisimple, so $L$ is semisimple with root system $\Phi$. This gives us our conclusion:
\begin{thm}
	A semisimple Lie algebra $L$ exists with \textbf{any} specified root system $\Phi$ and base
	field $k$.
\end{thm}

\subsection{Extending this construction}
This begs to be embedded in a larger construction: Give $k$ and our generators $x_i$, and an ideal $I$ 
of $F$ (the free $k$ Lie algebra on the $x_i$), we can form $T/\langle I\rangle$ ($T$ the free tensor algebra), the quotient of $T$ by the 
\textbf{ordinary two-sided} generated by $I$ in $T$. This will be a larger associative algebra containing $F/I$ such that
$[x,y]=xy-yx$.

More generally, let $L$ be a finite dimensional Lie algebra with basi $x_i$. Then we have 
\[[x_ix_j]=\sum m_{ijk}x_k\]
and imposing the relations on the tensor algebra, and this is called the universal enveloping algebra of $L$.

Sick.

\section{February 25, 2019}
Notice that (having classified all complex semisimple Lie algebras) we have thus far used lots of linear algebra
but relatively little abstract algebra. But now the theory of associative algebras is going to come into the fold
as we proceed to classify the finite dimensional $L$ modules.

By Weyl's theorem, it is enough to classify the finite-dimensional \textit{irreducible} $L$-modules $M$.
For this we need an asoociaitive algebra containing $L$ as a Lie algera , and then we use the method of associative algebras:
\begin{ex}
	If $A$ is such an algebra, then a typical irreducible left $A$-module takes the form $A/M$
	for some maximal left ideal $M$ of $A$.
\end{ex}

\subsection{Universal Enveloping Algebras}
But we need to choose $A$ in such a way that all irreducible finite-dimensional $L$-modules are captured
in this way (as $A$-modules). For this purpose,
\begin{defn}
	Let $\mathcal U:=\mathcal U(L)$ be the associative algebra generated by $L$ (or any basis) with relations exactly tose of $L$,
	that is $xy-yx=[xy]$.

	There are only finite many relations since it is enough to take $x$ and $y$ above to be elements of a basis of $L$.
\end{defn}

How big is $\mathcal U$? What is a basis? To this we answer
\begin{thm}[Poincar\'e-Birkhoff-Witt (PBW)]\label{thm-pbw}
	If $x_1,\dots,x_n$ is an ordered basis for $L$, then a basis of $U$ (over $k$) is given by 
	monomials in the $x_i$: 
	\[x_1^{m_1}\cdots x_n^{m_n},\quad m_i\ge 0\].
\end{thm}
\begin{rmk}
	The proof of this can be found in the book (p.93-94) (or other places). The idea is that, to show span,
	using that whenever $j>i$, $x_jx_i=x_ix_j+[x_jx_i]$, converting an element in ``wrong'' order into the
	``right'' order.
\end{rmk}
A more sophistochated statement of PBW: Let $\mathcal U_m=\mathcal U_m(L)$ be the span of all products of at most $m$ geneartors $x_i$ (allowing
repetitions) so that 
\[0=\mathcal U_{-1}\subseteq\mathcal U_{0}\subseteq\mathcal U_{1}\subseteq\cdots\]
where $\cup \mathcal U_i=\mathcal U$.

Then $\mathcal U_m\mathcal U_n\subseteq \mathcal U_{m+n}$, so we can form the associated graded algebra
$G=\oplus_0^\infty \mathcal U_m/\mathcal U_{m-1}$.

In particular, we have $x_1,\dots,x_n$ as a basis for $L$ representa by elements labelled in the same way in $\mathcal U_1/\mathcal U_0$.
Then $x_ix_j=x_jx_i$ for all $i,j$, regarding the $x_i$ as lying in $G$. Moreover, the images in $G$ generate $G$, so 
we have a map $\varphi$ from $S$, the polynomial ring in the $x_i$ to $G$, sending each $x_i$ to its image in $G$.
\begin{thmprime}{thm-pbw}[PBW']
	In the above construction, the map $\varphi$ is an isomorphism.
\end{thmprime}
\begin{rmk}
	An interesting consequence of this is that $\mathcal U$ has no zero divisors. For if $x,y$ are nonzero elements just choose $m,n$ minimal so that 
	$x\in\mathcal U_m$ and $y\in\mathcal U_n$. Then the product is nonzero in $\mathcal U_{m+n}/\mathcal U_{m+n-1}$!
\end{rmk}

It is clear from the construction of $|mathcal U$ that any left $L$ module may be regarded as a left $\mathcal U$-module. Thus we can construct an irreducible such module
as $\mathcal U/M$ for a maximal (left) module $M$. Now let $N$ be an irreducible $L$-module,
so that $N$ is an irreducible $\mathcal U$-module. 

If $L$ is semisimple, it has root space decomposition $L=\h\oplus_{\alpha\in\Phi}L_\alpha$. Choose a positive root system
and we ahve seen that the Borel subalgebra $B$ is a (maximal) solvable subalgebra for $L$. Thus by Lie's theorem, there is a 
one-dimensional submodule $K_v$ of $L$, acted on semisimple (thus diagonally) by $\h$,
while $x_\alpha v=0$ for all $\alpha\in\Phi^+$, since $L_\alpha\in [BB]$.

Accordingly, $v$ is a weight vector in $N$ (simultaneous eigenvector for $\h$) and so now we bring in PBW:
we know by irreducibility that $\mathcal U v=N$. Now fix a basis of $L$ consisting of root vectors, together 
with a basis for $\h$, in which all negative root spaces come first, then the basis for $\h$, then the basis for $[BB]=\oplus_{\alpha\in\Phi^+}L_\alpha.$

We have observed that $N$ (since it is irreducible) is a quotient of $\mathcal U$ by a maximal left ideal. We 
know this ideal contains all $x_\alpha$ for $\alpha\in\Phi^+$. It follows that if $N'=\oplus_{\alpha\in\Phi^-}$,
then $\mathcal Uv=\mathcal U(N')v$, so all vectors in $N$ are obtained from $v$ by applying suitable products of 
negative roots vectors and taking the span.

It follows that our module $N$, the sum of its weight spaces, is such that any of its weights $\mu\in\h^*$
is such that $\mu$ is obtained from $\lambda$ (where $hv=\lambda(h) v$ for all $h\in\h$) by subtracting anonnegative integer combination
of simple roots. This says that the weight $\lambda$ of $v$ occurs with multiplicity one (the only product $P$ of negative root vectors such that
$Pv$ has weight $\lambda$ is the empty product) and all other weights occurring (whoops, it got erased.)

\section{February 27, 2019}
Last time we saw that, given any semisimple Lie algebra, we can identify a \textbf{highest weight vector} of any $L$-module $V$
which is killed by the action of any positive root. We also saw (via $\sl_2$ theory) that this highest weight $\lambda$ satisfies
$\langle\lambda, \alpha_i\rangle\in\N$ for all $i$.

We also saw the enveloping algebra $\mathcal U$ of $L$ and that $V=\mathcal Uv$ and $\h v=kv$. Let $I$ be the ideal of $\mathcal U$ gnerated by $N^+$,
the sum of the positive weight spaces, and all $h-\lambda(h)$. Notce $I\subseteq M=\operatorname{Ann}(V).$
Notice that any proper ideal of $\mathcal U$ cannot contain a vector of weight zero (since the only one is 1)
and thus the sum of every ideal containing $I$ is proper, whence there is a unique maximal $M$ over $A$.

Hence immediately if $V$ and $W$ are irreducible modules generated by vectors of the same highest weight $\lambda\in\h^*$,
then $V\cong W\cong \mathcal U/M$. 

\subsection{Integral Weights}
It remains to show the following:
\begin{lem}
	If (conversely) $\lambda$ is such that $\langle\lambda,\alpha_i\rangle\in\N$ for all $i$, 
	then there is a unique irreducible highest weight module $V$ with highest weight $\lambda$ 
	that, furthermore, is finite dimensional!
\end{lem}
\begin{rmk}
	Such a $\lambda$ (regardless of the context in which we see it) is called a \textbf{dominant (integral) weight.}
\end{rmk}
\begin{prf}
	We will construct $M$ explicitly and show $\mathcal U/M$ is finite dimensional. Set
	\[m_i=\langle\lambda,\alpha_i\rangle\]
	and let $J$ be the ideal generated by the $y_i^{m+i+1}$ where $y_i\in L_{-\alpha_i}$. We claim that $M=I+J$. To see this, note first that
	$I+J$ is generated as a left $\mathcal U$=module by $I$  and $\mathcal U(N^+)J$ since $x_jy_j^{m+j+1}=y_j^{m_j+1}$ if $i\ne j$ where $x_j\in L_{\alpha_j}$
	and $x_jy_j^{m_j+1}\in I+J$ by a $\sl_2$ calculation. 

	Thus $I+J$ is proper and $U/(I+J)$ has a finite-dimensional $S_{\alpha_i}$-module spanned by $v, y_iv,\dots, y_i^{m_i}v$
	where each $S_{\alpha_i}$ is spanned by the $x_i,y_i,h_i$. Eventually we can prove that $U/(I+J)$ is stable under 
	the Weyl group, having the same dimenion of its weight space.

	Then given any weight $\eta$ smaller than $\lambda$ in the partial order, recall that, if the dot product with any simple root 
	is negative, one can apply a reflection to "bump up" $\eta$ in the partial order. But of course you can only do this so many times!

	Thus any dominant weight $\mu$ of $V'=\mathcal U/(I+J)$ has length at most that of $\lambda$ and it lies on the integer lattice,
	so as a discrete subset of a compact space there are only finitely many.

	Thus any weight occuring in $V'$ is $W$-conjugate to some dominant weight, so only finite many weights occur in $V'$ 
	and therefore $V$ is finite-dimensional. Furthermore, we see that $V$ has only one highest weight vector (if it is irreducible and finite-dimensional). 
\end{prf}

Next time we will try to explain the structure of $V$ as it decomposes into $V_\lambda$ which is analgous to, but not
exactly the same as what we see in the adjoint representation: in this case, we will see that one can get higher multiplicities
occuring for certain weights, but the pattern will be that the highest weights will occur with multiplicity one and the rest 
will lie in the convex hull of these vectors, with multiplicity increasing as you go further inside the hull.

\section{February 28, 2019}
Had an appointment. :(

\section{March 4, 2019}
Notice that we should read sections 21 and 24 of the text, but not 22 or 23. There are notoriously hard sections and it will end 
up that we don't actually need 23, using the appendix to section 24.

Last itme we identitified the highest weight of the definitng representation $V=k^n$ of any classical Lie algebra
$L\subseteq \gl_n(k)$. The highest weight is $e_1$ in essentially all cases of type $A_{n-1}$ -- modify to
($\frac{n-1}{n},\frac{-1}{n},\dots, \frac{-1}{n}$).

\subsection{Creating lager representations}
We have see the tensor product $\otimes V_i$ of representations $V_i$ over a Lie algebra $L$ that has a Lie sructure defined by 
\[x(v_1\otimes v_n)=\sum_{i=1}^nv_1\otimes\cdots\otimes xv_i\otimes\cdots\otimes v_n.\]

Certain quotients of $T^nV$, the $n$-fold tensor product, also inherit an $L$ module structure. Specifically, $S^nV=T^nV$ modulo
the ideal generated by $v\otimes w-w\otimes v$ intersected with $T^nV$. These are the homogeneous polynomials of degree $n$ on a basis of $V$
as an $L$-module. Similarly one can corrects the exterior power $\Lambda^nV$.

Recall that $S^nV$ grows in dimension while the exterior power grows like binomial coefficients.

In particular, any classical Lie algebra $L\subseteq\gl_n(k)$ acts on $S^mk^n$ and $\Lambda^mk^n$. Luckily
there is one phrase that sums up how $L$ acts:

\textit{All summetric and exterior powers of $k^n$ in this situation are irreducible, modulo the obvious relations.}

\begin{itemize}
	\item For $A_{n-1}$ ($\sl_n(k)$), all powers $S^mk^n$ and $\Lambda^mk^n$ are irreducible.
	\item For $B_n$ and $D_n$, recall that the corresponding group $\so_n(k)$preserves a symmetric bilinear nondegenerate
	form, which may be thought of as an element of $S^2k^n$. Thus $S^mk^n$ contains a copy of $S^{m-2}k^n$ and then
	$S^mk^n/S^{m-2}k^n$ is irreducible under the $L$-action, identifying $S^{m-2}k^n$ with its copy in $S^mk^n$. the highest weight is $me.$
	\item For type $C$, the exterior power $\Lambda^mk^n$, as long as $2\le m\le\frac{n}{2}$. Then the underling group $\sp_n(k)$ preserves
	a skew-symmetric nondegenerate bilinear form, whence $\Lambda^mk^n$ contains acopy of $\Lambda^{m-2}k^n$ and the corresponding 
	quotient is irreducible. The highest weight is $m$ 1's followed by $n-m$ zeros.
\end{itemize}

More generally, there is a purely multilinear algebraic construction starting from $V=k^n$ which produces all irreducible finite dimensional representations in 
types $A$ and $C$. For $B$ and $D$, we need an additional ingredient of the half-splin representations seen in hw.

\subsection{A Road Map}
We now head towards a road map of all finite dimensional irreducible represetnations of a semisimple Lie algebra $L$. That is, recall 
that every such representation $V$ is the sum $\oplus V_\mu$ of its weight spaces, each occuring with some multiplicity $m_\mu=\dim V_\mu$.

We want an implicite formulat for all the multiplicities. We introduce some formalisms:
\begin{defn}
	$\ch V$ is the formal character of $V$ and is defined to be
	\[\ch V=\oplus_\mu m_\mu e^\mu\]
	where the sum is over all weights of $V$ (the weight lattice).
\end{defn}

We add any two finite formal sums in the obvious way and multiplication is also obvious in that $e^\mu s^\sigma=e^{\mu+\sigma}$.

Then we can actually take the formal characters $\ch M_\lambda$ of \textbf{Verma modules} $M_\lambda :=\mathcal U/(L_\alpha,h-\lambda(h),\alpha\in\Phi^+)$
where $\ch M_\lambda=\sum a_\mu e^\mu$ where (as long as $\lambda$ is a weight) all $\mu$ occuring will lie in the same lattice.

We will work out an expression for $\ch L_\lambda$, the formal character of the irreducible highest weight module of heighest weight $\lambda$, a dominant integral weight $\lambda$
in terms of various $M_\mu$. We will see that $\ch L_\lambda$ will be a quotient of two simple alternating sums of ${e^\mu}'s$. To do this, we will have to dive into the structure of $M_\lambda$
for a general $\lambda$.

Notice that any subquotient (quotient of a submodule) of $M_\lambda$ has a weight that is maximal in the partial order, so also a highest weight vector of that highest weight. 
Then that vector generates a highest weight module with a unique irreducible quotient $L_\mu$ for some $\mu$ so every irreducible subquotient 
of $M_\lambda$ takes the form $L_\mu$ for some $\mu$.

Now only finitely many $L_\mu$ can occur as subquotients of $M_\lambda$ for a fixed $\lambda$. 
To prove theis, we need to revisit and generalize the idea of a \textbf{Kasimir element.} Recall the Killing form,
which is nondegenerate on $L$. Set up dual basis of this form so that $\kappa(x_\alpha,z_\alpha)=1$ for $x_\alpha\in L_\alpha$ and $z\in L_{-\alpha}$.

Set
\[c=\sum_{\alpha\in\Phi^+}(x_\alpha z_\alpha+z_\alpha x_\alpha)+\sum_{i=1}^nh_i^2\]
and notice that $c$ lies in the center of $\mathcal U$ -- it commutes with $L$. Then consider 
(for $\lambda$ a highest weight vector)
\[c v_\lambda=\sum_1^r\lambda(h_i)^2+\sum_{\alpha\in\Phi^+}(z_\alpha x_\alpha+[x_\alpha,z_\alpha]+z_\alpha x_\alpha)v_\lambda=(\lambda\cdot \lambda+\lambda(2\rho))v_\lambda\]
where $2\rho$ is the sum of the positive roots (called the ``rho shift'')

\section{March 6, 2019}
Last time we stated that in the classical case, if $V=k^n$ is the defining (or natural) representation of
$L\in\gl_n(k)$, then the symmetric and exterior owers are irreducible under $L$, except for the ``obvious relations.''

The obvious relations were that, if $L$ is orthogonal, $S^2V$ has an $L$-invariant vector (sent to zero)
so $S^mV$ is not irreducible for $m\ge 2$, but $\Lambda^mV$ is irreducible of highest weight $(1,\cdots,1,0\cdots,0)$  (with $m$ ones)
for $m\le\lfloor\frac{n}{2}\rfloor$ where $n=\dim V$.

Likewise if $L$ is symplectic, the $S^mV$ are all irreducible with highest weight $(m,0,\dots,0)$.

\subsection{Return to (Formal) Characters}
Recall that last time we studyies the Verma modules $M_\lambda=\mathcal U/(L_\alpha, h-\lambda(h): h\in H, \alpha>0)$.
These all have irreducible quotients $L_\lambda$. SO we introduced the formal character $\ch M_\lambda$ of $M_\lambda$
to be $\ch M_\lambda=\sum n_\mu e^\mu$ where $|mu$ runds over the $H$ weights of $M_\lambda$ 
and $n_\mu=(M_\lambda)_\mu$, the dimension of the corresponding weight space. 

\begin{rmk}
	We'll do a quick aside to explain the connection between the character of a group representation vs 
	one of a Lie algebra. Recall that in the group case, we are just taking the determinant of a matrix 
	(the image of $g$ under the representation).

	In general, there is an exponential map from any Lie algebra to a Lie group $G$ with Lie algebra $L\subseteq \gl_n(k)$.
	Here $G=GL(n,k)$ and $\exp \mu=\sum_{i=0}^\infty \frac{\mu^i}{i!}$ (we can work here over $\C$ so the series 
	always converges.)

	Then if $h\in H$, the Cartan subalgebra and if $N$ is a finidte simensional subquotient, then
	\[\tr e^h)=\sum n_\mu e^{\mu(h)}.\]
	we are extending to the case of infinite dimensional $M_\lambda$ without worrying about convergence.

	We saw last time that any $M_\lambda$ admits a finite composition series $0=M_0\subseteq\cdots\subseteq M_m=M_\lambda$
	where successive quotients are irreducible and each is isomorphic to some $L_\mu$. Then we are saying
	\[\ch M_\lambda=\sum_{1}^m\ch(M_i/M_{i-1}).\]
\end{rmk}

Let's specialize this discussion to the case when $\lambda$ is \textbf{dominant integral}, which is the case of 
greatest interest, so that $L_\lambda$ is finite-dimensional. Now
\[\ch L_\lambda=\sum_\mu m_\mu\ch M_\mu\]
and we will work out exactly which $\ch M_\mu$ occur and the $m_\mu$ in each case. More precisely,
\[\ch M_\mu=e^\mu\prod_{\alpha\in\Phi^+}(1+e^{-\alpha}+e^{-2\alpha}+\cdots)=e^\mu\prod_{\alpha\in\Phi^+}\frac{1}{1-e^{-\alpha}}=\frac{e^{\mu+\rho}}{\prod(e^{\alpha/2}-e^{-\alpha/2})}\]

Now if $|lambda$ is dominant integral we know that $\dim(L_\lambda)_\mu=\dim(L_\lambda)_{w\mu}$
for any $w\in W$. So $w(\ch L_\lambda)=\ch L_\lambda$. Now applying any simple reflection $s_\alpha$ to the product above
just premutes the terms except the one including $\alpha$, which is sent to its negative. Thus if $\Delta$ is 
the denominator, $s_\alpha\Delta=-\Delta.$

Hence $w\Delta=(\det w)\Delta$ where $\det w$ is computed with respect to the $W$ action on $H^*$. If 
$\ch L_\lambda=\sum m_\mu \ch M_\mu=(\sum m_\mu e^{\mu+\rho})/\Delta$, it must be the case that 
$w(\sum m_\mu e^{\mu+\rho})=(\det w)\sum m_\mu e^{\mu+\rho}$. THus we get that the coefficent of 
$e^{w\lambda+\rho}$ in the sum must be $\det w$. But no other coefficients can appear -- $m_\mu=0$
if $\mu\ne w(\lambda+\rho)-\rho$ for if any $m_\mu\ne -$ then $m_{w\mu}\ne 0$ for some
\textbf{dominant} $w\mu$ lying below $|lambda$< but $w\mu+\rho$ would have the same length as $\lambda+\rho$.

This leads us to
\begin{thm}
	The Weyl Charcter formula:
	\[\ch L_\lambda=\frac{\sum_{w\in W}(\det w)e^{w(\lambda+\rho)}}{\prod_{\alpha\in \Phi^+(e^{\alpha/2}-e^{-\alpha/2}}}.\]
\end{thm}
\begin{rmk}
	If $\lambda=0$ then $L_\lambda$ is the trivial representation having weight zero with multiplicity one and no other weight. So 
	\[\prod(e^{\alpha/2}-e^{-\alpha/2}=\sum_W(\det w)e^{w\rho}\]
	so 
	\[\ch L_\lambda=\frac{\sum_{w\in W}(\det w)e^{w(\lambda+\rho)}}{\sum_{w\in W}(\det w)e^{w\rho}}\]

	There is one small annoyance: that there is no product expression for the nmerator, except for the case when $\lambda=n\rho$.
	Then we see that the weights of $L_n\rho$, with heighest weight $n\rho$ are of the form $n\rho-\sum n_\alpha\alpha$ for positive roots $\alpha$

	The multiplicitiy of any weight is just the number of ways it can be written in this form.
\end{rmk}

\section{March 8, 2019}
We will begin by mopping up some minor technical issues in the Weyl character formula from last time.

We have Kostant's multiplicity formula: in computing $\ch L_\lambda$, we used that it is the sum over $w\in W$
of $(\det w)\ch M_{w(\lambda+\rho)-\rho}$, then $m_\mu$, the multiplicity of $\mu$ in $L_\lambda$ is 
\[m_\mu=\sum_{w\in W}(\det w)P(w(\lambda+\rho)-(\mu+\rho))\]
where $P(\nu)$ is the number of ways to write $\nu$ as nonnegative ineger combination of positive roots.

\subsection{Computing Dimension of $L_\lambda$}
We use the Weyl character formula to compute a finite dimensional representation of highest weight $\lambda$, assumed dominant integral.
Start with $R$, the rings of $\Z$ combinations of $e^\mu$ for $\mu$ in the root lattice. We have a homomorphism $v:R\to \Z$ taking $e^\mu\to 1.$
The basic idea here is to apply this to the top and bottom of the Weyl character formula. We get zero in both
cases. So we use something like L'H\^opital's rule.

More precisely, given any positive root $\alpha$, there is a derivation $\delta_\alpha$ of $R$ sending $e^\mu$ to $(\mu \alpha)e^\mu$ (or something similar).
Let $\delta=\prod_{\alpha>0}\delta_\alpha$ , which is not a derivation. But we can compute $\delta$
on any product, term by term:
\[\Delta(\ch L_\lambda)=\sum_{w\in W}(\det w)e^{w(\lambda+\rho)}\]
where $\Delta$ is the Weyl denominator from before. Apply $v\delta$ to each side.

On the left side, the nly remaining terms are from $v(\delta\Delta)\ch L_\lambda$ since $v(\Delta)=0$.
On the other hand when we apply $\delta$ to a single $(\det w) e^{w(\lambda+\rho)}$, you get the 
same result, independent of $w$ since any $w$ is the product of simple reflections and 
\[\det(ws_\alpha)=\det w\prod_{\beta>0}(w(\lambda+\rho)\cdot \beta)=-\det w\prod_{\beta>0}(ws_\alpha(\lambda+\rho)\cdot \beta)\]

With a little more computation we get 
\begin{thm}
	\textbf{Weyl's Dimension Formula}
	\[\dim L_\lambda =\frac{\prod_{\beta>0}(\lambda+\rho)\cdot\beta)}{\prod_{\beta>0}\rho\cdot \beta}\]
\end{thm}

In general when applying this formula, we usually multiply each term $w(\lambda+\rho)\cdot\beta$ or $\rho\cdot \beta$ by 2, 
divide by $\beta\cdot \beta$ so all terms on top and bottom are integers.

\subsection{What this means}
Recall that for any root system $\Phi$ there is a dual root system $\Phi^v=\{2\alpha/(\alpha\cdot\alpha)|\alpha\in\Phi\}$
which is again a root system with a positive rootsystem going to a positive roots system and a simple subsystem going to a simple subsystem,

Applying this formula to $G_2$, notice that the dual system is isomorphic to $G-2$, but the duality
interchanges long and short roots. The simple roots are $\alpha=(0,1,-1)$ and $\beta=(1,-2,1)$
and the positive roots are $\alpha,\beta,\alpha+\beta,\beta+2\alpha,\beta+3\alpha,2\beta+3\alpha.$

In the dual system, we get simple roots the same thing but with $\alpha,\beta$ swapped for $\beta^v$ and $\alpha^v$, respectively. THen applying the formula 
with $\lambda a^v=1$ and $\lambda\beta^v=0$, we get $\dim L_\lambda=7$. Thus $G_2$ has a seven dimensional representation. This tells us that a Lie algebra $L$ 
of type $G_2$ embeds in $\gl_7$, and in fact in $\so_7$. So $L$ may be realized as a subalgebra of $\so_7$ (which is type $B_3$)
and you can see these computations explicitly in $\S 19$ of the book.

If instead we have $\lambda\cdot a^v=0$ and $\lambda\cdot \beta^v=1$, then $L_\lambda$ is 
14 dimensional, and this is the adjoint representation of $G_2$. The largest exception simple Lie algebra s of type $E_8$
and it is the unique simple Lie algebra $L$ such that $L$ itself is the unique nontrivial $L$-module of smallest dimension (which is 248).

\subsection{Next Week}
We are going to talk about simple Lie algebras over $\R$ -- we will find that it will somehow 
mirror the classification we mentioned for Lie groups (where they were parameterized by subgroups between the weight
and integer lattices). This time we will find that we get two types: complex simple Lie algebras which are just 
considered to be real and real forms of complex simple lLie algebras such that $L\otimes_R\C$ is simple.

\section{March 11, 2019}
Today we are going to say something about semisimple Lie algebras over non algebraically closed fields (in particular $\R$).
In the first week we defined the classical Lie algebras and gave bases for them. We say the commutator of any two basis vectors was an 
integer combination of the the others. 

Now let $k$ be a field of characteristic zero (this isn't needed everywhere so we will point out where) and let $\Phi$
be a root system. We have already seen the construction of an abstract semisimple Lie algebra over $k$ with root system $\Phi$
by generators and relations. We didn't need anything about the field except that we needed formulas (power series)
to implement the action of the Weyl group, which involves denominators. In fact, we only get twos and threes.

This doesn't work over $\Z$, but let $L$ be the complex semisimple Lie algebra with root system $\Phi$. Let $\sigma$ be a
Chevalley automorphism of $L$ (recall it acts on a Cartan subalgebra by $-1$ and interchanges the root spaces $L_{\alpha}$ and $L_{-\alpha}$).
Now for all $\alpha\in\Phi^+,$ pick $0\ne x_\alpha\in L_\alpha$. We know that $[x_\alpha,-\sigma(x_\alpha)]$ is a multiple
of $h_\alpha(w(2\alpha)/(\alpha,\alpha))$. If we replace $x_\alpha$ by $cx_\alpha$ for any $c\in\C$,
then $-\sigma(x_\alpha)$ is also replaced by $-c\sigma(x_\alpha)$, so $[x_\alpha,-\sigma(x_\alpha)]$ is replaced by $c^2$
times itself. Thus we can choose $c$ to scale $x_\alpha$ such that htis bracket is exactly $h_\alpha$.

Choose a basis of $L$ as follows: $x_\alpha\in L_\alpha$ for $\alpha>0$ and $-\sigma(x_\alpha)\in L_{-\alpha}$ for $\alpha>0$. 
Then pick $h_i=h_{\alpha_i}$ for all the simple roots $\alpha_i$. THen 
\[[x_\alpha,x_{-\alpha}]=h_\alpha\quad [h_i,x_{\alpha}]=\alpha(\alpha_i)h_\alpha\in\Z h_\alpha\quad [x_\alpha x_\beta]=c_{\alpha\beta}x_{\alpha+\beta}\]
and by applying $\sigma$, we get that $[x_{-\alpha}x_{-\beta}]=c_{-\alpha-\beta}$, so $c_{-\alpha-\beta}=-c_{\alpha\beta}$

Then by thm 25.2(a) in the text, this implies $c_{\alpha\beta}=\pm(r+1)$ where $r$ is the largest integer 
such that $\beta-r\alpha\in\Phi$. The upshot here is that any bracket of basis vectors is an integer combination of basis vectors.

\brk

So we see that htere is a Lie algebra (usually semisimple) over any basefield $k$ with root system $\Phi$, called \textbf{split}
(having a root space decomposition and a lage abelian subalgebra consisting of semisimple elements). We now want to look at semisimple Lie algebras 
\textit{not} admitting root space decompositions -- the main example is $\R$.

Let $\Phi$ be a root system, $L$ the corresponding complex semisimple Lie algebra, and $\{h_i,x_\alpha\}$
a Chevalley basis. Look at the $\R$-subalgebra $L'$ of $L$ spanned by $x_\alpha-x_{-\alpha}$ and $i(x_\alpha+x_{-\alpha})$ as well as $ih_1,\dots,ih_l,$ for all $\alpha\in\Phi^+$.
Now $L'$ is closed under the bracket and the Killing form $\kappa$ on $L'$ is negative definite. 

But $L'$ doesn't admit a root space decomposition because $L'$ is toral -- it consists \textit{entirely} of semisimple elements:
possible since in fact the semisimple elements are \textbf{not} diagonalizable (in a non-algebraically-closed field semisimple and diagonalizable don't have to coincide)
and in fact, they will all have purely imaginariy eigenvalues.

We call $L'$ \textbf{compact} -- note that it is not compact as a topological space, but its adjoint group $\Int L$, generated 
by $\exp\ad x$, is a compact Lie group. It is, in fact, a closed subgroup of $O(n,\R)$ (a compact group)
and $n=\dim_\R L'=\dim_\C L$, Since it preserves $\kappa$, a negative definite symmetric bilinear form, the \textbf{compact real form of $L$.}

\subsection{Unitary matrices}
An astute reader will notice that we haven't mentioned the Lie groups formed by unitary matrices, for instance (and 
serving as a guide to understand the rest) $\mathfrak{su}(n,\C)$, the set of all $n\times n$ complex matrices that are skew-adjoint
with respect to the Hermetian form $((v_i),(w_i))=\sum_1^n v_i\bar w_i$ and of trace zero.
Notice that although these matrices are complex, the algebra is not stable under complex multiplication, so it is not a \textit{complex} Lie algebra.
The correspoinding group $SU(n,\C)$ consists of all $g\in GL_n(k)$ such that $(gv,gw)=(v,w)$ for all $v,w\in\C^n$.

\subsection{Orthogonal matrices}
$\mathfrak{so}_n(\R)=\mathfrak{o}_n(\R)$, the skew-adjoint matrices with respect to the dot product on $\R^n$ has correspinding group $SO(n,\R)$.
The lie algebra is all the $n\times n$ skey-symmetric matrices in the ordinary sense.

Over $\R$, the group that we would get from our original construction is not $SL(n,\R)$, but something else.

\subsection{Symplectic Case}
Starting with $\sp_{2n}(\C)$, take the intersection with $\mathfrak{su}_{2n}(\C)$ -- this is still a real Lie algebra.
We can think of this as right-linear transformations of $\mathbb{H}^n$ (quaternions) preserving a skew-Hermitian form:
\[((v_1,\dots,v_n)(w_1,\dots,w_n))=\sum_1^nv_ij\bar w_i.\]

Notice that we say ``right-linear'' because $\mathbb{H}^n$ admits (different) left- and right-vector space structures.

Then the group in question is the symmetry group of this form, the Lie algebra consists of matrices skew-adjoint with respect to it.

\subsection{Other Real Forms}
The above concludes our discussion of the \textit{compact} Lie algebra over $\R$, but there are even 
more real forms that are neither compact nor split. There are only finitely many of these, however, and each is partly compact and partly not.
These are of the form $L=K+P$ where $K$ is compact and $[PP]\subseteq K$. All the elements in $K$ will be semisimple
with purely imaginary eigenvalues and $P$ will consist of both semisimples and nilpotents, but the semisimples will have 
real eigenvalues.

One such example is defined by the form 
\[(v_1,\dots,v_n)(w_1,\dots,w_n)=\sum_1^m v_i\bar w_i-\sum_{m+1}^n v_j\bar w_j.\]
This is called the \textbf{symmetric group} $U(n,n-m)$.

\section{March 13, 2019}
One correction from last week: the compact from of $\sp_{2n}(\C)$ is given by the skew-adjoing matrices of trace zero relative to the \textit{Hermitian}
form $(v,w)=\sum_iv_i\bar w_i$ on $\mathbb{H}^n$ (not skew-Hermitian).

\subsection{Continuing with Real Forms}
Now we will address the non-compact forms. We've seen the compact and split real forms, which form 
the opposite ends of a spectrum and now we consider the ones that lie in the middle. 

\subsubsection{Real forms of type $A$}
In type $A$, we have already
seen $\su_n(\C)$ consisting of skew-adjoint matrices of trace zero relative to the positive definite Hermitian form on $\C$.
We've also seen $\sl_n(\R)$.

In between, we have $\su(p,q)$, consisting of the skey adjoint matrixes with respect to the Hermitian form $(-,-)_{pq}$ where
\[(v,w)=\sum_1^pv_i\bar w_i-\sum_{p+1}^{p_q}v_i\bar w_i\]
which is Hermitian of \textbf{signature} $(p,q)$ on $\C^{p+q}=\C^n$. WHen $p=n$ and $q=0$, we get the compact form $\su_n(\C)$.

We get one more real form -- only in even dimensions -- $\su^*(2n)$: all $n\times n$ matrices over the quaternions $\mathbb{H}$ whose trace is a real
combination of $i,j,k$.

\subsubsection{Types $B$ and $D$}
Here the story is mostly the same for these types with a small difference: we have $\so(p,q)$
in either, which is all skew-adjoint matrices with respect of the symmetric form $(-,-)_{pq}$
on $\R^{p+q}$ where
\[(v,w)=\sum_1^pv_iw_i-\sum_{p+1}^{p+q}v_iw_i.\]

But we get one extra in type $D$: $\so^*(2n)$. This consits of all skew-adjoint matrices with respect to the skey Hermitian form
\[(v,w)=\sum_1^n v_ij\bar w_i\]
where $j\in\mathbb{H}$.

\subsubsection{Type $C$}
Here we get $\sp(p,q)$ -- the skew-adjoint matrices with respect to the form as usual ($v_i\bar w_i$).
We also have $\sp(n,\R)$, the compact form.

Finally we get $\sp(2n,\R)$, the real matrices skew adjoint with respect to a skew-sypmmetric form on $\R^{2n}$.

\subsection{Cartan Involution}
In general, recall from last time that any real semisimple Lie algebra admits a Cartan decomposition
$L=K+P$ where $K$ is compact and $[P,P]\subseteq K$. Furthermore, the Killing form $\kappa$ restricts
to $K$ as a negative definite form and to $P$ as a positive definite form. Finally, $P$ is a $K$-sumbodule:
$[KP]\subseteq P$.

It follows that we get an involution on $L$ by decreeing that $\theta=1$ on $K$ and $\theta=-1$ on $P$ (???) the \textbf{Cartan
involution.} Moreover $L$ admits a complexification $L\otimes_\R\C=L'$, which is complex semisimple, and
$\theta$ extends to an involution of $L'$. We classify all real forms up to isomorphism by classifying the involutions
$\theta$ up to conjugacy. It ends up there are only finitely many real forms.

\subsection{Exceptional Cases}
It turns out htat in the exceptional cases we can label real forms explicitly as $X_N(a)$ where the real form of type $X_N$ has $a=\dim P-\dim K$.
Here the compact form is always $X_N(-\dim_CX_N)$
\subsubsection{$E_6$}
This ends up being the most interesting type (although it sits inside of its bigger brother).
Here we get the real forms:
\begin{figure}[h!]
	\centering
	\begin{tabular}{c|c|c|c}
		$E_6(6)$ & $E_6(2)$ & $E_6(-14)$ & $E_6(-26)$\\\hline
		$C_4$ & $A_5\times A_1$ & $D_5\times \C$ & $F_4$
	\end{tabular}
\end{figure}
where here the $\C$ refers to the fact that there will be a (one-dimensional) nontrivial center.
\subsubsection{$E_7$}
\begin{figure}[h]
	\centering
	\begin{tabular}{c|c|c}
		$E_7(7)$ & $E_7(-5)$ & $E_7(-25)$ \\\hline
		$A_7$ & $D_6\times A_1$ & $E_6\times \C$
	\end{tabular}
\end{figure}
\subsubsection{$E_8$}
\begin{figure}[h]
	\centering
	\begin{tabular}{c|c}
		$E_8(8)$ & $E_8(-24)$ \\\hline
		$D_8$ & $E_7\times A_1$
	\end{tabular}
\end{figure}
\subsubsection{$F_4$}
\begin{figure}[h!]
	\centering
	\begin{tabular}{c|c}
		$F_4(4)$ & $F_4(-20)$ \\\hline
		$C_3\times A_1$ & $B_4$
	\end{tabular}
\end{figure}
So again we get two.
\subsubsection{$G_2$}
Here we only have one form: $G_2(2)=A_1\times A_1$.

\subsection{Diagram Automorphisms}
There are three types of Lie algebras where diagram automorphisms of order two occur: namely 
$A_n$, $D_n$ and $E_6$. In these cases, we create a different kind of extended Dynkin diagram:
we add a vertex representing the lowest weight of the quotient representation that is formed
by beginning by looking at $L$ and then quotienting out by the fixed points of the diagram 
automorphism.

\subsection{Next time...}
The last lecture of the quarter will be dedicated to what happens when $k$ has positive characteristic.
There is a classification, but it is much longer and mroe complicated than the one we have given. 

In particular, one might at first think that at least the split form of a complex semisimple Lie algebra
$L$ would have a characteristic $p$ analog. While this is true, it is not always (even) semisimple. 
The problem is that we completely lose the notion of positivity in positive characteristic, so 
the Killing form, even when restricted to a Cartan subalgebra, needn't be nondegenerate.

In the simplest case, consider $\ch k=p$ and the Lie algebra $\sl_n(k)$. The identity matrix
then has trace zero! Problems abound. Finally we will talk about the finite simple groups of Lie type,
which form the building blocks for the Lie groups we are interested in.

\section{March 15, 2019}
Let $L$ be a complex simple Lie algebra. We have seen that $L$ has a Chevalley basis
consisting of a basis of $H$, a Cartan subalgebra, with root system $\Phi$, and for each
(nonzero) root $\alpha\in\Phi$ a vector $x_\alpha\in L_\alpha$ such that whenever $\alpha,\beta,\alpha+\beta$
are all roots, $[x_\alpha,x_\beta]=c_{\alpha\beta}x_{\alpha+\beta}$. Then $c_{\alpha\beta}=-c_{-\alpha.-\beta}$
and $c_{\alpha\beta}=\pm(r+1)$ where $r$ is the largest integer (possibly zero) such that $\beta-r\alpha\in\Phi$.

Then the span $L_\Z$ over the integers of a Chevalley basis is a Lie algebra ``over $\Z$'', which is 
simple in the sense that any nonzero ideal comtains $nL_\Z$ for some nonzero $n\in\Z$. We have seen however that
the reduction $L_p$ of $L_\Z$ modulo $p$ need not be simple.

So we focus on the \textbf{group} generated by $\exp\ad cx_\alpha$ for $c\in\Z$ and $\alpha\in\Phi$.
Suppose that $\alpha,\beta,\beta+\alpha,\beta+2\alpha,\dots,$ are all roots. Then
\[\exp\ad x_\alpha(x_\beta)=\pm(r+1)(r+2)\cdots x_{\beta+n\alpha}\]
(since all root strings are actually finite, the series actually can be considered to be
finite in any one case). Thus $\pm(r+1)\cdots(r+n)/w!$ is an integer, so the terms in the series
expainsion of $\exp\ad cx_\alpha$ all preserve the $\Z$-span $L_\Z$ whence there reductions modulo
a prime $p$ preserve the corresponding span $L_p$ so that if $F$ is any field containing $\Z_p$
and if we set $L_F=L_p\otimes_{\Z/p}F$ then the group $G_F$ generated by $\exp\ad cx_\alpha$ acts on $L_F$.

Specializing down to the case when $F=\mathbb{F}_q$, the finite field of order $q=p^n$,
we therefore get a finite group acting on $L_q:=L_{\mathbb{F}_q}$. Then Chevalley proved that $L_q$ is always simple as a group, 
with just a few exceptions occurring for very small fields $F$, thereby exhibiting a number 
of families of finite simple groups, now said to be \textbf{of Lie Type.} Many of the resulting simple groups were known
before but Chevalley was the first to give a unified treatment.

\subsection{Examples}
An explicit example comes from $GL_n(\mathbb{F}_q)$. It can be shown (see many old prelims) to 
have order $q^{\binom{n}{2}}(q^n-1)\cdots(q-1)$. This is actually never simple! But luckily we want some smaller
groups! We have $SL_n(\mathbb{F}_q)$ which has order $q^{\binom{n}{2}}(q^n-1)\cdots(q^2-1)$.

Still not simple, unfortunately, but if we mod out scalars to $PSL_n(\mathbb{F}_q)$, we get a simple
group (whenever $q\ge 4$) of order $q^{\binom{n}{2}}(q^n-1)\cdots(q^2-1)/g$ where $g=\operatorname{gcd}(n,q-1)$.

We notice that the formula for the order of the simple group $PSL_n(\mathbb{F}_q)$ is given 
by a power of $q$ times a product of $q^i-1$ for varius $i$, divided the by the order of a central
subgroups.

Chevalley showed this pattern always holds: the exponent of $q$ is the dimension of the Borel
subalgebra of the Lie algebra giving rise to the group. The $q^i-1$ powers are called the \textbf{exponents}
of the Weyl group of the Lie algebra.

\subsection{Classificiations}
Let us look at all finite simple roups of nonprime order: we have
\begin{itemize}
	\item $A_n$, the alternating groups, for $n\ge 5$
	\item Finite groups of Lie type 
	\item Chevalley groups
	\item Further anaglogues of these arising from a nontrivial diagram automorphisms fo the Lie algebra
	\item Steinberg analogues on characteristic 2 and 3 on types $B_2$ and $G_2$
	\item Ree, Suzuki, and Tits (are these the discoverers of above?)
	\item Only finitely many others! 26 sporadic groups with the monster group being the largest.
\end{itemize}
\begin{rmk}
	There is also a monster Lie algebra.
\end{rmk}

Moreover, one can form an analgof of $L_\Z$ for the enveloping algebra: it has a $\Z$-form
$U_\Z$ consisting of all sums of products $\frac{x^{n_\alpha}_{-\alpha}}{n_\alpha!}\cdots\frac{x_\alpha^{n_\alpha}}{n_\alpha!}$
and this is stable under the action of the Chevalley group.

I missed the last bit because Monty was talking, but that wraps it up for the quarter!
\end{document}