\documentclass[12pt]{article}

\usepackage{setspace}

\usepackage{amsmath, amsfonts, amssymb, graphicx, color, fancyhdr, lipsum, scalerel, stackengine, mathrsfs, tikz-cd, mdframed, enumitem, framed, adjustbox, bm, upgreek, x	color}
\usepackage[framed,thmmarks]{ntheorem}
\usepackage[mathscr]{euscript}

%Replacement for the old geometry package
\usepackage{fullpage}

%set up theorem/definition/etc envs
%Problems will be created using their own counter and style
\theoremstyle{break}
\theoreminframepreskip{0pt}
\theoreminframepostskip{0pt}
\newframedtheorem{prob}{Problem}[section]

%solution template
\theoremstyle{nonumberbreak}
\theoremindent0.5cm
\theorembodyfont{\upshape}
\theoremseparator{:}
\theoremsymbol{\ensuremath\spadesuit}
\newtheorem{sol}{Solution}

%Theorems
\definecolor{thmcol}{RGB}{120,100,50}
\theoremstyle{changebreak}
\theoremseparator{}
\theoremsymbol{}
\theoremindent0.5cm
\theoremheaderfont{\color{thmcol}\bfseries} 
\newtheorem{thm}{Theorem}[subsection]

%Lemmas and Corollaries
\theoremheaderfont{\bfseries}
\newtheorem{lem}[thm]{Lemma}
\newtheorem{cor}[thm]{Corollary}
\newtheorem{prop}[thm]{Proposition}

%Create a new env that references a theorem and creates a 'primed' version
%Note this can be used recursively to get double, triple, etc primes
\newenvironment{thm-prime}[1]
  {\renewcommand{\thethm}{\ref{#1}$'$}%
   \addtocounter{thm}{-1}%
   \begin{thm}}
  {\end{thm}}

\setlength\fboxsep{15pt}

%Example
\theoremstyle{break}
\def\theoremframecommand{\colorbox[rgb]{0.9,0.9,0.9}}
\newshadedtheorem{ex}{Example}[section]

%Man, that's really good! Let's use the same thing for definitons.
\newenvironment{def-prime}[1]
  {\renewcommand{\thethm}{\ref{#1}$'$}%
   \addtocounter{thm}{-1}%
   \begin{def}}
  {\end{def}}

%proofs
\theoremstyle{nonumberbreak}
\theoremindent0.5cm
\theoremheaderfont{\sc}
\theoremseparator{}
\theoremsymbol{\ensuremath\spadesuit}
\newtheorem{prf}{Proof}

\theoremstyle{nonumberplain}
\theoremseparator{:}
\theoremsymbol{}
\newtheorem{conj}{Conjecture}

%remarks
\theoremstyle{change}
\theoremindent0.5cm
\theoremheaderfont{\sc}
\theoremseparator{:}
\theoremsymbol{}
\newtheorem{rmk}[thm]{Remark}

%Put page breaks before each part
\let\oldpart\part%
\renewcommand{\part}{\clearpage\oldpart}%

% Blackboard letters
\newcommand*{\bbA}{\mathbb{A}}
\newcommand*{\bbB}{\mathbb{B}}
\newcommand*{\bbC}{\mathbb{C}}
\newcommand*{\bbD}{\mathbb{D}}
\newcommand*{\bbE}{\mathbb{E}}
\newcommand*{\bbF}{\mathbb{F}}
\newcommand*{\bbG}{\mathbb{G}}
\newcommand*{\bbH}{\mathbb{H}}
\newcommand*{\bbI}{\mathbb{I}}
\newcommand*{\bbJ}{\mathbb{J}}
\newcommand*{\bbK}{\mathbb{K}}
\newcommand*{\bbL}{\mathbb{L}}
\newcommand*{\bbM}{\mathbb{M}}
\newcommand*{\bbN}{\mathbb{N}}
\newcommand*{\bbO}{\mathbb{O}}
\newcommand*{\bbP}{\mathbb{P}}
\newcommand*{\bbQ}{\mathbb{Q}}
\newcommand*{\bbR}{\mathbb{R}}
\newcommand*{\bbS}{\mathbb{S}}
\newcommand*{\bbT}{\mathbb{T}}
\newcommand*{\bbU}{\mathbb{U}}
\newcommand*{\bbV}{\mathbb{V}}
\newcommand*{\bbW}{\mathbb{W}}
\newcommand*{\bbX}{\mathbb{X}}
\newcommand*{\bbY}{\mathbb{Y}}
\newcommand*{\bbZ}{\mathbb{Z}}
%Fraktur letters
\newcommand*{\frakA}{\mathfrak{A}}
\newcommand*{\frakB}{\mathfrak{B}}
\newcommand*{\frakC}{\mathfrak{C}}
\newcommand*{\frakD}{\mathfrak{D}}
\newcommand*{\frakE}{\mathfrak{E}}
\newcommand*{\frakF}{\mathfrak{F}}
\newcommand*{\frakG}{\mathfrak{G}}
\newcommand*{\frakH}{\mathfrak{H}}
\newcommand*{\frakI}{\mathfrak{I}}
\newcommand*{\frakJ}{\mathfrak{J}}
\newcommand*{\frakK}{\mathfrak{K}}
\newcommand*{\frakL}{\mathfrak{L}}
\newcommand*{\frakM}{\mathfrak{M}}
\newcommand*{\frakN}{\mathfrak{N}}
\newcommand*{\frakO}{\mathfrak{O}}
\newcommand*{\frakP}{\mathfrak{P}}
\newcommand*{\frakQ}{\mathfrak{Q}}
\newcommand*{\frakR}{\mathfrak{R}}
\newcommand*{\frakS}{\mathfrak{S}}
\newcommand*{\frakT}{\mathfrak{T}}
\newcommand*{\frakU}{\mathfrak{U}}
\newcommand*{\frakV}{\mathfrak{V}}
\newcommand*{\frakW}{\mathfrak{W}}
\newcommand*{\frakX}{\mathfrak{X}}
\newcommand*{\frakY}{\mathfrak{Y}}
\newcommand*{\frakZ}{\mathfrak{Z}}
\newcommand*{\fraka}{\mathfrak{a}}
\newcommand*{\frakb}{\mathfrak{b}}
\newcommand*{\frakc}{\mathfrak{c}}
\newcommand*{\frakd}{\mathfrak{d}}
\newcommand*{\frake}{\mathfrak{e}}
\newcommand*{\frakf}{\mathfrak{f}}
\newcommand*{\frakg}{\mathfrak{g}}
\newcommand*{\frakh}{\mathfrak{h}}
\newcommand*{\fraki}{\mathfrak{i}}
\newcommand*{\frakj}{\mathfrak{j}}
\newcommand*{\frakk}{\mathfrak{k}}
\newcommand*{\frakl}{\mathfrak{l}}
\newcommand*{\frakm}{\mathfrak{m}}
\newcommand*{\frakn}{\mathfrak{n}}
\newcommand*{\frako}{\mathfrak{o}}
\newcommand*{\frakp}{\mathfrak{p}}
\newcommand*{\frakq}{\mathfrak{q}}
\newcommand*{\frakr}{\mathfrak{r}}
\newcommand*{\fraks}{\mathfrak{s}}
\newcommand*{\frakt}{\mathfrak{t}}
\newcommand*{\fraku}{\mathfrak{u}}
\newcommand*{\frakv}{\mathfrak{v}}
\newcommand*{\frakw}{\mathfrak{w}}
\newcommand*{\frakx}{\mathfrak{x}}
\newcommand*{\fraky}{\mathfrak{y}}
\newcommand*{\frakz}{\mathfrak{z}}
% Caligraphic letters
\newcommand*{\calA}{\mathcal{A}}
\newcommand*{\calB}{\mathcal{B}}
\newcommand*{\calC}{\mathcal{C}}
\newcommand*{\calD}{\mathcal{D}}
\newcommand*{\calE}{\mathcal{E}}
\newcommand*{\calF}{\mathcal{F}}
\newcommand*{\calG}{\mathcal{G}}
\newcommand*{\calH}{\mathcal{H}}
\newcommand*{\calI}{\mathcal{I}}
\newcommand*{\calJ}{\mathcal{J}}
\newcommand*{\calK}{\mathcal{K}}
\newcommand*{\calL}{\mathcal{L}}
\newcommand*{\calM}{\mathcal{M}}
\newcommand*{\calN}{\mathcal{N}}
\newcommand*{\calO}{\mathcal{O}}
\newcommand*{\calP}{\mathcal{P}}
\newcommand*{\calQ}{\mathcal{Q}}
\newcommand*{\calR}{\mathcal{R}}
\newcommand*{\calS}{\mathcal{S}}
\newcommand*{\calT}{\mathcal{T}}
\newcommand*{\calU}{\mathcal{U}}
\newcommand*{\calV}{\mathcal{V}}
\newcommand*{\calW}{\mathcal{W}}
\newcommand*{\calX}{\mathcal{X}}
\newcommand*{\calY}{\mathcal{Y}}
\newcommand*{\calZ}{\mathcal{Z}}
% Script Letters
\newcommand*{\scrA}{\mathscr{A}}
\newcommand*{\scrB}{\mathscr{B}}
\newcommand*{\scrC}{\mathscr{C}}
\newcommand*{\scrD}{\mathscr{D}}
\newcommand*{\scrE}{\mathscr{E}}
\newcommand*{\scrF}{\mathscr{F}}
\newcommand*{\scrG}{\mathscr{G}}
\newcommand*{\scrH}{\mathscr{H}}
\newcommand*{\scrI}{\mathscr{I}}
\newcommand*{\scrJ}{\mathscr{J}}
\newcommand*{\scrK}{\mathscr{K}}
\newcommand*{\scrL}{\mathscr{L}}
\newcommand*{\scrM}{\mathscr{M}}
\newcommand*{\scrN}{\mathscr{N}}
\newcommand*{\scrO}{\mathscr{O}}
\newcommand*{\scrP}{\mathscr{P}}
\newcommand*{\scrQ}{\mathscr{Q}}
\newcommand*{\scrR}{\mathscr{R}}
\newcommand*{\scrS}{\mathscr{S}}
\newcommand*{\scrT}{\mathscr{T}}
\newcommand*{\scrU}{\mathscr{U}}
\newcommand*{\scrV}{\mathscr{V}}
\newcommand*{\scrW}{\mathscr{W}}
\newcommand*{\scrX}{\mathscr{X}}
\newcommand*{\scrY}{\mathscr{Y}}
\newcommand*{\scrZ}{\mathscr{Z}}

%Section break
\newcommand*{\brk}{
\rule{2in}{.1pt}
}

%General purpose stuff
\DeclareMathOperator{\Aut}{Aut}
\DeclareMathOperator{\ch}{char}
\DeclareMathOperator{\rank}{rank}
\DeclareMathOperator{\End}{End}
\let\Im\relax
\DeclareMathOperator{\Im}{Im}

%Category Theory
\DeclareMathOperator{\Hom}{Hom}
\let\hom\relax
\DeclareMathOperator{\hom}{hom}
\DeclareMathOperator{\id}{id}
\DeclareMathOperator{\coker}{coker}
\DeclareMathOperator{\colim}{colim}
\DeclareMathOperator{\invlim}{\lim_{\leftarrow}}
\DeclareMathOperator{\dirlim}{\lim_{\rightarrow}}

%Commutative Algebra
\DeclareMathOperator{\gldim}{gldim}
\DeclareMathOperator{\projdim}{projdim}
\DeclareMathOperator{\injdim}{injdim}
\DeclareMathOperator{\findim}{findim}
\DeclareMathOperator{\flatdim}{flatdim}
\DeclareMathOperator{\depth}{depth}

%Common Categories
%\newcommand*{\modR}{\mathbf{mod}\text{-}R}
%\newcommand*{\Rmod}{R\text{-}\mathbf{mod}}
\newcommand{\rmod}[1]{\mathbf{mod}\text{-}#1}
\newcommand{\lmod}[1]{#1\text{-}\mathbf{mod}}
\DeclareMathOperator{\Vectk}{\mathbf{Vect}_k}
\DeclareMathOperator{\Ch}{\mathbf{Ch}}
\newcommand*{\Ab}{\mathbf{Ab}}
\newcommand*{\Grp}{\mathbf{Grp}}
\newcommand*{\Alg}{\mathbf{Alg}_k}
\newcommand*{\Ring}{\mathbf{Ring}}
\newcommand*{\K}{\mathbf{K}}
\newcommand*{\D}{\mathbf{D}}
\newcommand*{\Db}{\mathbf{D}^b}
\newcommand*{\Dpos}{\mathbf{D}^+}
\newcommand*{\Dneg}{\mathbf{D}^-}
\newcommand*{\Dbperf}{\mathbf{D}^b_{\text{perf}}}
\newcommand*{\Dsing}{\mathbf{D}_{sing}}
\newcommand{\CRing}{\mathbf{CRing}}
\DeclareMathOperator{\stmod}{\mathbf{stmod}}
\DeclareMathOperator{\StMod}{\mathbf{StMod}}
\DeclareMathOperator{\sHom}{\underline{Hom}}

%Homological algebra
\DeclareMathOperator{\cone}{cone}
\DeclareMathOperator{\HH}{HH}
\DeclareMathOperator{\Der}{Der}
\DeclareMathOperator{\Ext}{Ext}
\DeclareMathOperator{\Tor}{Tor}

%Lie algebras
\DeclareMathOperator{\ad}{ad}
\newcommand*{\gl}{\mathfrak{gl}}
\let\sl\relax
\newcommand*{\sl}{\mathfrak{sl}}
\let\sp\relax
\newcommand*{\sp}{\mathfrak{sp}}
\newcommand*{\so}{\mathfrak{so}}

% Hacks and Tweaks
% Enumerate will automatically use letters (e.g. part a,b,c,...)
\setenumerate[0]{label=(\alph*)}
% Always use wide tildes
\let\tilde\relax
\newcommand*{\tilde}[1]{\widetilde{#1}}
%raise that Chi!
\DeclareRobustCommand{\Chi}{{\mathpalette\irchi\relax}}
\newcommand{\irchi}[2]{\raisebox{\depth}{$#1\chi$}} 



%class-specific macros
\DeclareMathOperator{\sExt}{\mathscr{E}xt}

%header stuff
\setlength{\headsep}{24pt}  % space between header and text
\pagestyle{fancy}     % set pagestyle for document
\lhead{Notes on Homological Algebra} % put text in header (left side)
\rhead{Nico Courts} % put text in header (right side)
\cfoot{\itshape p. \thepage}
\setlength{\headheight}{15pt}
\allowdisplaybreaks

\begin{document}
%make the title page
\title{Homological Algebra\vspace{-1ex}}
\author{A course by: Prof. Julia Pevtsova\\
Notes by: Nico Courts}
\date{Spring 2019}
\maketitle

\renewcommand{\abstractname}{Introduction}
\begin{abstract}
	These are the notes I took in class during the Winter 2019 topics course
	\textit{Math 509 - Homological Algebra} at University of Washington, Seattle. 
	
	The course description follows:

	\brk

	This course is an introductory course on homological algebra. We will be following the 
	book \textit{An Introduction to Homological Algebra} by Charles Weibel. We will be
	covering the following topics:

\begin{itemize}
	\item Chain complexes, homotopies, homology and long exact sequence in homology
	\item Resolutions, derived functors, Ext and Tor. Koszul complexes
	\item Group (co)homology
	\item Triangulated and derived categories
	\item Spectral sequences or open topic depending on the class interests
\end{itemize}
\end{abstract}

\section{April 1, 2019}

The idea here is that we are going to spend a significant amount of time doing traditional lectures.
After a while we are going to be doing some presentations. We have two weeks to pick groups
(and topics, if desired) before we are assigned to them randomly.

Again, we are using Weibel's book. At the beginning we are going to be following the text quite closely,
so that is a perfect reference.

\subsection{Chain Complexes, Maps, Homotopies, and Homology}
Let $R$ be a ring. In general we will say it is unital and associative, but not much more.
We will be generally working in the category $\mathbf{Mod}_R$ of (left) $R$-modules.

\begin{defn}
	A chain complex $C_\bullet$ is a sequence of $R$-modules and $R$-module homomorphisms
	\[d_n:C_n\to C_{n-1}\]
	where the $C_n$ are $R$-modules and $f_n$ are $R$-module homomorphisms, subject to the relation
	\[d_n\circ d_{n-1}=0.\]
\end{defn}
The first HW problem:
\begin{prob}
	Show that $\Ch$, the category of chain complexes forms an abelian category.
\end{prob}

\begin{defn}
	Given two chain complexes $C_\bullet$ and $D_\bullet$, then $f_\bullet:C_\bullet\to D_\bullet$
	is a \textbf{chain map} if
	\[f_n:C_n\to D_n\]
	is a map of $R$-modules and the obvious diagram commutes. (figure~\ref{fig-commute})
\end{defn}
\begin{figure}\label{fig-commute}
	\centering
	\begin{tikzcd}
		C_n \ar[r,"d_n"]\ar[d,"f_n"] & C_{n-1}\ar[d,"f_{n-1}"]\\
		D_n\ar[r,"d'_n"] & D_{n-1}
	\end{tikzcd}
	\caption{Commuting Figure for Chain Maps}
\end{figure}

\begin{defn}
	Let $C_\bullet$ be a chain complex. Then $Z_n=Z_n(C_\bullet)$ is defined to be
	\[Z_n=\ker d_n\]
	called the $n$-\textbf{cycles} of $C_\bullet$.

	Similarly, define
	\[B_n=\Im d_{n+1}\]
	the $n^{th}$ \textbf{boundary} of $C_\bullet$. 

	Then $H_n=Z_n/B_n$, the $n^{th}$ \textbf{homology group}. 
\end{defn}

\subsection{Some Notation}
The category $\mathbf{Ch}_\bullet$ (double underline!!!) is the category of (unbounded) complexes,
but we are also sometimes interested in $\mathbf{Ch}_\bullet^b$, the category of \textbf{bounded}
complexes. Furthermore, we may consider $\mathbf{Ch}_\bullet^{\ge 0}$ or $\mathbf{Ch}_\bullet^{\le 0}$

Furthermore, there are cochain complexes, where we switch the direction and the notation is changed as
we do for any dual notion.

\subsection{Types of Complexes}
\begin{defn}
	A complex $C_\bullet$ is called \textbf{exact} if it is an exact sequence. Equivalently the 
	homology groups are trivial.
\end{defn}
\begin{rmk}
	A complex with trivial homology is also called \textbf{acyclic}. This is obviously equivalent to being 
	exact.
\end{rmk}
\begin{lem}
	Let $f:C_\bullet\to D_\bullet$ be a map of chain complexes. Then it induces a well-defined map on 
	cycles, boundaries, and homology.
\end{lem}
\begin{prf}
	Homework.
\end{prf}

\begin{defn}
	Let $f:C_\bullet\to D_\bullet$ be a chain map. It is called a \textbf{quasi-isomorphism}
	if it induces isomorphisms on homology groups.
\end{defn}
\begin{rmk}
	An acyclic complex $C_\bullet$, then $C_\bullet\to 0$ is a quasi-isomorphism.
\end{rmk}
\subsection{Snake Lemma}
\begin{lem}[Snake]
	Given a commutative diagram with exact rows as in figure~\ref{fig-snake},
	we have a long exact sequence
	\[\ker f\to \ker g\to \ker h\xrightarrow{\delta} \coker f\to \coker g\to \coker h\to 0\]
\end{lem}
\begin{figure}[h!]\label{fig-snake}
	\centering
	\begin{tikzcd}
		& A'\ar[r]\ar[d,"f"] &B'\ar[r]\ar[d,"g"] &C\ar[d,"h"]\ar[r] & 0\\
		0\ar[r]& A\ar[r] & B\ar[r] & C &
	\end{tikzcd}
	\caption{Snake Lemma}
\end{figure}

\subsection{An Example}
We talked about singular homology. Cool!

\section{April 3, 2019}
\subsection{Long Exact Sequence of Homology}
\begin{defn}
	A short exact sequence $0\to A_\bullet\xrightarrow{f} B_\bullet\xrightarrow{g} C\bullet\to 0$
	of chain complexes is a width four double complex with exact rows.
\end{defn}
\begin{thm}
	Let $0\to A_\bullet\to B_\bullet\to C_\bullet\to 0$ be a short exact sequence of complexes. Then 
	there exists a (natural) long exact sequence in homology:
	\[\cdots\to H_n(A)\to H_n(B)\to H_n(C)\to H_{n-1}(A)\to\cdots\]
\end{thm}
\begin{rmk}
	The statement of naturality just means that the maps above are induced from those in the SES
	under application of $H_n$ functor.
\end{rmk}
\begin{prf}
	Snaaaaaaaake.

	(Sketch) Use first the fact that $d:A_n\to A_{n-1}$ factors through both $A_n/B_n(A_\bullet)$ and $Z(A_\bullet)$,
	so using the induced maps between these quotients to get a nice little snake-like diagram:

	\begin{center}
	\begin{tikzcd}
		& A_n/B(A_n)\ar[r]\ar[d] & B_n/B(B_n)\ar[r]\ar[d] & C_n/B(C_n)\ar[r]\ar[d] & 0\\
		0\ar[r] & Z_{n-1}(A)\ar[r] & Z_{n-1}(B)\ar[r] & Z_{n-1}(C)\ar[r] & 0
	\end{tikzcd}
	\end{center}	

	where the exactness of the rows above follows from applications of snake lemma to the original
	SES of chains.

	A final application of snake gets us the SES we desire.
\end{prf}
\subsection{Some Topology}
\begin{defn}
	Let $f:C_\bullet\to D_\bullet$ be a chain map. We say $f$ is \textbf{nullhomotopic} if there are $s_n:C_n\to D_{n+1}$
	for all $n$ such that
	\[f_n=d_{n+1}\circ s_n+s_{n-1}\circ d_n.\]

	Sometimes we just write $f=ds+sd$ and let the indicies figure themselves out.
\end{defn}
\begin{defn}
	If $f,g:C_\bullet\to D_\bullet$ are chain maps, we say that they are \textbf{chain homotopic}
	if $f-g$ is nullhomotopic.

	Here we write $f\sim g$, so $f\sim 0$ means $f$ is nullhomopotic.
\end{defn}
\begin{defn}
	We say $C_\bullet$ and $D_\bullet $ are \textbf{chain homotopy equivalent} if there
	exist maps $f:C\to D$ and $g:D\to C$ such that $f\circ g\sim\id_D$ and $g\circ f\sim \id_C$.
\end{defn}
\begin{defn}
	We say a complex $C_\bullet$ is \textbf{contractible} if $\id_C\sim 0$.
\end{defn}
\begin{rmk}
	Some motivation: say we have $X,Y$ topological spaces and $f,g:X\to Y$ continuous maps. Then $f\sim g$
	if there is a continuous map $H:X\times [0,1]\to Y$ such that $H(x,0)=f(x)$ and $H(x,1)=g(x)$.

	Now if $f\sim g$, then the natural (induced) maps $S_\bullet(f)$ and $ S_\bullet(g)$ on singular 
	chain complexes are chain homotopic.
\end{rmk}
\begin{thm}
	If $f,g:C_\bullet\to D_\bullet$ are chain homotopic, then they induce the same map on $H_\bullet$.
\end{thm}
\begin{prf}
	We will be using that homology is an additive functor: $H_n(f)-H_n(g)=H_n(f-g)$. Then it suffices to show 
	that if $f\sim 0$ then $H_n(f)=0$ for all $n$.

	Assume $f\sim 0$ with homotopy $s_\bullet:C_\bullet\to D_\bullet[1]$. Then we have
	\begin{center}
		\begin{tikzcd}
			C_{n+1}\ar[r,"d"] & C_n\ar[r,"d"]\ar[ld,"s"]\ar[d,"f"] & C_{n-1}\ar[ld,"s"]\\
			D_{n+1}\ar[r,"d"] & D_n\ar[r,"d"] & D_{n-1}
		\end{tikzcd}
	\end{center}

	\noindent and so if $a\in Z_n(C)$, then say $H_n(f)(a)=f(a)$. But 
	\[f(a)=(ds+sd)(a)=d_{n+1}(s(a))+s(d_n(a))=d_{n+1}(s(a))\in B_n(D).\]
\end{prf}
\begin{rmk}
	We can define the \textbf{homotopy category} $K_\bullet(\mathbf{Mod}_R)$ or $K_\bullet(\mathbf{Ch}_\bullet)$
	where the objects are chain complexes and the morphisms are chain homotopy equivalence classes of maps.

	This category is additive and in our homework we will investigate whether it is abelian. So then 
	our homology functors $H_n:\Ch(R)\to \mathbf{Ab}$ factor through $K_\bullet(R)$.
\end{rmk}

\subsection{Split exact complexes}
\begin{defn}
	$C_\bullet$ is a \textbf{split exact sequence} if its an exact sequence that splits. That is,
	\[C_n=B_{n+1}(C_\bullet)\oplus B_n(C_\bullet)\]
	or since the sequence is exact, the sum of cycles.

	The boundary maps simply map isomorphic summands to one another.
\end{defn}
\begin{ex}
	$R=k$. Then $\mathbf{Mod}_R$ is actually $\mathbf{Vect}_k$. All exact sequences here split.
\end{ex}

\section{April 5, 2019}
\begin{ex}
	Continuing from our example on Wednesday, let $R=k$ and consider $\Ch(k)$. Everything in $\Vectk$ splits, 
	so we can write $C_n=Z_n\oplus C_n/Z_n\cong Z_n\oplus B_{n-1}$.

	But then $C_n\cong B_n\oplus H_n\oplus B_{n-1}$, so $C$ is exact iff $H_n=0$ (it is acyclic)
	iff $C$ is split exact.
\end{ex}
\begin{rmk}
	In general, split exact implies acyclic, but the converse is not true.
\end{rmk}
\begin{ex}
	Consider $\Ch(\bbZ)$ and consider the chain of $\cdots \to\bbZ/4\to\bbZ/4\to\bbZ/4\to\cdots$
	with differential $\ast 2$.

	Notice that $H_n(C)=0$, but it is NOT nullhomotopic/contractible! This is because $\id=sd+ds$ maps
	only to even numbers!
\end{ex}

\subsection{Operations with complexes}
We can do some operations on complexes $C\in\Ch$:
\begin{itemize}
	\item $(C[d])_i=C_{i+d}$
	\item Total complexes
\end{itemize}
\begin{defn}
	A \textbf{double (chain) complex} is a collection of $R$-modules and maps $C_{pq}$ 
	\begin{center}
		\begin{tikzcd}
			C_{pq}\ar[r,"d^h"]\ar[d,"d^v"] & C_{p-1\,q}\ar[d,"d^v"]\\
			C_{p\,q-1}\ar[r,"d^h"] & C_{p-1\,q-1}
		\end{tikzcd}
	\end{center}

	such that $(d^h)^2=(d^v)^2=d^h\circ d^v+d^v\circ d^h=0.$
\end{defn}
\begin{ex}
	Let $f:C\to C'$ be a chain map. Then 

	\begin{center}
		\begin{tikzcd}
			C_{n+1}\ar[r,"-f"] \ar[d,"-d"] & C_{n+1}'\ar[d,"d'"]\\
			C_n\ar[r,"-f"]\ar[d,"-d"] & C_n'\ar[d,"d'"]\\
			C_{n-1}\ar[r,"-f"] & C_{n-1}'
		\end{tikzcd}
	\end{center}

	Is a double complex. Notice we also could use $f$ instead of $-f$ but for some reason this 
	is usually what we want.
\end{ex}
\begin{defn}
	Let $(C_{pq},d)$ be a double complex. Then
	\[\operatorname{Tot}_n(C_{\bullet\bullet})=\oplus_{p+q=n}C_{pq}\]
	with differential $d=d^v+d^h$. This is a chain complex.
\end{defn}
\begin{rmk}
	Notice that $(d^h+d^v)^2=d^h\circ d^h+d^h\circ d^v+d^v\circ d^h+d^v\circ d^v=0.$ So this
	is indeed a chain complex.
\end{rmk}
\begin{rmk}
	Notice that we have technically used the fact that our category contains coproducts to
	define $\operatorname{Tot}=\operatorname{Tot}^\oplus$ as we just did. These should exist 
	in an abelian category.
\end{rmk}
\begin{ex}
	Let $f:B\to C$ be a chain map. Then $\operatorname{Cone}(f)$ is the total complex of the associated
	double complex of $f$. 

	Here $(\operatorname{Cone}(f))_n=B_{n-1}\oplus C_n$ (careful, the indices are a bit wonky here.)
	Then the differential is given by $(\begin{smallmatrix}
		-d_B & 0\\ -f & d_C
	\end{smallmatrix})$.
\end{ex}
\begin{rmk}
	Notice that
	\[B_\bullet\xrightarrow{f} C_\bullet\to (\operatorname{Cone}(f))_\bullet\to B_\bullet[-1]\]
	is a short(?!) exact sequence of complexes.
\end{rmk}
\begin{rmk}
	This idea comes from an idea in topology: given a continuous map $f:X\to Y$
\end{rmk}

\section*{April 8, 2019}
Some quick comments about the topological analogy we discussed last time (I trailed off because
there was a question about what the map between topological spaces induces this map on homology):

The idea is that although the image of $f$ is a deformed copy of $X$ in $\operatorname{Cone}(f)$, 
the suspension $\Sigma X$ is somehow recovered in the cone over $\Im f$.

This is a bit hand-wavy, but we then recover via the long exact sequence in homology the sequence
we were cosidering last time.

\subsection*{Derived Functors, $\Ext$, and $\Tor$}
Recall that we are working over $\Rmod$.
\begin{defn}
	An $\Rmod$ $P$ is projective if one of the following equivalent conditions hold:
	\begin{itemize}
		\item This diagram commutes:
		\begin{center}\begin{tikzcd}
		& P\ar[ld,swap,dashed,"\tilde f"]\ar[d,"f"]\\
		M\ar[r,two heads,"\varphi"]& N
		\end{tikzcd}\end{center}
		\item $\Hom_R(P,-):\Rmod\to\Ab$ is an exact functor.
		\item $P$ is a direct summand of a free module. \textit{Notice that this only works in $\Rmod$}
	\end{itemize}
\end{defn}
\begin{rmk}
	Notice that we are generally assuming (as in the definition of an exact functor) that our
	functors are additive -- that is, it preserves sums. 
\end{rmk}

A question one may ask: what are the projective objects in $\Ch(R)$? This will be the topic of a homework 
question.
\begin{defn}
	A \textbf{projective resolution} $P_\bullet$ of an $R$ module $M$ is a complex
	\[\cdots\to P_2\to P_1\to P_0\]
	of projective $R$ modules such that if we tack on $\varepsilon:P_0\to M$, it becomes exact.

	Equivalently, the chain map induced by $(\cdots,0,\varepsilon):P_\bullet\to (\cdots\to 0\to M\to 0)$ 
	is a quasi-isomorphism.
\end{defn}
\begin{defn}
	A category $\mathcal A$ has \textbf{enough projectives} if, for all $A\in\mathcal A$, there exists
	a projective $P\in\mathcal A$ and an epimorphism $P\to A$.
\end{defn}
\begin{lem}
	$\Rmod$ has enough projectives.
\end{lem}
\begin{prf}
	Let $P_0$ be any free module over $M$ (say the free module on all elements of $M$) and say $\varepsilon:P_0\to M$ is the covering map.
	and let $\Omega M=\ker \varepsilon$, the zeroth syzygy of $M$. Then let $P_1$ be a projective cover of $\Omega_0M$
	and continue in the fashion.
\end{prf}
\begin{cor}
	Any $R$-module has a projective resolution.
\end{cor}

\subsection*{Fundamental Theorem of Homological Algebra}
\begin{thm}[Fundamental Theorem]
	If $P_\bullet$ is a projective complex with map $\varepsilon_M:P_0\to M$ (not necessarily projective) and $Q_\bullet\to N$ is a projective 
	resolution of $N$ and $f:M\to N$ is an $R$-linear map, then there exists an $F:P_\bullet \to Q_\bullet$ 
	extending $f$. That is it makes following diagram commute:
	\begin{center}
		\begin{tikzcd}
			P_\bullet\ar[r,"\varepsilon_M"]\ar[d,"F",dashed] & M\ar[d,"f"]\\
			Q_\bullet \ar[r,"\varepsilon_N"] & N
		\end{tikzcd}
	\end{center}

	Furthermore $F$ is unique up to chain homotopy.
\end{thm}
\begin{rmk}
	Notice that while any two extensions are chain homotopic, this doesn't mean there exists
	a \textit{unique} homotopy between them and this is the cause of many problems in homological algebra.
\end{rmk}
\begin{prf}
	Begin by lifting $f\circ \varepsilon_M$ to a map $f_0:P_0\to Q_0$ using the definition of projectivity.
	Then construct $f_n$ by induction. Consider
	\begin{center}
		\begin{tikzcd}
			P_n\ar[r,"d^P"] \ar[rd] & P_{n-1}\ar[r]\ar[d,"f_{n-1}"] & P_{n-2}\ar[d,"f_{n-2}"]\\
			Q_n\ar[r] & Q_{n-1}\ar[r,"d^Q"] & Q_{n-2}
		\end{tikzcd}
	\end{center}
	and by the induction hypothesis, $d^Q\circ f_{n-1}\circ d^P=0$ whence $\varphi:P_n\to Q_{n-1}$ 
	maps to $\ker d^Q_{n-1}=\Im d^Q_n$. But this gives us a map $P_n\to \Im d^Q$ onto which $Q_n$ surjects,
	so again we leverage the projectivity of $P_n$ to lift to $f_n$.

	To see the uniqueness, assume $F$ and $F'$ are two liftings of $f$. We want to show $H=F-F'\sim 0$.
	It will be sufficient once we see that any lifting of $0:M\to N$ is nullhomotopic. We will see this next time.
\end{prf}

\section{April 10, 2019}
\begin{thm}
	... and it is unique up to chain homotopy.
\end{thm}
\begin{prf}
	Say that $F$ and $F'$ are two chain maps extending $f:M\to N$ in the above scenario.
	Notice that since we want to show $F\sim F'$, it suffices to show that $0:M\to N$ induces 
	a nullhomotopic map $H$.

	Let $H$ be a lifting of $0$ and define $s_i$ such that
	\begin{center}
		\begin{tikzcd}
			P_1\ar[r]\ar[d,"h_1"] & P_0\ar[dl,"s_0"]\ar[d,"h_0"],\ar[r] & M\ar[d,"0"]\ar[dl,"s_{-1}=0"]\\
			Q_1\ar[r] & Q_0 \ar[r,"\varepsilon_N"] & N
		\end{tikzcd}
	\end{center}
	where we need to have that $h_0=s_0\circ d_Q$ for some $s_0$. we know that $h_0(P_0)\subseteq\ker\varepsilon_N=d_Q(Q_1)$
	so lifting $h_0$ through $Q_1\twoheadrightarrow d(Q_1)$, we get a map $s_0:P_0\to Q_1$ (since $P_0$ is projective).

	Continuing my induction, consider
	\begin{center}
		\begin{tikzcd}
			P_{n+1}\ar[r]\ar[d,"h_{n+1}"] & P_n\ar[dl,"s_n"]\ar[d,"h_n"],\ar[r] & P_{n-1}\ar[d,"h_{n-1}"]\ar[dl,"s_{n-1}"]\\
			Q_1\ar[r] & Q_0 \ar[r,"\varepsilon_N"] & Q_{n-1}
		\end{tikzcd}
	\end{center}
	and consider the map $s_n=h_n-(s_{n-1}\circ d_P)$, which maps $P_n$ into $d_Q(Q_{n+1})$. Now $d_Q(Q_{n+1})=\ker d_Q$ (easy lemma)
	This establishes that $s_n$ exists and satisfies all the requisite properties.
\end{prf}
\begin{cor}
	Any two projective resolutions of $M$ induce the same homology.
\end{cor}
\begin{rmk}
	Lift the identity map $M\to M$ in both directions to get chain maps $(P\to M)\to(Q\to M)$
	and backwards. These compose to the identity map! Then apply the theorem!
\end{rmk}

\begin{lem}[Horseshoe]
	(\textbf{Weibel 2.2.8}) Say that we have a short exact sequence $0\to A\to B\to C\to 0$ 
	and projective resolutions $P^A_\bullet$ and $P_\bullet^C$ of $A$ and $C$, respectively.
	Then $P_\bullet^B=(P_i^A\oplus P_i^C)_{i\in\bbZ}$ is a resolution of $B$ and furthermore we have 
	maps for the following diagram (which commutes everywhere):

	\begin{center}
		\begin{tikzcd}
			P_2^A\ar[r] & P_1^A\ar[r]\ar[d] & P_0^A\ar[r]\ar[d] & A\ar[d]\\
			 & P_1^A\oplus P_1^C\ar[r]\ar[d] & P_0^A\oplus P_0^B\ar[r]\ar[d] & B\ar[d]\\
			P_2^C\ar[r] & P_1^C\ar[r] & P_0^C\ar[r]s & C
		\end{tikzcd}
	\end{center}
\end{lem}

\subsection{Left Derived Functors}
\begin{defn}
	Let $\mathcal A, \mathcal B$ be Abelian categories and let $F:\mathcal A\to \mathcal B$
	be an additive functor. We say that $F$ is \textbf{right exact} if for every exact sequence
	$A\to B\to C\to 0$ in $\mathcal A$, 
	\[F(A)\to F(B)\to F(C)\to 0\]
	is exact in $\mathcal B$.
\end{defn}
\begin{defn}
	Assume $\mathcal A$ is an abelian category with enough projective and that $\mathcal B$ is abelian.
	Let $F:\mathcal A\to \mathcal B$ be a right exact functor. Then $L_iF$ for $i\ge 0$ are called
	\textbf{the left derived functors of $F$} which are defined as follows:

	For all $A\in\mathcal A$, let $P_\bullet\to A$ be a projective resolution. Then we have an 
	exact sequence
	\[\cdots\to F(P_2)\to F(P_1)\to F(P_0)\]
	and we define $(L_iF)(A):=H_i(F(P_\bullet))$.
\end{defn}
\begin{rmk}
	\begin{enumerate}
		\item $L_i F$ are functors because of the fundamental theorem.
		\item The functors are well-defined because of the lemma.
	\end{enumerate}
\end{rmk}
The derived functors have some nice properties:
\begin{itemize}
	\item $L_i F= 0$ for $i<0$.
	\item Say $0\to A\to B\to C\to 0$ is short exact and $F$ is right exact. Then we get a long exact sequence
	\[\cdots \to L_1F(A)\to L_1F(B)\to L_1F(C)\to F(A)\to F(B)\to F(C)\to 0\]
\end{itemize}

\section{April 17, 2019}
Last time we wrote down the definition for (left) derived functors, but never wrote out why $\Tor$ is an example.
\subsection{Examples of $L_i$}
Let $R$ be a ring and $M$ and $N$ be right- and left-$R$-modules, respectively. Then $M\otimes_RN$ 
is the set of $m\otimes n$ for $m\in M$ and $n\in N$ where $m\cdot a\otimes n=m\otimes a\cdot n$
for all $a\in R$.

This can also be defined formally as a bifunctor $-\otimes_R-:\Rmod\times\modR\to\Ab$, or as a regular
functor by currying. So then

\begin{defn}
	\[\Tor_i^R(M,N):=L_i(M\otimes_R-)(N)\]
	and
	\[\Tor_i^R(M,N)=L_i(-\otimes_R M)(N)\]
\end{defn}
\begin{rmk}
	Notice that the two definitions are completely equivalent and compatible. That is, $\Tor_i^R(M,N)$ 
	is what we call \textbf{balanced.}

	To see this you can do a few computations including computing the homology of the total complex to show that 
	you always get the same thing.
\end{rmk}

\subsection{Injective Modules}
\begin{defn}
	Let $\mathcal A$ be an abelian category. Then $E\in \mathcal A$ is injective if we have the diagram
	\begin{center}
		\begin{tikzcd}
			M\ar[d,"f"]\ar[r,hook] & N\ar[dl,dashed,"\tilde f"]\\
			E &
		\end{tikzcd}
	\end{center}
\end{defn}
\begin{lem}
	If $\mathcal A=\Rmod$, then in the above definition it suffices to check that maps $f:I\to E$ from
	every \textit{ideal} lift to maps $\tilde R\to E$ from $R$.
\end{lem}
\begin{ex}
	Injective $\bbZ$-modules are precisely the divisible abelian groups. For instance, $\bbQ$, $\bbQ/\bbZ$, $\bbZ_{p^\infty}=\bbZ_{(p)}/\bbZ$.
\end{ex}
In fact:
\begin{lem}
	The indecomposable injective $\bbZ$ moduels are $\bbQ$ and $\bbZ_{p^\infty}$.
\end{lem}
\begin{thm}
	For any ring $R$, $\modR$ has enough injectives.
\end{thm}
\begin{cor}
	Every $M\in\modR$ has an injective resolution.
\end{cor}
\begin{prf}
	It suffices to show that for all $M\in\modR$, there is an injective $R$-module $E$ 
	such that $M\hookrightarrow E$.

	To begin, we will see that $\bbZ$-$\mathbf{mod}=\Ab$ has enough injectives. Let $A\in\Ab$
	and define the notation
	\[A^{\vee}=\Hom_{\Ab}(A,\bbQ/\bbZ).\]
	Then $\bbZ^\vee=\bbQ/\bbZ$, since any map is determined uniquely by its image. So then if $F=\bbZ^n$,
	we use linearity of $\Hom$ to prove $F^\vee=(\bbQ/\bbZ)^n$, so all free groups are divisible (whence injective).

	Finally, for all $A$, there is a natural map $A\to A^{\vee\vee}$. Now for any $A\in\Ab$, there exists a 
	surjective map $F\to A^{\vee}$ from a free module onto the dual of $A$. Thus we get an embedding
	$A^{\vee\vee}\hookrightarrow F^\vee$ (note that $\Hom(-,\bbQ/\bbZ)$ is exact). Since $A\hookrightarrow A^{\vee\vee}$,
	this gets us an embedding of $A$ into $F^\vee$.

	\brk

	Now let $R$ be any ring. Let $A\in\Ab$, then the claim is that $\Hom_\bbZ(R,A)$ is a (right) $R$-module via
	the action $(f\cdot a)(b)=f(ba)$ (\textbf{check this}).

	Next, we claim that there exists a natural isomorphism for all $M\in\modR$ and $A\in\Ab$
	such that
	\[\Hom_\bbZ(M,A)\cong\Hom_R(M,\Hom_\bbZ(R,A)).\]
	We claim that one can ``follow their nose'' here.

	Using these facts, we will finish the proof of the main theorem: Let $A$ be a divisible abelian group.
	Then $\Hom_\bbZ(R,A)$ is injective. This is because 
	\[\Hom_R(-,\Hom_\bbZ(R,A))\]
	is exact since $A$ is injective (by the last claim).

	Let $M$ be any right $R$-module. Then in particular it is an abelian group, so 
	$M\hookrightarrow A\in\Ab$, where $A$ is divisible. Thus (again by the last fact) we have a map
	$\varphi:M\to \Hom_\bbZ(R,A)$ is a map of $R$-modules. Then it suffices to show that $\varphi$ is injective.

	This follows from the explicit construction of the natural isomorphism between the functors.
\end{prf}
\begin{rmk}
	The map referred to above takes $f\in \Hom_Z(M,A)$ to a map $\varphi_f\in\Hom(M,\Hom_\bbZ(R,A))$
	where
	\[\varphi_f(a)(r)=(f\cdot a)(r).\]

	To show it is injective we just construct the inverse map: let $\psi_g$ for any $g:M\to\Hom_\bbZ(R,A)$
	to be
	\[\psi_g(a)=(g(a))(1).\]
	One can easily check that this composes to the identity in each direction.
\end{rmk}

\section{April 19, 2019}
I came late to the discussion, but it looks like we defined left exact contravariant functors --
those are functors that are left exact as functors $F:C^{op}\to D$.

\begin{rmk}
	$\Hom_R(-,W)$ is left exact.
\end{rmk}

From here, we can continue the sequence to a long exact sequence of homology. The point of this 
section was to show that we can compute $\Ext_R^i(V,W)$ \textit{either} from a projective resolution of $V$ or 
an injective resolution of $W$.

\begin{lem}
	$P$ is a projective $R$ module if and only if $\Ext_R^i(P,-)=0$ for all $i>0$.
\end{lem}
\begin{lem}
	$I$ is an injective $R$ module if and only if $\Ext_R^i(-,I)=0$ for all $i>0$
\end{lem}
\subsection{Adjoint Functors}
\begin{defn}
	We say the functors $\cal F:\cal A\to \cal B$ and $\cal G:\cal B\to \cal A$ are \textbf{adjoint functors}
	if there exists a natural isomorphisms
	\[\eta_{AB}:\Hom_{\cal B}({\cal F}(A),B)\to \Hom_{\cal A}(A,{\cal G}(B))\]
	and in this case we say $\cal F$ is left adjoint to $\cal G$.
\end{defn}
\begin{ex}
	Recall we hat the functors $\cal F$ and $\cal G$ from $\bbZ$-modules to $R$-modules and back, respectively,
	where $\cal F=\Hom_\bbZ(R,-)$, assigning to each $\bbZ$ module an $R$-module with the action we discussed last time.

	Then $\cal G=\operatorname{Res}_\bbZ^R$, the forgetful functor, is its left adjoint.
\end{ex}
\begin{ex}
	In rep theory we have Frobenius Reciprocity! Woo.
\end{ex}

\section{April 22, 2019}
Recall the tensor-hom adjunction: $-\otimes_S B$ and $\Hom_R(B,-)$ are adjunctions between $\modR$ and $\mathbf{mod}$-$S$.
There is some shadiness here (mirroring what we saw before) but the fact that 
\[(f.s)(b)=f(sb)\]
defines a right $S$ action on $\Hom_R(B,C)$ is not immediately evident to me.

Again, many of the common adjunctions we see are actually just special cases of tensor-hom.

\subsection{Why Adjointness Matters}
\begin{lem}
	If $\cal F$ and $\cal G$ are adjoints ($\cal F$ is left adjoint to $\cal G$), then $\cal F$ is
	right exact and $\cal G$ is left exact.
\end{lem}
\begin{prf}
	Let $0\to A' \to A\to A''\to 0$ be a short exact sequence. Then consider the diagram
	\begin{center}
		\begin{tikzcd}
			\Hom_{\cal A}(A',\cal G(B))\ar[d,"\sim"]& \ar[l] \Hom_{\cal A}(A,\cal G(B))\ar[d,"\sim"] &\ar[l]\Hom_{\cal A}(A'',\cal G(B)\ar[d,"\sim"])& \ar[l]0\ar[d,"\sim"]\\
			\Hom_{\cal B}(\cal F(A'),B) & \ar[l] \Hom_{\cal B}(\cal F(A),B) & \ar[l] \Hom_{\cal B}(\cal F(A''),B) & \ar[l] 0
		\end{tikzcd}
	\end{center}
	and since the top sequence is exact (by left exactness of $\Hom$), and by the naturality of the isomorphism
	between the functors, we get exactness below.
\end{prf}
\begin{rmk}
	There is a fact that $C'\to C\to C''\to 0$ is exact \textbf{if and only if} $\Hom_{\cal B}(C_\bullet,B)$ is 
	exact \textit{for all $B$.} One direction is just left exactness, the other basically uses Yoneda lemma.
\end{rmk}
\begin{thm}
	If $\cal F$ and $\cal G$ are adjoint (left and right, resp.), then
	\begin{enumerate}
		\item $\cal F$ commutes with colimits (e.g. coproducts, $\oplus$, cokernels).
		\item $\cal G$ commutes with limits (e.g. products and kernels).
	\end{enumerate}
\end{thm}
\begin{cor}
	Let $\cal A$ be an abelian category with enough projectives and let $\cal F:\cal A\to \cal B$ be right exact.
	Then
	\[(L_i\cal F)\left(\bigoplus_j A_j\right)\cong \bigoplus_j L_i\cal F(A_j).\]
\end{cor}

\begin{ex}
	Let $R$ be a PID. Consider torsion modules (free ones are uninteresting when computing $\Tor$).
	Let's compute $\Tor_\ast^R(R/(a),M)$ for any $M$. Take the resolution
	\[0\to R\xrightarrow{\cdot a} R\to R/(a)\]
	which is free whence projective. Then compute the homology of
	\[0\to R\otimes_R M\xrightarrow{\cdot a} R\otimes_RM\to 0\]
	and since $R\otimes_RM\cong M$, we get $\Tor_{\ge 2}^R(R/(a),M)=0$. Then $\Tor_0^R(R/(a),M)=R/(a)\otimes_R M\cong M/aM$.

	Finally $\Tor_1^R(R/(a),M)={_a}M$, the $a$-torison in $M$.
\end{ex}
\begin{rmk}
	Because of the fact that derived functors of right exact functors preserve colimits, we have actually effectively computed
	$\Tor_i^R(N,M)$ for all $N$.
\end{rmk}
\begin{ex}
	In the case when $R=Z$ and $N=\bbQ/\bbZ$, we can use the fact that $\bbQ/\bbZ$ is the colimit (directed limit)
	of its finite subgroups. Then you can pull this out and say that $\Tor_*^R(\bbQ/\bbZ,M)$ is the ``total torsion module'' of $M$,
	the directed limit of all finite torsion subgroups (is this last sentence correct?).
\end{ex}

\section{April 24, 2019}
We're going to start by doing some more computations.
\subsection{Computing $\Ext$}
Begin by letting $A=\bbZ$. Then $\Ext_\bbZ^{i>0}(\bbZ,B)=0$ since $\bbZ$ is free whence projective.
When $A=\bbZ/n$, we can compute $\Ext$ via the free resolution
\[0\to \bbZ\xrightarrow{\cdot n}\bbZ\to \bbZ/n\]
and then applying $\Hom_\bbZ(-,B)$, we get the complex 
\[0\to \Hom_\bbZ(\bbZ,B)\xrightarrow{\cdot n}\Hom_\bbZ(\bbZ,B)\to 0\]
so $\Ext_\bbZ^{i\ge 2}(\bbZ/n,B)=0$ and $\Ext_\bbZ^1(\bbZ/n,B)=B/nB$ and $\Ext_\bbZ^0(\bbZ/n,B)={_n}B/$.

\begin{thm}
	$\Ext_{\bbZ}^{i\ge 2}(A,B)=0.$
\end{thm}
\begin{prf}
	Take an injective resolution of $B$:
	\[B\to I^0\to I^1\cong\Omega^{-1}B\to 0\]
	and every resolution can end there! This is because the cokernel of $B\to I^0$ is the 
	quotient of a divisible group, whence divisible.
\end{prf}

\begin{rmk}
	Notice that $\operatorname{gldim}(\bbZ)=1$.
\end{rmk}
Notice that when $B=\bbZ$, we can consider the injective resolution
\[\bbZ\to \bbQ\to \bbQ/\bbZ\to 0\]
amd then compute from the complex
\[\Hom_\bbZ(A,\bbQ)\to\Hom_Z(A,\bbQ/\bbZ)\]
where the first is zero and the latter is $A^\vee$, the \textbf{Pontragin dual} of $A$. Here
we get that $\Ext_\bbZ^1(A,\bbZ)\cong \Hom_\bbZ(A,\bbQ/\bbZ)=A^\vee$.

Now let $A=Z/p^\infty=\lim_\to \bbZ/p^n$. Then 
\[\Ext_\bbZ^1(\lim_\to\bbZ/p^ny,\bbZ)=\lim_\leftarrow(\bbZ/p^n,\bbZ)=\lim_\leftarrow \bbZ/p^n=\widehat\bbZ_p\]
the $p$-adic numbers.

Recall:
\begin{lem}
	$P$ is projective iff $\Ext^1(P,-)$ is exact if and only if $\Ext^i(P,-)$ is exact for all $i\ge 1$.
\end{lem}
\begin{defn}
	$M$ is a flat module if $-\otimes_R M$ is exact.
\end{defn}
\begin{thm}
	THe following are equivalent
	\begin{itemize}
		\item $M$ is flat
		\item $-\otimes_RM$ is exact
		\item if $N'\hookrightarrow N$ then $N'\otimes_RM\hookrightarrow N\otimes_RM$.
		\item $\Tor_i^R(-,M)=0$ for $i\ge 1$
		\item $\Tor_1^R(-,M)=0$
	\end{itemize}
\end{thm}
\begin{rmk}
	Any projective module is flat.
\end{rmk}

\subsection{Local Properties of Modules}
Here let $R$ be a commutative ring, $\mathfrak p\lhd R$ is a prime, and $R_{\mathfrak p}$ is localization at $\mathfrak p$
and $(-)_{\mathfrak p}$ is a functor from $R$-mod to $R_{\mathfrak p}$-mod.
\begin{prop}
	$(-)_{\mathfrak p}$ is exact.
\end{prop}
\begin{defn}[Informal]
	We say that a property $P$ is local if $M$ satisfies $P$ if and only if $M_{\mathfrak p}$ satisfies $P$ for all $\mathfrak p\lhd R$.
\end{defn}
\begin{prop}
	$M\cong 0$ if and only if
	\begin{itemize}
		\item $M_{\mathfrak p}=0$ for all $\mathfrak p$
		\item $M_{\mathfrak m}=0$ for all maximals $\mathfrak m$
	\end{itemize}
\end{prop}
\begin{rmk}
	Other examples of local properties are flatness of a module and exactness of a short exact sequence.
\end{rmk}
\begin{prop}
	If $R$ is commutative, and if $S$ is flat as an $R$-algebra (flatness is just referring to the module structure), then
	\[S\otimes_R\Tor^R_i(A,B)\cong \Tor_i^S(S\otimes_R A,S\otimes_R B).\]
\end{prop}
\begin{rmk}
	The idea above is ``flat base changes commute with $\Tor$.''
\end{rmk}
\begin{cor}
	The following are equivalent:
	\begin{itemize}
		\item $\Tor_i^R(M,N)=0$
		\item For all $\mathfrak p$, $\Tor_i^{R_{\mathfrak p}}(M_{\mathfrak p},N_{\mathfrak p})=0$
		\item For all maximal $\mathfrak m$, $\Tor_i^{R_{\mathfrak m}}(M_{\mathfrak m},N_{\mathfrak m})=0$
	\end{itemize}
\end{cor}

\section{April 26, 2019}
\subsection{A bit more about flat modules}
\begin{prop}
	Let $R$ be a local ring and let $M$ be a finitely generated $R$-module. Then the following are equivalent:
	\begin{itemize}
		\item $M$ is flat.
		\item $M$ is free.
		\item $M$ is projective.
	\end{itemize}
\end{prop}
\begin{rmk}
	In other words, every flat module is locally free.
\end{rmk}
\begin{rmk}
	In general we have the inclusions: free, projective, flat, torsion free.
\end{rmk}

\subsection{$\Ext$ as Extensions}
Recall the five lemma 
\begin{lem}[Weak 5-lemma]
	If we have the commutative diagram with exact rows:
	\begin{center}
		\begin{tikzcd}
			0\ar[r] & A\ar[r]\ar[d,"\sim"] & B\ar[r]\ar[d,"f"] & C\ar[r]\ar[d,"\sim"] & 0\\
			0 \ar[r] & A'\ar[r] & B'\ar[r] & C'\ar[r] & 0
		\end{tikzcd}
		then $f$ is an isomorphism.
	\end{center}
\end{lem}

\begin{defn}
	Define $\sExt_R^1(M,N)$ to be the set of short exact sequences $0\to N\ to E\to M\to 0$
	modulo the relation that if there is a map (necessarily an ismomorphism) $\varphi:E\to E'$ extending 
	the identity maps on $M$ and $N$, then the two sequences are equivalent.

	We call $0\to N\to E\to M\to 0$ a \textbf{extension of $M$ by $N$ of length 1.}
\end{defn}
\begin{prop}
	$\sExt_R^1(N,M)$ has the structure of an abelian group.
\end{prop}
\begin{rmk}
	Let $\alpha:M'\to M$ and let $E\times^MM'$ be the pullback over
	\[E'\xrightarrow{g} M\xleftarrow{\alpha}M'\]
	where where $0\to N\to E\to M\to 0$ is an extension.

	Then setting $E'=E\times^MM'$, we can pull back along $\alpha$ to a short exact sequence 
	\[0\to N\to E'\to M'\to 0.\]
	Here we usually refer to this map (of short exact sequences) as $\alpha^*:\sExt^1(M,N)\to \sExt^1(M',N)$.

	Dually, we can use a $\beta:N\to N'$ and the pushforward $E''=N'\times_NE$ under
	\[N'\xleftarrow{\beta} N\xrightarrow{f}E\]
	and this gives us a map to the sequence $0\to N'\to E''\to M\to 0$. This gives us 
	the map $\beta_*:\sExt^1(M,N)\to \sExt^(M,N')$.
\end{rmk}

\begin{rmk}
	A natural thing to want to do is sum SES's by using direct sums of the elements in the sequence.
	Call this $\xi\oplus \xi'$.
	That is great, but it doesn't quite work. To do so, you use the (co)diagonal maps $\Delta:M:M\oplus M$
	sending $m\mapsto(m,m)$ and $\nabla:N\oplus N\to N$ via $(n,n')\mapsto n+n'$.

	Then we can define a sum of extensions:
	\[\xi+\xi':= \nabla_*\Delta^*(\xi\oplus\xi')=\Delta^*\nabla_*(\xi\oplus\xi')\]
	The last equality isn't completely trivial.
\end{rmk}
From here we talked about how one proves that this is indeed a group. The zero element is the split exact sequence.
\begin{prop}
	There exists a natural group isomorphism
	\[\sExt^1_R(M,N)\cong\Ext_R^1(M,N).\]
\end{prop}
\begin{rmk}
	The proof isn't too hard. Check Weibel.
\end{rmk}

\brk

In general, for $i\ge 1$, we use an analogous definition but now we define two extensions
to be equivalent if there exist a collection of ($i$) maps $f_k:E_k\to E_k'$ making the diagram commute.

The general group law via pushbacks and -forwards like before. Then we define the addition the same way.
\begin{prop}
	There exists a natural isomorphism $\sExt^i_R(M,N)\to \Ext^i_R(M,N)$.
\end{prop}
\begin{prf}
	(Sketch) Take a projective resolution $P_\bullet\to M\to 0$ and for any extension 
	\[0\to N\to E_1\to \cdots\to E_i\to M\to 0\]
	lift the identity map on $M$ to a map of resolutions (not both projective) of $M$.

	Say that $f:P_i\to N$ is the map you get in this chain map. We claim that $f$ is a cycle
	in $\Hom(P_\bullet,N)$. If $d_i^*$ is the differential on this complex, then since $d_i^*(f)=f\circ d_i$,
	and using the extension since the map $f\circ d_i:P_{i+1}\to P_i\to N$ is the same as $0:P_{i+1}\to 0\to N$,
	we have it.

	But then $\bar f$ is in $\Ext_R^i(M,N)$.
\end{prf}

\subsection{Yoneda Product}
There is a relatively natural pairing
\[\Ext_R^i(L,M)\times\Ext_R^j(M,N)\to \Ext_R^{i+j}(L,N)\]
where we just concatenate the two exact sequences (eliminating $M$ along the way). It is easy to 
check that it is still exact.

\section{April 29, 2019}
Today we will finish up the last bit on $\sExt$. Let $M=N$ and consider $\Ext_R^\ast(M,M)$.
Then the Yoneda product gives the the structure of a graded ring.

Usually when you are working in another context (e.g. group cohomology) where you have another 
product (e.g. the cup product) this ends up being the same as the Yoneda product. There is something 
to prove here.

\begin{ex}
	Recall $\Ext_\bbZ^1(\bbZ/p,\bbZ/p)\cong\bbZ/p$. Then consider the SES
	\[0\to \bbZ/p\hookrightarrow\bbZ/p^2\to \bbZ/p\]
	where the embedding on the first map takes (written multiplicatively) $a\mapsto ap$.
	Then there are $p-1$ different choices we have for the surjection where $a\mapsto ia$ for $1\le a\le p-1$.
	
	The one remaining expension comes from the split exact sequence.
\end{ex}
\begin{rmk}
	These are the same embedings we say in $\bbZ/p^\infty$!
\end{rmk}

\subsection{The K\"unneth formula and the Universal Coefficient Theorem}
The motivation here is that, given a space $X$, it is most natural to consider $H_\ast(X,\bbZ)$, 
but if there is some kind of an action of $M$ on these cycles, one may want to compute $H_\ast(X,M)$.

\begin{defn}
	Let $P_\bullet$ and $Q_\bullet$ be complexes. Then $P_\bullet\otimes Q_\bullet$ be defined by 
	\[(P_\bullet\otimes_RQ_\bullet)_n=\bigoplus_{i+j=n}P_i\otimes_R Q_j\]
	and the differential 
	\[d_n=\sum_{i+j=n}d_{ij}:(P\otimes Q)_n\to (P\otimes Q)_{n-1}\]
	is such that if $p$ is in degree $i$ and $q$ is in degree $j$, then 
	\[d_n(p\otimes q)=d_i(p)\otimes q+(-1)^ip\otimes d_j(q).\]
\end{defn}

A natural question to ask: what is the homology of this complex?
\begin{rmk}
	We will be using the fact that, to compute $\Tor$, it suffices to take flat (rather than projective)
	resolutions. 

	There is a more general principle here that one can use modules that are ``acyclic'' (higher derived functors are zero)
	for a functor and these resolutions give us homology.
\end{rmk}

\begin{thm}
	Let $P$ and $Q$ be complexes of (right- and left-) $R$ modules such that the $P_n$ and 
	\[B_n^P=d_{n+1}(P_{n+1})\]
	are flat. 

	Then there exists a short exact sequence
	\[0\to \bigoplus_{i+j=n}H_i(P)\otimes H_j(Q)\to H_n(P\otimes Q)\to \bigoplus_{i+j=n}\Tor_1^R(H_i(P),H_j(Q))\to 0\]
\end{thm}
\begin{prf}
	We will not use boundaries and cycles for $Q$, so throughout $Z$ and $B$ refer to those for $P$. 
	Now since we have the short exact sequence
	\[-\to Z_i\to P-i\to B_{i-1}to 0\]
	and the long exact sequence of $\Tor$ gives us that $Z_i$ are flat as well. Furthermore (again by the long exact sequence)
	we get 
	\[0\to Z_i\otimes Q_j\to P_i\otimes Q_j\to B_{i-1}\otimes Q_j\to 0\]
	is short exact.

	Now consider
	\[0\to\bigoplus_{i+j=n}Z_i\otimes Q_j\to \bigoplus_{i+j=n} P_i\otimes Q_j\to\bigoplus_{i+j=n-1}B_i\otimes Q_j\to 0\]
	which gives us a SES of complexes
	\[-\to Z_\bullet\otimes Q_\bullet \to P_\bullet\otimes Q_\bullet \to B_\bullet\otimes Q_\bullet[-1]\to 0\]
	and by taking the long exact sequence in homology (note the index shift because of the shift on $Q$):
	\[\cdots\to H_n(B\otimes Q)\to H_n(Z\otimes Q)\to H_n(P\otimes Q)\to H_{n-1}(B\otimes Q)\to H_{n-1}(Z\otimes Q)\to\cdots\]
	Let $\partial:H_n(B\otimes Q)\to H_n(Z\otimes Q)$ be the map above. Then we get the SES
	\[0\to \coker\partial\to H_n(P\otimes Q)\to \ker \delta\to 0\]
	where $\coker d_{ij}$ is $H_i(P)\otimes H_j(Q)$ (this needs some work -- see the book!)

	Finally $ker\partial$ can be computed via using the (flat!) resolution $0\to B_i\to Z_i\to H_i(P)$ and computing $\Tor$.
\end{prf}
\begin{rmk}
	Notice that it is \textit{not true} that if $A$ and $B$ are flat that $A/B$ are flat! 
\end{rmk}

\section{May 1, 2019}
\begin{cor}[Universal Coefficient Theorem -- Corollary of K\"unneth Formula]
	Let $P_\bullet$ be a chain complex of flat modules and that $d_n(P)$ are flat, and that $M$ is an $R$ module.
	Then we have a short exact sequence
	\[0\to H_n(P_\bullet)\otimes M\to H_n(P_\bullet\otimes M)\to \Tor_1^R(H_{n-1}(P_\bullet),M)\to 0\]
	and in particular if $M$ is flat, then $H_n(P_\bullet)\otimes M\cong H_n(P_\bullet\otimes M)$ 
\end{cor}
\begin{prop}[UCT for $\bbZ$]
	Let $P_\bullet$ be a complex of free abelian groups. Then there exists a (non-canonical) isomorphism
	\[H_n(P\otimes M)\cong H_n(P_\bullet)\otimes M\bigoplus \Tor_1^R(H_{n-1}(P_\bullet),M)\]
	where the statement that this isomorphism is non-canonical amounts to saying that this splitting
	in the sequence from the UCT doesn't behave well with respect to maps.
\end{prop}
\begin{rmk}
	What does this come from? Well in topology, if $X$ is a topological space, then ``homology with coefficients''
	is just $H_n(X,M)=H_n(S(X)\otimes M)$. Then the universal coefficient theorem gives us a way to compute this.
\end{rmk}


\begin{prop}[UCT for $H^n$]
	Let $P_\bullet$ be a chain complex of projective $R$-modules where the $d_n(P_\bullet)$ are projective and
	$M$ is any $R$-module. Then we can apply $\Hom(-,M)$ to $P_\bullet$ to compute cohomology, yielding the short exact sequence
	\[0\to \Ext_R^1(H_{n-1}(P_\bullet),M)\to H^n(\Hom(P_\bullet,M))\to \Hom(H_n(P_\bullet,M))\to 0\]
\end{prop}
\begin{prop}[K\"unneth for $H^n$]
	Let $P_\bullet$ be a chain complex of projectives with $d_n(P_\bullet)$ projective. Let $Q_\bullet$ be a cochain complex. 
	Then we have the short exact sequence
	\[0\to\prod_{i+j=n-1}\Ext_R^1(H_i(P),H^j(Q))\to H^n(\Hom(P,Q))\to \prod_{i+j=n}\Hom(H_i(P),H^j(Q))\]
\end{prop}

\subsection{A reminder on dimension theory}
\begin{defn}
	Let $M$ be a (left) $R$-module. Then
	\[\projdim_RM:=\inf_n\{0\to P_n\to\cdots\to P_0\to M\}\]
	Notice that $\projdim$ can be infinite.
\end{defn}
\begin{rmk}
	A module has zero projective dimension if and only if it is projective.
\end{rmk}
\begin{rmk}
	One can define (analogously) injective and flat dimension.
\end{rmk}
\begin{lem}
	Let $M$ be an $R$-module. The following are equivalent:
	\begin{enumerate}
		\item $\projdim_RM\le d$
		\item $\Ext_R^i(M,N)=0$ for all $i\ge d+1$, for all $N$.
		\item $\Ext_R^{d+1}(M,N)=0$ for all $N$.
		\item If $0\to\Omega^dM\to P_{d-1}\to P_{d-2}\to\cdots\to P_0\to M$ is a resolution, then $\Omega^d(M)$ is projective.
	\end{enumerate}
\end{lem}
\begin{rmk}
	Everything can be dualized (e.g. use injective resolutions and injective dimension and $\Omega^{-d}$) 
	and get a similar set of equivalences.
\end{rmk}
\begin{rmk}
	Notice here that the syzygies are not uniquely defined, but they are uniquely defined \textit{up to projective summands.}
	In fancier parlance, they are uniquely defined in the stable module category.
\end{rmk}
\begin{rmk}
	Notice that $\Omega(\Omega^{d-1}(M))=\Omega^d(M)$. Using this, we can compute
	\[\Ext_R^{d+1}(M,N)\cong \Ext_R^1(\Omega^dM,N)\cong\underline{\Hom}_R(\Omega^{d+1},N)\]
	where $\underline{\Hom}$ is $\Hom$ is the stable module category.
\end{rmk}
\begin{rmk}
	Sometimes the \textbf{syzygy} is called \textbf{Heller shift} in representation theory or just \textbf{shift}
	when using a categorical viewpoint.
\end{rmk}

\begin{thm}
	The following numbers are the same:
	\begin{enumerate}
		\item $\sup_{M\in\Rmod}\projdim_R M$
		\item $\sup_{M\in\Rmod}\injdim_R M$
		\item $\sup_{I\text{ a left ideal}}\projdim_RR/I$
		\item $\sup_{M,N\in\Rmod}\{d|\Ext_R^d(M,N)\ne 0\}$
	\end{enumerate}
\end{thm}
\begin{defn}
	The above number is called the \textbf{left global dimension of $R$.}
\end{defn}
\begin{rmk}
	This is where commutative algebra diverges: we have notions of \textit{right}
 global dimension (just change to $\mathbf{mod}$-$R$) as well as global dimension proper.
\end{rmk}

We also have a notion of \textit{weak global dimension} or $\Tor$ global dimension:
\begin{defn}
	The $\Tor$ dimension of $R$ is any of the following numbers:
	\begin{enumerate}
		\item $\sup_{M\in\Rmod}\operatorname{flatdim}_RM$
		\item $\sup_{M\in\modR}\operatorname{flatdim}_RM$
		\item $\sup_{I\text{ left ideal}}\operatorname{flatdim}_R R/I$
		\item $\sup_{J\text{ right ideal}}\operatorname{flatdim}_R R/J$
		\item $\sup_{M\in\modR,N\in\Rmod}\{d|\Tor_d^R(M,N)\ne 0\}$
	\end{enumerate}
\end{defn}
\begin{rmk}
	If $R$ is (left) Noetherian, then $\operatorname{flatdim}_R(M)=\projdim_R(M)$ for all finitely
	generated left $R$-modules. Furthermore the \textbf{left} $\gldim R=\operatorname{Tordim}R$.
\end{rmk}

\begin{rmk}
	If $R$ is (right) Noetherian, you get all the same stuff but flipped.
\end{rmk}
\begin{rmk}
	Rings of global dimension zero are exactly the semisimple ones!
\end{rmk}

\section{May 3, 2019}
Recall that semisimple rings are precisely those of global dimension zero. 
\subsection{Frobenius Algebras}
\begin{defn}
	Let $R$ be a finite dimensional algebra over $k$. Then $R$ is \textbf{Frobenius} if there exists a non-degenerate,
	associative, bilinear form $\sigma:R\times R\to k$.
\end{defn}
\begin{defn}
	If $R$ is Frobenius and furthermore $\sigma$ is a symmetric form, then $R$ is \textbf{symmetric}.
\end{defn}
\begin{prop}
	If $R$ is Frobenius, then
	\[R^\ast=\Hom_k(R,k)\cong R\]
	as (left) $R$-modules.
\end{prop}
\begin{prf}
	Consider the map $R\to R^\ast$  where we send $a\mapsto \varphi_a=\sigma(-,a)$. One can check that this
	actually gives you an isomorphism.
\end{prf}
\begin{cor}
	$R$ is injective as an $R$-module.
\end{cor}
\begin{rmk}
	The idea here is that $R^*$ is projective, and dualizing takes projectives to injectives.
\end{rmk}
\begin{cor}
	When $R$ is Frobenius, all projectives are injective (and vice versa).
\end{cor}
\begin{rmk}
	$R$ is Frobenius implies that $R$ is ``self-injective'', or that $\injdim_RR=0$. Note that this 
	then implies that $R$ is \textit{Gorenstein!}

	It often happens in representation theory that one proves something for finite dimensional algebras, 
	then Frobenius algebras, then sees that it generalizes to Gorenstein ones.
\end{rmk}
\begin{defn}
	$R$ is called \textbf{quasi-Frobenius} (according to Wikipedia) if the collection of injectives and 
	projectives are the same.
\end{defn}
\begin{ex}
	(This is homework) If $G$ is a finite group then $kG$ is symmetric. Note that it is uninteresting when 
	we have Artin-Wedderburn, so the fun stuff happens in the modular case.
\end{ex}
\begin{ex}
	Consider the Lie algebra $\frakg$ over $k$, a field of characteristic $p$. Then $[p]:\frakg\to \frakg$
	is a $p^{th}$ power map or \textbf{restriction operation} if
	\begin{enumerate}
		\item $\ad x^{[p]}=(\ad(x))^p$
		\item $(\lambda x)^{[p]}=\lambda^{[p]}x^{[p]}$
		\item $(x+y)^{[p]}=x^{[p]}+y^{[p]}+\sum_1^{p-1}s_i(x,y)$
	\end{enumerate}
	where $is_i(x,y)$ is the coefficient of $\lambda^{i-1}$ in 
	\[(\ad(\lambda x+y))^{p-1}(x)=[\lambda x+y,[\lambda x+y,[\cdots,[\lambda x + y,x]\cdots,x],x],x]\]
\end{ex}
\begin{rmk}
	If $\frakg$ is linear (sits in $\gl_n$), then we have the $p^{th}$ power map that takes every matrix 
	to its $p^{th}$ power. This is what we are trying to capture without having to rely on what is going to be 
	an inherently coordinate-dependent definition.
\end{rmk}

\begin{defn}
	The universal enveloping algebra $\mathfrak{U}(\frakg)$ is the universal object representing the smallest
	associative algebra capturing the relations of $\frakg$. Look to my Lie algebra notes for more details.
\end{defn}
\begin{defn}
	The \textbf{restricted enveloping algebra} $\mathfrak{u}(\frakg)$ is 
	\[\mathfrak{u}(\frakg)=\mathfrak{U}(\frakg)/\langle x^p-x^{[p]}\rangle\]
\end{defn}
\begin{prop}
	$\mathfrak{u}(\frakg)$ has a ``PBW'' style basis. The monomials take a basis for $\frakg$ and 
	then the monomials look like 
	\[x_1^{i_1}\cdots x_n^{i_n}\]
	for $0\le i_1,\dots\le p-1$.
\end{prop}
In general, the category of representations of $\frakg$ is equivalent to $\mathfrak{U}(\frakg)$-$\mathbf{mod}$
and the restricted representations of $\frakg$ are equivalent to $\mathfrak{u}(\frakg)$ modules.
\begin{prop}
	$\mathfrak{u}(\frakg)$ is Frobenius.
\end{prop}
\subsection{Hereditary Rings}
\begin{defn}
	A ring $R$ is \textbf{(left) hereditary} of every submodule of a free module is projective.
\end{defn}
\begin{rmk}
	Equivalently, the (left) $\gldim R\le 1$.
\end{rmk}
\begin{ex}
	PIDs are hereditary, of course. Furthermore, a commutative integral domain $R$ is hereditary if and only if $R$
	is a Dedekind domain.
\end{ex}

\begin{prop}
	If $f:R\to S$ is a ring homomorphism and $M$ is an $S$-module, then 
	\[\projdim_R(M)\le\projdim_R(S)+\projdim_S(M).\]
\end{prop}
\begin{prf}
	The proof of this statement (which we may have seen with S\'andor?) uses Cartan-Eilenberg resolutions.
	Let $\cal A$ be an abelian category with enough projectives and let $C$ be a chain complex in $\cal A$.
	Then $P=P_{\bullet,\bullet}$ (a double complex) is a \textbf{Cartan-Eilenberg Resolution of $C$} if we have 
	a commutative diagram
	\begin{center}
		\begin{tikzcd}
			& \vdots\ar[d] & \vdots\ar[d] & \\
			& P_{n+1,1}\ar[d] & P_{n,1}\ar[d] & \\
			& P_{n+1,0}\ar[d] & P_{n,0}\ar[d] & \\
			\cdots\ar[r] & C_{n+1}\ar[r] & C_n\ar[r] & \cdots
		\end{tikzcd}
	\end{center}
	where we have the properties: 
	\begin{itemize}
		\item $P_{n\bullet}\to C_n$ is a projective resolution.
		\item $P_{nj}=0$ for $j<0$
		\item $B(P_{n\bullet})\to B(C_\bullet)$ and $H(P_n)\to H_n(C_\bullet)$ are projective resolutions.
	\end{itemize}
\end{prf}

\section{May 8, 2019}
\begin{lem}
	Cartan-Eilenberg resolutions exist for any complex $C$.
\end{lem}
\begin{prop}
	Let $f:R\to S$ be ar ring homomorphism and let $M$ be an $S$-module and let $f^\ast(M)$ be the $R$-module
	taken by acting on $M$ via $f$. Then 
	\[\projdim_S(f^\ast(M))=\projdim_S(M)+\projdim_R(S)\]
\end{prop}
\begin{prf}
	\textit{Sketch.} Let $n=\projdim_S(M)$ and let $P_\bullet\to M$ be a projective resolution of $M$ (as an $S$-module)
	of length $n$. Now we use the fact (exercise) that $\projdim_R(P_i)\le \projdim_R(S)$. 

	Let $Q_{\bullet\bullet}\to P_\bullet$ be a Cartan-Eilenberg resolution of $P_\bullet$ as $R$-modules. Now we can ``chop off''
	higher terms of the CE complex, replacing the $m^{th}$ row with $\Omega^m(Q_{n\bullet})$ and the higher rows with zeros. One can 
	show without too much work that this is still a CE resolution of $P_\bullet$.

	Now we have already that $P_\bullet\sim M$ is a quasi isomorphism and the total complex $\operatorname{Tot}(Q_{\bullet\bullet})$ 
	is quasi isomorphic to $P_\bullet$, so we have, in effect, created a projective resolution of $M$ using $R$-modules of length $n+m$.
\end{prf}
\subsection{Koszul Complex/Resolution}
As motivation, let $R$ be a regular local ring of dimension $n$. Then the Koszul resolution is a resolution of the 
residue field $k=R/\mathfrak{m}$ of length $n$.

\begin{ex}
	Let $x$ be a central element in $R$. Consider the map $R\xrightarrow{x}R$. This is a complex! We can compute 
	$H_1={_x}R$ and $H_0=R/xR$. In particular, if $x$ is not a zero divisor, then $K_\bullet(x)=R\to R$ is a 
	projective resolution of $R/xR$.

	This gives us the Koszul complex on a single element.
\end{ex}

\begin{defn}[Koszul Complex]
	Let $\underline x=(x_1,\dots,x_n)$ be a sequence (not yet regular) of central elements. Then 
	\[K_\bullet(\underline{x})=K_\bullet(x_1)\otimes\cdots\otimes K_\bullet(x_n)\]
	is the Koszul complex of $\underline{x}$.
\end{defn}
\begin{ex}
	Let's consider what happens when $n=2$. Then the length of this complex is going to be 2. This comes 
	from the diagonals of the double complex:
	\begin{center}
		\begin{tikzcd}
			R\ar[r,"x_1"]\ar[d,"-x_2",swap] & R\ar[d,"x_2"]\\
			R\ar[r,"x_1"] & R
		\end{tikzcd}
	\end{center}
	giving us the complex 
\[R\xrightarrow{\begin{pmatrix}-x_2 & x_1\end{pmatrix}} R\oplus R\xrightarrow{\begin{pmatrix}x_1\\x_2\end{pmatrix}} R\]
\end{ex}

Now in general if $K_p(\underline{x})$ is the $p^{th}$ component of the Koszul complex, we get $\dim_R(K_p(\underline{x})))=\binom{n}{p}$.
We denote the generators for $K_p(\underline{x})$ as $e_{i_1}\wedge\cdots\wedge e_{i_p}$ for $1\le i_1\cdots i_p\le n$.

Then we claim that the differential on this complex is 
\[d_p(e_{i_1}\wedge \cdots\wedge e_{i_p})=\sum (-1)^{k+1}x_k e_{i_1}\cdots\wedge \widehat{e_{i_k}}\wedge\cdots\wedge e_{i_p}\]

\begin{prop}
	$K_\ast(\underline{x})$ is a differential graded algebra with product formed by concatenation modulo
	the relation $e_i\wedge e_j=-e_j\wedge e_i$ and $e_i\wedge e_j=0$. If $V=Rx_1\oplus \cdots\oplus Rx_n$, then
	$K_\ast(\underline{x})\cong \Lambda^\ast(V)$ and $K_p(\underline{x})\cong\Lambda^p(V)\cong\Lambda^p(R^n)$.
\end{prop}
\begin{rmk}
	Note that in the proposition above, $K_\ast(\underline{x})\cong \Lambda^\ast(V)$ is meant to imply isomorphism \textit{on the level of graded algebras.}
\end{rmk}
\begin{defn}
	If $M$ is any (left) $M$ module then $K_\bullet(\underline{x})\otimes M$ is a Koszul complex of $M$.
\end{defn}
\begin{rmk}
	Then, as a matter of notation, $H_i(\underline{x},M)=H_i(K_\bullet(\underline{x}\otimes M))$.
\end{rmk}

\section{May 10. 2019}
\subsection{Koszul Complexes Continued}
Let $R$ be a commutative local ring $(R,\frakm,k)$. 
\begin{thm}
	Let $(x_1,\dots,x_n)\in \frakm$ is an $M$-regular sequence for $M$, then
	\begin{itemize}
		\item $H_0(\underline{x},M)=M/((x_1,\dots,x_n)M)$.
		\item $H_i(\underline{x},M)=0$ when $i>0$.
	\end{itemize}
\end{thm}
\begin{rmk}
	Equivalently, $K_\bullet(\underline{x})\otimes M$ is a projective resolution of $M/(x_1,\dots,x_n)M$.
\end{rmk}
In particular, when $M=R$, and when $\dim R=n$, then the (length $n$) sequence is just a regular sequence. Then $K_\bullet(\underline{x})\to k$ is a free resolution of $k=R/\frakm$.
\begin{cor}
	If $R$ is a regular ring, $\projdim_Rk=n=\dim R$
\end{cor}
\begin{lem}
	Let $x$ be a non-zero-divisor in the center of $R$ and let $C\in\Ch(\Rmod)$. Then there 
	exists a natural short exact sequence
	\[0\to H_i(C)/xH_i(C)\to H_i(K_\bullet(x)\otimes C)\to {_x}H_i(C)\to 0\]
\end{lem}
\begin{prf}
	K\"unneth formula!
\end{prf}
\begin{thm}
	Let $R$ be a commutative local ring, $M$ an $R$-module, and a sequence $(x_1,\dots,x_n)\in\frakm$. Then 
	\begin{itemize}
		\item $(x_1,\dots,x_n)$ is $M$-regular.
		\item $H_i(\underline{x},M)=0$ for $i>0$.
		\item $H_1(\underline{x},M)=0$
	\end{itemize}
\end{thm}
\begin{cor}
	If $(x_1,\dots,x_n)$ is a regular sequence in a local ring, then any permutation of it is also regular!
\end{cor}
\begin{rmk}
	Definitely not true when $R$ is non-local. We did an example with S\'andor.
\end{rmk}

\subsection{Group (co)homology}
Let $G$ be a group. Then we can consider $\bbZ G$ modules and compute 
\[\Ext_G^i(M,N)=R^i\Hom_G(M,N).\]

Recall that we can consider representations of $G$ over $R$ (a commutative ring). Then if $M$ and $N$ 
are two representations of $G$ ($RG$-modules), then $M\otimes_R N$ can be defined to have the diagonal action of $G$.
Similarly $\Hom_R(M,N)$ becomes a representation via $(g\cdot f)(m)=g\cdot(f(g^{-1}m))$.
\begin{rmk}
	Notice here that $\Hom_R(-,-)$ is \textit{internal hom} since it becomes again a $G$-module.
\end{rmk}
\begin{prop}
	$\Hom_R(M,N)^G\cong \Hom_{RG}(M,N)$ where the later is the external (categorical) $\Hom$.
\end{prop}

The upshot of all of this is that we can compute the group cohomology:
\[H^i(G,M):=\Ext_{RG}(R,M)=R^i\Hom_{RG}(R,M)=R^i\Hom_R(Z,M)^G=R^iM^G\]
so we can define the group cohomology functor 
\[H^i(G,-)=R^i(-)^G.\]

Now $H_i(G,M)=\Tor_i^{RG}(R,M)$. But $R\otimes_{RG}M$ is going to be isomorphic to $M_G$, the \textbf{coinvariance} of $M$,
which is $M/\langle m-gm|m\in M,g\in G\rangle$. So then $H_i(G,-)=L_i(-)_G$.

\begin{rmk}
	How do we compute this in general? This is not actually known. For instance, consider $GL_n(\bbF_p)$ is a relatively simple group (finite!)
	and we don't know the minimal $i>0$ such that $H^i(GL_n(\bbF_p),\bbF_p)\ne 0$.
\end{rmk}

Na\"ively, we can use that $H^i(G,M)=\Ext_G^i(\bbZ,M)$ (here setting $R=\bbZ$) and then compute this via a resolution 
either of $P_\bullet\to\bbZ$ and compute $H^i(\Hom(P_\bullet,M))$ or $H^i((I^\bullet)^G)$.

\section{May 13, 2019}
\subsection{Sample Computations}
\begin{ex}
	Let $G=\bbZ/p=C_p$ (not necessarily a prime yet). For instance, let $p=0$ first and $G=\bbZ$.
	Then to compute $H^i(C_p,A)=\Ext_{C_p}^i(\bbZ,A)$ is computed either with a resolution of $\bbZ$ or $A$. Since the 
	latter is allowed to be general, we need to do the former. Consider 
	\[\cdots\to\bbZ C_p\xrightarrow{N}\bbZ C_p\xrightarrow{\sigma-1}\bbZ C_p\xrightarrow{\varepsilon}\bbZ\to 0\]
	where $N=(\sigma^p-1)/(\sigma-1)$.

	Let's begin by noticing that $\Hom_{\bbZ C_p}(\bbZ C_p,A)\cong A$, so we get a resolution 
	\[\cdots\leftarrow A\xleftarrow{\sigma-1}A\xleftarrow{N}A\xleftarrow{\sigma-1}A\]
	Doing our computations, we get $H^0(C_p,A)=\ker\{\sigma-1:A\to A\}=A^{C_p}$. As we expected!

	Then $H^1(C_p,A)=\ker(N)/(\sigma-1)A$ and $H^2(C_p,A)=A^{C_p}/NA$, and this just cycles.
\end{ex}
\begin{rmk}
	The above is an example of a \textbf{periodic resolution of $\bbZ$}. We will not talk much about this, but there is a measure of 
	\textit{complexity of a module} which refers to the length of the period in such a resolution.
\end{rmk}

\begin{ex}
	In a specific case, consider $A=\bbZ$ (the trivial $\bbZ C_p$ module). Then $H^0$ is $\bbZ$, $H^1=0$ and $H^2=\bbZ/p$.
\end{ex}
\begin{ex}
	Now let's let $k$ be a field of characteristic $p$ (now $p$ is a prime, obviously) and let $A$ be a $k$-module. 
	Then the same thing goes through in computing $H^i(C_p,A)=\Ext_{kC_p}(k,A)$, but we just replace $\bbZ$ with $k$.

	If instead we compute $H^i(C_p,k)$, we get $k$ every time (since $\sigma-1$ is the zero map as well as $1+\sigma+\cdots+\sigma^{p-1}$).
\end{ex}
\begin{ex}
	Let $\langle t\rangle=C$ be a cyclic group and consider the group algebra $\bbZ C\cong \bbZ[t,t^{-1}]$.
	But then a resolution of $\bbZ$ is 
	\[0\to \bbZ\langle t\rangle\xrightarrow{t-1}\bbZ\langle t\rangle\xrightarrow{\varepsilon}\bbZ\]
	And so you get a result about Poincar\'e duality that I missed but don't have time to figure out.
\end{ex}
\begin{ex}
	Now if $A=\bbZ\langle t\rangle$ and $C=\langle t\rangle$. THen $H^0(C,\bbZ C)=0$ and $H^1(C,\bbZ C)=\bbZ$.
	Replacing $\bbZ$ with $k$, we get $H^0(C,\bbZ C)=0$ and $H^1(C,\bbZ C)=k$.
\end{ex}
\begin{rmk}
	This gives us one of the examples of the differences between finite and infinite groups: if $C$ were finite,
	$kC$ would be Frobenius, whence self-injective, whence $H^1(C,kC)=0$.
\end{rmk}

\subsection{Products in Cohomology}
We will begin by discussing the idea of a ``coss-product.'' Let $G$ and $H$ be groups. Then we want to define 
a map 
\[H^p(G,\bbZ)\otimes H^q(H,\bbZ)\xrightarrow{\times} H^{p+q}(G\times H,\bbZ).\]
Eventually we can let $G=H$ and using the coproduct (diagonal) map $\Delta(g)=(g,g)$, whose image under $H^{p+q}$ is 
$\Delta^\ast:H^{p+q}(G\times G,\bbZ)=H^{p+q}(G,\bbZ)$ to get our cup product.

\begin{prop}
	Let $P_\bullet\to\bbZ$ be a projective resolution as $\bbZ G$ modules and $Q_\bullet\to\bbZ$ a resolution as $\bbZ H$ modules.
	Then $P_\bullet\otimes Q_\bullet\to\bbZ$ is a projective resolution as $\bbZ G\otimes \bbZ H=\bbZ(G\times H)$ modules.
\end{prop}

Let's begin by defining a map
\[f:\Hom_G(P_\bullet,\bbZ)\otimes \Hom_H(Q,\bbZ)\to \Hom_{G\times H}(P\otimes Q,\bbZ)\]
via
\[f(\mu\otimes \nu)(x\otimes y)=\mu(x)\mu(y)\]
for all $x\in P$ and $y\in Q$. Notice here that if $|\mu|\ne |x|$, then we set $\mu(x)=0$.

Now using that $H^i(G,\bbZ)=H^i(\Hom_G(P_\bullet,\bbZ)),$ and using the first map from the K\"unneth formula,
\[H^i(\Hom_G(P_\bullet))\otimes H^j(\Hom_H(Q_\bullet,\bbZ))\to H^{i+j}(\Hom_G(P_\bullet,\bbZ)\otimes \Hom_H(Q_\bullet,\bbZ))\]
and then it can be checked that this map behaves well with respect to cycles and boundaries, so we 
get a map 
\[H^{i+j}(\Hom_G(P_\bullet,\bbZ)\otimes \Hom_H(Q_\bullet,\bbZ))\to H^{i+j}(\Hom_{G\times H}(P\otimes Q,\bbZ))\cong H^{i+j}(G\times H,\bbZ).\]

\section{May 17, 2019}
Recall that today we will have extra class at 5pm today. Cody is going to give his talk!

\subsection{The Bar Resolution}
We begin with a non-standard presentation of the Bar resolution. Consider 
\[\cdots\to (\bbZ G)^2\to \bbZ G\to \bbZ\]
where the differential takes $(g_0,\dots,g_n)\in(\bbZ G)^{n+1}$ to $\sum_i(-1)^i(g_0,\cdots,\widehat{g_i},\cdots,g_n)$.
\begin{prop}
	This is a free resolution of $\bbZ$ as a $\bbZ G$ module.
\end{prop}
\begin{prf}
	It suffices to show that this is a resolution (i.e. that it is contractible). Thus we want to 
	construct the homotopy:
	\[s(g_0,\cdots,g_n)=(e,g_0,\cdots,g_n)\]
	which amounts to taking an $n$-simplex and embedding it into an $(n+1)$-simplex as a face.

	Then we can compute 
	\begin{align*}
		ds+sd(g_0,\cdots,g_n)&=d(e,g_1,\cdots,g_n)+s\left(\sum_i(-1)^i(g_0,\cdots,\widehat{g_i},\cdots,g_n)\right)\\
		&= (g_0,\cdots,g_n)-\sum_{i}(-1)^i(e,g_1,\cdots,\widehat{g_i},\cdots,g_n)+s\left(\sum_i(-1)^i(g_0,\cdots,\widehat{g_i},\cdots,g_n)\right)\\
		&= (g_0,\cdots,g_n)
	\end{align*}
	proving the statement.
\end{prf}
\begin{rmk}
	Notice that $(\bbZ G)^{n+1}$ is a free $\bbZ G$ module of rank $|G|^n$ with basis 
	\[[g_1|\cdots|g_n]=(e,g_1,g_1g_2,\dots,g_1\cdots g_n).\]
\end{rmk}
\begin{rmk}
	We can define the differential on the basis elements:
	\[d:=\sum_0^{n}(-1)^id_i:(\bbZ G)^{n+1}\to(\bbZ G)^n\]
	where $d_i:\bbZ G^{n+1}\to\bbZ G^n$ is given by
	\[d_i([g_1|\dots|g_n])=\left\{\begin{array}{lr}
		g_1[g_2|\dots|g_n],& i=0\\
		\left[g_1|\dots |g_ig_{i+1}|\dots|g_n \right], & i = n\\
		\left[g_1|\dots|g_{n-1}\right], & i = n.
	\end{array}\right.\]
\end{rmk}
\begin{defn}
	If we set $B_n=\bbZ G^{n+1}$, then $(B_\bullet, d)$ is called the \textbf{Bar resolution} of $\bbZ$.
\end{defn}
\begin{defn}
	If $D_\bullet$ is the subcomplex of $B_\bullet$ generated by all elements $[g_1|\dots|g_n]$ such that some $g_i=e$,
	then the quotient $B_\bullet/D_\bullet=\overline{B}_\bullet$ is called the \textbf{normalized Bar complex.}
\end{defn}
\begin{prob}
	Write down both the Bar and normalized Bar resolutions of $\bbZ$ using $G=\bbZ/2$ and compare.
\end{prob}

\subsection{Hochschild Complex for $H^\ast(G,M)$}
Begin with the Bar resolution $B_\bullet\to \bbZ$ and apply $\Hom_G(-,M)$, giving us a cochain complex. It will help to notice 
that $\Hom_G(B_n,M)\cong\Hom_{\bbZ G}(\bbZ G^{n+1},M)\cong\Hom_\bbZ(\bbZ G^n,M)$, which is equal to the set of 
(set!) functions $f:G^n\to M$ extended $\bbZ$-linearly. 
\begin{defn}
	$C^n(G,M)$ is the set of all linear combinations of set maps $f:G^n\to M$.
\end{defn}
\begin{rmk}
	Notice $C^n(G,M)\cong \Hom_G(\bbZ G^{n+1},M)$.
\end{rmk}
Then we can compute the induced differential $d_n:C^n(G,M)\to C^{n+1}(G,M)$ to be 
\begin{align*}
	(df)(g_1,\dots,g_{n+1})&- g_1f(g_2,\dots,g_{n+1})+\cdots+(-11)^if(g_1,\dots,g_ig_{i+1},\dots,g_{n+1})\\
	&\qquad\qquad+\cdots+(-1)^nf(g_1,\dots,g_n)
\end{align*}

\begin{cor}
	The Hochschild complex $C^\bullet$ computes $H^\ast(G,M).$
\end{cor}

\begin{defn}
	Let $R$ be a ring, $M$ an $(R,R)$-bimodule. Then a map of abelian groups $d:R\to M$ is called 
	a \textbf{derivation} if
	\[d(rs)=rd(s)+d(r)s.\]
\end{defn}
\begin{defn}
	If $M$ is a $\bbZ G$ (left) module and $d:\bbZ G\to M$ is such that 
	\[rd(s)+d(r)\]
	$d$ is called a \textbf{crossed homomorphism.}
\end{defn}
\begin{rmk}
	This essentially arises by considering a derivation on a module where the right action is defined to be trivial! This is sort of what is happening
	with the derivation in the Hochschild complex.
\end{rmk}
\begin{rmk}
	Notice that $d^1f(g_1,g_2)=g_1f(g_2)-f(g_1g_2)+f(g_1)$. If $d^1f=0$ then we call $f$ a cocycle (as per usual).
	So then the \textbf{cocycle condition on $C^1(G,M)$} is 
	\[f(g_1g_2)=g_1f(g_2)+f(g_1)\]
	or exactly the crossed homomorphisms/derivations.

	This gives us a nice characterization:
	\[H^1(G,M)\cong\{\text{crossed homomorphisms}\}/\{\text{principal crossed homomorphisms}\}.\]
\end{rmk}	

\section{Talk: Lie Algebra Cohomology}
\subsection{Geometric Notions}
Let $G$ be a Lie group, a smooth manifold with smooth multiplication and inverse maps. Let $\frakg$ be $\operatorname{Lie}(\frakg)$
And consider the complex $\Omega^\bullet(G),$ the complex of $k$-forms on $G$ with differential given by the exterior derivative.

\begin{defn}
	$H^\ast_{dR}(G)$ is just the homology of $\Omega^\bullet(G)$.
\end{defn}

\begin{defn}
	Let $\Omega^\bullet_L(G)$ be the complex of left-invariant forms, $\{\omega\in\Omega^\ast(G)|L_g^\ast\omega=\omega\}$.
\end{defn}
\begin{rmk}
	A key fact to showing that this is indeed a complex is that the exterior derivative commutes with pullbacks.
\end{rmk}

\begin{thm}
	$\iota:\Omega_L^\bullet(G)\to\Omega^\bullet(G)$ is a quasi-isomorphism when $G$ is compact and connected.
\end{thm}
\begin{prf}
	(Sketch) For injectivity, suppose $\iota^\ast[\omega]=0\in H_{dR}^\ast(G)$. Then $\omega=d\mu$ for some $\mu$.
	Set $I(\omega)=\int_G L_g^\ast\omega dg$ (where this is called the Haar measure). But then after checking lots 
	of fiddly facts, you can show 
	\[\omega=I(\omega)=I(d\mu)=dI(\mu)\]

	For sujectivity, notice that the natural pairing $\langle[z],[w]\rangle=\int_z w$ where $[z]$ is a class in the homology  and $[w]$ is in the cohomology,
	then since any other representative for $[z]$ differs by a boundary the integral in unaffected and since any other representative in $[w]$ 
	differs by a coboundary (exact form), the integral is also unafected by this.

	Then there is some integration computations that go through lots of replacing things with other representatives (since $[z]=[L_g(z)]$ and $\iota^\ast[I(\omega)]=[\omega]$).
\end{prf}

For the next step we want to restrict to the identity. Notice that $\varepsilon:\Omega_L^\ast(G)\to(\wedge^\ast\frakg)^\ast$
is a well-defined map (here we send each form to its restriction to the identity: $\varepsilon(\omega)=\omega_e$).

Notice 
\begin{align*}
	\varepsilon(d\omega)(X_1|_e,\dots,X_n|_e)&=d\omega(X_1,\dots,X_n)(e)\\
	&=\sum_1^nX_i(\omega(X_1,\dots,\hat X_i,\dots,X_n))(e)+\sum_{i<j}([X_i,X_j],X_1,\dots,\hat X_i,\dots,\hat X_j,\dots,X_n)(e)
\end{align*}
and since the first term in the sum is zero (why?) the latter part is exactly the differential we want:
\[\delta\alpha(X_1,\dots,X_n)=\sum_{i<j}([X_i,X_j]_e,X_1|_e,\dots,\hat X_i|_e,\dots,\hat X_j|_e,\dots,X_n|_e)\]

\begin{thm}
	$H_L^\ast(G)\cong H^\ast(\frakg)$. Furthermore $H_{dR}^\ast(G)\cong H^\ast(\frakg)$.
\end{thm}
\begin{rmk}
	Note that $H^\ast(\frakg)$ is the cohomlogy of the $(\wedge^\ast\frakg)^\ast$
\end{rmk}
\subsection{Algebraic Notions}
Let $k$ be a commutative ring, $\frakg$ be a Lie algebra, and $\calU(\frakg)$ its universal enveloping algebra.

Notice that there is a functor from $k$ modules to $\frakg$ modules that gives any module $M$ a trivial $\frakg$ action. Recall the definition 
of invariants $(-)^\frakg$ ($M^\frakg=\{m\in M|xm=0\}$) and coinvariants $(-)_\frakg$ ($M_\frakg=M/\frakg M$).

The plan of attack is to show that these functors have adjoints and thus are exact on one side, and then we can talk about derived functors.

\begin{lem}
	$(-)^\frakg$ is right adjoint to the trivial $\frakg$-module functor. $(-)_\frakg$ is left adjoint to the same functor.
\end{lem}
\begin{cor}
	Thus $(-)^\frakg$ and $(-)_\frakg$ are left and right exact, respectively.
\end{cor}
\begin{lem}
	Let $\frakg$ be a Lie algebra. Then we have an equivalence of categories (isomorphism actually!)
	\[\frakg\text{-}\mathbf{mod}\cong \mathcal U\frakg\text{-}\mathbf{mod}\]
\end{lem}
\begin{rmk}
	Since $\mathcal U\frakg$-$\mathbf{mod}$ has enough projectives, so does $\frakg$-$\mathbf{mod}$.
\end{rmk}
\begin{defn}
	We write $H_\ast(\frakg,M)$ to be the left erived functor $L_\ast(-)_\frakg(M)$ and $H^\ast(\frakg,M)=R^\ast(-)^\frakg(M)$.
	These are the (co)homology groups of $\frakg$ with coefficients in $M$.
\end{defn}

How does this relate to $\Ext$ and $\Tor$?
\begin{rmk}
	If $M$ is a $\frakg$-module, then we can give $M$ a $\calU\frakg$ action in the following way:
	\[(x_1\otimes\cdots\otimes x_n)\cdot m=(x_1(x_2(\cdots(x_nm))))\]
\end{rmk}
\begin{thm}
	sIf $M$ is a $\frakg$-module, then 
	\[H_\ast(\frakg,M)=\Tor_\ast^{\calU\frakg}(k,M)\]
	and 
	\[H^\ast(\frakg,M)\cong\Ext_{\calU\frakg}^\ast(k,M).\]
\end{thm}

Here we need the PBW theorem, so we need to assume now that $\frakg$ is free as a $k$-module.
\begin{defn}
	Let $V_p(\frakg)=\calU\frakg\otimes_k\wedge^p\frakg$, where $V_0=\calU\frakg$ and $V_1=\calU\frakg\otimes\frakg$.
	
	Let $\varepsilon:\calU\frakg\to k$ be the augmentation map. Let $d:V_1(\frakg)\to V_0(\frakg)$
	to be the multiplication map $d(u\otimes x)=ux$ and more generally, 
	\[d(u\otimes x_1\wedge\cdots\wedge x_n)=\theta_1+\theta_2\]
	where
	\[\theta_1=\sum_1^n(-1)^{i+1}(ux_1\wedge\cdots\wedge\hat x_i\wedge\cdots\wedge x_n)\]
	and 
	\[\theta_2=\sum_{i<j}(-1)^{i+j}u\otimes[x_i,x_j]\wedge x_1\wedge\cdots\wedge\hat x_i\wedge\cdots\wedge \hat x_j\wedge\cdots\wedge x_n\]

	Then $V_\bullet(\frakg)$ with the above differential is called the \textbf{Chevalley-Eilenberg complex} for $\frakg$
\end{defn}

\begin{thm}
	$V_\bullet(\frakg)\to k$ is a projective resolution.
\end{thm}
\begin{rmk}
	This needs spectral sequences as well as the PBW theorem.
\end{rmk}
\begin{thm}
	If $M$ is a left $\frakg$-module, then the cohomology modules $H^\ast(\frakg,M)$ are the cohomology of the complex 
	\[\Hom_\frakg(V(\frakg),M)=\Hom_\frakg(\calU\frakg\otimes\wedge^\ast\frakg,M)\cong\Hom)k(\wedge^\ast\frakg,M)\]
\end{thm}

\section{May 20, 2019}
Recall that we hat just said that $H^1(G,M)$ was crossed homomorphisms module principal crossed homomorphisms.

Notice that when $M$ is trivial, a crossed homomorphism is precisely a group homomorphism and since the 
principle ones are zero, 
\[H^1(G,M)=\Hom_\Grp(G,M)=\Hom_\Ab(G/[G,G],M)\]
\begin{cor}
	If $G$ is perfect ($G=[G,G]$) then $H^1(G,\bbZ)=0$.
\end{cor}
\begin{rmk}
	In fact, $H_1(G,\bbZ)=G/[G,G]$.
\end{rmk}

\subsection{$H^2$}
Now let $A$ be an ableian group and $G$ module. Recall that 
\[0\to A\to E\to G\to 1\]
(a short exact sequence in $\Grp$) is called an extension of $G$ by $A$. Then we can show that (or let Weibel or Dummit \& Foote show us)
\[H^2(G,A)\cong \{0\to A\to E\to G\to 1\}/\sim\]
Check out Weibel 6.6.3.

\subsection{Triangulated Categories}
Recall the definition of triangulated categories from James' class or from my notes o
n Brown Representability:
\begin{defn}
	Let $\cal C$ be a $k$-linear additive category with a \textbf{suspension} or \textbf{shift} functor (auto-equivalence, actually)
	\[\Sigma:\cal C\to\cal C.\]

	Then $({\calC}, \Sigma ,D)$ is \textbf{triangulated} category where $D$ is a full, nonempty subcategory $D\subseteq\Delta(\cal C)$
	with shift functor $\Sigma$ of $D$ and we have the following axioms:
	\begin{itemize}
		\item (\textbf{TR0}) $0\to X\xrightarrow{\id} X\to 0$ is in $D$ for each $X\in \cal C$ and furthermore $D$ is closed under both shifts and triangle isomorphisms.
		\item (\textbf{TR1}) [\textit{Mapping Cone Axiom}] For any $f:X\to Y$ in $\cal C$, there is a triangle
		\[X\xrightarrow{f} Y\to Z\to\Sigma X\]
		\item (\textbf{TR2}) [\textit{Rotation Axiom}] If $F\in D$, then $\mathscr R(F),\mathscr R^{-1}(F)\in D$
		\item (\textbf{TR3}) [\textit{Morphism Axiom}] Given two triangles 
		\begin{center}
			\begin{tikzcd}
				X\ar[r]\ar[d,"f"] & Y\ar[r]\ar[d,"g"] & Z\ar[r]\ar[d,"h",dashed] & \Sigma X\ar[d,"\Sigma f"]\\
				X'\ar[r] & Y'\ar[r] & Z'\ar[r] & \Sigma X'
			\end{tikzcd}
		\end{center}
		with maps $f$ and $g$, there exists an $h:Z\to Z'$ such that the above diagram commutes.
		\item (\textbf{TR5}) [\textit{Octahedral Axiom}] Given three triangles: $X\xrightarrow{u}Y\to Z'\to\Sigma X$, $Y\xrightarrow{v}Z\to X'\to\Sigma Y$
		and $X\xrightarrow{v\circ u} Z\to Y'\to\Sigma X$. Then there is a triangle $Z'\to Y'\to X'\to\Sigma Z'$ such that
		\begin{center}
			\begin{tikzcd}
				X\ar[dr,"u"]\ar[rr,"v\circ u",bend left] && Z\ar[rr,bend left]\ar[dr] && X'\ar[rr,dashed,bend left]\ar[dr] && \Sigma Z'\\
				& Y\ar[ur,"v"]\ar[dr] && Y'\ar[ur,dashed]\ar[dr] && \Sigma Y\ar[ur] &\\
				&& Z'\ar[rr,bend right]\ar[ur,dashed] && \Sigma X\ar[ur] &&
			\end{tikzcd}
		\end{center}
	\end{itemize}
\end{defn}
\begin{rmk}
There are two other representations of the octahedral axiom that are sometimes more helpful:
\begin{center}
	\begin{tikzcd}
		X\ar[r,"u"]\ar[d,"\sim"] & Y\ar[r]\ar[d,"v"] & Z'\ar[r]\ar[d,dashed] &\Sigma X\ar[d,"\sim"]\\
		X\ar[r,"v\circ u"] & Z\ar[r]\ar[d] & Y'\ar[r]\ar[d,dashed] &\Sigma X\ar[d,"\Sigma u"]\\
		& X'\ar[r,"\sim"] \ar[d] & X'\ar[r]\ar[d,dashed] & \Sigma Y\\
		&\Sigma Y\ar[r] & \Sigma Z' &
	\end{tikzcd}
\end{center}

\begin{center}
	\begin{tikzcd}
		&& & X\ar[dl,"u"]\ar[dd,"v\circ u"] &\\
		&& Y\ar[dr,"v"]\ar[ddll] & &\\
		&& & Z\ar[d]\ar[dr] &\\
		Z'\ar[rrr,dashed] && & Y'\ar[r,dashed] & X'
	\end{tikzcd}
\end{center}

\begin{center}
	\begin{tikzcd}
		A\ar[rr,"f"]\ar[rrrr,bend left,"g\circ f"] & & B\ar[rr,"g"]\ar[dl] & & C\ar[ddll,bend left,dotted]\ar[dl]\\
		& C'\ar[ul,dotted]\ar[dr] & & A'\ar[ul,dotted]\ar[ll,dotted] &\\
		& & B'\ar[ur]\ar[uull,dotted, bend left] & &
	\end{tikzcd}
\end{center}
Above dotted lines refer to shifting.
\end{rmk}
\begin{lem}
	Say that $A\xrightarrow{u}B\xrightarrow{v}C\to\Sigma A$ is exact. Then $v\circ u=0$.
\end{lem}
\begin{prf}
	Just complete the morphism $(\id_A,u)$ to a triangle morphism. This gets it!
\end{prf}
\begin{ex}
	$\K(\Ch(\Rmod))$ is triangulated with triangles being all those in $\Delta(\calC)$ isomorphic to $A\xrightarrow{f}B\to \cone(f)\to \Sigma A$.
\end{ex}

\section{May 22, 2019}

%%%%%%%%%%%%%%%%%%%%%%%%%%%%%%%%%%%%%%%%%%%%%%%%%%%%%%%%%%%%%%%%%%%%
%%%%%%%%%%%%%%%%%%%%%%%%%%%%%%%%%%%%%%%%%%%%%%%%%%%%%%%%%%%%%%%%%%%%
%%%%%%%%%%%%%%%%%%%%%%%%%%%%%%%%%%%%%%%%%%%%%%%%%%%%%%%%%%%%%%%%%%%%
%%%%%%%%%%%%%%%              Homework                %%%%%%%%%%%%%%%
%%%%%%%%%%%%%%%%%%%%%%%%%%%%%%%%%%%%%%%%%%%%%%%%%%%%%%%%%%%%%%%%%%%%
%%%%%%%%%%%%%%%%%%%%%%%%%%%%%%%%%%%%%%%%%%%%%%%%%%%%%%%%%%%%%%%%%%%%
%%%%%%%%%%%%%%%%%%%%%%%%%%%%%%%%%%%%%%%%%%%%%%%%%%%%%%%%%%%%%%%%%%%%

\newpage
\section{HW1 -- Due April 17}
Throughout the assignment, $R$ is an associative ring with 1.
\begin{prob}
	Show that $\Ch(R)$ is an abelian category.
\end{prob}
\begin{sol}
	We begin by proving a lemma:
	\begin{lem}\label{lem-chain-additive} 
		$\Ch=\Ch(R)$ is additive. 
	\end{lem}
	\begin{prf}
		Let $C,D\in\Ch$ and $f\in\Hom_{\Ch}(C,D)$. Then 
		$f=(f_i)_{i\in\bbZ}$ is a collection of maps of $R$-modules. Then, leveraging that $\Rmod$ is Abelian
		(\textit{the} Abelian category in some ways) and the fact that addition of chain maps is perfomed 
		component-wise, this gives $\Hom_{\Ch}(C,D)$ the structure of an Abelian group as the direct product
		of Abelian groups. That the component-wise sum is actually a chain map is evident since 
		\[d((f+g)(x))=d(f(x)+g(x))=d(f(x))+d(g(x))=f(d(x))+g(d(x))=(f+g)(d(x)).\]

		The chain $0_\bullet=\cdots 0 \to 0\to 0\to \cdots$ is the zero object in $\Ch$ since there is a 
		unique map $\mathbf{0}=(0)_{i\in\bbZ}$ both from and to $0_\bullet$ from any chain (again, since maps
		are component-wise $R$-module maps, so each component must be $0$). This map is indeed a chain map
		since $0=0\circ d=d'\circ 0$. Therefore $\Ch$ is additive.
	\end{prf}

	\brk

	Now for the final three properties (that (co)kernels exist, that every monic is the kernel of its cokernel,
	and that every epi is the cokernel of its kernel) follow from the fact that $\Rmod$ is abelian (and thus satisfy
	these properties) and that (co)kernels in $\Ch$ are ``what we would hope they would be'' component-wise. 
	Therefore the properties pass from $\Rmod$ up to $\Ch$.

	\begin{lem}\label{lem-chain-ker}
		Let $C,D\in\Ch$ and $f\in\Hom_{\Ch}(C,D)$. Then there is a complex $K$ and a map $i:K\to C$ 
		that is the kernel of $f$ and furthermore corresponds to the ($\Rmod$) kernel 
		\[i_n:K_n=\ker f_n\to C_n\]
		on the components.
	\end{lem}
	\begin{prf}
		\begin{rmk}
		I have just noticed that I am basically re-proving part of snake lemma below for the existence of the maps between
		kernels. Oh well, good practice.
		\end{rmk}
		Consider the following diagram:

	\begin{center}
		\begin{tikzcd}
			\ker f_n \ar[r,"\partial_n"]\ar[d,hook,"i_n"] & \ker f_{n+1}\ar[d,hook,"i_{n+1}"]\\
			C_n\ar[r,"d_n"]\ar[d,"f_n"] & C_{n+1}\ar[d,"f_{n+1}"]\\
			D_n\ar[r,"d_n'"] & D_{n+1}
		\end{tikzcd}
	\end{center}

	Let $f\in\Hom_{\Ch}(C,D)$ as above. Define $K=(\ker f_i,\partial)\in\Ch$ where 
	$\partial^i:\ker f_i\to \ker f_{i+1}$ is defined as follows: let $x\in\ker f_n$. Then since $f$
	is a chain map, 
	\[0=d_n'(0)=d_n'(f_n(x))=f_{n+1}(d_n(x))\]
	so in particular $d_n(x)\in\ker f_{n+1}$. So $d_n(\ker f_n)\subseteq \ker f_{n+1}$, so we can lift
	$d_n\circ i_n$ through $i_{n+1}$ to a map $\partial_n:\ker f_n\to \ker f_{n+1}$. Then $K$ is a chain 
	and $i$ is a chain map by construction.
	
	That this is the kernel of $f$ in $\Ch$ follows since if $g\in\Hom(A,C)$ and $f\circ g=0$, each component
	map $g_n$ must factor through $\ker f_n$. It remains to see that this map is a chain map. Consider the diagram

	\begin{center}
		\begin{tikzcd}
			& A_n\ar[rr,"d_A"]\ar[ld,"\tilde g_n",swap]\ar[ldd,"g_n"] & & A_{n+1}\ar[ld,"\tilde g_{n+1}",swap]\ar[ldd,"g_{n+1}"]\\
			\ker f_n\ar[rr,"d_K"]\ar[d,hook,"i_n",swap] & & \ker f_{n+1}\ar[d,hook,"i_{n+1}",swap] &\\
			C_n\ar[rr,"d_C"] & & C_{n+1} &
		\end{tikzcd}
	\end{center}

	\noindent where $\tilde g_n$ is the canonical map through the kernel of $f_n$. Pick any $a\in A_n$ and notice
	\begin{align*}
		(i_{n+1}\circ d_K)\circ \tilde g_n(a)&=(d_C\circ i_n)\circ \tilde g_n(a) \\
		&= d_C\circ g_n(a)\\
		&=g_{n+1}\circ d_A(a)\\
		&=i_{n+1}\circ \tilde g_{n+1}\circ d_A(a)
	\end{align*}
	and so since $i_{n+1}$ is mono, this establishes
	\[d_K\circ\tilde g_n = \tilde g_{n+1}\circ d_A\]
	so $\tilde g$ is a chain map and therefore $i:K\to C$ is the kernel of $f$, which lies in $\Ch$.
	\end{prf}

	\brk
	\begin{lem}\label{lem-chain-coker}
		Let $C,D\in\Ch$ and $f\in\Hom_{\Ch}(C,D)$. Then there is a complex $E$ and a map $q:D\to E$ 
		that is the cokernel of $f$ and furthermore corresponds to the ($\Rmod$) cokernel 
		\[q_n:D_n\to E_n=\coker f_n\]
		on the components.
	\end{lem}
	\begin{prf}
	Consider
	
	\begin{center}
		\begin{tikzcd}
			C_n\ar[r,"d_n"]\ar[d,"f_n"] & C_{n+1}\ar[d,"f_{n+1}"]\\
			D_n\ar[r,"d_n'"]\ar[d,two heads,"q_n"] & D_{n+1}\ar[d, two heads,"q_{n+1}"]\\
			\coker f_n\ar[r,"\partial_n"] &\coker f_{n+1}
		\end{tikzcd}
	\end{center}

	\begin{rmk}
		This also follows from snake lemma and is less ``elementy''.
	\end{rmk}
	Given $\hat x\in\coker f_n$, let $x$ be any preimage of $\hat x$ under $q_n$ and define
	\[\partial_n(\hat x)=q_{n+1}\circ d_n'(x).\]

	To show that this map is well-defined, let $y$ be any other preimage of $x$ under $q_n$.
	Then since $q_n(x-y)=0$, we can pull $x-y$ back through $f_n$ to some $c\in C_n$. Then
	\begin{align*}
		(q_{n+1}\circ d_n')(x-y)&=q_{n+1}\circ (d_n'\circ f_n)(c)\\
		&=q_{n+1}\circ(f_{n+1}\circ d_n)(c)\\
		&=(q_{n+1}\circ f_{n+1})\circ d_n(c)\\
		&=0\circ d_n(c)=0
	\end{align*}
	so $q_{n+1}\circ d_n'(x)=q_{n_1}\circ d_n'(y)$, so $\partial_n(\hat x)$ is well-defined. Then 
	$E=(\coker f_n)_{n\in\bbZ}$ is a chain complex with differential $\partial$. To show that $q:D\to E$ 
	is the cokernel of $f$, let $A\in\Ch$ and $g\in\Hom_{\Ch}(D,A)$  with $g\circ f = 0$ then from the universal property in 
	$\Rmod$, the component maps must factor through the $E_n$:

	\begin{center}
		\begin{tikzcd}
			& A_n\ar[rr,"d_A"] & & A_{n+1}\\
			\coker f_n\ar[rr,"d_E"]\ar[ur,"\tilde g_n"]& & \coker f_{n+1}\ar[ur,"\tilde g_{n+1}"] &\\
			D_n\ar[rr,"d_D"]\ar[uur,"g_n",swap]\ar[u,two heads,"q_n"] & & D_{n+1}\ar[uur,"g_{n+1}",swap]\ar[u,two heads,"q_{n+1}"] &
		\end{tikzcd}
	\end{center}

	Then for any $d\in D_n$, 
	\begin{align*}
		\tilde g_{n+1}\circ d_E\circ q_n(d) &= \tilde g_{n+1}\circ q_{n+1}\circ d_D(d)\\
		&= g_{n+1}\circ d_D(d)\\
		&= d_A\circ g_n(d)\\
		&= d_A\circ \tilde g_n\circ q_n(d)
	\end{align*}
	and since $q_n$ is epi, we conclude that 
	\[\tilde g\circ d_E=d_A\circ \tilde g\]
	so since $E$ is the (unique) object satisfying this property, it is the cokernel of $f$ in $\Ch$.
	\end{prf}
\end{sol}

\begin{prob}
	Show that a chain map $f:C_\bullet\to D_\bullet$ induces a well-defined map $H_n(f)$ on homology.
\end{prob}
\begin{sol}
	Since $f$ is a chain map, it is easy to see it induces a map on cycles: let $c\in Z_n(C)$.
	Then 
	\[d_D(f(c))=f(d_C(c))=f(0)=0\]
	so $f(Z_n(C))\subseteq Z_n(D).$ Thus it suffices to show that if $x$ and $y$ differ by a boundary,
	$f(x)$ and $f(y)$ do as well.

	To see this, assume that $x-y\in B_n(C)$, so there is a $z\in C_{n+1}$ such that $d_C(z)=x-y$.
	But then 
	\[f(x)-f(y)=f(x-y)=f(d_C(z))=d_D(f(z))\in B_n(D)\]
	so the map $H_n(f)$ is well-defined.
\end{sol}

\begin{prob}
	(\textbf{Weibel 1.4.5}) 
	\begin{enumerate}
		\item Show that the chain homotopy equivalence is an equivalence relation on the set of all chain maps from $C$
		to $D$. Let $\Hom_\mathbf{K}(C,D)$ denote the equivalence classes of such maps. Show that $\Hom_\mathbf{K}(C,D)$
		is an Abelian group.
		\item Let $f\sim g$ where $f,g:C\to D$. If $u:B\to C$ and $v:D\to E$ are chain maps, show $vfu$ and $vgu$ are 
		chain homotopic. Deduce that there is a category $\mathbf K$ whose objects are chain complexes and whose morphisms
		are given in (a).
		\item Let $f_0,f_1,g_0,$ and $g_1$ be chain maps $C\to D$ such that $f_i$ is chain homotopic to $g_i$. Show that 
		$f_0+f_1\sim g_0+g_1$. Deduce that $\mathbf K$ is an additive category and that $\Ch\to\mathbf K$ is an 
		additive functor.
		\item (Optional) Is $\mathbf K$ an abelian category? Explain.
	\end{enumerate}
\end{prob}
\begin{sol}
	\textbf{(a):} Let $f,g,h\in\Hom_{\Ch}(C,D).$ It is clear that $f\sim f$ since $f-f=0$ is nullhomotopic via $s=0$. Thus $\sim$ is reflexive.
	If $f\sim g$, then there is a chain homotopy $s$ such that $f-g=ds+sd$. Setting $s'=-s$, we get
	\[ds'+s'd=-(ds+sd)=-(f-g)=g-f\]
	so $g\sim f$, whence $\sim$ is symmetric.

	Finally, assume $f\sim g$ via $s$ and $g\sim h$ via $t$. But then notice that
	\begin{align*}
		f-h &= f-g + g-h\\
		&= ds+sd+dt+td\\
		&= ds+dt+sd+td\\
		&= d(s+t)+(s+t)d
	\end{align*}
	so $f-h\sim 0$ via the chain homotopy $s+t$. Therefore $\sim$ is an equivalence relation.

	The last statement follows since 
	\[\Hom_{\mathbf K}(C,D)=\Hom_{\Ch}(C,D)/\sim\]
	and every quotient of an Abelian group is Abelian.

	\vspace{2ex}\noindent \textbf{(b):} Say $s$ is the homotopy from $f-g$ to $0$. Then notice since everything is linear:
	\begin{align*}
		vfu-vgu &= v(f-g)u\\
		&= v(sd_C+d_Ds)u\\
		&= vs(d_Cu)+(vd_D)su\\
		&= vsud_B+d_Evsu
	\end{align*}
	so $vfu\sim vgu$ via the chain homotopy $vsu$. This means that if $[f]$ is the chain homotopy
	equivalence class of $f$, that the composition of chain maps is well-defined. To see this, we can let
	either $B=C$ and $u=\id_C$ or else $D=E$ and $v=\id_D$. 
	
	The identity map in $\Hom_{\mathbf K}(C,C)$ is $[\id_C]$, containing all nullhomotopic self-maps of $C$.
	Since $\mathbf K$ is a collection of objects and morphisms with an identity map for each object and that is closed 
	under composition, we get that $\mathbf K$ is a category.

	\vspace{2ex}\noindent \textbf{(c):} Let $f_0,f_1,g_0$ and $g_1$ be defined as provided. Then
	\begin{align*}
		f_0+f_1-g_0-g_1 &= (f_0-g_0)+(f_1-g_1)\\
		&= s_0d_C+d_Ds_0 +s_1d_C+d_D s_1\\
		&= (s_0+s_1)d_C+d_D(s_0+s_1)
	\end{align*}
	so $f_0+f_1\sim g_0+ g_1$. Then $[f]+[g]=[f+g]$ for all $f,g\in\Hom_{\Ch}(C,D)$. Now $0_\bullet$
	is the zero object in $\mathbf K$ since there is a unique map in $\Hom_{\Ch}(0,C)$ and $\Hom_{\Ch}(C,0)$,
	so the same is true in $\mathbf K$. Therefore $\mathbf K$ is additive.

	The functor $Q:\Ch\to\mathbf K$ sending complexes to themselves and chain maps $Q(f)=[f]$ (which is
	a functor since composition is respected by (b) and identity maps are respected by definition)
	is additive since for any $A,B\in \Ch$,
	\[\Hom_{\Ch}(A,B)\to \Hom_{\mathbf K}(A,B)\]
	is a group homomorphism as shown above.

	\vspace{2ex}\noindent\textbf{(d):} I suspect the answer to this is no.
\end{sol}

\begin{prob}
	Show that a chain complex $C$ is split exact if and only if it is nullhomotopic.
\end{prob}
\begin{sol}
	Begin by assuming that $C$ is split exact. Then consider the diagram
	
	\begin{center}
		\begin{tikzcd}
			C_{n-1}^0\oplus C_{n-1}^1\ar[r,"d"]\ar[d,"\id"] & C_n^0\oplus C_n^1\ar[r,"d"]\ar[d,"\id"]\ar[ld,"s"] & C_{n+1}^0\oplus C_{n+1}^1\ar[d,"\id"]\ar[ld,"s"]\\
			C_{n-1}^0\oplus C_{n-1}^1\ar[r,"d"] & C_n^0\oplus C_n^1\ar[r,"d"] & C_{n+1}^0\oplus C_{n+1}^1
		\end{tikzcd}
	\end{center}

	\noindent where $C_n=C_n^0\oplus C_n^1$ and $d_n|_{C_n^0}$ is the zero map and $d_n|_{C_n^1}$ is 
	an isomorphism onto $C_{n+1}^0.$ 
	
	Define $s$ to be 
	\[(0,\pi_0)_i:C_i^0\oplus C_i^1\to C_{i-1}^0\oplus C_{i-1}^1\]
	where 
	\[\pi_0:C_i^0\oplus C_i^1\to C_i^0\cong C_{i-1}^1\]
	is projection onto the first coordinate. Then clearly 
	\[f((a,b))=(a,b)=(a,0)+(0,b)=d((0,a))+s((b,0))=d(s((a,b)))+s(d((a,b)))\]
	whence $\id\sim 0$ so $C$ is nullhomotopic.

	\brk

	Now assume there exists a map $\id_C\sim 0$ via chain homotopy $s:C\to C[-1]$. The exactness of 
	$C$ is immediate since homotopic maps induce isomorphisms on homology. Consider the SES

	\begin{center}
		\begin{tikzcd}
			0\ar[r] & \ker d \ar[r,"i"] & C_n\ar[r,swap,"d"] & d(C_n)\ar[r]\ar[l,bend right,"s'",swap] & 0
		\end{tikzcd}
	\end{center}
	
	\noindent where 
	\[s'=s_{n+1}|_{d(C_n)}\]

	But then consider the restriction of $\id =\id_{C_{n+1}}$, to $d(C_n)\subseteq C_{n+1}$:
	\[\id|_{d(C_n)}=(s_{n+2}\circ d+d\circ s_{n+1})|_{d(C_n)}=(s_{n+2}\circ d)|_{d(C_n)}+(d\circ s_{n+1})|_{d(C_n)}=(d\circ s_{n+1})|_{d(C_n)}\]
	so $d\circ s'$ is the identity on $d(C_n)$ whence $s$ is a splitting of the above short exact sequence.

	Since this holds for an arbitrary $n$, this establishes that $C_n\cong \ker d\oplus d(C_n)$, 
	and thus by exactness $C$ is split exact.
\end{sol}

\begin{prob}
	Show that a chain complex $P_\bullet$ is a projective object in the category of
chain complexes over a ring $R$ if and only if $P_\bullet$ is a split exact complex of projective
modules.
\end{prob}
\begin{sol}
	Assume first that $P$ is a projective object in $\Ch$. Following a hint in Weibel, we consider the surjection
	$\operatorname{cone}(P)\xrightarrow{\delta} P[-1]\to 0$, which admits a splitting map $\sigma:P[-1]\to\operatorname{cone}(P)$
	via the standard argument -- lift $\id_P:P\to P$ to a map $\sigma:P\to \operatorname{cone}(P)$ such that
	$\delta\circ\sigma=\id_P$.

	From Weibel, we know $\delta(a,b)=-a$, so if $\sigma_i=(s,t)_i:P_{i-1}\to P_{i-1}\oplus P_i,$ we get that
	\[\delta\circ(s,t)(p)=-s(p)=p=\id_P(p)\]
	by the property of the lift $\sigma$ and therefore $s=-\id_P$.

	Then using the fact that $\sigma$ is a chain map, we can compute (if $C=\operatorname{cone}(P)$):
	\[
		(s\circ d_P,t\circ d_{P[-1]})=(s,t)\circ d_P=d_C\circ(s,t)=(d_P\circ s, d_P\circ t-\id_P\circ s)
	\]
	and therefore by focusing on the second component:
	\[t\circ d_{P[-1]}-d_P\circ s=-t\circ d_P-d_P\circ s=\id_P\]
	whence $-\id_P\sim\id_P\sim 0$, so by problem three above $P$ is split exact.

	To see every element is injective, we can ``localize'' by considering a lift of $P\to M_\bullet=(\cdots\to0\to M\to 0\to \cdots)$
	through the surjection $(\cdots\to 0\to N\to 0\to\cdots)\twoheadrightarrow M_\bullet$. Since a lift always exists for $M$ and $N$ 
	in any component, this shows all the terms in the complex $P$ are projective.

	\brk

	Now assume that $P$ is a split exact. Then since the $P_i$ are projective, for any chain map $f:P\to M$ and 
	surjection $g:N\to M$, we get maps $\tilde f_i:P_i\to N_i$. Recall that since $P$ is split exact, there 
	exists a splitting map $\sigma_n:P_n\to P_{n+1}$ for every $n$ such that $d_P\circ \sigma_n=\id_{P_n}.$

	Now define $F_n:P_n\to N_n$ by 
	\[F_n=d_N\circ \tilde f_{n+1}\circ \sigma_n+\tilde f_n\circ\sigma_{n-1}\circ d_P\]
	and so we can compute
	\begin{align*}
		d_N\circ F_n&=d_N^2\circ\tilde f_{n+1}\circ \sigma_n + d_N\circ\tilde f_n\circ \sigma_{n-1}\circ d_P\\
		&= d_N\circ\tilde f_n\circ \sigma_{n-1}\circ d_P\\
		&= d_N\circ\tilde f_n\circ \sigma_{n-1}\circ d_P + \tilde f_{n-1}\circ\sigma_{n-2}\circ d_P^2\\
		&= F_{n-1}\circ d_P
	\end{align*}
	so $F$ is a chain map, and therefore $F$ is a lift of $f$, so since $P$ satisfies the usual 
	diagram, it is a projective object.
\end{sol}


%%%%%%%%%%%%%%%%%%%%%%%%%%%%%%%%%%%
%%%%%%%%%%%%%%%%%%%%%%%%%%%%%%%%%%%
%%%%%%%%%%%%%%%%%%%%%%%%%%%%%%%%%%%
%%%%%%%%%%%%%%%%%%%%%%%%%%%%%%%%%%%

\newpage
\section{HW2 -- Due XXXXX}
\begin{prob}
	Let $R$ be a commutative ring, and $N$ be an $R$-module. Show that the following are equivalent:
	\begin{enumerate}
		\item $N$ is flat 
		\item $\Tor_i^R(M,N) = 0$ for any $i>0$ and any $R$-module $M$
		\item $\Tor_1^R(M,N) = 0$ for any $R$-module $M$
		\item $\Tor_1^R(R/I, N) = 0$ for any ideal $I \subset R$.
	\end{enumerate}
\end{prob}
\begin{sol}
	(a)$\Rightarrow$(b)$\Rightarrow$(c) is clear since $\Tor_i^R(-,N)$ is a left derived functor of $-\otimes N$. But when $N$ is flat, $-\otimes N$ is exact.

	That (c)$\Rightarrow$(a) can be seen by using the long exact sequence of $\Tor$ arising from any short exact sequence $0\to A\to B\to C\to 0$:
	\[\cdots\to\Tor_2^R(C,N)\to 0\to 0\to 0\to A\otimes N\to B\otimes N\to C\otimes N\to 0\]
	but then in particular $0\to A\otimes N\to B\otimes N\to C\otimes N\to 0$ is exact, so $-\otimes N$ is exact whence $N$ is flat.

	Clearly (c)$\Rightarrow$(d), so it remains to see that (d) implies any of (a),(b), or (c).

\end{sol}

\begin{prob}
	Let $(R, \frakm, k)$ be a commutative local ring and $M$ be a finitely generated $R$-module. Prove that the following are equivalent: 
	\begin{enumerate}
		\item $M$ is free
		\item $M$ is flat
		\item The map $\frakm \otimes_R M \to R \otimes_R M =M$ induced by the embedding $\frakm \subset R$ is injective
		\item $\Tor_1(k,M)=0$.
	\end{enumerate}
\end{prob}
\begin{sol}
	(a)$\Leftrightarrow$(b): First let $M$ be free (and finitely generated). Then $M\cong R^k$ for some finite $k$. 
	Then take an injection
\end{sol}

\begin{prob}
	Let $R$ be a Frobenius algebra over a field $k$. Show that the global dimension of $R$ is either zero or infinity.
\end{prob}
\begin{sol}
	Assume that $\gldim(R)=n<\infty$. Then for any $R$-module $M$ both $\projdim_R(M)$ and $\injdim_R(M)$
	are less than or equal to $n$. Take minimal projective and injective resolutions $P_\bullet\xrightarrow{\varepsilon} M$ and $M\xrightarrow{\eta} I_\bullet$
	where in particular we have $I_k=P_k=0$ for all $k>n.$ But then consider the diagram 
	\begin{center}
		\begin{tikzcd}
			P_n\ar[r] & \cdots\ar[r] & P_1\ar[r] & P_0\ar[dr,"\varepsilon"]\ar[rr,"\eta\circ\varepsilon"] & & I_0\ar[r] & \cdots\ar[r] & I_n\ar[r] & 0\\
			& & & & M\ar[ur,"\eta"] & & & &
		\end{tikzcd}
	\end{center}
	where, since $\varepsilon$ is epi and $\eta$ is mono, the top row is still exact. Therefore since $R$ is Frobenius, 
	the $I_k$ are all projective as well, so this is a projective resolution of $I_n$. Then $\Ext^i_R(I_n,N)$ (for any $N$)
	is zero everywhere except (possibly) the zeroth position.

	Here we consider two cases. Notice that $n=0$ if and only if $M$ is projective and injective. In this case we are done since $\projdim_R(M)=\injdim_R(M)=0$. If this is not the case, 
	$P_1$ and $I_1$ are both not trivial. Then we can continue by noticing that the homology at $\Hom_R(P_1,N)$ is unaffected 
	by this splicing of sequences. Observe 
	\[\Ext_R^{n+1}(I_n,N)=\Ext_R^1(M,N)=0\]
	which implies that $M$ is projective (since $N$ was arbitrary) and injective. But then all $R$-modules are projective and injective,
	so $\gldim(R)=0$.
\end{sol}

\begin{prob}
	Let $k$ be a field of positive characteristic $p$. 
	\begin{enumerate} 
		\item Show that the group algebra $kG$ for a finite group $G$ is Frobenius (in fact, symmetric). 
		\item Show that the restricted enveloping algebra $\fraku(\gl_n)$ is Frobenius. 
	\end{enumerate} 
\end{prob}
\begin{sol}

\end{sol}

\begin{prob}
	Let $\underline{x} = (x_1, x_2,\ldots, x_n)$ be a sequence of 
	elements in a commutative ring $R$, and let $K(\underline{x})=K(x_1) \otimes \ldots \otimes K(x_n)$ 
	be the corresponding Koszul  complex. 

	\begin{enumerate}  
		\item Prove the explicit formula for the differential in $K(\underline{x})$. 
		\item Show that $K(\underline{x})$ 
		is a graded commutative DGA and, moreover, that $K(\underline{x}) \simeq \Lambda^*(V)$ where $V = \bigoplus Rx_i$.
	\end{enumerate}
\end{prob}
\begin{sol}

\end{sol}

\end{document}
