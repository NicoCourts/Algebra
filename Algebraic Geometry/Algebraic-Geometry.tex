\documentclass[12pt]{article}

\usepackage{setspace}

\usepackage{amsmath, graphicx, color, fancyhdr, tikz-cd, mdframed, enumitem, framed, adjustbox, bbm, upgreek, xcolor, hyperref, manfnt, wasysym}
\usepackage[framed,thmmarks]{ntheorem}
\usepackage[style=alphabetic, bibencoding=utf8]{biblatex}
%Set the bibliography file
\bibliography{sources}

\usepackage[T1]{fontenc}
\usepackage[urw-garamond]{mathdesign}
\usepackage{garamondx}

%Replacement for the old geometry package
\usepackage{fullpage}

%Input my definitions
%set up theorem/definition/etc envs
%Problems will be created using their own counter and style
\theoremstyle{break}
\theoreminframepreskip{0pt}
\theoreminframepostskip{0pt}
\newframedtheorem{prob}{Problem}[section]

%solution template
\theoremstyle{nonumberbreak}
\theoremindent0.5cm
\theorembodyfont{\upshape}
\theoremseparator{:}
\theoremsymbol{\ensuremath\spadesuit}
\newtheorem{sol}{Solution}

%Theorems
\definecolor{thmcol}{RGB}{120,100,50}
\theoremstyle{changebreak}
\theoremseparator{}
\theoremsymbol{}
\theoremindent0.5cm
\theoremheaderfont{\color{thmcol}\bfseries} 
\newtheorem{thm}{Theorem}[subsection]

%Lemmas and Corollaries
\theoremheaderfont{\bfseries}
\newtheorem{lem}[thm]{Lemma}
\newtheorem{cor}[thm]{Corollary}
\newtheorem{prop}[thm]{Proposition}

%Create a new env that references a theorem and creates a 'primed' version
%Note this can be used recursively to get double, triple, etc primes
\newenvironment{thm-prime}[1]
  {\renewcommand{\thethm}{\ref{#1}$'$}%
   \addtocounter{thm}{-1}%
   \begin{thm}}
  {\end{thm}}

\setlength\fboxsep{15pt}

%Example
\theoremstyle{break}
\def\theoremframecommand{\colorbox[rgb]{0.9,0.9,0.9}}
\newshadedtheorem{ex}{Example}[section]

%Man, that's really good! Let's use the same thing for definitons.
\newenvironment{def-prime}[1]
  {\renewcommand{\thethm}{\ref{#1}$'$}%
   \addtocounter{thm}{-1}%
   \begin{def}}
  {\end{def}}

%proofs
\theoremstyle{nonumberbreak}
\theoremindent0.5cm
\theoremheaderfont{\sc}
\theoremseparator{}
\theoremsymbol{\ensuremath\spadesuit}
\newtheorem{prf}{Proof}

\theoremstyle{nonumberplain}
\theoremseparator{:}
\theoremsymbol{}
\newtheorem{conj}{Conjecture}

%remarks
\theoremstyle{change}
\theoremindent0.5cm
\theoremheaderfont{\sc}
\theoremseparator{:}
\theoremsymbol{}
\newtheorem{rmk}[thm]{Remark}

%Put page breaks before each part
\let\oldpart\part%
\renewcommand{\part}{\clearpage\oldpart}%

% Blackboard letters
\newcommand*{\bbA}{\mathbb{A}}
\newcommand*{\bbB}{\mathbb{B}}
\newcommand*{\bbC}{\mathbb{C}}
\newcommand*{\bbD}{\mathbb{D}}
\newcommand*{\bbE}{\mathbb{E}}
\newcommand*{\bbF}{\mathbb{F}}
\newcommand*{\bbG}{\mathbb{G}}
\newcommand*{\bbH}{\mathbb{H}}
\newcommand*{\bbI}{\mathbb{I}}
\newcommand*{\bbJ}{\mathbb{J}}
\newcommand*{\bbK}{\mathbb{K}}
\newcommand*{\bbL}{\mathbb{L}}
\newcommand*{\bbM}{\mathbb{M}}
\newcommand*{\bbN}{\mathbb{N}}
\newcommand*{\bbO}{\mathbb{O}}
\newcommand*{\bbP}{\mathbb{P}}
\newcommand*{\bbQ}{\mathbb{Q}}
\newcommand*{\bbR}{\mathbb{R}}
\newcommand*{\bbS}{\mathbb{S}}
\newcommand*{\bbT}{\mathbb{T}}
\newcommand*{\bbU}{\mathbb{U}}
\newcommand*{\bbV}{\mathbb{V}}
\newcommand*{\bbW}{\mathbb{W}}
\newcommand*{\bbX}{\mathbb{X}}
\newcommand*{\bbY}{\mathbb{Y}}
\newcommand*{\bbZ}{\mathbb{Z}}
%Fraktur letters
\newcommand*{\frakA}{\mathfrak{A}}
\newcommand*{\frakB}{\mathfrak{B}}
\newcommand*{\frakC}{\mathfrak{C}}
\newcommand*{\frakD}{\mathfrak{D}}
\newcommand*{\frakE}{\mathfrak{E}}
\newcommand*{\frakF}{\mathfrak{F}}
\newcommand*{\frakG}{\mathfrak{G}}
\newcommand*{\frakH}{\mathfrak{H}}
\newcommand*{\frakI}{\mathfrak{I}}
\newcommand*{\frakJ}{\mathfrak{J}}
\newcommand*{\frakK}{\mathfrak{K}}
\newcommand*{\frakL}{\mathfrak{L}}
\newcommand*{\frakM}{\mathfrak{M}}
\newcommand*{\frakN}{\mathfrak{N}}
\newcommand*{\frakO}{\mathfrak{O}}
\newcommand*{\frakP}{\mathfrak{P}}
\newcommand*{\frakQ}{\mathfrak{Q}}
\newcommand*{\frakR}{\mathfrak{R}}
\newcommand*{\frakS}{\mathfrak{S}}
\newcommand*{\frakT}{\mathfrak{T}}
\newcommand*{\frakU}{\mathfrak{U}}
\newcommand*{\frakV}{\mathfrak{V}}
\newcommand*{\frakW}{\mathfrak{W}}
\newcommand*{\frakX}{\mathfrak{X}}
\newcommand*{\frakY}{\mathfrak{Y}}
\newcommand*{\frakZ}{\mathfrak{Z}}
\newcommand*{\fraka}{\mathfrak{a}}
\newcommand*{\frakb}{\mathfrak{b}}
\newcommand*{\frakc}{\mathfrak{c}}
\newcommand*{\frakd}{\mathfrak{d}}
\newcommand*{\frake}{\mathfrak{e}}
\newcommand*{\frakf}{\mathfrak{f}}
\newcommand*{\frakg}{\mathfrak{g}}
\newcommand*{\frakh}{\mathfrak{h}}
\newcommand*{\fraki}{\mathfrak{i}}
\newcommand*{\frakj}{\mathfrak{j}}
\newcommand*{\frakk}{\mathfrak{k}}
\newcommand*{\frakl}{\mathfrak{l}}
\newcommand*{\frakm}{\mathfrak{m}}
\newcommand*{\frakn}{\mathfrak{n}}
\newcommand*{\frako}{\mathfrak{o}}
\newcommand*{\frakp}{\mathfrak{p}}
\newcommand*{\frakq}{\mathfrak{q}}
\newcommand*{\frakr}{\mathfrak{r}}
\newcommand*{\fraks}{\mathfrak{s}}
\newcommand*{\frakt}{\mathfrak{t}}
\newcommand*{\fraku}{\mathfrak{u}}
\newcommand*{\frakv}{\mathfrak{v}}
\newcommand*{\frakw}{\mathfrak{w}}
\newcommand*{\frakx}{\mathfrak{x}}
\newcommand*{\fraky}{\mathfrak{y}}
\newcommand*{\frakz}{\mathfrak{z}}
% Caligraphic letters
\newcommand*{\calA}{\mathcal{A}}
\newcommand*{\calB}{\mathcal{B}}
\newcommand*{\calC}{\mathcal{C}}
\newcommand*{\calD}{\mathcal{D}}
\newcommand*{\calE}{\mathcal{E}}
\newcommand*{\calF}{\mathcal{F}}
\newcommand*{\calG}{\mathcal{G}}
\newcommand*{\calH}{\mathcal{H}}
\newcommand*{\calI}{\mathcal{I}}
\newcommand*{\calJ}{\mathcal{J}}
\newcommand*{\calK}{\mathcal{K}}
\newcommand*{\calL}{\mathcal{L}}
\newcommand*{\calM}{\mathcal{M}}
\newcommand*{\calN}{\mathcal{N}}
\newcommand*{\calO}{\mathcal{O}}
\newcommand*{\calP}{\mathcal{P}}
\newcommand*{\calQ}{\mathcal{Q}}
\newcommand*{\calR}{\mathcal{R}}
\newcommand*{\calS}{\mathcal{S}}
\newcommand*{\calT}{\mathcal{T}}
\newcommand*{\calU}{\mathcal{U}}
\newcommand*{\calV}{\mathcal{V}}
\newcommand*{\calW}{\mathcal{W}}
\newcommand*{\calX}{\mathcal{X}}
\newcommand*{\calY}{\mathcal{Y}}
\newcommand*{\calZ}{\mathcal{Z}}
% Script Letters
\newcommand*{\scrA}{\mathscr{A}}
\newcommand*{\scrB}{\mathscr{B}}
\newcommand*{\scrC}{\mathscr{C}}
\newcommand*{\scrD}{\mathscr{D}}
\newcommand*{\scrE}{\mathscr{E}}
\newcommand*{\scrF}{\mathscr{F}}
\newcommand*{\scrG}{\mathscr{G}}
\newcommand*{\scrH}{\mathscr{H}}
\newcommand*{\scrI}{\mathscr{I}}
\newcommand*{\scrJ}{\mathscr{J}}
\newcommand*{\scrK}{\mathscr{K}}
\newcommand*{\scrL}{\mathscr{L}}
\newcommand*{\scrM}{\mathscr{M}}
\newcommand*{\scrN}{\mathscr{N}}
\newcommand*{\scrO}{\mathscr{O}}
\newcommand*{\scrP}{\mathscr{P}}
\newcommand*{\scrQ}{\mathscr{Q}}
\newcommand*{\scrR}{\mathscr{R}}
\newcommand*{\scrS}{\mathscr{S}}
\newcommand*{\scrT}{\mathscr{T}}
\newcommand*{\scrU}{\mathscr{U}}
\newcommand*{\scrV}{\mathscr{V}}
\newcommand*{\scrW}{\mathscr{W}}
\newcommand*{\scrX}{\mathscr{X}}
\newcommand*{\scrY}{\mathscr{Y}}
\newcommand*{\scrZ}{\mathscr{Z}}

%Section break
\newcommand*{\brk}{
\rule{2in}{.1pt}
}

%General purpose stuff
\DeclareMathOperator{\Aut}{Aut}
\DeclareMathOperator{\ch}{char}
\DeclareMathOperator{\rank}{rank}
\DeclareMathOperator{\End}{End}
\let\Im\relax
\DeclareMathOperator{\Im}{Im}

%Category Theory
\DeclareMathOperator{\Hom}{Hom}
\let\hom\relax
\DeclareMathOperator{\hom}{hom}
\DeclareMathOperator{\id}{id}
\DeclareMathOperator{\coker}{coker}
\DeclareMathOperator{\colim}{colim}
\DeclareMathOperator{\invlim}{\lim_{\leftarrow}}
\DeclareMathOperator{\dirlim}{\lim_{\rightarrow}}

%Commutative Algebra
\DeclareMathOperator{\gldim}{gldim}
\DeclareMathOperator{\projdim}{projdim}
\DeclareMathOperator{\injdim}{injdim}
\DeclareMathOperator{\findim}{findim}
\DeclareMathOperator{\flatdim}{flatdim}
\DeclareMathOperator{\depth}{depth}

%Common Categories
%\newcommand*{\modR}{\mathbf{mod}\text{-}R}
%\newcommand*{\Rmod}{R\text{-}\mathbf{mod}}
\newcommand{\rmod}[1]{\mathbf{mod}\text{-}#1}
\newcommand{\lmod}[1]{#1\text{-}\mathbf{mod}}
\DeclareMathOperator{\Vectk}{\mathbf{Vect}_k}
\DeclareMathOperator{\Ch}{\mathbf{Ch}}
\newcommand*{\Ab}{\mathbf{Ab}}
\newcommand*{\Grp}{\mathbf{Grp}}
\newcommand*{\Alg}{\mathbf{Alg}_k}
\newcommand*{\Ring}{\mathbf{Ring}}
\newcommand*{\K}{\mathbf{K}}
\newcommand*{\D}{\mathbf{D}}
\newcommand*{\Db}{\mathbf{D}^b}
\newcommand*{\Dpos}{\mathbf{D}^+}
\newcommand*{\Dneg}{\mathbf{D}^-}
\newcommand*{\Dbperf}{\mathbf{D}^b_{\text{perf}}}
\newcommand*{\Dsing}{\mathbf{D}_{sing}}
\newcommand{\CRing}{\mathbf{CRing}}
\DeclareMathOperator{\stmod}{\mathbf{stmod}}
\DeclareMathOperator{\StMod}{\mathbf{StMod}}
\DeclareMathOperator{\sHom}{\underline{Hom}}

%Homological algebra
\DeclareMathOperator{\cone}{cone}
\DeclareMathOperator{\HH}{HH}
\DeclareMathOperator{\Der}{Der}
\DeclareMathOperator{\Ext}{Ext}
\DeclareMathOperator{\Tor}{Tor}

%Lie algebras
\DeclareMathOperator{\ad}{ad}
\newcommand*{\gl}{\mathfrak{gl}}
\let\sl\relax
\newcommand*{\sl}{\mathfrak{sl}}
\let\sp\relax
\newcommand*{\sp}{\mathfrak{sp}}
\newcommand*{\so}{\mathfrak{so}}

% Hacks and Tweaks
% Enumerate will automatically use letters (e.g. part a,b,c,...)
\setenumerate[0]{label=(\alph*)}
% Always use wide tildes
\let\tilde\relax
\newcommand*{\tilde}[1]{\widetilde{#1}}
%raise that Chi!
\DeclareRobustCommand{\Chi}{{\mathpalette\irchi\relax}}
\newcommand{\irchi}[2]{\raisebox{\depth}{$#1\chi$}} 



%Shade definitions
\theoremindent0cm
\theoremheaderfont{\normalfont\bfseries} 
\def\theoremframecommand{\colorbox[rgb]{0.9,1,.8}}
\newshadedtheorem{defn}[thm]{Definition}

%%%%%%%%%%%%%%%%%%%%%%%%%%%%%%%%%%%%%%%%%%%%%%%%%%%%%%%%%%%%%%%%%%%%%%
%%%%%%%%%%%%%%%%%%%%%%% Customize Below %%%%%%%%%%%%%%%%%%%%%%%%%%%%%%
%%%%%%%%%%%%%%%%%%%%%%%%%%%%%%%%%%%%%%%%%%%%%%%%%%%%%%%%%%%%%%%%%%%%%%

%header stuff
\setlength{\headsep}{24pt}  % space between header and text
\pagestyle{fancy}     % set pagestyle for document
\lhead{Algebraic Geometry} % put text in header (left side)
\rhead{Notes by Nico Courts} % put text in header (right side)
\cfoot{\itshape p. \thepage}
\setlength{\headheight}{15pt}
%\allowdisplaybreaks

% Document-Specific Macros
\newcommand{\SpcFun}{\mathbf{SpcFun}}

\begin{document}
%make the title page
\title{Algebraic Geometry\vspace{-1ex}}
\author{A course by Max Lieblich\\
Notes by Nico Courts}
\date{Autumn 2019/ Winter and Spring 2020}
\maketitle

\begin{abstract}
	A three-quarter sequence covering the basic theory of affine and projective 
	varieties, rings of functions, the Hilbert Nullstellensatz, localization, and 
	dimension; the theory of algebraic curves, divisors, cohomology, genus, and the 
	Riemann-Roch theorem; and related topics. 
\end{abstract}

\part{Quarter 1: Spaces with functions}
\section{September 25, 2019}
The first thing that one asks is ``what is geometry?'' One needs to be able to answer this 
question before they define AG. One idea is that geometry is topology + structure.

\subsection{What is Geometry?}

\begin{ex}
	Exotic differentiable structures on a sphere. There are many different smooth structures,
	all of which are independent of the topology,

	$S^1\times S^1$ has infinitely many complex structures (remember the parallelograms)!
\end{ex}

How to you go about defining the geometry of a thing? One idea from manifolds: charts. These 
describe the local models and the interesting part is how this comes together to a whole space.

There is another idea to capture the ``local'' model of geometry that underlies modern algebraic geometry:
consider the map $\varphi:W\to W'\in\bbC\bbP^n$ and then say that this map is $C^\infty$ if and only if its coordinate
functions are. But the coordinate functions are problematic, so we can replace it with the following idea:

$\phi:W\to W'$ is $C\infty$ if and only if for all $C^\infty$ functions $f:W'\to \bbR$,
the composition
\[\varphi^\ast f=f\circ \varphi:W\to \bbR\]
is $C^\infty$.

To capture the manifold structure on $M$, it is equivalent to know the set of $C^\infty$ fucnctions $U\to \bbR$ for every open $U\subseteq M$.

\subsection{The Big Idea}
So then the idea we are talking away here is that \textit{geometry is in the functions} that exist on a particular space!

Fix a field $k$. 
\begin{defn}
	A \textbf{space with functions} is a topological space $X$ along with a collection (a $k$-algebra!) $\calO(U)$ of maps $U\to k$ for each open 
	$U\subseteq X$. 

	$\calO(U)$ are called \textbf{regular functions} and must satisfy:
	\begin{itemize}
		\item Given an open cover $U_\alpha$ of $U$, a function is regular if and only if its restrictions to each element of the cover is regular.
		\item If $f:U\to k$ is regular, then $D(f)=\{u\in U|f(u)\ne 0\}$ is an open set and $\frac{1}{f}\in\calO(D(f))$.
	\end{itemize}
\end{defn}

For the next time, try to think of as many examples of this as you can. Next time will be a mind blowing example of a variety.

\section{September 27, 2019}
\begin{prob}
	Do all the exercises in Kempf chapter 1!
\end{prob}

For now we assume that $k$ is algebraically closed.

\subsection{Examples of spaces with functions}
There were lots of suggestions, but here are a couple:
\begin{ex}
	Let $X=\bbS^2$ and let $\calO_X^{cts}$ be the continuous $\bbC$-valued functions. Alternatively
	we could consider a different sheaf $\calO_X^{an}$, the holomorphic functions. Or we could 
	consider $\calO_X^\infty$, the $C^\infty$ functions (under some smooth structure).
\end{ex}

\begin{defn}
	A \textbf{morphism} of spaces with functions between $(X,\calO_X)$ and $(Y,\calO_Y)$ is a continuous map 
	$\varphi:X\to Y$ such that for all $U\subseteq Y$ open and $f\in\calO_Y(U)$, the function 
	\[\phi^\ast f=f\circ\phi|_{\phi^{-1}(U)}:\phi^{-1}(U)\to k\in \calO_X(\phi^{-1}(U))\]

	In other words, a morphism of spaces with functions is a map of spaces that \textit{respects the regular functions.}
\end{defn}
\begin{ex}
	Let $X,Y$ be topological spaces and let $\calO_X$ and $\calO_Y$ be the continuous functions. Then morphisms are just continuous maps.
\end{ex}
\begin{ex}
	When $X$ and $Y$ are manifolds and $\calO_\bullet$ are complex-valued functions, then the maorphisms are maps 
	of manifolds.
\end{ex}

So now we return to the examples we saw before: $(\bbS^2,\calO^\infty)$, $(\bbS^2,\calO^{cts})$, and $(\bbS^2,\calO^{an})$.
A natural question to ask is when we have morphisms between these spaces to see if there exist ones that are the identity on $\bbS^2$.

Consider the identity map from the continuous to the analytic functions. Then take any map $f\in\calO^{an}$ and consider that 
\[f=f\circ id_{id^{-1}(U)}:U\to k\in\calO^{cts}(U)\]
and there is no map in the other direction.
\begin{rmk}
	Notice that since we are pulling functions back, the maps go in the opposite direction as you may think at first.
\end{rmk}

We can also talk about \textbf{open subspaces}. If $V\subseteq X$ is an open subset, we can let the induced space with functions 
be $(V,\calO_V)$ where if $U\subseteq V$ then $\calO_V(U):=\calO_X(U)$.

\subsection{Varieties}
\begin{defn}
	An \textbf{affine $k$-variety} is a space with functions $(Y,\calO_Y)$ such that for every space with functions $(X,\calO_X)$,
	the natural map 
	\[\Hom((X,\calO_X),(Y,\calO_Y))\to\Hom_{\Alg}(\calO_Y(Y),\calO_X(X))\]
	is a bijection and furthermore $\calO_Y(Y)=:k[Y]$ is a finitely generated $k$-algebra.
\end{defn}
\begin{rmk}
	The idea here is that the algebra maps (on the right) are precisely the same as the geometry maps (on the left).
	Algebraic geometry, baby.
\end{rmk}
So then this leads to a very simple (loose) definition:
\begin{defn}
	A \textbf{variety} is something that is covered by affine varieties.
\end{defn}

\begin{ex}
	$\bbA^1=k$. Give this space the cofinite topology. Then if we have $U=k\setminus\{x_1,\dots,x_n\}\subset \bbA^1$,
	\[\calO_{\bbA^1}(U)=\{f(t)\in k(t)| \text{poles are in }\{x_i\}\}\]
\end{ex}
\begin{prob}
	Show that $\bbA^1$ is an affine variety!
\end{prob}
\begin{rmk}
	Notice that this staetment is equivalent to saying that any morphism of spaces with functions gives us a regular map $X\to k$.
\end{rmk}

\section{September 30th, 2019}
One question that was asked: if we have fixed the underlying topological space in a space with functions, must there be a morphism 
between them somehow? Might there instead be a common cover of the two?

\begin{ex}
	Let $k$ be a field with some topology on it such that every point is closed (you could do the discrete topology).
	Let $\tilde\O(U)$ be the continuous funcitons $U\to k$. In other words, these functions are locally constant.

	Locally constant functions behave nicely under restrictions to opens, of course. The other axioms are also great.

	Have we really found an initial object in our category? This would be enough to establish a ``tent'' (as in 
	localization of categories). Try this out and see what happens!
\end{ex}

\subsection{The question of affine space}
Recall the question about whether $\A 1$ is an affine variety. The idea here is that $\phi:X\to k$ is a morphism of 
spaces with functions if and only if it is regular (that is, in $\O_{\A 1}$).

One direction is tautological (a morphism to $\A1$ has a polynomial underlying it), so let $\phi$ be regular. Then to see that $\phi$ is continuous 
can be checked by pulling back all closed sets. The important observation is that $D(f-a)=X\setminus\phi^{-1}(a)$, which is closed (an axiom for spaces with functions).

The last thing to check is where $\phi$ pulls back regular functions to regular functions. This relies on the facts that $\O_X$ is a $k$-algebra and that 
$\phi(x)-b_j$ is regular on $U$ when $b_j\notin U$.

\subsection{Algebra maps}
Notice that since we have a condition that $\O_X(X)$ must be finitely generated as a $k$-algebra, this means that 
\[\Hom(X,Y)=\Hom_k(\O_Y(Y),\O_X(X))=\Hom_k(k[x_1,\dots,x_n]/(f_1,\dots,f_m),O_X(X))\]
and
\[\Hom(X,Y)=\{(\gamma_1,\dots,\gamma_n)\in(\O_X(X))^n:f_j(\gamma_i)=0,\forall j=1,\dots,m\}\]
In other words, we are looking at maps that factor through $Z$:
\begin{center}
	\begin{tikzcd}
		(\gamma_1,\dots,\gamma_n):X\ar[r]\ar[dr] & k^n\\
		& Z=Z(f_i)\ar[u,hook]
	\end{tikzcd}
\end{center}
Now what we want to say is that $Y=Z$. That is, \textit{affine varieties are closed subsets of affine spaces.}

Now this is all good, but the problem is that we had to \textit{choose} a presentation of $\O_Y(Y)$ to get this picture.
of course we want something more canonical! We will see in this class (and in Kempf) that this can be done.

\section{October 2, 2019}
\subsection{Questions without (complete) answers}
\subsubsection{Morphisms and stuff}
A question to get things started for the day. Let $X$ and $Y$ be spaces with functions and let $Y$ be an affine variety and let $f:Y\to X$ be a map of sets 
(but with no further assumption on $f$). This naturally induces amap from $\calO_X(X)$ to the functions $\Hom_{\Set}(Y,k)$ (which 
clearly contains the regular functions on $Y$).

Further assume that there exists a $\gamma:\O_X(X)\to\O_Y(Y)$. We know that since $Y$ is affine, $\gamma$ corresponds to a morphism $\varphi:Y\to X$.
Then the question is: when does $f=\varphi$? We've already answered this question for $\A1$, notice.
\subsubsection{Algebraic closure}
Where do we use algebraic closure of the base field? It has been swept under the rug a bit, but consider the function
\[\frac{1}{x^2+1}:\bbR\to \bbR.\]

This certainly seems like it should be a regular function (e.g. it is rational and defined everywhere on $\bbR$) but this 
conflicts with the idea that we want to identify $\O_{\A1}(\A1)=k[t]$, but that is clearly not the case here. Think about this.
\subsubsection{Yet another}\label{subsubsec:third}
Consider the set $R$ of all continuous maps $k\to k$ under the cofinite topology. Someone asked if $R$ is a $k$-algebra. The answer is 
a bit convoluted, but the short answer is no. Specifically if we are using the product topology on $k\times k$, the addition map isn't continuous! 
This also points to the question of what topology is the correct one to use on these things.

\subsection{Back to affine varieties}
Recall that we constructed a (highly-non-canonical) picutre of how any affine variety arises as a closed subset of some affine space $k^n$.

We want to remove this dependence on presentation, however, and that is what we are working toward.

\subsubsection{Affine Space}
Now we focus in on $\A n=k^n$. We really want that the projection functions $x_i:k^n\to k$ should be regular. But since we want this (eventually)
to form a $k$-algebra, we want that each $f\in[x_1,\dots,x_n]$ should be regular!

The axioms of a space with functions tells us that the \textbf{vanishing locus}
\[Z(f)=\{a|f(a)=0\}\subseteq k^n\]
and furthermore $Z(S)$ should be closed for all $S\subseteq k[x_1,\dots,x_n]$. This leads us to a definition:
\begin{defn}
	A subset $Z\subseteq k^n$ is \textbf{Zariski-closed} if there exists an $S\subseteq k[x_1,\dots,x_n]$
	such that $Z=Z(S)$.
\end{defn}
\begin{lem}
	The Zariski closed sets are the closed sets of a topology (called the \textbf{Zariski Topology}).
\end{lem}
\begin{prf}
	Just do it. Nike. \checked
\end{prf}
\begin{rmk}
	Notice that here the set $\{(a,-a)\}\subseteq k^2$ (the pullback of zero under the addition map) is Zariski closed!
	This fixes the problem we were running into in the third question (sec.~\ref{subsubsec:third}) above.
\end{rmk}
Now since $Z(S)=Z(I_S)$ where $I_S$ is the ideal generated by $S$, it is enough to consider vanishing loci of ideals. Furthermore we have the 
map that extracts the ideal of functions that vanish on a set $Z\subseteq k^n$. There are a ton of great identiies you can prove here. Go to your 
favorite algebra book (e.g. Dummit \& Foote) to see them.

\subsubsection{Functions}
What about functions on these spaces? If we take $f\in k[x_1,\dots,x_n]$ these seem like they should be regular functions $k^n\to k$.
\begin{thm}[(Weak) Nullstellensatz]
	Say $k=\bar k$. Then every maximal ideal $\frakm\lhd k[x_1,\dots,x_n]$ has the form $(x_1-a_1,\dots,x_n-a_n)$.
\end{thm}
\begin{rmk}
	Equivalently, it is the kernel of a $k$-algebra morphis $k[x_1,\dots,x_n]\to k$.
\end{rmk}
\begin{cor}[Nullstellensatz]\label{nullstellensatz}
	Let $J$ be an ideal of $k[x_1,\dots,x_n]$. Then $I(Z(J))=\sqrt{J}$.
\end{cor}
\begin{prf}
	\textit{Notice this only works when $k$ is uncountable!} Suppose that $\frakm$ is a maximal ideal with residue field 
	$L=k[x_i]/\frakm$. This gives us a surjection of $k[x_1,\dots,x_n]\to L$. Thus $\dim_k L$ is countable!

	But $\dim_k k(t)$ is uncountable! The proof here is that the $\frac{1}{t-\lambda}$ for $\lambda\in k$ is a linearly-independent collection.
	So then $L/k$ is algebraic, and since $k=\bar k$ $L=k$.
\end{prf}

\section{October 4, 2019}
Today we are going to be talking a bit more about the existence of affine varieties. Max talked a bit about the philosophy of work in this course:
he made this extended metaphor concerning butterflies but the take-away is to take learning onto ourselves. :)

\subsection{Questions from last time}
\subsubsection{Maps and elements}
In the book we did this silly thing where we defined $\Spec A\eqdef\Hom_{\Alg}(A,k)$ and then idendified 
$A$ with $k[\Spec A]$ by $a(f)=f(a)$. This seems a bit silly at first, but it may have something to do with the fact that 
we are looking for a natural way to construct affine varieties without having to choose a presentation. We will hopefully see 
something about this by the end of the day. 

\subsection{Back to the Nullstellensatz}
Recall that we defined the operators $Z$ and $I$ that ``do the work'' of the Nullstellensatz. We then wrote (cor.~\ref{nullstellensatz}) $I(Z(J))=\sqrt{J}$. The idea is that 
this will gives us the function structure on an affine variety.
\begin{prf}[Cor.~\ref{nullstellensatz}]
	One way is not too hard. For the more difficult direction:
	Let $g\in I(Z(J))$. Then $Z(J)\subseteq Z(g)$. Now notice taht $D(g)$ can be naturally identified 
	with $\Spec k[x_i][1/g]$. Then consider
	\[J'=Jk[x_i][1/g]\]
	and the key realization is that $J'$ cannot be contained in any maximal ideal. The idea is that you can work by contradiction:
	this implies that $J$ is contained in an element of $D(g)$, but it isn't!

	Thus $J'=(1)$. So we can write $1=\frac{f}{g^N}$. Thus $g^k(f-g^N)=0$ in $k[x_i]$ and since $g$ isn't nilpotent, $f=g^N$.
\end{prf}
\begin{cor}
	There is a lattice anti-isomorphism between the radiacal ideals in $k[x_i]$ and Zariski-closed 
	subsets $Z\subseteq k^n$ via the maps $J\to Z(J)$ and $Z\mapsto I(Z)$.
\end{cor}
\begin{cor}
	For any ideal $J\subseteq k[x_i]$, 
	\[\sqrt{J}=\bigcap_{\text{maximal } \frakm\supset J}\frakm\]
\end{cor}
\begin{rmk}
	``The functions that vanish at the zero locus of $J$ are precisely those that vanish at all the points of $J$''.
\end{rmk}
\begin{cor}
	$D(g)\subseteq k^n$. Then the map 
	\[k[x_i][1/g]\to \Hom(D(g),k)\]
	via the map 
	\[\frac{f}{g^N}\mapsto \left(x\mapsto \frac{f(x)}{g(x)^N}\right)\]
	is injective.
\end{cor}

\subsection{Affine space}
Let's define $\A n\eqdef k^n$ with the Zariski topology. Let 
\[\calO_{\A n}(U)=\{f\in k(x_1,\dots,x_n)|\operatorname{poles}(f)\subseteq\A n\setminus U\}\subseteq\Hom(U,k)\]
Then, for example,
\[\O_{\A n}(D(g))=k[x_1,\dots,x_n][1/g].\]
\begin{prop}
	$\A n$ is an affine variety.
\end{prop}
\begin{prf}
	$\phi:X\to \A n$ gives us maps $\phi_1,\dots,\phi_n:X\to k$. Then that $\A n$ is affine relies on the fact:
	$\phi$ is a morphis if and only if the $\phi_i$ are regular. One direction is not too bad since coordinate functions 
	are regular by the axioms of morphisms. The other direction needs to be completed! DO IT!
\end{prf}

\section{October 7th, 2019}
\subsection{Questions/Discussions}
\subsubsection{Initial and final objects}
We asked before whether the space with functions $(X,\O^{\text{loc. constant}})$ is an initial or terminal 
object. Adam asserts that it is a final object in the category of spaces with functions where the underlying space is $X$ (I believe this 
is sheaves over $X$). 

Notice we can't use all continus maps where $k$ has the discrete topology, since $0\in k$ is not closed. 

\subsubsection{Subalgebras and rings of functions}
Assume that $f:B\to A$. This gives us a nice map $\tilde f:\Spec A\to \Spec B$. One question we may have is 
``if $f$ is injective, does this imply that $\tilde f$ is surjective?''

Consider an example: Say $B=k[t]\hookrightarrow k[s,t]=A$. Then any function on $A$ looks like $(s-a,t-b)$
and the map induced on functions is just projection, so this gives us the map $(t-b)$, which is all the maps on $B$.

Another example: consider the map $\bbC[t]\to\bbC[s]$ sending $t\to s^2$. Then this induces a map $z\to z^2$ from $\bbC\to \bbC$ (why?)
which is again surjective.

Next consider $k[x,u]/(xy-1)=k[x][1/x]=A$, which is a hyperbola over $\bbR$. Then the localization map $B=k[x]\hookrightarrow A$
induces a map that is basically the identity everywhere \textit{except zero}. So it is \textbf{not surjective.}

\brk

Some properties to notice: examples one and two are \textit{flat} extensions. The second is a \textbf{finite} extension. The 
third is neither. We will investigate what is going on further later on.

One idea: consider whether the map $X=\Spec B\to \A 2\setminus(\A1\setminus\{0\})$ exists. One of the things that he keeps questioning is 
whether the target spaces is open or closed as a subset of affine space. (Note that a map is proper if it sends closed to closed).

\subsection{Back to affine varieties}
Continuing our proof/discussion from last time, we were considering $\varphi:X\to\A n$ which we said was a morphism iff each coordinate $\phi_i:X\to k$ is regualr.
(This basically follow since $\varphi$ is continuous and sends regular functions to regular functions.

For the regularity, consider $U\subseteq\A n$. Then $U$ admits a cover of $D(g)$, so it suffices to check where $\phi$
pulls back regular maps on $D(g)$. Of course this is $k[x_1,\dots,x_n][1/g]$! So consider the image of 
\[\frac{1}{g^N}\sum a_ix^i\]
and its image in $\phi^{-1}(D(g))=D(\phi^\ast(g))$ is 
\[\frac{\sum a_i(\phi^\ast x)^i}{(\phi^\ast g)^N}\]

\begin{rmk}
	Big idea: We started with the ``dream'': that there is a correspondence between the algebra and the geometry. This is 
	our main guiding principle, so we know we've found the ``right'' topology when we have found one that supports this dream.
	This is an answer to the question ``why is the Zariski topology not just a degenerate case?''
\end{rmk}

\subsubsection{Affine varieties in general}
Let $J\subseteq k[x_1,\dots,x_n]$ be a radical ideal. We know this corresponds (uniquely!) 
to a subset $Z\subseteq \A n$. Here we can consider $(Z,\O_Z)$ where $Z$ is a subspace of $\A n$.

Then take any closed $W\subseteq Z$, which is the intersection (by definition of the topology)
\[\cap_{i\in I}Z(f_i)\quad f_i\in h[x_i]/J\]
and then $\O_Z(D(g))=\frac{k[x_i]}{J}[1/g]$ (note we used the nullstellensatz here!).

So then we claim that $Z$ is an affine variety. To see this, consider a map $\varphi:X\to Z$ and the composition
\[X\xrightarrow{\varphi} Z\hookrightarrow \A n\]
so topologically $X$ cfactors through $Z$ if and only if $J\subseteq \O^{\A n}(\A n)$ maps to zero in $\O_X(X)$.

The takeaway here is that a morphism $X\to\A n$ factors through $Z$ topologically if and only if it factors in the categorification of 
spaces with functions.

\section{October 11, 2019}
Today we are going to talk some more about varieties. Coming up on the horizon is a discussion of 
the functor-of-points perspective and we'll talk about Yoneda.

\subsection{Questions/Discussion}
A group of students met up on Wednesday (which we skipped for Yom Kippur) and were talking about 
how to prove the statement: ``points in an affine variety are closed.''

The Nullstellensatz gives us that a point $(x_1,\dots,x_n)\in X$ correspinds to the vanishing locus of $(x_i-a_i)$. 
What if we use the definition in the book, though? If $X=\Spec A=\Hom_{\Alg_k}(A,k)$ (where $A$ is a reduced finitely generated 
$k$-algebra). But then the points are the maximal ideals $\frakm\in k[X]=A$, which are exactly the points we want!

\subsection{Back to Varieties}
Recall the definition of a variety: we are considering $(X,\O_X)$ is a space of functions. Then the idea we want is that we want to say 
that this thing is locally affine. But if you consider infinitely many copies of $\A1$ intersecting pairwise, there is 
an obvious cover by (affine) $\A1$'s. This shouldn't be affine. It doesn't embed in an affine space, for example.

So the definition we used (and the one in Kempf) is that we must have a \textit{finite} cover by affine spaces. We wanted to discuss some examples.
\begin{itemize}
	\item Clearly all affine varieties are varieties.
	\item $\P1$ is our first nontriial example. As a space it is the one-point compactification of $\A1$. The functions 
	$\O_X(U)$ are the rational functions in $t$ with poles not in $U$. You can also construct it by taking the morphism 
	\[\Gm\xrightarrow{t\mapsto 1/t}\Gm\]
	and gluing along this morphism to get a copy of $\P1$! We've already shown in Kempf that it is not affine. 
	What happens if we were to pick the identity above?! We get the line with two origins. It's a variety! But notice that 
	(under the Euclidean topology) the space isn't Hausdorff! 
\end{itemize}
\begin{prob}
	If $X$ and $Y$ are affine varieties, is $X\sqcup Y$ affine?
\end{prob}
\begin{prob}
	How can we expresse the non-Hausdorffness of the line/plane with two origins in the Zariski topology?!
\end{prob}

\subsection{Varieties Glue}
We've been throwing things around, but this is important to write down: Start with $U_i, i\in I$, where $I$ is finite 
(although we could drop finiteness if we don't care about the thing being a variety). For all $i,j$, we have open subsets 
$V_{ij}\subseteq U$ such that $V_{ii}=U_i$ and isomorphisms $\varphi_{ij}:V_{ij}\xrightarrow{\sim}V_{ji}$ of varieties such that 
\begin{enumerate}
	\item $\phi_{ii}=\id$
	\item $\varphi_{ij}(V_{ik}\cap V_{ij})=V_{ji}\cap V_{jk}$
	\item $\forall i,j,k,\, \varphi_{jk}\circ\varphi_{ij}=\varphi_{ik}$ on $V_{ij}\cap V_{ik}$
\end{enumerate}
then there exists a unique package $(X,\iota_i:U_i\hookrightarrow X)$ where $XS$ is a variety and each $\iota_i$ is an open embedding.

One idea is you can rephrase this in categorical language as the colimit of a diagram in a category. Hmmm

\section{October 14th, 2019}
We're going off-book a bit to talk about 
\subsection{Yoneda Lemma}
Recall that we said that $Y$ was affine if 
\[\Hom(X,Y)\cong \Hom_k(\O_Y(Y),\O_X(X)).\]
Now wer are going to take some time to put this into a broader context. Let $\calC$ be a category. We get naturally two functors 
\[h_a:\calC^{op}\to\Set\qquad h^a:\calC\to\Set\]
where 
\[h_a(b)=\Hom_\calC(b,a)\qquad h^a(b)=\Hom(a,b).\]

Now given a map $f:a\to a'$, we get natural transformations $f\circ -:h_a\to h_{a'}$ and $-\circ f:h^{a'}\to h^a$. Now notice that 
if $\ast$ is a one-point variety, then $h_X(\ast)=|X|$, the underlying point set of $X$. Notice that in a similar way 
$h_X(\A1)$ is something like the ``line space'' of $X$.

Notice that we then get a functor
\[h_{(-)}:\calC\to\mathbf{Func}(\calC^{op},\Set)\]
and we should \textbf{think} that a functor $\calC^{op}\to\Set$ is a \textbf{space over $\calC$}. That is, we can think of 
$\calC$ as the category of open sets of a topological space. We can say 
\[\Hom(U,V)=\left\{\begin{array}{lr}
	\varnothing, & U\not\subset V\\
	\{\varnothing\}, & \text{otherwise}.
\end{array}\right.\]

Notice the category $\mathbf{Open}(X)$ gives us a functor $\phi:\mathbf{Open}(X)^{op}\to\Set$ where we map 
an open $U\subseteq X$ to $\phi(U)$. For instance if $F$ is a vector bundle over $X$, we can define
\[\phi_F(U)\]
to be the sections over $U$ of the covering map. Cool.

So we're working with the slice category $\mathbf{Top}/X=\{X\to Y,\text{cts}\}$. Here morphisms are maps $Y\to X$ satisfying 
\begin{center}
	\begin{tikzcd}
		Y\ar[rr]\ar[rd] & &Z\ar[ld]\\
		& X &
	\end{tikzcd}
\end{center}

So now fix $F\to X$. Then the map $\phi_F(Y\to X)$ which is the collection of of diagrams of the form above ($Z=F$).

Then if we look at points $x\hookrightarrow X$, we get that $\phi_F(x\to X)$ is the fiber of $F$ over $x$. One ould hole that somehow 
you could recover all of $F$ from these maps, and that is precisely the content of 
\begin{lem}[Yoneda]
	The functor $h:\calC\to\mathbf{Func}(\calC^{op},\Set)$ is fully faithful.
\end{lem}
\begin{rmk}
	That is, 
	\[\Hom_\calC(a,a')\to\Hom_{\mathbf{Func}(\calC^{op},\Set)}(h_a,h_{a'})\]
	is a bijection.
\end{rmk}
\begin{prf}
	This proof is so tautological it is sometimes confusing to prove. We can look it up in any of our old notes or books but the idea is 
	to look at the image of the identity map.
\end{prf}
Next time we will see a bunch of examples and exercises.

\section{October 16}
We're going to finish up with with Yoneda! The milestone for the next week or so: we should have seen and digested some of the 
algebra in the early sections of chapter 2. Max will assume we'll have seen it already starting Friday(ish). We should shoot for having all the problems 
in chapter 2 done by a week from Friday (or so).

\subsection{Old Questions} 
We spoke a bit about lines glued together and the automorphisms of one of the lines that extend to the entire space. 
This necessitates talking about what the automorphisms of $\A1$ are.

We also talked about whether the coordinate axes in $\A3$ are isomorphic as varieties to the projection 
of these axes onto a plane. We discussed that this must induce an isomorphism between the coordinate algebras since they are affine.

\subsection{Back to Yoneda}
Some useful notation for encoding Yoneda: given a functor $F:\calC^{op}\to\Set$, we say that an object $a\in\calC$
\textit{represents} $F$ if $h_a\simeq F$. Then another formulation of Yoneda says that a representing object of a representable functor is unique up to isomorphism of
representing objects.
\begin{center}
	\begin{tikzcd}
		h_a\ar[r,"\sim"]\ar[rr,bend right,"\sim"] & F & h_{a'}\ar[l,"\sim",swap]\\
		& & \\
		& a\to a'&
	\end{tikzcd}
\end{center}
\begin{rmk}
	$h_a$ is called the \textit{functor of points of $a$.}
\end{rmk}
\begin{ex}
Given $f_1\dots,f_n\in k[x_i]$, consider the $k$-algebra $k[x_i]/(f_i)$. Then Yoneda gives us that $A$ is uniquely
determined by the functor 
\[h^A:\Alg_k\to\Set\]
via
\[B\mapsto \Hom(A,B)=\{(b_1,\dots,b_n)\in B^n|f_i(b_1,\dots,b_n)=0\}.\]
\end{ex}
\begin{ex}
	The functor $h_{\P n}:\Alg_k\to\Set$ is 
	\[A\mapsto \{A\twoheadrightarrow L| L\text{ invertible $A$-module}\}/\cong\]
	This is appartently very important.
\end{ex}

\subsection{Representable or not?}
\begin{itemize}
	\item $\Ga$ as a functor from $\SpcFun^{op}\to\Set$ is represented by $\A1$.
	\item $\Gm$ as a functor from $\SpcFun^{op}\to\Set$ is represented by $\A1\setminus\{0\}$.
	\item $\GL_n:\SpcFun^{op}\to\Set$ which is represented by $\Spec k[x_{ij}]_{\det}$
	\item $|\cdot|:\SpcFun\to\Set$ underlying set represented by $\ast$
	\item $\varnothing:\SpcFun^{op}\to\Set$ sending $X\mapsto\varnothing$. Not representable!
	\item $\{\varnothing\}:\SpcFun^{op}\to\Set$ sending $X\mapsto\{\varnothing\}$. Representable by $\ast$.
	\item $\mu_n:\SpcFun^{op}\to\Set$ sending $X\mapsto\{f\in\O_X(X)|f^n=1\}$ represented by $\Spec k[t](t^n-1)$
\end{itemize}

\section{October 18th, 2019}
This is something that Max just mentioned: what is $\O_X(\varnothing)?$ There is a single map here! 

\subsection{Old Questions}
Is $U=\A2\setminus (0,0)$ affine? Well first notice 
\[\O_{U}(U)\subseteq \bigcap_{f\in k[x,y]\text{ irred}}k[x,y][1/f]=k[x,y]\]
\textbf{Why is the last equality true?! We're working over a UFD.} But then of course 
$U$ can't be affine since $U\ne \A2=\Spec k[U].$ Another way to think of this is that every isomorphism of 
coordinate rings yields an isomorphism of spaces!

\subsection{Back (again) to affine varieties}
\begin{thm}
	\begin{itemize}
		\item Let $\calA\subseteq\Alg_k$ be the subcategory of finitely-generated, reduced $k$-algebras. Then $\Spec$ defines an equivalence of categories
		\[\Spec:\calA^{op}\to \mathbf{Aff}\]
		\item The functor
		\[X\mapsto\Spec \O_X(X):\SpcFun^\ast\to\mathbf{Aff}\]
		is left adjoint to the canonical inclusion $\mathbf{Aff}\subseteq\SpcFun^\ast$, where $\SpcFun^\ast$
		is the subcategory of $\SpcFun$ such that the global sections are finitely generated.

		Thus $X\to\Spec \O_X(X)$ is universal for maps to affine varieties.
	\end{itemize}	
\end{thm}
\begin{prob}
	The last thing is saying that every morphism $X\to \Spec A$ factors (uniquely) through $X\to\Spec \O_X(X)$. Show this!
\end{prob}

\section{October 21st, 2019}
Today we are moving on from Yoneda to speak a bit about the topological properties of algebraic varieties.

Later we will do a lot more examples with the functor of points perspective, but for now the idea to keep around 
is that this enables us to study something \textit{in relation to something else}.

\subsection{Questions}
We thought about $\mu_n:\SpcFun^{op}\to\Set$. Notice that (at least when $\ch k\nmid n$), we get that 
$k[t]/(t^n-1)$, the representing algebra, is $\prod_{\zeta\in\mu_n(k)}k$. Probably the best way to understand 
this is that it is $n$ points, but moreso that it is embedded in a natural way in $\Gm$ via the SES:
\[\mu_n\hookrightarrow\Ga\to\Gm\]

\subsection{Topological properties of varieties}
\begin{defn}
	A topological space $X$ is \textbf{quasi-compact} if every open cover has a finite subcover.
\end{defn}
\begin{rmk}
	In France, compactness requires a space be Hausdorff. Thus in the development of AG we used the French 
	definition and it stuck!
\end{rmk}
\begin{lem}
	An affine variety is quasi-compact.
\end{lem}
\begin{prf}
	The idea was that if $X=\cup U_i$, then since $D(f)$ generate the topology for $X$, we can refine
	the cover so that $U_i=D(f_i)$. By the Nullstellensatz, $X=\cup D(f_i)$, we know that $(f_i)\lhd \O_X(X)$
	is the whole ring!

	But then $1=\sum_{j=1}^n a_jf_{i_j}$, so we can restrict to a finite cover!
\end{prf}
But note that we didn't need that our ring was Noetherian! So the exact same proof shows that $\Spec \bbC[x_1,x_2,\dots]$ is 
quasi-compact! But $\Spec \bbC[x_i]/(x_1,\dots,x_n,\dots)$ is not quasi-compact!!!

\begin{rmk}
	A very cool (``AMAZING'' according to Max) fact is the following: if $U\subset X$ is any open in an affine space, then $U$ is quasi-compact.
	Compare a similar idea in Hausdorff spaces, where this is actually false.
\end{rmk}

Let $X=\Spec A$ where $A=k[x_1,\dots,x_n]/(f_1,\dots,f_m)$, which is Noetherian by Hilbert basis and correspondence theorem.
Now we have an anti-isomorphism of lattices between the radical ideals in $A$ and the closed sets in $X$.
Thus 
\begin{defn}
	A topological space $X$ is Noetherian if any \textbf{descending} chain of closed subspaces stabilizes.
\end{defn}
\begin{rmk}
	Assuming the axiom of choice, this is equivalent to the statement ``A non-empty set of closed subsets in $X$ has 
	a minimal element.''
\end{rmk}
\begin{rmk}
	Note that $\Spec A$ is Noetherian for any (finitely-generated) $k$-algebra.
\end{rmk}
So let $X$ be a Noetherian topological space.
\begin{lem}
	$Z\subseteq X$ is closed implies that $Z$ is Noetherian and quasi-compact.
\end{lem}
\begin{rmk}
	The Noetherian bit is pretty clear! The quasi-compact part needs a bit of arguing. Use the definition that $X$ is quasi-compact 
	if and only if $\cap Z_i=\varnothing\twoheadrightarrow \cap_{j=1}^n Z_{i_j}=\varnothing$ for some $i_j$.

	Then using choice and the contrapositive it falls out!
\end{rmk}
\begin{defn}
	A topological space $X$ is \textbf{irreducible} if $X\ne\varnothing$ and if $X=X_1\cup X_2$ where 
	the $X_i$ are both closed imples that either $X=X_1$ or $X=X_2$.
\end{defn}
\begin{rmk}
	Equivalently, if $U_1$ and $U_2$ are open and nonempty then $U_1\cap U_2\ne\varnothing$.
	We also have that any non-empty open set is dense.
\end{rmk}

\section{October 23rd, 2019}
The plan moving forward: we are going to talk about some more ideas not from the book about topology and other topics.
Then we will rejoin the book with chapter three (skipping the discussion of dimension theory).

Here's where we're aiming: last time we talked about what it meant for a Noetherian topological space (or variety) to be irreducible.
In particular for any Noetherian topological space.
\begin{thm}\label{thm:decomp}
	$X$ admits a decomposition 
	\[X=X_1\cup\cdots\cup X_n\]
	with each $X_i$ irreducible and for all $i\ne j$, $X_i\not\subseteq X_j$. 

	Furthermore, given two such decompositions 
	\[X=X_1\cup\cdots\cup X_n=Y_1\cup\cdots\cup Y_m\]
	then there exists a function $\iota:[n]\to[m]$ such that $X_i=Y_{\iota(i)}$.
\end{thm}
Recall that when $X=\Spec A$, the irreducible components $X_i$ of $X$ correspond precisely to the minimal primes $\frakp\lhd A$.

\subsection{Noetherian Induction}
Let $X$ be a Noetherian topological space and let $P$ be a set of closed subsets of $X$. Suppose that for all $Y\subseteq X$ closed, if 
\[\text{for all closed $Z\subsetneq Y$, $Z\in P$ implies that $Y\in P$ }\]
Then $X\in P$
\begin{prf}
	By contrapositive: assume that $X\in P^c.$ But now we can use axiom of choice: there exists a minimal $W\in P^c$. 
	But then for any closed $Z\subsetneq P$, we have $Z\in P$, so $W\in P$, a contradiction. But then $P^c=\varnothing=X$.
\end{prf}

Now we can prove theorem~\ref{thm:decomp}
\begin{prf}
	$\varnothing$ has the empty decomposition. Thus $\varnothing\in P$. Now if $Y\subset X$ is closed, then either $Y$ is irreducible and $Y\in P$ 
	(``admits a decomposition'') or else $Y=Z\cup W$ and assuming each are in $P$ we get that $Y\in P$ by concatenating (and eliminating). Thus by Noetherian induction we get our decomposition.

	To get the last bit, take two decompositions. Then consider 
	\[X_1=X_1\cap X=(Y_1\cap X_1)\cup\cdots\cup(Y_m\cap X_1).\]
	Then since $X_1$ is irreducible, $X_1\subseteq X_1\cap Y_j$ for some $j$ and thus $X_1\subseteq Y_j$. But then similarly $Y_j\subseteq X_i$ for some $i$.
	But by assumption on the $X_k$, $X_1=Y_j=X_i$, giving us our $\iota$.
\end{prf}

\subsection{Important Morphisms}
\begin{defn}
	A morphism $f:X\to Y$ of varieties is 
	\begin{itemize}
		\item \textbf{affine} if there exists an affine covering $U_i\subseteq Y$ such that $f^{-1}(U_i)\subseteq X$ is affine.
		\item \textbf{finite} if there exists an affine covering $U_i\subseteq Y$ such that $f^{-1}(U_i)$ is affine and 
		\[\O_Y(U_i)\to\O_X(f^{-1}(U_i))\]
		is a finite ring extension (finitely generated as a module over the base).
	\end{itemize}
\end{defn}
\begin{prop}
	$f:X\to Y$ is affine (resp. finite) if and only if for all $U\subseteq Y$ affine, $f^{-1}(U)$ is affine (resp. $f^{-1}(U)$ is affine 
	and $\O_X(U)\to \O_X(f^{-1}(U))$ is finite).
\end{prop}

\section{October 25th, 2019}
Another takeaway from this class: \textit{algebraic geometry is relative and not absolute.} One way we are starting to dip our toes into this idea 
is via the definition of affine and finite morphisms. Notice that the definition of finite is a bit clunky because it requires us 
to restate the definition of affine withing the definiton of a finite morphism. Gross. We should be able to write it as \textit{``affine + $\square$.''}

For some examples, let $Y=\Spec k$. Then $X\to \Spec k$ is affine if and only if $X$ is affine. So when is $X\to\Spec k$ finite? Well since $X$ is affine, $X=\Spec A$ 
where each $A$ is a finite dimensional (reduced!) $k$-algebra. Therefore it is Artinian and we get a decomposition 
\[A=A_1\times\cdots\times A_n\]
where each $A_i$ are finite dimensional local rings with maximals $\frakm_i\subset A_i$ that are nilpotent (use Nakayama!). So by this theorem (since $A$ is reduced),
we get $A_i=k$, so 
\[A\cong k\times\cdots\times k.\]
So then $\Spec A=\sqcup \Spec k$, just a handful of points!

Let's look at another example we saw on the first day: $k[x]\to k[t]$ sending $x\mapsto t^2$. This corresponds to a map $\A1\to\A1$, 
and we can talk about the fibers of this morphism. They are all (well, almost all) of size two. But they are definitely finite! Similarly if 
we have a finite map $A\to B$, we can ask if it has finite fibers. Here if $\varphi:A\to B$ is our morphism and $f:\Spec B\to\Spec A$ is the map of rings,
then since
\[f^{-1}(\frakm)=\{\frakn\subseteq B|\varphi(n)=m\}\]
we get a lemma 
\begin{lem}
	Given a map $f:\Spec B\to \Spec A$ and $\frakm\in\Spec A$, 
	\[f^{-1}(\frakm)=\{\frakn\subseteq B/\frakm B,\text{ maximal}\}\]
\end{lem}
And so (think this through) the answer is \textbf{yes, the fibers are finite.}

A natural question involves the following 
\begin{defn}
	A morphism $f:X\to Y$ is \textbf{quasi-finite} if for all $y\in U$, $f^{-1}(y)$ is finite.
\end{defn}
and we ask: is this equivalent to finiteness? The answer is no. Consider the map $\A1\setminus 0\hookrightarrow\A1$. Both of these are affine, so the map is affine. 
But definitely the $k[t]\to k[t,t^{-1}]$ is not finite!

\begin{prop}
	Let $X\to Y$ be finite and $Z\subseteq X$ clsoed. Then $f(Z)\subseteq Y$ closed.
\end{prop}
Try it!

\section{October 28th, 2019}
Max heard some things from the feedback: he is interested in seeing if he can't help us 
put things together at the end of class. That should help!

\subsection{Products}
Today we are going to be talking about products (and fiber products) in the category of varieties. 
\begin{defn}
	Let $\calC$ be a category and $X,Y\in\calC$. Then the product is the usual thing with the usual universal property.
	\begin{center}
		\begin{tikzcd}
			& X\prod Y\ar[dl,"\pi_X"]\ar[dr,"\pi_Y"] &\\
			X & & Y\\
			& A\ar[ul]\ar[ur]\ar[dashed,uu] &
		\end{tikzcd}
	\end{center}
\end{defn}
\begin{rmk}
	Notice that we can also ``think in terms of functors.'' THen we can define the product of $h_X$ and $h_Y$ as $h_X\times h_Y$ 
	in terms of the (Cartesian) product of sets.
\end{rmk}
Also recall the definition of a fiber product. Notice that if we translate back to functors, we are again trying to figure out what to 
make of the set 
\[h_X\times_{h_Z}h_Y.\]
If $f$ and $g$ are maps from $h_X$ and $h_Y$, respectively, into $h_Z$, then the fiber over $x\in h_Z$ in $h_X\times_{h_Z}h_Y$
is precisely the product of the fibers (whhaaaaat \textit{like a fibered product}) in $f$ and $g$.

\begin{prob}
	If $\calC$ has a final object $\Omega$, then $X\times Y=X\times_\Omega Y$. THink about this.
\end{prob}
\begin{thm}
	Fiber products exist in the category of varieties.
\end{thm}
Today we are going to be discussing the affine case: given $X\to Z\leftarrow Y$ of affines, constructing 
$X\times_Z Y$. To start, let $Z=\Spec k$. We are looking to study the functor $h_X\times h_Y$ and then show representability.

Let's look at this pointwise: let $X=\Spec A$ and $Y=\Spec B$. Then
\[h_X(S)\times h_Y(S)=\Hom(S,X)\times \Hom(S,Y)=\Hom(A,\O_S)\times\Hom(B,\O_S)=\Hom(A\otimes B,\O_S)=h_{\Spec A\otimes B}(S)\]
\begin{prob}
	Show that the coproduct in $\Alg_k$ is the tensor product.
\end{prob}
\begin{rmk}
	For the above problem, notice that $A\otimes B$ is odd notation because it is not just the underlying vector space, but carries with it 
	an algebra structure.
\end{rmk}
\begin{prob}
	Is the tensor product of a finitely-generated algebra finitely generated? Can you do it without coordinates?
\end{prob}

But notice for what we wrote above to be ``okay'' we need that $A\otimes_k B$ is a reduced finitely generated $k$-algebra. The finite generation 
isn't too hard to see. The reducedness comes from the following (in Kempf):

We ahve a map $A\otimes B\to \{|\Spec A|\times |\Spec B|\to k\}$ as set of functions. The claim is that this map is injective.
The idea is that if $\sum_i f_i\otimes g_i\mapsto 0$, we can assume that the $f_i$ and $g_i$ are linearly independent.

But then for all $x,$ we get $\sum_i f_i(x)g_i=0:\Spec B\to k$. But then since the $g_i$ are linearly independent,
this means the $f_i(x)$ are all zero. Thus the nullstellensatz tells us that all $f_i=0$.

\section{October 30th, 2019}
Let's start by discussing the idea that for any final object $\Omega$, $X\times_\Omega Y\cong X\times Y$. One way to do this 
is to show that $X\times Y$ also has the universal property corresponding to the limit of $X\to \Omega\leftarrow Y$. But we can 
also do things functorially! Notice that we have a map 
\[h_{X\times_\Omega Y}=h_X\times_{h_\Omega}h_Y\xrightarrow{\cong}h_X\times h_Y=h_{X\times Y}\]
where we can show the isomorphism of these things on points.

Let's talk about the fiber product 
\[\Spec A\times_{\Spec C}\Spec B\]
which intuitively should be something like $\Spec A\otimes_C B$. But is this going to be reduced still? Let's do an example. Consider $A=k[t]$ and $C=k[x]$ 
and consider the map $x\mapsto t^2$ as a map $C\to A$. Then let $B=k$ with the map that sends $x\mapsto 0$. Then $A\otimes_C B\cong k[\varepsilon]/\varepsilon^2$.

The takeaway here is that the tensor product is sometimes non-reduced! It ends up that if we take $(A\otimes_C B)_{red}$ is the algebra we want, which is the 
appropriate object \textit{in the category of reduced $k$-algebras.}
\begin{prob}
	Consider the maps $\Spec(-)$ and $\O(-)$, between the categories $\Alg_k^{op}$ and varieties. These form an adjoint pair! Which object is 
	left adjoint? Do left adjoints commute with limits? colimits?
\end{prob}

\subsection{A series of fun exercises}
Here's some work to do!
\begin{thm}
	Fiber products exist in the category of varieties.
\end{thm}
\begin{rmk}
	You should prove this! We will discuss the idea here. The only thing we know how to do is do this with affines! 
	reduce to the affine case (for all three) and glue!
	\begin{itemize}
		\item Begin by showing: Suppose $U\subseteq Z$ is open and $X\times_Z Y$ exists. Then we can look at 
		\begin{center}
			\begin{tikzcd}
				(X\times_ZY)_U\ar[d]\ar[r,hookrightarrow]& X\times_Z Y\ar[d]\\
				U\ar[r,hookrightarrow] & Z
			\end{tikzcd}
		\end{center}
		and we claim that $(X\times_ZY)_U$ representas $h_{X_u}\times_{h_u}h_{Y_u}$. This gives us a way to shrink over the base.
		\item Now suppose $W\subseteq Y$ is open. Then looking at a similar diagram, we want to show that $(X\times_ZY)_W$ represents $h_X\times_{h_Z}h_W$. The same thing can be done with a subset of $X$.
	\end{itemize}
	Now we want to reduce to $Z$ being affine, then reduce to $Y$ is affine, then $X$. Then we are golden.
	For the first, assume it holds when $Z$ is affine and then show it works for arbitrary $Z$: pick a covering $U_i$ of $Z$ by affines. Consider $U_i$ and $U_j$:
	
	Each $(X_{U_i}\times_{U_i}Y_{U_i})_{U_i\cap U_j}\cong (X_{U_j}\times_{U_j}Y_{U_j})_{U_i\cap U_j}$ via a canonical isomorphism $\phi_{ij}$. This is the crux of it. THINK. :)
\end{rmk}

\section{November 1st, 2019}
Let's look at some examples of things. $\A1\times\A1=\Spec k[x]\otimes k[y]\cong\Spec k[x,y]\cong \A2$. Another way to see this 
is to think about the functors:
\[\Hom(X,\A1\times\A1)\cong\Hom(X,\A1)\times\Hom(X,\A1)=\Hom(k[s,t],\O(X))=\Hom(X,\A2).\]
Similarly you can do something along these lines with $\A n$.

What about $\P1\times\P1$? It's not $\P2$. Why? Because there are curves $C_1$ and $C_2$ in the product that do not intersect.
But Bezout's theorem says that any two curves in $\P2$ intersect!

The problem with solving the above problem is that we really need sheaf theory to solve the above problem. 
We will circle back around to this.

Question: Is the zariski topology on $X\times Y$ the product topology? NO! For instance $\A2$: the product topology is not the cofinite one!

\subsection{Projective spaces}
\begin{defn}
	A variety is \textbf{projective} if there exists a closed immersion $X\hookrightarrow \P n$.
\end{defn}
\begin{prop}
	Products of projective varieties are projective.
\end{prop}
\begin{prob}
	Classify all projective objects in the category of varieties! You can do it. :)
\end{prob}
\begin{rmk}
	Notice that right now we are working in the \textit{Italian} projective space. This is the usual $\A n\setminus\{0\}$.
	Then we say that $Z$ is closed if the closure of its inverse image in $\A n\setminus \{0\}$ is invariant under scaling.
\end{rmk}
Now think about the functions on $\P n$. There is a ring $R\subseteq k(x_0,\dots,x_n)$ called the field of homogeneous rational functions. These are the 
ratio of homogeneous polynomials of the same degree. It's a field baby. Then 
\[\O_{\P n}(U)=\{\phi\in R|\operatorname{poles}(\phi)\subseteq\P n\setminus U\}.\]

\subsection{The Point}
There is a famous map called the Segre embedding. We get a map 
\[\P n\times\P m\hookrightarrow \P {(n+1)(m+1)}\]
where we can just imagine taking $(v_i)$ and $(w_i)$ and map it to $\mathbf{v}\mathbf{w}^T$.

This is just the projectivization of a matrix! It dends up that it identifies the product with a closed subvariety equal to \textbf{the set of matrices of rank 1.}
There was something at the end about $2\times 2$ subdeterminants!

\section{November 4th, 2019}
We're going to talk a bit more about projective varieties because we really haven't done a lot of examples and that is a shame.

\subsection{What we want to do}
By the end of the quarter (we have four more weeks, but only three more weeks worth of meetings with holidays), we want to 
have talked a bit about sheaves and dimension/intersection theory. Continuing into next quarter, we are going to dip into schemes, although perhaps not immediately.

\subsection{Some tangible examples}
One idea is to read Harris' \textit{Algebraic Geometry: a first course}. Now we're talking to talk about $\P n$. 

First things first, we form $\P n$ as a set as the lines in $k^{n+1}$ or $k^{n+1}\setminus \{0\}/k^\times$. A point in $\P n$ 
has homogeneous coordinates $[x_0:\cdots:x_n]$, which is only well-defined up to scaling. But if any $x_i$ is nonzero, we can normalize this 
coordinate to get well-defined coordinates. The collection of sets $\{x_i\ne 0\}=\A n\subset \P n$ is called the standard affine open 
cover of $\P n$.

A thought: if we have an affine variety $Z\subseteq \A n\subseteq \P n$, how can we view the closure of $Z$ in $\P n$?
Given $f\in k[x_0,\dots,x_n]$ homogeneous, $Z(f)\subseteq \P n$ is well-defined. More generally, if $I$ is a homogeneous ideal (generated by homogeneous polys)
$Z(I)$ makes sense and furthermore $Z((f_1,\dots,f_k))=\cap_i Z(f_i)$. It ends up that these are precisely the closed sets in the Zariski topology.

\begin{thm}[Projective Nullstellensatz]
	$I(Z(J))=\sqrt{J}$ for all homogeneous $J$ excipt $J$ such thtat $\sqrt{J}$ is the irrelevant ideal $(x_0,\dots,x_n)$.
\end{thm}

\subsection{Homogenization/dehomogenization}
Given a homogenous polynomial $f(x_0,\dots,x_n)$ of degree $d$, we can \textit{dehomogenize} things by setting $x_0=1$. For instance $x_0^2+x_1^2+x_2^2\mapsto 1+x_1^2+x_2^2$.
Given a polynomial $g(x_1,\dots,x_n)$, we just look at 
\[\tilde g(x_0,\dots,x_n)=x_0^{\deg g}g(x_1/x_0,\dots,x_n/x_0).\]
One can see relatively easily that $\tilde g$ is a homogeneous polynomial of degree $d$.

\begin{prob}
	What happens when you homogenize then dehomogenize? How about the other way around?
\end{prob}

\section{November 6, 2019}
Today we are going to focus on the pickles and not the meat. Go Max.

\subsection{Questions and Thoughts}
First things first, we wanted to talk about the problem above. It is easy to see that homogenizing and then dehomongenizing gives you the same polynomial.
For the other direction, it seems that a (homogeneous) polynomial comes back to itself if it is not divisible by a power of $x_0$.

What is happening geometrically? Dehomogenizing is intersecting with an open affine subset. Homogenizing is taking the closure.
\begin{rmk}
	The idea came up that taking the closure might be the same as taking the cone containing the original thing. But consider 
	the hyperbola $xy-1$ and its homogenization $xy-z^2$. Now $Z=Z(xy-z^2)$ is closed and contains $H=Z(xy-1)$, so it contains $\bar H$.

	Conjecture: $Z\cong \P1\subseteq\P2$ as a conic. How can we realize this as a map from $\P1$ to $\P2$? Well,
	we need a map 
	\[(x_0,x_1)\mapsto(f_0(x_i),f_1(x_i),f_2(x_i))\]
	where the $f_i$ are homogeneous of the same degree and have no common zeros in $\P1$.
\end{rmk}
\begin{rmk}
	More generally, given varieties $X$ and $Y$, a \textbf{rational map} from $X$ to $Y$ $f:X\dashrightarrow Y$ is an equivalence class of morphisms $U\to Y$ where $U$ is open in $X$ 
	and $U\xrightarrow{\alpha} Y$ and $V\xrightarrow{\beta}X$ are equvialent if $\alpha|_{U\cap V}=\beta|_{U\cap V}$.

	Then we can choose $f_0,\dots,f_m\in k[x_0,\dots,x_n]$ that are homogeneous of degree $d$ and these coordinate functions give us a rational map (in fact, all the rational maps) from $\P n$ to $\P m$.
\end{rmk}
\begin{prob}
	Find a morphism $f:\P1\to\P2$ whose image is $Z(xy-z^2)$.
\end{prob}
I am thinking $x_0^2$, $x_1^2$, and $x_0x_1$.
\begin{prob}
	Suppsoe that $Q$ is a non-singular quadratic form in three variables, so $Z(Q)\subseteq \P2$. Then show that $Z(Q)\cong \P1$.
	\textit{Hint: Linear changes of coordinates are algebraic!}
\end{prob}

\section{November 8th, 2019}
Last time we were thinking about conic curves in $\P2$, that is $Z(Q)$ where $Q$ is a nondegenerate quadratic form in $x_0,x_1,x_2$. One thing that people tried 
was to classify the ways that $C=Z(Q)$ intersects with copies of $\A2$. Then using these options, you can proceed. This leverages automorphisms of 
$\P2$ to reduce everything to a very finite set of cases.

Another idea was to do a linear change of coordinates that takes $C$ to $Z(xy-z^2)$. This amounts to doing some work on the matrix representing $Q$. Then answer it for a hyperbola. 

For the other question, we were trying to find a good morphism $\P1\to\P2$ with image the hyperbola. The idea was the one I said. It can be seen relatively easily to be bijective.
\begin{rmk}
	Is a bijective morphism an isomorphism?! (No) The idea to consider is $R=k[x,y]/(y^2-x^3)\to k[t]$ where $k[t]$ is the normal closure of the former. Then if we are looking at the fiber over $0$, 
	we compute $R/(x,y)\cong k\to k[t]/(t^2,t^3)\cong k[t]/t^2$
\end{rmk}
\begin{rmk}
	Another idea that came up: if we are thinking about the geometry behind the homogenization/dehomogenization maps, notice that this tells us something about how 
	dehomogenizing is not a unique process!
\end{rmk}

\subsection{The Veronese Embedding}
Given $n$ and $d$, there is a (somewhat) canonical map 
\[\P n\to \P{\binom{n+d}{d}-1}\]
where the coordinate functions are the degree $d$ monomials in $x_0,\dots, x_n$. The image here is ``cut out by quadrics.'' These quadrics 
come from $MN=PQ$ for (not necessarily distinct) coordinate functions.

Recall the Segre embedding. When $n=m=1$, we get an embedding 
\[\P1\times \P1\to\P3\]
where 
\[(x_0:x_1),(y_0,y_1)\mapsto (x_0y_0,x_0y_1,x_1y_0,x_1y_1)=(z_0,z_1,z_2,z_3).\]
Here the relation is $z_0z_3=z_1z_2$.

Some problems:
\begin{prob}
	Show that the image of $\P1\times\P1$ in $\P3$ above is $Z(z_0z_3=z_1z_2)$.
\end{prob}
\begin{prob}
		Show that $\P1\times \P1\not\cong \P2$.
\end{prob}
\begin{prob}
	For any non-degenerate quadratic form $z_0,\dots,z_3$, do we have that $Z(Q)\cong \P1\times\P1$?
\end{prob}
\begin{prob}
	Are all the smooth quadratic hypersurfaces of a fixed dimension isomorphic? Are they all isomorphic to $(\P1)^d$?
\end{prob}
\begin{prob}
	What about cubic hypersurfaces?
\end{prob}

\section{November 15th, 2019}
Thinking about quadradic forms (avoiding characteristic 2 for a moment): Any quadratif form $Q(x_0,\dots,x_n)$ is equivalent
(via a linear change of coordinates) to $x_0^2+\cdots+x_m^2$ (when $k=\bar k$) for some $0\le m\le n$. Thus $Q$ is nondegenerate 
if and only if $m=n$.

Think about the rational map $\pi:\P n\dashrightarrow \P m$ sending $(x_0:\cdots:x_n)\mapsto (x_0:\cdots:x_m)$. Notice this is not 
defined at $x_0=\cdots=x_m=0$, a linear subspace $L\subseteq \P n$ called the \textit{center of $\pi$}.
For instance, the map $\P3\dashrightarrow\P2$ is not defined at $(0:0:0:1)$. In affine coordinates $\A3\subseteq \P3$, 
where we set $x_3=1$. Here we think of setting $z=1$ (since we want to send $(x,y,z)\mapsto(x/z,y/z)=(x:y:z)$).

Looking at where points go in the plane $\A2\subseteq \A3\subseteq \P3$, we see that it is somehow mapped via the line through the origin. 
This is a ``pinhole camera'' projection of $\P3$ onto $\P2$.
\begin{prob}
	Given $\A3\to\A2$ via the map $(x,y,z)\mapsto (x,y)$, can this be recognized as the restiction of a projective morphism?
\end{prob}

Notice that given $\pi:\P3\dashrightarrow\P2$, the center is a conic given by $x_0^2+x_1^2+x_2^2=0$. This is a cone! Sweet!

Check this out: If we have the affine map $\A3\to\A1$ with the quadratic form $(x/y)^2+1=0$, consider the pullback 
under this map: the solutions to this in $\bbC$ are $x=\pm iy$, and for each of these points $\alpha$, the point $(x/y)-\alpha=0$ 
pulls back to $x-\alpha y=0$, a plane! But we get two roots, so two planes in $\P3\supset \A3$. What is the intersection of these? The center!

\subsection{Cremona}
We have the Cremona map $\P2\dashrightarrow\P2$ sending $(x,y,z)\mapsto(1/x,1/y,1/z)$. This is ``obviously'' a birational involution. 
At least involutivity is apparent.

But we can scale by $xyz$ and get the map $(x,y,z)\mapsto(yz,xz,xy)$. Notice it is undefined at $(1,0,0),(0,1,0)$, and $(0,0,1)$.
Now notice that the original map wasn't defined where \textit{any of the coordinates were zero} while the latter only requires that no more 
than two are zero. Why is this kosher?!

The idea here is that each of the lines $x=0$, etc are sent to POINTS $(1,0,0)$, which are then not in the domain of the cremona map.
Thus the idea of involution is no longer meaningful!

\begin{prob}
	Consider the form $Q=xy-zw$ and its zero set in $\P3$. Then look at its image under the map $(x,y,z,w)\mapsto(y,z,w)$. SHow $\pi$ is birational and write down an 
	inverse then compute $\pi\circ \pi^{-1}$ using cremona.
\end{prob}

\section{November 18th, 2019}
Apparently Hitler loved cake.

Today we are going to do some more examples. However, first we will consider 
\subsection{Last week's problems}
Consider the standard projection $\A3\to\A2$. We wanted to think of it as a linear map on projective spaces. Recall 
\begin{defn}
	A \textbf{rational map} $f:X\dashrightarrow Y$ is an equivalence of morphisms $f_U:U\to Y$ where $\varnothing\ne U\subseteq X$ is open. The equivalence 
	relation is that $f_U=f_V$ if $f_U|_{U\cap V}=f_V|_{U\cap V}$.
\end{defn}
What does it mean for a map to be birational? One idea could be
\begin{prob}
	Make a category $\mathbf{Bir}$ of irreducbile varieties with mapes being rational maps (eq. classes). Try to think about inverses in this category.
\end{prob}
\begin{defn}
	A rational map $f:X\dashrightarrow Y$ is \textbf{birational} if there exists a $g:Y\dashrightarrow X$ such that 
	\[fg=\id_Y\qquad gf =\id_X\]
	
	$f$ is \textbf{dominant} if there exists an open $V\subseteq Y$ that is contained in $\Im \tilde f$ for some representative $\tilde f$ of $f$.
\end{defn}
\begin{prob}
	Show that if $f:X\dashrightarrow Y$ is birational then there exist nonempty opens $U$ and $V$ in $X$ and $Y$, respectively such that $f$ induces an isomorphism $U\xrightarrow{\sim} V$ of varieties.
\end{prob}
\begin{defn}
	An irreducible variety is \textbf{rational} if it is birational to $\P n$ for some $n$.
\end{defn}
\begin{rmk}
	Roughly speaking, we think that ``most of $X$ can be uniquely described by $n$ algebraic parameters.'' Try to look at some examples! The points where this doesn't make sense is jsut the 
\end{rmk}
\begin{prob}
	I am repeating some of the old problems to refresh their importance: try figuring out how a quadric projects from $\P2$ to $\P1$ by the usual projection map. There is actually an isomorphism!
\end{prob}
We've been sniffing around the idea that any non-degenerate quadric $Q\subseteq \P n$ is rational. What about cublics? Which are rational? The cubic curves (defined by cubic equations in $\P2$)
are genus one and NOT rational. The cubic surfaces (those in $\P3$) are apparently.

\section{November 25th, 2019}
Last time we talked about quadrics and cubics and stuff! We asked whether smooth quadrics are all rational (birational to $\P n$). We also asked when cubics are rational.

We saw that when $Q\subseteq \P2$ that $Q\cong \P1.$ In general we can see that these are rational! This involves taking any point on the quadric 
and then project onto the other one. But the important thing to notice here is that this required us to pick a point and we break the symmetry of having 
two points in each fiber.

What about cubics? Here the center of a projection $\P3\dashrightarrow\P1$ is a \textit{line}. Let $S=Z(\text{cubic form})$. Then the fiber over $p\in\P1$ is 
a plane: $\P2\hookrightarrow\P3$ via $(x_0:x_1:x_2)\mapsto(L_0:L_1:L_2:L_3)$ where the $L_i$ are linear forms in the $x_i$. Then if $F$ is our cubic form,
we can consider the slice by this plane: $F(L_0,\dots,L_3)=G(x_0,\dots,x_3)$. For next time: what if the center $l\subseteq S$ (the center of projection is in $S$). One question:
does the intersection $G$ of a plane containing such a line factor as a product of a quadric and a linear factor?

\section{December 2nd, 2019}
Today (and through the rest of the week) we are going to talk about sheaves! Woot.

We're going to have a long time in between our last class this week and our first class in January. Here's a tidbit to tide you over:
\begin{prob}
	Homework for the break: Do all the problems in section 1 of chapter II of Hartshorne.
\end{prob}
\subsection{Sheaves}
There is a mantra! \textit{Sheaves are a geometric manifestation of the flow of information from local to global.}
\begin{ex}
	Let $X\in\SpcFun$. Then $\O$ is the sheaf of regular functions on $X$.
\end{ex}
\begin{rmk}
	Recall that we had that $\O(U)$ was defined for every open $U$ and that regulartity was preserved by 
	restrictions and finally that regularity is local (it glues). These give us the ideas we will use to define a sheaf.
\end{rmk}
I love this little gluing diagram: given an open covering $U_i$ of $U$,
\begin{center}
	\begin{tikzcd}
		\O(U)\ar[r] & \prod_i \O(U_i)\ar[r,shift left]\ar[r,shift right] & \prod_{i,j}\O(U_i\cap U_j)
	\end{tikzcd}
\end{center}
where the top map sends
\[(\varphi_i)_i\mapsto (\varphi_i|_{U_i\cap U_j})_{(i,j)}\]
and the bottom does 
\[(\varphi_i)_i\mapsto (\varphi_j|_{U_i\cap U_j})_{(i,j)}.\]
That is, $\O(U)$ is the equalizer of these maps.

We took some time to talk about limits as (right) adjoints to the constant diagram functor. Hadn't seen that before.
We concluded with a bunch of cateory theory, which was really fun. Ind and Pro categories.
\subsection{More examples}
\begin{itemize}
	\item Let $X$ and $Y$ be spaces
	\item $h_Y:U\mapsto h_Y(U)$ is a sheaf on $X$ for all open $U$ in $X$.
	\item If $X$ is a manifold and $V\to X$ is a vector bundle over $X$, then you get a variety of sheaves (discrete, cts, flat, holomorphic sections).
\end{itemize}

\section{December 4th, 2019}
Let $\calC$ be a category (think of $\Set$).
\subsection{More Sheafiness}
\begin{defn}
	A \textbf{presheaf} with values in $\calC$ on $X$ is an $\calF$ such that 
	\begin{itemize}
		\item $\forall U\subseteq X$ (open) we have $\calF(U)\in\calC$
		\item $\forall V\subseteq U\subseteq X$, we get a restriction map $\rho_{VU}:F(U)\to F(V)$ which composes with other restrictions.
		\item $\rho_{UU}=\id_{F(U)}$
	\end{itemize}
\end{defn}
\begin{rmk}
	Of course this is all equivalent to saying a presheaf is a functor 
	\[\calF:\mathbf{Open}(X)^{op}\to \calC.\]
\end{rmk}

Now let's say that $\calC$ has products (including the empty product/final object). Then definitely the sheaf condition makes sense. 
We get the coordinate kind of definition if we imaging $\calC$ is $\Set$ (or at least concrete).
\begin{defn}
	A \textbf{sheaf} is a presheaf that satisfies the sheaf condition for all open coverings of all open sets.
\end{defn}
Some examples:
\begin{itemize}
	\item $\O=h_{\A1}$
	\item $h_Y$ is a sheaf on $X$ as we saw above.
\end{itemize}
\begin{rmk}
	There are two categories:
	\[\PreSh=\Func(\Open(X)\op,\calC)\]
	and a full subcaltegory $\Sh$ of sheaves.
\end{rmk}
\subsection{Stalks}
\begin{defn}
	Given a $p\in X$W and $\calC\in \PreSh(X)$, the \textbf{stalk of $\calF$ at $p$} is
	\[\calF_p=\colim_{p\in U}\calF(U)\]
\end{defn}
\begin{rmk}
	If $\calC$ is $\Set$, then we can realize this as $\sqcup_{p\in U}\calF(U)$ modulo the relation that two maps are the same
	if they agree on a small neighborhood of $p$.
\end{rmk}
As a matter of notation, we say that if $p\in V$ and $f\in\calF(V)$, then there is a value $f_p\in \calF_p$,
called the \textbf{germ of $f$ at $p$.}

\begin{thm}[Basic]
	A map of sheaves $\calF\to \calG$ is an isomorphism (resp. mono/epi) if and only if for all $p\in X$ the induced map $\calF_p\to\calG_p$ is an iso (resp. mono, epi).
\end{thm}
\begin{prob}
	For next time: Classify all (pre)sheaves on the space with two points $x$ and $y$ with topology $\{\varnothing,\{x\},\{x,y\}\}$. Also think about monos, epis, and isos of (pre)sheaves).
\end{prob}

\part{Quarter 2: Sheaves and Schemes}
I missed the first five lectures of this quarter because I was travelling for a conference. Sorry about that!

\section{January 17th, 2020}
\subsection{SCHEMES}
Let's just dive right in. Last time they were talking about things:
\begin{itemize}
	\item Is the category of spaces with functions a full subcategory of locally ringed spaces?
	\item Thinking about $\Hom(\Spec A,\O_A),(\Spec B,\O_B))$
\end{itemize}
In the later case, we noticed that if we had a pair of maps $(f,f^\sharp)$ between $\Spec$s and the structure sheafs. Then $\Gamma(f^\sharp):B\to A$
gives us a local condition: if $\varphi=\Gamma(f^\sharp)$, $f$ must be the map induced by $\varphi$.

\subsection{Talking about affine schemes}
Let $A$ be a (commutative unital) ring. Define a locally ringed space $\Spec A$ where 
\begin{itemize}
	\item The underlying space $|\Spec A|$ is the set of prime ideals of the ring.
	\begin{itemize}	
		\item Notice before we used the maximals but getting all the primes gives us something extra...
		\item The topology is the Zariski topology.
		\item This looks like a closed point for every prime in $\bbZ$ as well as the dense generic point $(0)$. This is the French model for the Italian idea of a ``general point.''
		\item A basis for this topology is $D(f)\subseteq |\Spec A|$ defined as the nonvanishing locus:
		\[D(f)=\{\frakp|f\notin\frakp\}\]
		or in other words, $f(\frakp)\ne 0$.
	\end{itemize}
	\item We want 
	\[\Gamma(|\Spec A|,\O)=\O(|\Spec A|)=A.\]
	as well as $\O(D(f))=A_f$, the localization. We know that $D(f)$ defines a basis for the topology for $\Spec A$, so this implies
	this gives us a way to define the sheaf locally.

	So it suffices to show that $\O:\calB\op\to\Ring$ forms a sheaf (satisfies the sheaf condition). Here's the (``bestest'') proof:
	\begin{prf}[Bestest]
		Algebraic formulation: Begin by reducing to $U=|\Spec A|$. Then take an open cover $U_i$ of $U$. Look at descent theory! It's also in SGA.
	\end{prf}
\end{itemize}
\begin{rmk}
	We can think of the value of $f$ at a prime $\frakp$ as follows: $f\in A$ is an element of a ring and 
	\[f(\frakp)\eqdef \bar f\in \kappa(A/\frakp)\]
	the residue field.
\end{rmk}
\begin{rmk}
	To see why it suffices to define the sheaf on a basis: let $\calB$ be a basis for a topological space $X$. Then let $F:\calB\op\to \Ab$ be a sheaf 
	on $\calB$. Then there is some work to do, but the statement is that the category of sheaves on $X$ is equivalent to the category of sheaves on $\calB$.

	The idea is as follows: to get essential surjectivity, you take a sheaf on $\calB$ and take the sheafification to get a sheaf on $X$. 
\end{rmk}

\section{January 22, 2020}
Recall that we have the set $\Spec A$ of primes of $A$ and we know the value of the structure sheaf $\O$ on special sets:
\[\O(D(f))=A_f\]
Then we interpolate between these sets to get the value of the sheaf at specific sets.

\subsection{Connection with Locally Ringed Spaces}

\begin{prop}
	Say $(X,\O_X)$ is a LRS. Then the map
	\[\Hom_{LRS}((X,\O_X),\Spec A)\to \Hom_{\Ring}(A,\O_X(X))\]
	where we send $(f,f^\sharp)\mapsto f^\sharp(\Spec A)$, is a bijection.
\end{prop}
\begin{prf}
	Why is this surjective? Given a map $\varphi:A\to \O_X(X)$ we want to concoct a map of schemes $A\to\Spec A$.
	Let $p\in X$ and $f(p)=(\rho_p\circ\varphi)^{-1}(m_p)$ where $\rho_p$ is the localization at $p$ map.

	Why is $f$ continuous under this definition? It is enough to chekc that $f^{-1}(D(a))\subseteq X$ is open,
	which is the set of all $x\in X$ such that $[\varphi(x)]\notin m_x\in \O_{X,x}$ at x. In other words, we are finding the $x$ such that $\overline{\varphi(a)}\ne 0\in k(x)$ (the residue field.)

	Then to get the map $f^\sharp:\O_{\Spec A}\to f_\ast\O_X$.
\end{prf}

\begin{lem}
	Let $(X,\O_X)$ be a locally ringed space. Let $f\in\O_X(X)$. Then 
	\[D(f)=\{x\in X|f\notin m_x\in\O_{X,x}\}\subset X\]
	is open. (Note we are writing the germ $g_x$ as $g$).
\end{lem}
\begin{prf}
	Since $g\notin m_x$, so $g$ is a unit in $\O_{X,x}$. Thus we have an $h\in\O_{X,x}$
	such that $gh=1$. Thus there exists an $x\in U$ and $\tilde h\in \O_X(U)$ such that $g|_U\tilde h=1\in\O_X(U)$
	so $U\subseteq D(g)$.
\end{prf}

\begin{rmk}
	Look at ``sheaf on a base.'' This is the idea of interpolation after defining the sheaf on the distinguished 
	opens $D(f)$. It is in Vakil and EGA chapter zero and many others.
\end{rmk}	

\subsection{Schemes}
So what is a scheme?
\begin{defn}
	A \textbf{scheme} is a locally ringed spaces such that there exists an open covering $U_i\subseteq X$ with $(U_i, \O_X(U_i))$ is an affine scheme for all $i$.
\end{defn}
Some examples:
\begin{itemize}
	\item $\Spec \bbZ$. $\bbZ$ is special because it is an initial object in $\Ring$. So for any $X$ we have a 
	canonically-defined map $f:X\to \Spec \bbZ$, where we assign to every point in $X$ a prime number: this is the characteristic of the residue field $k(x)$!
	\item $\Spec k$ where $k$ is a field. Here we have a single point and $\O(\ast)=k$. Then the endomorphisms of this scheme are 
	the endomorphisms $\Hom(k,k)$, which is often interesting! There is a unique one when $k=\bbQ$, but not over $\bbC$!
\end{itemize}

\begin{defn}
	For a fixed scheme $S$, the category of $S$ schemes $\Sch/S$ has objects $X\to S$ and 
	$\Hom(X\to S,Y\to X)$ are given by diagrams.
\end{defn}

\section{January 24th, 2019}
Today we are going to continue talking about $S$-schemes (or schemes over $S$). This is just the slice category $\Sch/S$!
Notice to really make things work we have to make the following definition:
\[(g,g^\sharp)\circ(f,f^\sharp)=(g\circ f,g_\ast(f^\sharp)\circ g^\sharp)\]
\begin{rmk}
	There is a section in Hartshorne where he talks about the difference between $k$-varieties and $k$-schemes. You should read it.
\end{rmk}

\subsection{Examples of Schemes}
Of course we have affine schemes by construction. Other examples we have seen are $\Spec R$ for $R$ a DVR or product of fields. These both 
have two points. 

\begin{rmk}
	Look up the ``Punctual Hilbert Scheme'' that talks about schemes with one point with different-sized tangent spaces. For instance, $k[x,y]/(x,y)^2$.
\end{rmk}

Remember the idea of the line with two origins? Well you can take two copies of $\Spec k[[t]]$ (which contains both an open and a closed point) 
and glue along the open point and get something analogous. This is no loner affine, but that is not obvious!

\subsection{Proj}
Let $A=\oplus A_i$ be a graded ring. Then an element of $A$ is homogeneous if it lives in some $A_i$. An ideal is homogeneous if $I$ 
is generated by homogeneous elements.

Some examples: $A=k[x_0,\dots,x_n]$ with $A_i$ the monomials of degree $i$. The irrelevant ideal $A^+$ is the 
ideal of all stuff in positive degree.

Then $\Proj$ makes a schemes from a graded ring. As a set,
\[|\Proj A|=\{\frakp\subseteq A| \frakp\text{ is a homogeneous prim such that } A^+\not\subset\frakp\}\]
and we assign to this set the Zariski topology where if $I\subseteq A$ is homogeneous, then
\[V(I)=\{\frakp\in|\Proj A||\frakp\supset I\}\]
and, in particular, notice that $V(A^+)=\varnothing.$

Then we need to assign to this space a sheaf. Let $f\in A$ be a homogeneous element. Then 
\[\O(D(f))=A_{(f)}=(A_f)_0\]
which is the degree zero part of the usual ring $A_f$.

For example, if $A=k[x_0,x_1]$, and if $f=x_0$, then $A_{(f)}=k\left[\frac{x_1}{x_0}\right]$.

Our main result: 
\begin{thm}
	$(|\Proj A|,\O)$ is a locally ringed space. Furthermore given a homogeneous element $f$ with $\deg f>0$, the map 
	\[A_{(f)}\to\O(D(f))\]
	induces an isomorphism 
	\[(D(f),\O_{D(f)})\xrightarrow{\sim}\Spec A_{(f)}.\]
\end{thm}
Another example: consider $\bbP_\bbZ^n=\Proj\bbZ[x_0,\dots,x_n]$.

\section{January 27th, 2020}

\subsection{Absolute properties of schemes}
\begin{defn}
	A schemes is:
	\begin{enumerate}
		\item \textbf{connected} (resp. \textbf{irreducible}) if the topological space $|X|$ is.
		\item \textbf{reduced} if for all $U\subseteq X$ $\O_X(U)$ is reduced.
		\item \textbf{integral} if for all $U\subseteq X$ $\O_X(U)$ is an integral domain (iff $X$ is reduced and irreducible).
		\item \textbf{quasicompact} if $|X|$ is quasicompact.
		\item \textbf{locally Noetherian} if there is an open cover $\Spec A_i=U_i\subseteq X$ with $A_i$ Noetherian.
		\item \textbf{Noetherian} if $X$ is locally Noetherian and quasicompact (iff there is a finite Noetherian open cover).
		\item \textbf{normal} if there is an open cover $\Spec A_i=U_i\subseteq X$ with $A_i$ normal (integrally closed domain).
	\end{enumerate}
\end{defn}

A question: if $\Spec A$ is locally Noetherian, is $A$ Noetherian? The answer is yes! We need a sublemma:
\begin{lem}
	Suppose that $\Spec B\subseteq \Spec A$ is open with $f\in A$ such that $D(f)\subseteq\Spec B$. Then $D(f)=D(g)$
	where $g=f|_{\Spec B}\in\O(\Spec B)=B.$
\end{lem}
Then we may replace the $\Spec A_i$ by $\Spec A_{f_i}$ such that $A_{f_i}$ is Noetherian. But then $A\to \prod_1^n A_{f_i}$ is a faithfully flat 
extension of a Noetherian ring, so it follows from the following:
\begin{lem}
	Let $A\to B$ be faithfully flat. Then $B$ being Noetherian implies $A$ is.
\end{lem}
You can see this by tensoring up a chain $I_\bullet$ in $A$ to a sequence $I_\bullet\otimes B\cong (I_iB)_{i\in\bbN}$. This is a sequence in $B$.

\subsection{Examples}
Consider $f\in k[x,y]$. Then $\Spec k[x,y]/(f)$ is integral.
Consider $n$ copies of $\A1$ glued along $\A1\setminus 0$. This is also integral.

Think about $\Spec k[x,y]/(xy)$. Draw a picture. Is it reduced/ireducible/connected?

\section{January 29th, 2020}
\subsection{Example}
Consdier the tangent space (the $\Hom$ from $\Spec k[\varepsilon]/(\varepsilon^2)$) to the space $\Spec k[x,y]/(xy)$ (the axes in $\bbR^2$).

Looking at the algebra maps, you can map $x$ and $y$ freely to multiples of $\varepsilon$, giving us a two dimensional thing! Otherwise you can compute you only have one 
degree of freedom. It is (locally) Noetherian, reducible, integral, and other stuff.

You can do this with higher-order vectors as well! Think about what it means to map $\Spec k[\varepsilon]/(\varepsilon^n)$ into a scheme.

\brk

Think more about the countable number of lines glued together in a criss-cross. This is a scheme! It is locally Noetherian but not Noetherian. Also think about the line with infinitely many origins.

\subsection{AG is relative}
\begin{defn}
	A \textbf{morphism $f:X\to Y$ of schemes} is 
	\begin{itemize}
		\item \textbf{quasicompact} if there is an open cover $U_i\subseteq Y$ with $U_i$ and $f^{-1}(U_i)$ quasicompact.
		\item \textbf{locally of finite type} if there exists an open cover $\Spec A_i=U_i\subseteq Y$ such that each $f^{-1}(U_i)$ has an open cover $\Spec B_{ij}$ such that each map $A_i\to B_{ij}$ 
		makes $B_{ij}$ into a finitely generated $A_i$ algebra.
		\item \textbf{of finite type} if $f$ is locally of finite type and quasicompact.
		\item \textbf{affine} if there exists an open covering $\Spec A\subseteq Y$ such that $f^{-1}(U_i)$ is affine.
		\item \textbf{finite} if it is affine and there is an open cover $\Spec A_i\subseteq Y$ such that $f^{-1}(U_i)=\Spec B_i\to \Spec A_i$ is (module-)finite.
		\item \textbf{quasifinite} if the map of topological spaces has finite fibers.
		\item \textbf{locally quasifinite} if there exists a cover $U_i\subseteq Y$ and $V_{ij}\subseteq X$ such that $V_{ij}\to U_i$ is quasifinite.
	\end{itemize}
\end{defn}

This is a theorem with a thousand different formulations so you may not find this one online:
\begin{thm}[Zariski's Main Theorem]
	Finite morphisms are precisely the proper and quasifinite ones.
\end{thm}

\subsection{An observation}
Let $X$ a variety. Then how do we understand this in the language of schemes? It is an \textit{integral} $k$-scheme such that the map 
$X\to \Spec k$ is
\begin{itemize}
	\item of finite type,
	\item and separated (will see this later).
\end{itemize}
So finite type is aprt of ``relative variety.'' We can imagine a family of varieties given by the zero sets 
of $y^^2=x^3+tx+4$ and consider this as a closed subvariety in $\A 2$ over $\A1$. Then we could specialize at any prime and recover a variety in $\A2_\bbZ$.
Furthermore varying $4$ can get us more.

\subsection{Another observation}
If $f:X\to Y$ is finite, then $f$ is quasifinite. To see this, let $y\in \Spec A\subset Y$. Then since $f$ is finite,
$f^{-1}(\Spec A)=\Spec B$ and $A\to B$ is a finite ring map (some ordering needs to be fixed there on choosing $A$ and $B$).
Thus without loss of generality, we can assume $X=\Spec A$ and $Y=\Spec B$. So then $y\in Y$ if and only if $p\lhd A$ is prime.

\section{January 31, 2020}
Today we are going to talk about fiber products. Max thinks they should be ``fibered products'' but he's wrong.

\subsection{Fiber products are real good}
\begin{defn}
	A fiber product $X\times_Z Y$ is the limit of $X\to Z\leftarrow Y$.
	You know the drill.
\end{defn}
Equivalently the fiber product of schemes is the fiber product of the functor $\Sch\op\to \Set$ that it represents.
The reason this is worth mentioning is that one can use the fuctor of points perspective to define this as a fiber product 
on sets! 

\begin{thm}
	Fiber products exist in $\Sch$.
\end{thm}
\begin{prf}
	We gonna do this.
	\begin{defn}
		A functor $\calF:\Sch\op\to\Set$ is a \textbf{sheaf} if for all $W\in\Sch$, the induced functor 
		$\calF_W:\mathbf{Open}(W)\op\to\Set$ is a sheaf.
	\end{defn}
	Kinda obvious. Then the idea here is that the Yoneda embedding actually maps into sheaves on schemes (not just presheaves). Furthermore 
	since the category of sheaves has all limits (CHECK THIS! It's easy), this means we are just trying to show that the sheaf $h_X\times_{h_Y}h_Z$ is representable 
	by a scheme.

	\begin{defn}
		A natural transformation $\calG\to \calF$ in $\Func(\Sch\op,\Set)$ is an \textbf{open subfunctor} if, 
		for all schemes $W$ and all $h_W\to F$, the pullback $G\times_F h_W\to h_W$ is representable by 
		an open immersion $h_U\to h_W$ ($U\subseteq W$ open).
	\end{defn}
	We're trying to define a thing like a topology on the category of schemes.
	\begin{defn}
		Given a fucntor $\calF\in\PreSh(\Sch)$, an \textbf{open covering} is a collection of open 
		subfunctors $\calG_i\hookrightarrow\calF$ such that for all $W\in\Sch$ $h_{U_i}=\calG_i\times\calF h_W\to h_W$.
	\end{defn}
	\begin{prop}
		A sheaf $\calF:\PreSh(\Sch)$ is representable by a scheme if and only if $\calF$ admits an open covering by functors 
		representable by affine schemes.
	\end{prop}
	You can try to prove it! It also is done in Demazure and Gabriel although I am pretty sure that is in French and/or inscrutable. 
	Max says it's not too hard, though. You can kinda follow the argument in Hartshorne for the analogous thought coming from the category of locally ringed spaces.

	\begin{lem}
		Let $U,V,W$ be opens in $X,Y,Z$, resp. such that $f(U)\subset W\supset g(V)$. Then $h_U\times_{h_W}h_V\to h_X\times_{h_Z}h_Y$ is an open subfunctor.
	\end{lem}
	Again easy to check. This gives us (from affine coverings of $X,Y,Z$) an open covering by $h_{\Spec A}\times_{h_{\Spec C}}h_{\Spec B}$.

	From here we can use (among other things) that the restricted Yoneda $\Sch\to\PreSh(\Aff)$ is fully faithful.
\end{prf}
You can read all of today's lecture in Vakil chapter 9.
\section{February 3, 2020}
Today Adam is giving the lecture! The title is 

\subsection{Category theory for algebraic geometry}
This lecture is assuming that we know about categories, functors, and natural transformations.
We will be focusing on limits and colimits.

A motivating question: say we have an open cover $\cup U_i=X$. How can we relate the sheaves on $X$ to those on the $U_i$? Saw we have sheaves $\calF_i$ on $U_i$ along with 
isomorphisms $\phi_{ij}:\calF_i|_{U_i\cap U_j}\to \calF_j|_{U_i\cap U_j}$ (along with a cocycle condition we'll discuss later). How can a sheave $\calF$ on $X$ 
be related to these?

Take an open $V\subseteq X$. Then for all $i,j,$ we have maps
\begin{figure}[h]
	\centering
	\begin{tikzcd}
		& \calF(V)\ar[dl]\ar[dr] &\\
		\calF(V\cap U_i)\ar[d] & & \calF(V\cap U_j)\ar[d]\\
		\calF_i(V\cap U_i)\ar[d] & & \calF_j(V\cap U_j)\ar[d]\\
		\calF_i(V\cap U_i\cap U_j)\ar[rr,"\varphi_{ij}"] & & \calF_j(V\cap U_i\cap U_j)
	\end{tikzcd}
\end{figure}

The gluing idea is to define $\calF(V)$ to be the limit of the bottom portion of the diagram. To do this in general,
let $\calC$ be a category and $K:\calJ\to\calC$ for some other category $\calJ$. Define a cone over $K$ with summit $c\in \calC$ 
to be the usual thing. Same for a cone under a functor. Then limits and colimits. Most of this is 
basically the same as Emily Riehl.

He uses the fact that $\operatorname{cone}(-,K):\calC\op\to\Set$ (this assumes $\calJ$ is small) is a functor,
and furthermore that 
\[\Hom_\calC(-,\lim K)\simeq \operatorname{cone}(-,K)\]
which is essentially a restatement of the universal property of a limit.

Some examples:
\begin{itemize}
	\item In $\Set$, the limit of $A\hookrightarrow X\hookleftarrow B$ is $A\cap B$.
	\item Equalizers are limits!
\end{itemize}

\section{February 5th, 2020}
Today Adam is talking more about limits!

Recall that Yoneda says that the natural transformations 
\[\Hom(\Hom_\calC(-,\lim k),\operatorname{cone}(-,K))\simeq\operatorname{cone}(\lim K,K)\]

\subsection{Equilizers}
Some clarity here: an equilizer in $\Ab$ can be a couple things:
\begin{itemize}
	\item The limit of a pair of parallel functors.
	\item It is the same as saying that the following is exact:
	\[0\to A\to B\xrightarrow{f-g} C\]
\end{itemize}

\subsection{Base change}
This is one of the most important (co)limits in this subject (and in number theory and representation theory).
If $A$ is a $k$-algebra and there is an inclusion $k\hookrightarrow L$. How can we extend $A$ to an $L$-algebra? 

Say that $B$ is an extension of $A$ that is also an $L$-algebra. Then if the universe loves us, we should get 
\begin{figure}[h]
	\centering
	\begin{tikzcd}
		k\ar[r]\ar[d] & A\ar[d]\\
		L\ar[r] & B
	\end{tikzcd}
\end{figure}
so if we want the ``best'' one, we need to find the colimit of $L\leftarrow k\to A$. To define this thing we just 
trace through and show that the object with the usual axioms of the tensor product falls out. :)

\begin{prop}
	The following is a pullback (in schemes, locally ringed spaces, etc)
	\begin{figure}[h]
		\centering
		\begin{tikzcd}
			\Spec A\otimes_k L\ar[r]\ar[d] & \Spec L\ar[d]\\
			\Spec A\ar[r] & \Spec k
		\end{tikzcd}
	\end{figure}
\end{prop}
\begin{rmk}
	Why is it called a fiber product? Look at what happens in $\Set$. Look at the limit of the diagram 
	$\ast\stackrel{y}{\hookrightarrow} Y \xleftarrow{f}X$. The pullback is literally the fiber over $y\in Y$.
\end{rmk}
Proving the following is a good excercise. Go try it! It's a fun categorical proof. You can do it by looking at cones. 
\newpage
\begin{lem}
	If $\pi:X\to Y$ is a monomorphism, then in the pullback square 
	\begin{figure}[h]
		\centering
		\begin{tikzcd}
			X'\ar[r]\ar[d,"\pi'"] & X\ar[d,"\pi"]\\
			Y'\ar[r] & Y
		\end{tikzcd}
	\end{figure}

	\noindent $\pi'$ is also mono.
\end{lem}
As a corollary of this, if we have an open $U\subseteq X$ in a scheme and $\pi:Y\to X$ a morphism, then 
$\pi^{-1}(U)$ is open in $Y$.

\subsection{Gluing, baby}
For the setup, remember that $X$ is a topological space and $X=\cup_i U_i$ open cover. We are given sheaves 
$\calF_i$ on $U_i$ together with isomorphisms 
\[b_{ij}:\calF_i|_{U_i\cap U_j}\to\calF_j|_{U_i\cap U_j}.\]

We saw that the right way to glue to an $\calF$ on $X$ is to take 
$\calF(V)$ to be a limit of the diagram we drew yesterday (just the bottom two rows).
Then there are two questions we ask:
\begin{itemize}
	\item Is $\calF|_{U_i}=\calF_i$? This is answered by the \textit{cocycle condition}.
	\item Is $\calF$ a sheaf? This is due to the fact that limits commute with limits.
\end{itemize}

\section{February 7th, 2020}
Max's back. Time to talk about 
\subsection{Separatedness and Properness}
These are replacements for the Hausdorff and Compact conditions, neither of which we get in algebraic geometry. Let's get some definitions 
\begin{defn}
	A morphism $f:X\to Y$ is \textbf{separated} if the diagonal $\Delta:X\to X\times_Y X$ is a closed immersion. This 
	replaces Hausdorff.
\end{defn}
Recall that a closed immersion is a homeomorphism onto a closed subspace such that the map of rings is surjective. 
The prototype you should be thinking of is $\Spec k[x_1,\dots,x_m]/(f_1,\dots,f_n)$ as a closed subspace of $\A m$.
\begin{defn}
	A morphism $f:X\to Y$ is \textbf{closed} if the closed subsets of $X$ are sent to closed subsets of $Y$ (as topological spaces).

	It is \textbf{universally closed} if for all $T\to Y$, the base change map $f_T:X\times_Y T\to T$ is closed.
\end{defn}
\begin{ex}
	Consider $\A1_k\to\Spec k$. This is closed, but not universally closed! Consider the base change via $\A1\to\Spec k$:
	$\A1\times\A1=\A2\to \A1$. In coordinates, we get the map of algebras $k[s]\hookrightarrow k[t,s]$. But the hyperbola 
	cut out by $ts-1$ in $\A2$ projects onto $\A1\setminus 0$ under this map, which is super not closed.

	Is $\A1\to \Spec k$ separated? Hell yes! Look at the ring map $A\times A\to A$ by multiplication. This is surjective.
	And unravelling the definitions gets you the result: every morphism of affines is separated! So we don't really need to check 
	that the inclusion is a topological embedding onto a closed set.
\end{ex}

I (well, Max) claim(s): If $f:X\to Y$ is separated and $\sigma:Y\to X$ is a section, then $\sigma$ is closed.
In fact, check this out: \textbf{any section is a pullback of the diagonal} or even \textbf{``The diagonal is the universal section.''}

\begin{figure}[h]
	\centering
	\begin{tikzcd}
		X\ar[r]\ar[d,bend right,"f",swap]\ar[rr,bend left,"\id"] & X\times_Y \tilde X\ar[r]\ar[d,swap,"f_X"] & \tilde X\ar[d,"f"]\\
		Y\ar[u,"\sigma",swap] \ar[r,"\sigma"] & X\ar[u,bend right,swap,"\Delta"] \ar[r,"f"] & Y
	\end{tikzcd}
\end{figure}
Check it out. Assuming everything is a pullback square (remember the neat two of three thing for pullbacks) we get that $\Delta$ is 
a pullback through itself of the diagonal! This gives us another result: $\Delta$ is closed if and only if sections are universally closed.

So sections are closed, but ot universally closed! For instance the line with two origins is not separated!

\begin{defn}
	A morphism $f:X\to Y$ is \textbf{proper} if it is of finite type, separated, and universally closed.
\end{defn}
Note: $f$ is separated if and only if ``sections of $f$'' are universally closed. An example, $\P n$ is proper over $\Spec\bbZ$.
So of course it is proper over anything.

\section{February 10th, 2020}
Today Taffy is talking about base change!

\subsection{Base change}
Notice this cool thing that we see in topological spaces: let $f:U\hookrightarrow Y$ be an open embedding. Then consider the pullback
\begin{figure}[h]
	\centering
	\begin{tikzcd}
		X\times_Y U\ar[r]\ar[d,"g"] & U\ar[d,"f"]\\
		X\ar[r] & Y
	\end{tikzcd}
\end{figure}
where we see $X\times_Y U\cong f^{-1}(U)$ and $g:f^{-1}(U)\hookrightarrow X$ is the open embedding. Here we just did base change and we saw a property being preserved!

In $\Sch$, we can do the same kind of thing: Let $\calX$ and $\calY$ be schemes over $k$. 

I just listened to the rest. Good talk!

\section{February 12th, 2020}
Today we are going to talk more about seperatedness and properness and in particular the so-called 
\subsection{Valuative Criteria}
The idea here is that checking for separatedness or properness can be quite daunting sometimes. We will give a way to do this!
\begin{defn}
	A domain $R$ is a \textbf{valuation ring} if for every $x\in \kappa(R)^\times$, we have $x\in R$ or $x^{-1}\in R$.
\end{defn}
\begin{lem}
	Consider loal rings contained in a field $k$ (with fraction field $k$). Say that $(S,\frakm_S)$ \textit{dominates} $(R,\frakm_R)$ if $R\subset S$ 
	and furthermore $\frakm_S\cap R=\frakm_R$. This gives a partial ordering. 

	Then the valuation rings in $k$ are precisely the maximal elements for this ordering.
\end{lem}
\begin{defn}
	A morphism $f:X\to Y$ is \textbf{quasi-separated} if the diagonal $\Delta:X\to X\times_Y X$ is quasicompact.
\end{defn}
THis is always true if $X$ is locally Noetherian. This will be true in almost every case you see in real life, because it 
is really necessary to make the entire theory work.

\begin{defn}
	A \textbf{test diagram} for a morphism $f:X\to Y$ is a diagram as in the figure below	
	where $i:R\hookrightarrow k=\kappa(R)$ is a vaulation ring.
	A \textbf{filling} for a test diagram is an arrow $\gamma:\Spec R\to X$ making this diagram commute.
\end{defn}
\begin{figure}[h]\label{fig:testdiagram}
	\centering
	\begin{tikzcd}
		\Spec k\ar[r]\ar[d,"i^\ast"] & X\ar[d,"f"]\\
		\Spec R\ar[r] & Y
	\end{tikzcd}
	\caption{A test diagram for $f$}
\end{figure}
\begin{thm}\label{thm:fillings}
	\begin{enumerate}
		\item A quasiseparated morphism $f:X\to Y$ is separated if and only if every test diagram has at most one filling.
		\item A quasicompact morphism $f:X\to Y$ is universally closed if and only if every test diagram admits at least one filling.
		\item A quasisemparated morphism of finite type $f:X\to Y$ is proper sif and only if every test digram admints precisely one filling.
	\end{enumerate}
\end{thm}
\begin{lem}
	The entire theorem above follows from (b).
\end{lem}
\begin{prf}
	To get (a) from (b), $f$ is quasiseparated iff $\Delta:X\to X\times_YX$ is quasicompact. Note that $f$ is separated iff $\Delta$ is 
	universally closed. Further, note that $\Delta$ is always an immersion. One can show that $\Delta$ is a closed immersion into an open subscheme (look in Hartshorne and use a cover by open affines).

	But then the test diagram for $f$ with two fillings $\alpha,\beta$ gives us a test diagram for $\Delta$ with $(\alpha,\beta)$ along the bottom.
	What does it mean for this new diagram to have a filling? If we ahve $t:\Spec R\to X$, we get that 
	\[(\alpha,\beta)=\Delta\circ t=(t,t)\]
	so we have a filling if and only if $\alpha=\beta$!

	That (b) implies (c) follows from the definitions!
\end{prf}
\begin{defn}
	Given a scheme $X$, a point $s\in X$ is a \textbf{specialization} of a point $t\in X$ (equivalently, $t$ 
	is a generization of $s$) if $s\in\overline{\{t\}}$.
\end{defn}
This is a topological property, but we want to capture it algebraically. We need to pass to limiting algebraic object 
to do it most efficiently. As a mater of notation, we write $t\rightsquigarrow s$.
\begin{lem}
	Suppose $t\rightsquigarrow s$ in $X$ and we are given $K\supset \kappa(t)$ (residue field). Then there exists a vaulation ring $R\subset K=\operatorname{Frac} R$ and 
	morphism $\Spec R\to X$ such that $(0)\mapsto t$ and $\frakm_R\mapsto s$.
\end{lem}
A ``proof'' of this is to look at $\O_{\overline{\{t\}},s}\subset k$ and find $R$ using Zorn's lemma with respect to domination.

\section{February 14th, 2020}
Max recommends probelm 4.10 from Hartshorne. It is the proof of Chow's lemma. It is not an easy proof, 
but it is a very important one.

\subsection{More generization and specialization}

\begin{lem}
	Given a quasicompact morphism $f:X\to Y$, $f(X)$ is closed in $Y$ if and only if it is \textit{closed under specialization}.
\end{lem}
\begin{prf}
	The conclistion is stable udner localiation on $U_i$, so we may assume $Y=\Spec A$. Then since $f$ is quasicompact, 
	$X=\cup_1^n\Spec B_i$ where $\Spec B_i=V_i$. But then 
	\[\Spec\prod B_i=\sqcup V_i\to X\to Y\]
	has the same image. Therefore we can assume that $X=\Spec B$ is affine.

	So we have a morphism $\Spec B\to \Spec A$, giving us a ring map $\varphi:A\to B$. So now if $I=\ker\varphi$,
	we get that the map $X\to Y$ factors through $\Spec A/I\hookrightarrow Y$, which is closed. So we can assume that $\varphi$ is injective.
	We want to show that $f(X)=Y$. \textit{Notice that we haven't checked that the assumption is closed under these reductions.}

	Now take any $p\in \Spec A$ and take a minimal prime $p_0\subseteq p$ (which exists by the standard Zorn argument).
	So now $p_0\rightsquigarrow p$. If $p_0\in f(X)$, then $p\in f(X),$ so $f(X)=Y$. Consider $A\hookrightarrow B$ and localize at $p_0$ (which preserves injectivity).
	and since $A_{p_0}$ has a single prime, you get a map with $p_0$ in the image. Thus you get $p$ and surjectivity.
\end{prf}

Now we are going to prove:
\begin{prf}[of thm.~\ref{thm:fillings}(b)]
	Assume that $f:X\to Y$ is universally closed. We want to show that fillings exist for any test diagram. Notice that fillings for test diagrams are equivalent 
	to finding sections (since we can assume $Y=\Spec R$):
	\begin{figure}[h]
		\centering
		\begin{tikzcd}
			\Spec k\ar[r,"\iota"]\ar[d] & X=X\times_{\Spec R}\Spec R\ar[d]\\
			\Spec R\ar[r,"\id"] & \Spec R\ar[u,bend right,dashed]
		\end{tikzcd}
	\end{figure}
	Let $x\in X$ be in the image of $\iota(\Spec k)$ and let $k$ be the resitude field of $x$. Let $z=\overline{\{x\}}\subseteq X$
	Now $(0)\subseteq f(z)\subset\Spec R$ is closed, so there exists $z\in Z$ such that $f(z)=\frakm_R$. I got lost at the end, but there were some diagrams that 
	boiled down to our ring being a valuation ring.

	Now assume that we have our fillings. Using the lemma, we can show that the image is cloed under specialization (warning: implicit base change being hidden here. If 
	we have $f:X\to Y$, $X_T\to T$).
\end{prf}

\section{February 19th}
Today Christine is talking!
\begin{lem}[Affine Communication Lemma]
	Let $\scrP$ be a property of some affines of a scheme $X$ such that
	\begin{enumerate}
		\item If $\Spec A\hookrightarrow X$ has $\scrP$ then for all $f\in A$, $\Spec A_f\hookrightarrow X$ has $\scrP$.
		\item If $(f_1,\dots,f_n)=A$ and $\Spec A_f\hookrightarrow X$ have $\scrP$ then $\Spec A\hookrightarrow X$ has $\scrP$.
	\end{enumerate}
	Further if $X=\cup_{i\in\scrI}\Spec A_i$ and the $\Spec A\hookrightarrow X$ have $\scrP$, then so does $X$.
\end{lem}

\section{February 21, 2020}
Today we are going to talk about sheaves of modules. Let $(X,\O_X)$ be a ringed space. Then 
\begin{defn}
	An $\O_X$-module (``sheaf of $\O_X$ modules'') is a sheaf of abelian groups $\calF$ along with a morphism 
	\[\O_X\to \iHom(\calF,\calF)\]
	of sheaves of rings. Equivalently it is an action 
	\[\O_X\times\calF\to \calF\]
	or (locally) $\calF(U)$ is an $\O_X(U)$-module and the ``experience is functorial.''

	A homomorphism of $\O_X$-modules is a morphism $\varphi:\calF\to\calG$ of abelian sheaves that is compatible with the $\O_X$-action.
\end{defn}
\begin{lem}
	The category $\lmod{\O_X}$ of $\O_X$ modules is an abelian category.
\end{lem}

\section{February 24, 2020}
We talked about the sheaf of sections of a vector bundle on a manifold. I didn't write it down but it is a good thought.
Here are some 

\subsection{Moar sheavz}
Let $X=\Spec A$ and let $M\in \lmod A$. We create a sheaf of modules $\tilde M$ by saying $\tilde M(D(f))=M_f$. 
The right thing to do is $B=\prod A_{f_i}$. Then max did a bunch of shit. The idea is that this is almost exactly the 
same construction that you use to create the structure sheaf. There is some cochain stuff going on here that is quite cool. 

So the functor $\tilde -:\lmod A\to \lmod{\O_X}$ is a functor and is exact, commutes with $\otimes,$ and satisfies $\Gamma(\Spec A,\tilde M)=M$ and 
for all $p\in X$, $\tilde M_p=M_p$. It is also fully faithful but \textbf{not essentially surjective}.

To show this last idea, consider the space with two points: $R= k[[t]]$ and let $U\subset\Spec R$ be the open point. THen define 
a $\O_{K(R)}$ module on the localization $K(R)$ at $U$. Just let this sheaf be $\tilde K(R)$. But then consider the 
\[\Gamma(\Spec R,\tilde K(R))=0\quad\text{and}\quad\Gamma(K(R),\tilde K(R))=K(R)\]
which doesn't arise as the ``sheafification'' (DON'T read too much into this word although it may kinda be true) of an $R$-module.

\subsection{(Quasi-)Coherent sheaves}
Hey, don't read Hartshorne for coherent sheaves, apparently.

\begin{defn}
	Let $X$ be a scheme. Then an $\O_X$ modules $\scrF$ is \textbf{quasicoherent} if there exists an open affine cover $U_i=\Spec A_i$ and modules $M_i\in\lmod{A_i}$
	such that for all $i$, $\scrF|_{U_i}=\tilde M_i$.
\end{defn}
Better:
\begin{defn}
	$\scrF$ is quasicoherent if and only if there exists an open (not necessarily affine!) cover $U_i\subseteq X$ such that there exists a presentation 
	\[\O_{U_i}^{\oplus I}\to \O_{U_i}^{\oplus J}\to \scrF|_{U_i}\to 0\]
	for all $i$ and for some index sets $I$ and $J$.
\end{defn}

The right thing to read is FAC by Serre where he does this whole thing (in French).

\subsection{What does coherence mean?}
Quasicoherence gives us $\calO^I\to\calO^J\to \scrF\to 0$ is exact. We can say that $\scrF$ is quasicoherent finitely genreated if $J$ can be taken 
to be finite. It is quasicoherent finitely presented if both $I$ and $J$ can be taken to be finite.

A sheaf is \textbf{coherent} if for ANY finite $J$ and surjection $\O^J\to \scrF$, $\ker(\pi)$ is finitely generated.
In the Noetherian case, all of these coincide. 

WARNING! Coherence is not stable under pullback. So what kind of perations can we do? IF we have $f:X\to Y$ is a map of ringed spaces, we get the pullback $f_\ast:\O_X$-mod$\to\O_Y$-mod and we want to have a left 
adjoint $f^\ast$. Recal that we have $f^\sharp:\O_Y\to f_\ast\O_X$, where the codomain acts on $f_\ast\scrF$.

How do we define the adjoint? $f^\sharp$. Wow I just stopped in the middle of a thought.

\section{February 26th, 2020}
We had a functorial experience. We showed that these adjoint maps $f_\ast,f^\ast$ exist on the level of (abelian)
sheaves on $X$ and $Y$ and also they exist (THOUGH ARE DIFFERENT) when we're talking about $\O_X$ and $\O_Y$ modules. 
These also restrict to maps on the quasicoherent sheaves (and finitely presented ones) but not, in general, to the coherent sheaves.

\begin{thm}
	The map $\tilde -:\lmod{A}\to \mathbf{QCoh}_{\Spec A}$ is an equivalence of categories.
\end{thm}
\begin{cor}
	If $X$ is quasi-compact and quasi-separated, then for all $f\in \Gamma(X,\O)$ and $s\in \scrF(D(f))$, and $\scrF\in \mathbf{QCoh}_X$,
	there exists a positive integer $n$ such that $f^ns$ extends to a global section $\sigma\in\Gamma(X,\scrF)$. In other words, 
	all poles of $s$ have finite order.
\end{cor}

\subsection{Next time...}
\begin{prop}
	Let $f:X\to Y$ be a morphism of schemes. Then 
	\begin{enumerate}
		\item $f^\ast\mathbf{QCoh}_Y\subseteq\mathbf{QCoh}_X$ (we already showed this).
		\item If $f$ is quasi-compact, then $f_\ast\mathbf{QCoh}_X\subseteq\mathbf{QCoh}_Y$.
	\end{enumerate}
\end{prop}

\section{March 2nd, 2020}
Sorry, I missed Friday. Today we are continuing to talk about 
\subsection{Closed Subschemes}
We saw last time that there is a bijection between closed subschemes $\iota:Y\hookrightarrow X$ (modulo some relation?) and the collection of quasicoherent sheaves of ideals $\calI\subseteq\O_X$.
Last time Max showed that it is surjective.

To show injectivity, we go through a few steps: we begin by showing that the topological spaces underlying 
two closed immersions $Y$ and $Y'$ with an iso $g:\iota_\ast\O_Y\to \iota'_\ast\O_{Y'}$. Then we continued by 
constructing stuff.

\section{March 4th, 2020}
Today we are talking about sheaves on proj. The 

\end{document}