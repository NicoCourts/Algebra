\documentclass[12pt]{article}

\usepackage{setspace}

\usepackage{amsmath, graphicx, color, fancyhdr, tikz-cd, mdframed, enumitem, framed, adjustbox, bbm, upgreek, xcolor, hyperref, manfnt}
\usepackage[framed,thmmarks]{ntheorem}
\usepackage[style=alphabetic, bibencoding=utf8]{biblatex}
%Set the bibliography file
\bibliography{sources}

\usepackage[T1]{fontenc}
\usepackage[urw-garamond]{mathdesign}
\usepackage{garamondx}

%Replacement for the old geometry package
\usepackage{fullpage}

%Input my definitions
%set up theorem/definition/etc envs
%Problems will be created using their own counter and style
\theoremstyle{break}
\theoreminframepreskip{0pt}
\theoreminframepostskip{0pt}
\newframedtheorem{prob}{Problem}[section]

%solution template
\theoremstyle{nonumberbreak}
\theoremindent0.5cm
\theorembodyfont{\upshape}
\theoremseparator{:}
\theoremsymbol{\ensuremath\spadesuit}
\newtheorem{sol}{Solution}

%Theorems
\definecolor{thmcol}{RGB}{120,100,50}
\theoremstyle{changebreak}
\theoremseparator{}
\theoremsymbol{}
\theoremindent0.5cm
\theoremheaderfont{\color{thmcol}\bfseries} 
\newtheorem{thm}{Theorem}[subsection]

%Lemmas and Corollaries
\theoremheaderfont{\bfseries}
\newtheorem{lem}[thm]{Lemma}
\newtheorem{cor}[thm]{Corollary}
\newtheorem{prop}[thm]{Proposition}

%Create a new env that references a theorem and creates a 'primed' version
%Note this can be used recursively to get double, triple, etc primes
\newenvironment{thm-prime}[1]
  {\renewcommand{\thethm}{\ref{#1}$'$}%
   \addtocounter{thm}{-1}%
   \begin{thm}}
  {\end{thm}}

\setlength\fboxsep{15pt}

%Example
\theoremstyle{break}
\def\theoremframecommand{\colorbox[rgb]{0.9,0.9,0.9}}
\newshadedtheorem{ex}{Example}[section]

%Man, that's really good! Let's use the same thing for definitons.
\newenvironment{def-prime}[1]
  {\renewcommand{\thethm}{\ref{#1}$'$}%
   \addtocounter{thm}{-1}%
   \begin{def}}
  {\end{def}}

%proofs
\theoremstyle{nonumberbreak}
\theoremindent0.5cm
\theoremheaderfont{\sc}
\theoremseparator{}
\theoremsymbol{\ensuremath\spadesuit}
\newtheorem{prf}{Proof}

\theoremstyle{nonumberplain}
\theoremseparator{:}
\theoremsymbol{}
\newtheorem{conj}{Conjecture}

%remarks
\theoremstyle{change}
\theoremindent0.5cm
\theoremheaderfont{\sc}
\theoremseparator{:}
\theoremsymbol{}
\newtheorem{rmk}[thm]{Remark}

%Put page breaks before each part
\let\oldpart\part%
\renewcommand{\part}{\clearpage\oldpart}%

% Blackboard letters
\newcommand*{\bbA}{\mathbb{A}}
\newcommand*{\bbB}{\mathbb{B}}
\newcommand*{\bbC}{\mathbb{C}}
\newcommand*{\bbD}{\mathbb{D}}
\newcommand*{\bbE}{\mathbb{E}}
\newcommand*{\bbF}{\mathbb{F}}
\newcommand*{\bbG}{\mathbb{G}}
\newcommand*{\bbH}{\mathbb{H}}
\newcommand*{\bbI}{\mathbb{I}}
\newcommand*{\bbJ}{\mathbb{J}}
\newcommand*{\bbK}{\mathbb{K}}
\newcommand*{\bbL}{\mathbb{L}}
\newcommand*{\bbM}{\mathbb{M}}
\newcommand*{\bbN}{\mathbb{N}}
\newcommand*{\bbO}{\mathbb{O}}
\newcommand*{\bbP}{\mathbb{P}}
\newcommand*{\bbQ}{\mathbb{Q}}
\newcommand*{\bbR}{\mathbb{R}}
\newcommand*{\bbS}{\mathbb{S}}
\newcommand*{\bbT}{\mathbb{T}}
\newcommand*{\bbU}{\mathbb{U}}
\newcommand*{\bbV}{\mathbb{V}}
\newcommand*{\bbW}{\mathbb{W}}
\newcommand*{\bbX}{\mathbb{X}}
\newcommand*{\bbY}{\mathbb{Y}}
\newcommand*{\bbZ}{\mathbb{Z}}
%Fraktur letters
\newcommand*{\frakA}{\mathfrak{A}}
\newcommand*{\frakB}{\mathfrak{B}}
\newcommand*{\frakC}{\mathfrak{C}}
\newcommand*{\frakD}{\mathfrak{D}}
\newcommand*{\frakE}{\mathfrak{E}}
\newcommand*{\frakF}{\mathfrak{F}}
\newcommand*{\frakG}{\mathfrak{G}}
\newcommand*{\frakH}{\mathfrak{H}}
\newcommand*{\frakI}{\mathfrak{I}}
\newcommand*{\frakJ}{\mathfrak{J}}
\newcommand*{\frakK}{\mathfrak{K}}
\newcommand*{\frakL}{\mathfrak{L}}
\newcommand*{\frakM}{\mathfrak{M}}
\newcommand*{\frakN}{\mathfrak{N}}
\newcommand*{\frakO}{\mathfrak{O}}
\newcommand*{\frakP}{\mathfrak{P}}
\newcommand*{\frakQ}{\mathfrak{Q}}
\newcommand*{\frakR}{\mathfrak{R}}
\newcommand*{\frakS}{\mathfrak{S}}
\newcommand*{\frakT}{\mathfrak{T}}
\newcommand*{\frakU}{\mathfrak{U}}
\newcommand*{\frakV}{\mathfrak{V}}
\newcommand*{\frakW}{\mathfrak{W}}
\newcommand*{\frakX}{\mathfrak{X}}
\newcommand*{\frakY}{\mathfrak{Y}}
\newcommand*{\frakZ}{\mathfrak{Z}}
\newcommand*{\fraka}{\mathfrak{a}}
\newcommand*{\frakb}{\mathfrak{b}}
\newcommand*{\frakc}{\mathfrak{c}}
\newcommand*{\frakd}{\mathfrak{d}}
\newcommand*{\frake}{\mathfrak{e}}
\newcommand*{\frakf}{\mathfrak{f}}
\newcommand*{\frakg}{\mathfrak{g}}
\newcommand*{\frakh}{\mathfrak{h}}
\newcommand*{\fraki}{\mathfrak{i}}
\newcommand*{\frakj}{\mathfrak{j}}
\newcommand*{\frakk}{\mathfrak{k}}
\newcommand*{\frakl}{\mathfrak{l}}
\newcommand*{\frakm}{\mathfrak{m}}
\newcommand*{\frakn}{\mathfrak{n}}
\newcommand*{\frako}{\mathfrak{o}}
\newcommand*{\frakp}{\mathfrak{p}}
\newcommand*{\frakq}{\mathfrak{q}}
\newcommand*{\frakr}{\mathfrak{r}}
\newcommand*{\fraks}{\mathfrak{s}}
\newcommand*{\frakt}{\mathfrak{t}}
\newcommand*{\fraku}{\mathfrak{u}}
\newcommand*{\frakv}{\mathfrak{v}}
\newcommand*{\frakw}{\mathfrak{w}}
\newcommand*{\frakx}{\mathfrak{x}}
\newcommand*{\fraky}{\mathfrak{y}}
\newcommand*{\frakz}{\mathfrak{z}}
% Caligraphic letters
\newcommand*{\calA}{\mathcal{A}}
\newcommand*{\calB}{\mathcal{B}}
\newcommand*{\calC}{\mathcal{C}}
\newcommand*{\calD}{\mathcal{D}}
\newcommand*{\calE}{\mathcal{E}}
\newcommand*{\calF}{\mathcal{F}}
\newcommand*{\calG}{\mathcal{G}}
\newcommand*{\calH}{\mathcal{H}}
\newcommand*{\calI}{\mathcal{I}}
\newcommand*{\calJ}{\mathcal{J}}
\newcommand*{\calK}{\mathcal{K}}
\newcommand*{\calL}{\mathcal{L}}
\newcommand*{\calM}{\mathcal{M}}
\newcommand*{\calN}{\mathcal{N}}
\newcommand*{\calO}{\mathcal{O}}
\newcommand*{\calP}{\mathcal{P}}
\newcommand*{\calQ}{\mathcal{Q}}
\newcommand*{\calR}{\mathcal{R}}
\newcommand*{\calS}{\mathcal{S}}
\newcommand*{\calT}{\mathcal{T}}
\newcommand*{\calU}{\mathcal{U}}
\newcommand*{\calV}{\mathcal{V}}
\newcommand*{\calW}{\mathcal{W}}
\newcommand*{\calX}{\mathcal{X}}
\newcommand*{\calY}{\mathcal{Y}}
\newcommand*{\calZ}{\mathcal{Z}}
% Script Letters
\newcommand*{\scrA}{\mathscr{A}}
\newcommand*{\scrB}{\mathscr{B}}
\newcommand*{\scrC}{\mathscr{C}}
\newcommand*{\scrD}{\mathscr{D}}
\newcommand*{\scrE}{\mathscr{E}}
\newcommand*{\scrF}{\mathscr{F}}
\newcommand*{\scrG}{\mathscr{G}}
\newcommand*{\scrH}{\mathscr{H}}
\newcommand*{\scrI}{\mathscr{I}}
\newcommand*{\scrJ}{\mathscr{J}}
\newcommand*{\scrK}{\mathscr{K}}
\newcommand*{\scrL}{\mathscr{L}}
\newcommand*{\scrM}{\mathscr{M}}
\newcommand*{\scrN}{\mathscr{N}}
\newcommand*{\scrO}{\mathscr{O}}
\newcommand*{\scrP}{\mathscr{P}}
\newcommand*{\scrQ}{\mathscr{Q}}
\newcommand*{\scrR}{\mathscr{R}}
\newcommand*{\scrS}{\mathscr{S}}
\newcommand*{\scrT}{\mathscr{T}}
\newcommand*{\scrU}{\mathscr{U}}
\newcommand*{\scrV}{\mathscr{V}}
\newcommand*{\scrW}{\mathscr{W}}
\newcommand*{\scrX}{\mathscr{X}}
\newcommand*{\scrY}{\mathscr{Y}}
\newcommand*{\scrZ}{\mathscr{Z}}

%Section break
\newcommand*{\brk}{
\rule{2in}{.1pt}
}

%General purpose stuff
\DeclareMathOperator{\Aut}{Aut}
\DeclareMathOperator{\ch}{char}
\DeclareMathOperator{\rank}{rank}
\DeclareMathOperator{\End}{End}
\let\Im\relax
\DeclareMathOperator{\Im}{Im}

%Category Theory
\DeclareMathOperator{\Hom}{Hom}
\let\hom\relax
\DeclareMathOperator{\hom}{hom}
\DeclareMathOperator{\id}{id}
\DeclareMathOperator{\coker}{coker}
\DeclareMathOperator{\colim}{colim}
\DeclareMathOperator{\invlim}{\lim_{\leftarrow}}
\DeclareMathOperator{\dirlim}{\lim_{\rightarrow}}

%Commutative Algebra
\DeclareMathOperator{\gldim}{gldim}
\DeclareMathOperator{\projdim}{projdim}
\DeclareMathOperator{\injdim}{injdim}
\DeclareMathOperator{\findim}{findim}
\DeclareMathOperator{\flatdim}{flatdim}
\DeclareMathOperator{\depth}{depth}

%Common Categories
%\newcommand*{\modR}{\mathbf{mod}\text{-}R}
%\newcommand*{\Rmod}{R\text{-}\mathbf{mod}}
\newcommand{\rmod}[1]{\mathbf{mod}\text{-}#1}
\newcommand{\lmod}[1]{#1\text{-}\mathbf{mod}}
\DeclareMathOperator{\Vectk}{\mathbf{Vect}_k}
\DeclareMathOperator{\Ch}{\mathbf{Ch}}
\newcommand*{\Ab}{\mathbf{Ab}}
\newcommand*{\Grp}{\mathbf{Grp}}
\newcommand*{\Alg}{\mathbf{Alg}_k}
\newcommand*{\Ring}{\mathbf{Ring}}
\newcommand*{\K}{\mathbf{K}}
\newcommand*{\D}{\mathbf{D}}
\newcommand*{\Db}{\mathbf{D}^b}
\newcommand*{\Dpos}{\mathbf{D}^+}
\newcommand*{\Dneg}{\mathbf{D}^-}
\newcommand*{\Dbperf}{\mathbf{D}^b_{\text{perf}}}
\newcommand*{\Dsing}{\mathbf{D}_{sing}}
\newcommand{\CRing}{\mathbf{CRing}}
\DeclareMathOperator{\stmod}{\mathbf{stmod}}
\DeclareMathOperator{\StMod}{\mathbf{StMod}}
\DeclareMathOperator{\sHom}{\underline{Hom}}

%Homological algebra
\DeclareMathOperator{\cone}{cone}
\DeclareMathOperator{\HH}{HH}
\DeclareMathOperator{\Der}{Der}
\DeclareMathOperator{\Ext}{Ext}
\DeclareMathOperator{\Tor}{Tor}

%Lie algebras
\DeclareMathOperator{\ad}{ad}
\newcommand*{\gl}{\mathfrak{gl}}
\let\sl\relax
\newcommand*{\sl}{\mathfrak{sl}}
\let\sp\relax
\newcommand*{\sp}{\mathfrak{sp}}
\newcommand*{\so}{\mathfrak{so}}

% Hacks and Tweaks
% Enumerate will automatically use letters (e.g. part a,b,c,...)
\setenumerate[0]{label=(\alph*)}
% Always use wide tildes
\let\tilde\relax
\newcommand*{\tilde}[1]{\widetilde{#1}}
%raise that Chi!
\DeclareRobustCommand{\Chi}{{\mathpalette\irchi\relax}}
\newcommand{\irchi}[2]{\raisebox{\depth}{$#1\chi$}} 



%Shade definitions
\theoremindent0cm
\theoremheaderfont{\normalfont\bfseries} 
\def\theoremframecommand{\colorbox[rgb]{0.9,1,.8}}
\newshadedtheorem{defn}[thm]{Definition}

%%%%%%%%%%%%%%%%%%%%%%%%%%%%%%%%%%%%%%%%%%%%%%%%%%%%%%%%%%%%%%%%%%%%%%
%%%%%%%%%%%%%%%%%%%%%%% Customize Below %%%%%%%%%%%%%%%%%%%%%%%%%%%%%%
%%%%%%%%%%%%%%%%%%%%%%%%%%%%%%%%%%%%%%%%%%%%%%%%%%%%%%%%%%%%%%%%%%%%%%

%header stuff
\setlength{\headsep}{24pt}  % space between header and text
\pagestyle{fancy}     % set pagestyle for document
\lhead{Algebraic Groups} % put text in header (left side)
\rhead{Notes by Nico Courts} % put text in header (right side)
\cfoot{\itshape p. \thepage}
\setlength{\headheight}{15pt}
%\allowdisplaybreaks

% Document-Specific Macros
\DeclareMathOperator{\SL}{SL}


\begin{document}
%make the title page
\title{Algebraic Groups\vspace{-1ex}}
\author{A course by Jarod Alper and Julia Pevtsova\\
Notes by Nico Courts}
\date{Autumn 2019/ Winter 2020}
\maketitle

\begin{abstract}
	The topic of algebraic groups is a rich subject combining both group-theoretic and algebro-geometric-theoretic techniques. Examples include the general linear group $GL_n$, 
	the special orthogonal group $SO_n$ or the symplectic group $Sp_n$. Algebraic groups play an important role in algebraic geometry, representation theory and number theory.

	In this course, we will take the functorial approach to the study of linear algebraic groups (more generally, affine group schemes) equivalent to the study of Hopf algebras. 
	The classical view of an algebraic group as a variety will come up as a special case of a smooth algebraic group scheme. Our algebraic approach will be independent (even complementary) to the analytic approach taken in the course on Lie groups.
\end{abstract}

\section{September 25, 2019}
\subsection{Group objects}
Let $\calC$ be a category with a final object and finite products.
\begin{defn}
	A \textbf{group object $G$ in $\calC$} is an object in $\calC$ along with multiplication, identity, and inverse morphisms
	satisfying the usual axioms.
\end{defn}

One thing is that we are using that there is a final object $\ast$ along with our identity morphism $e:\ast\to G$.
Here Jarrod explictly used the fact that there is a unique map to $\ast$.

\begin{ex}
	If $\calC$ is $\operatorname{Set}$, then $G$ is a group. If $\calC=\operatorname{Top}$, then $G$ is a topological group, smooth manifolds give Lie groups, and finally (interesting to us):
\end{ex}
\begin{defn}
	Let $S$ be a scheme and let $\calC$ be the category of schemes over $S$. Then a group object $G$ in $\calC$ is 
	a \textbf{group scheme over $S$.}
\end{defn}

WHen $k$ is a field and $\calC$ is schemes of finite type over $k$, we get a group scheme of finite type over $k$. There is not a great consensus on what makes an \textbf{algebraic group}, 
but this is what we will use.

When we instead restrict to \textit{affine schemes} we get an affine groupe scheme of finite tipe over $k$, or a \textbf{linear algebraic group.}

\subsection{Examples}
$\bbG_m=\operatorname{Spec} k[t]_t$ is one. 

If we consider the map $f:\bbG_m\to \bbG_m$ which on the level of elements sends $t\mapsto t^p$, the kernel is 
\[\mu_p=\ker(f)=\operatorname{Spec}k[t]/(t^p-1)\]
and that's great, but when $\ch k=p$, this causes the group scheme to be \textbf{unreduced}. This is (apparently) a case when you need to use schemes.

\subsection{The Functorial Approach}
Let $\calC$ be a category with object $X$. Define the functor $h_X:\calC^{op}\to \mathbf{Set}$ where 
\[h_X(Y)=\Hom_\calC(Y,X).\]

Then we have 
\begin{lem}[Yoneda]
	Let $G:\calC^{op}\to\mathbf{Set}$ be a functor. There is a natural bijection
	\[G(X)\simeq \operatorname{Nat}(h_X,G).\]
\end{lem}
\begin{prop}
	A group object $G$ in $\calC$  is the same as an aobject $X\in\calC$ together with a choice of factorization of 
	$h_X:\calC\to\mathbf{Set}$ through $\mathbf{Grp}$.
\end{prop}

\subsection{Exercises}
\begin{enumerate}
	\item Spell out all the details of the proof of the above propositon.
	\item Given a group object $G$, define in two ways what it means for it to act on another object. (In coordinates and functorially).
\end{enumerate}

\subsection{Some Interesting Facts}
If we had to write down five results that we'd like to get out of this class:
\begin{prop}
	Every affine group scheme of finite type over a field embeds into $GL_n$ as a closed subgroup.
\end{prop}
\begin{thm}[Chevalley's Theorem]
	Let $G$ be a finite type group scheme over a field. Then it factors as 
	\[1\to H\to G\to A\to 1\]
	where $A$ is abelian and $H$ is affine (linear algebraic).
\end{thm}
\begin{prop}
	If $G$ is an affine group scheme of finite type over $k$, then we have af actorization
	\[1\to U\to G\to R\to 1\]
	where $U$ is unipotent and $R$ is reductive.
\end{prop}
\begin{prop}
	$H\subseteq G$  a subgroup scheme. Then $G/H$ is a projective scheme.
\end{prop}
Finally we want to talk about Tanakka duality and how the representations of $G$ define $G$ itself.

\section{September 27th, 2019}
Last time we defined a group scheme (a group object in the category of schemes over a base scheme). We also mentioned that 
You could define it as a map $h_G:\mathbf{Sch}/S\to \mathbf{Set}$ along with a factorization through $\mathbf{Grp}$.

We defined an \textbf{algebraic group} over $k$ as a group scheme over $\operatorname{Spec} k$ of finite type and a \textbf{linear algebraic group}
to be an \textit{affine} group scheme over $k$ of finite type.

\subsection{Hopf Algebras}
Let $G=\Spec A$ be a linear algebraic group over $k$. I have seen most of these before (see Waterhouse or my Hopf algebra notes)
\begin{rmk}
	One think I haven't seen explicitly before: Notice that the augmentation ideal $\ker \varepsilon$, where $\varepsilon$ is the counit,
	is the (maximal!) ideal corresponding in the algebro-geometric sense to the identity element in $G$.
\end{rmk}
\begin{defn}
	A \textbf{Hopf algebra} is ...
\end{defn}
\begin{defn}
	Let $G$ be an algebraic group over $k$. Then if $h_G$ factors through $\Ab$, $G$ is called \textbf{commutative.}
\end{defn}

\subsection{Some Examples}
\begin{rmk}
	Note that to define a functor from schemes over $k$, is suffices to define it on affine schemes, thereby defining 
	the (Zariski) local behavior of any such map. Thus we really only need to consider maps in $\Alg$.
\end{rmk}
\begin{itemize}
	\item $\bbG_a$. Here we can define it as a functor that sends $S\mapsto\Gamma(S,\calO_S)$. Geometrically, $\bbG_a=\bbA^1$ where the multiplication is addition, inverses send $x\mapsto -x$ and the unit is the zero map.
	The Hopf algebraic picture is the usual dual thing.
	\item $\bbG_m$ as a scheme isthe map $S\mapsto \Gamma(S,\calO_S)^\ast$. In the geometric picture, $\bbA^1\setminus\{0\}$ and the algebra structure comes from multiplciation. Hopf is pretty easy.
	\item $\GL_n$ is a scheme that sends
	\[S\mapsto \left\{A=(a_{ij}): a_{ij}\in\Gamma(S,\calO_S), \det(A)\in\Gamma(S,\calO_S)^\ast\right\}\]
	the algebra is $\bbA^{n\times n}\setminus \{\det = 0\}$ with the usual multiplication. The coalgebra structure can be seen in the book.
\end{itemize}
This one requires some more explaination so I am setting it apart.
\begin{ex}
	Let $V$ be a finite dimensionatl vector space over $k$. Then we can define the algebraic group $V_a$ which sends 
	\[S\mapsto \Gamma(S,\calO_S)\otimes_k V.\]
	Geometrically we are looking at $\bbA(V)=\Spec\Sym ^\ast V^\vee\simeq \Spec k[x_1,\dots,x_n]$ where $n=\dim V$.
\end{ex}

What about finite groups? As a scheme, we want $G=\sqcup_{g\in G}\Spec k$. The functor sends $S\mapsto \operatorname{Mor}_{\operatorname{Set}}(\pi_0(S),G)$,
or maps from the connected components into $G$.

\begin{ex}
	Now consider the $n^{th}$ roots of unity: as a scheme, $\mu_n=\Spec k[t]/(t^n-1)\subseteq \bbG_m$.
	If both $k=\bar k$ and $\ch k\nmid n$, then $\mu_n\cong\bbZ/n\bbZ$.

	But if (e.g.) $k=\bbQ$, then $\mu_3$ is $\bbQ[t]/(t^3-1)=\Spec\bbQ\sqcup \Spec\bbQ(\xi)$ where $\xi$ is a primitive third root of unity.

	If, on the other hand, $k=\bar \bbF_3$ and consider $\mu_3$, we get a single point with residye field $\bar\bbF_3$.
\end{ex}

\begin{ex}
	If we are in the case of positive characteristic, then we get an algebraic group $\alpha_p$. Here the scheme is $\Spec k[x]/x^p$ and functorially it 
	is the map $S\mapsto \{F\in\Gamma(S,\calO_S)|f^p=0\}$.
\end{ex}

\subsection{Matrix Groups}
We already defined $\GL_n$, but we can also define 
\[\operatorname{SL}_n:S\mapsto\{A=(a_{ij})|\det A=1\}\]
with scheme $\Spec k[x_{ij}]/(\det-1)$.

We also have the (upper) triangular matrices $T_n$ and unitary group $U_n$ and diagonal group $D_n$

\begin{defn}
	Let $G$ be a linear algebraic group. Then 
	\begin{itemize}
		\item $G$ is a \textbf{vector group} if $G\cong V_a$ for some finite dimensional $V$.
		\item $G$ is a \textbf{split torus} if $G\cong \bbG_m^n$.
		\item $G$ is a \textbf{torus} if there is a field extention $k\to k'$ such that 
		\[G\times_{\Spec k}\Spec k'\cong \bbG^n_{m,k'}\]
	\end{itemize}
\end{defn}

\section{September 30th, 2019}
Another example to consider:
\begin{ex}
	Let $G=\operatorname{PGL}_n$, the projective linear group. Recall we want to define this as $\GL_n/k^\ast$ (from group theory). To do this for algebraic groups,
	we define 
	\[\operatorname{PGL}_n=\operatorname{Proj}k[x_{ij}]_{det}:= \Spec (k[x_{ij}]_{det})_0\]

	The geometric picture is difficult since we haven't yet defined quotients, but as a functor we say $\operatorname{PGL}_n$ 
	is $\Aut(\bbP^n)$, the functor that sends $S\mapsto \Aut(\bbP_S^n)$ where $\bbP^n_S=\bbP_k^n\times_{\Spec k} S$.
\end{ex}

\subsection{Non-affine group schemes}
\begin{ex}
	Let $\lambda\ne 0,1$ be an element in $k$. Then we can define the elliptic curve 
	\[E_\lambda=V(y^2z-x(x-z)(x-\lambda z))\subset \bbP^2\]
	Which gives us a double cover over $(0,1)$ and $(\lambda,\infty)$ with singleton fiber (ramified) over $0,1,$ and $\lambda$.

	Then for any $\lambda\ne 0,1$, $E_\lambda$ is a \textbf{projective} group scheme.
\end{ex}
\begin{rmk}
	If you look at the $\bbC$-points, you get $E_\lambda(\bbC)=\Lambda_\lambda$, giving you a torus. Recall (from e.g. complex analysis) that the 
	moduli here is $\operatorname{SL}_2(\bbZ)$ of all elliptic curves.
\end{rmk}

\subsection{Abelian Varieties}
\begin{defn}
	An \textbf{abelian variety over $k$} is asmooth, geometrically connected ($A\times_{\Spec k}\Spec\bar k$ is connected), proper group scheme $A$ over $k$.
\end{defn}
\begin{ex}
	Over $\bbC$, $\bbC^g/\Lambda$ where $\Lambda\cong \bbZ^{2g}\subseteq \bbC^g$ gives us a genus $g$ example.
\end{ex}
\begin{thm}
	Any abelian variety over $k$ is commutative and projective.
\end{thm}
\begin{thm}[Chevalley]
	If $G$ is any group scheme, then the sequence 
	\[1\to H\to G\to A\to 1\]
	is exact, where $H$ is a linear algebraic group (affine!) and $A$ is an abelian variety.
\end{thm}
\begin{ex}
	Let $X\to \Spec k$ be a geometrically integral projective scheme (proper may suffice). The idea here is that over $\bbC$
	the rings over every open set are integral domains.

	Now consider the \textbf{Picard functor} $\operatorname{Pic}_X:\operatorname{Pic}:\mathbf{Sch/k}\to \Grp$ sending 
	\[S\mapsto \operatorname{Pic}(X_S=X\times_k S)/p^k\operatorname{Pic(S)}\]
\end{ex}
\begin{thm}
	$\operatorname{Pic}_X$ is represented by a scheme locally of finite type, thus $\operatorname{Pic}_X^0$, the connected 
	component of the identity in $[\calO_X]\in\operatorname{Pic}_X$ is an abelian variety.
\end{thm}
\subsection{Relative Group Schemes}
\begin{ex}
	Consider $\bbG_{m,\bbZ}=\Spec \bbZ[t]_t$. Then $G_{m,S}=\bbG_{m,\bbZ}\times_{\Spec \bbZ}S$. 
	In the case that $S=\Spec R$, $\bbG_{m,S}=\Spec R[t]_t$.
\end{ex}

\begin{ex}
	Let $\bbA^1=\Spec k[x]$ and define $G=\Spec k[x,y]_{xy+1}\subseteq\bbA^2$. Notice this is the plane minus a hyperbola.

	Define $\cdot:G\times_{\bbA^1}G\to G$ to be given by 
	\[(x,y)\cdot(x,y')=(x,xyy'+y+y')\]

	Then the thing here is the fiber (think vertical line in the plane!) over 0 is $\bbG_a$ and is isomorphic to $\bbG_m$ otherwise.
\end{ex}

\begin{ex}
	Let $\calE_\lambda=V(y^2z-x(x-z)(x-\lambda z))$ over $\Spec k[\lambda]$. Then when $\lambda=0$, we get the nodal cubic given by 
	$y^2z-x^2(x-z)$ (node at the origin). 

	Now if you look at the connected component around 0 of $\Aut(\calE_\lambda)/\bbA_\lambda$, you actually find (when $\lambda=0$) 
	that $\bbG_m\cong\Aut(\calE_0)^0$.
\end{ex}

\subsection{Some definitions}
\begin{defn}
	A \textbf{homomorphism} $\phi:G\to G$ of group schemes over $S$ is a map $\phi:H\to G$ of schemes such that 
	\begin{center}
		\begin{tikzcd}
			H\times_S H\ar[r,"m_H"]\ar[d,"\phi\times\phi"] & H\ar[d]\ar[d,"\phi"]\\
			G\times_S G\ar[r,"m_G"] & G
		\end{tikzcd}
	\end{center}
\end{defn}
\begin{prob}
	Show that this automoatically imples that the identity and inversion maps are respected as well (automatically).
\end{prob}

\begin{defn}
	A \textbf{subgroup of $G\to S$} is a subscheme $H\subseteq G$ such that $H(T)\le G(T)$ for all $T$ over $S$.
\end{defn}
\begin{prob}
	Show that $\ker(\phi)\subseteq H$ is a subgroup.
\end{prob}
\begin{rmk}
	This gives you a nice way to construct new group schemes. For example, the following are exact:
	\[1\to \operatorname{SL}_n\to GL_n\xrightarrow{\det} \bbG_m\to 1\]
	and 
	\[1\to \bbG_m\to \GL_n\to \operatorname{PGL}_n\to 1\]
\end{rmk}

\begin{prop}
	Let $G\to S$ be a group scheme. Then $G\to S$ is separated if andy only if $e:S\to G$ is a closed immersion.
\end{prop}
\begin{prf}
	The idea here is that $S\to G$ is a closed immersion. Then we consider the map $m\circ(\id,S):G\times_SG\to G$
	and consider this along with the diagonal map $\Delta:G\to G\times_S G$ and this is a pullback square!
\end{prf}
\begin{cor}
	Any group scheme over $k$ is separated.
\end{cor}
The idea is going to be that if $X$ is any scheme over $k$, then any point $X\in X(k)$ is closed.

\section{October 2, 2019}
Notice that a \textbf{relative group scheme} (referred to in last lecture) refers to a groups scheme over an arbitrary 
base scheme $S$.

\subsection{Properties of schemes}
Today we are going to be talking about reducedness, connectedness, irreduciblility, regularity, and smoothness.

Recall that a scheme $X$ is \textbf{reduced} if and only if $\forall x\in X$, $\calO_{X,x}$ is reduced. An example of a non-reduced 
scheme is $\Spec k[x]/(x^2)$.
\begin{defn}
	We say a scheme $X$ over $k$ is \textbf{geometrically reduced} if for all field extensions $k'/k$,
	\[X_{k'}=X\times_{\Spec k}\Spec k'\]
	is reduced.
\end{defn}
\begin{rmk}
	It is equivalent that $X_{\bar k}$ is reduced if and only if every $k'/k$ is purely inseperable (I think).
\end{rmk}
\begin{rmk}
	If $k$ is perfect, then $X$ is reduced if and only if $X$ is geometrically reduced.
\end{rmk}
\begin{defn}
	A local ring $(A,\frakm)$ is \textbf{regular} if $\dim _{\text{Krull}}A=\dim_{A/\frakm} \frakm/\frakm^2$
\end{defn}
\begin{defn}
	A scheme $X$ is regular if for all $x\in X$, $\calO_{X,x}$ is regular.
\end{defn}
\begin{rmk}
	If $X\to \Spec k$ and $x\in X(k)$, the tangent space at $x$ is 
	\[T_{X,x}=(\frakm/\frakm^2)^\vee=\{f:\Spec k[\varepsilon]/\varepsilon^2\to X|0\mapsto x\}\]
\end{rmk}
\begin{rmk}
	Notice that if $X\to \Spec k$ is regular and $k'/k$ is a field extension, then $X_{k'}$ is not necessarily regular.
\end{rmk}

\begin{defn} 
	A Scheme $X\to \Spec k$ of finite type is \textbf{smooth} if $X_{\bar k}$ is regular.
\end{defn}

\subsection{Facts about algebraic groups}
Then we can return to the proposition we want to prove:
\begin{prop}
	Let $G\to \Spec k$ be an algebraic group. Then $G$ is geometrically reduced if and only if $G$ is smooth over $\Spec k$.
\end{prop}
\begin{prf}
	Smoothness over $k$ implies reducedness. Now since we are only interested in the algebraic closure of $k$, we can say $k=\bar k$. Because $G$ 
	is reduced, there exists a nonempty open $U\subseteq G$ that is smooth. Then since $G(k)\subseteq |G|$ is dense in $G$ (as a topological space) 
	and Then $G=\cup_{g\in G(k)}m_g(U)$ for our smooth $U$, and this gives us a smooth cover of $G$.
\end{prf}

We will see next itme that all linear algebraic groups over $k$ where $\ch k=0$ are all 
geometrically reduced (and thus smooth).

\subsection{Connectedness}
Let $G$ be an algebraic group over $k$. Then we have our maps $e:\Spec k\to G$, so consider it as $e\in G(k)$.
Let $G^0\subseteq G$ be the connected component of $e$. It is both open and closed.
\begin{rmk}
	If $X\to \Spec k$ is of finite type and $x\in X(k)$, then $X$ being connected implies that $X$ is geometrically connected.
\end{rmk}
This establishes that $G^0$ is actually geometrically connected! We actually will see
\begin{prop}
	$G^0\subseteq G$ is an (open and closed) algebraic subgroup.
\end{prop}
The idea here is that $G^0\times G^0$ is connected, so the image of the multipication map on this set lands in a connected component (since it is connected).
Since $e\in G^0$, and $m(e,e)=e\in G^0$, this shows that the multiplication map restricts to a well-defined map $G^0\times G^0\to G^0$. A similar argument 
goes through for the inverset map, etc.

The upshot here is that if $G$ is an algebraic group, then there exists a factorization 
\[1\to G^0\to G\to \pi_0(G)\to 1\]
where $\pi_0(G)$ is given the structure of a discrete group.
\begin{rmk}
	Now we also have that $(G^0)_{k'}=(G_{k'})^0$ for all $k'/k$. The idea is to get a map of of one into the other and then use clopenness and connectedness 
	to show they are equal.
\end{rmk}

\begin{prop}
	A connected algebraic group over $k$ is irreducible.
\end{prop}
\begin{prf}
	We can assume $k=\bar k$. Suppose $G= X\cup Y$, where both are closed, $X$ is irreducible, 
	and $X\cap Y\ne\varnothing$. Thus there exists an element $g\in X\setminus Y$. That is, $g$ lies in a single irreducible component.

	But then using the multiplication by $h$ map on $G$, we get to every other point in $G$, so every point 
	is in a single irreducible component. But the intersection was nontrivial! Or something.
\end{prf}

\begin{prop}
	If $G_{\text{red}}$ is geometrically connected, then $G_{\text{red}}\subseteq G$ is a subgroup. In particular, 
	if $k$ is perfect, then $G_{\text{red}}$ is a subgroup of $G$.
\end{prop}
\begin{rmk}
	$X$ is geometrically reduced implies that $X\times X$ is geometrically reduced.
\end{rmk}

\section{October 4, 2019}
Some review. Let $G$ be an algebraic group and denote $e:\Spec k\to G$ be the identity. We saw a lot of propositions last time.

Now let $k$ be a nonperfect field and take $t\in k\setminus k^p$. Then define 
\[G\eqdef V(x^{p^2}-tx^p)\subseteq \bbG_a\]
which Milne claims is not reduced. We can see why it's not geometrically reduced, but we're missing the details here.

\subsection{Another special case}
\begin{thm}
	When $k=\bar k$, $G$ is smooth if and only if 
	\[\dim T_e \red G=\dim T_e G.\]
\end{thm}
\begin{rmk}
	When $G$ is smooth, it is reduced, so the equality is clear. For the other direction, we get that $k$ is pefect, so $\red G$
	which is geoemetrically reduced if and only if $G$ is smooth. But 
	\[\dim G\le \dim T_e G=\dim T_e\red G=\dim \red G=\dim G\]
\end{rmk}
\begin{thm}
	If $G$ is a linear algebraic group over $k$ and $\ch k=0$, $G$ is smooth,
\end{thm}
\begin{prf}
	We can assume $k=\bar k$. Then set $G=\Spec A$ where $A$ is a Hopf algebra. THen we get Hopf algebra maps $m^\ast$ and $e^\ast$.
	Notice that the augmentation ideal $\frakm=\ker(e^\ast)$ is a maximal ideal.

	Then we want to prove 
	\begin{enumerate}
		\item $A\cong\frakm\oplus k$ as a $k$-vector space (obvious).
		\item $\forall a\in\frakm$, $m^\ast(a)-a\otimes 1-1\otimes a\in \frakm\otimes \frakm.$
	\end{enumerate}
	To see the second, notice that $m^\ast(a)-a\otimes 1-1\otimes a$ is in the kernel of 
	\[e^\ast\otimes \id:A\otimes A\to k\otimes A.\]
	This is clear from the commutative diagram
	\begin{center}
		\begin{tikzcd}
			k\otimes A & A\otimes A\ar[l,"e^\ast\otimes\id"] \\
			& A\ar[ul,"\sim"]\ar[u,"m^\ast"]
		\end{tikzcd}
	\end{center}
	Then we conclude $f\in \ker(e^\ast\otimes\id)\cap\ker(\id\otimes e^\ast)=A\otimes\frakm\cap\frakm\otimes A$ by a symmetric argument.
	Finally we notice that $A\otimes\frakm\cap\frakm\otimes A=\frakm\otimes\frakm$, and so $f$ lies in this ideal.

	Now we want to show that $\dim T_e G=\dim_k\frakm/\frakm^2=\dim_k\frakm/(\sqrt{0}+\frakm^2)=\dim T_e\red G$. It suffices to show that 
	for all $a\in\sqrt{0}$, $a\in \frakm^2$. Suppose the opposite--so let $a\in\sqrt{0}\setminus \frakm^2$. Consider the diagram 
	\begin{center}
		\begin{tikzcd}
			A\ar[r]\ar[d] &A_\frakm\ar[d]\\
			A/\frakm^2\ar[r,"\sim"] & A_\frakm/(\frakm A_\frakm)^2
		\end{tikzcd}
	\end{center}
	Now the image of $a$ in $A_\frakm$ is nonzero, so there exists $n>0$ such that $a^n\in A_\frakm$ but $a^{n-1}\ne 0$ in $A_\frakm$.
	Thus there exists $f\notin \frakm$ such that $a^nf=0\in A$. Subsititute $af$ for $a$, and thus there is an $a\in\sqrt{0}$ such that $a^n=0$ in $A$ but $a^{n-1}\ne 0$ in $A_\frakm$.

	Then by fact 2,
	\[m^\ast(a)=1\otimes a+a\otimes 1+r,\quad r\in\frakm\otimes\frakm\]
	and since $m^\ast$ is a ring homomorphism, 
	\[0=m^\ast(a^n)=(m^\ast(a))^n=(a\otimes 1+(1\otimes a+r))^n=a^n\otimes 1+n(a^{n-1}\otimes a+(a^{n-1}\otimes 1)r)+X\]
	where $X\in A\otimes\frakm^2$. But since $a^n=0$, we get 
	\[n(a^{n-1}\otimes a+(a^{n-1}\otimes 1)r)\in A\otimes \frakm^2\]

	Now since $(a^{n-1}\otimes 1)r\in (a^{n-1})\frakm\otimes A$, so 
	\[n(a^{n-1}\otimes a)\in (a^{n-1})\frakm\otimes A+A\otimes\frakm^2\]
	and since $\ch k=0$, we get that $n$ is a unit, so 
	\[a^{n-1}\otimes a\in (a^{n-1})\frakm\otimes A+A\otimes\frakm^2\]

	Now since this lives in $A\otimes A$, consider the image of the quotient map $A\otimes A\to A\otimes A/\frakm^2$. Then 
	\[a^{n-1}\otimes\bar a\in (a^{n-1})\frakm\otimes A/\frakm^2\subseteq A\otimes A/\frakm^2\]
	And note that $a^{n-1}\notin a^{n-1}\frakm$ because otherwise $a^{n-1}=a^{n-1}q$ for $q\in \frakm$.
	Then $a^{n-1}(1-q)=0\in A_\frakm$, which implies that $a^{n-1}=0\in A_\frakm$ (since $1-q$ is a unit here).

	Then somehow we get that $\bar a=0\in A/\frakm^2$, so $a\in \frakm^2$.
\end{prf}

\section{October 7, 2019}
Today we will be primarly concerned with 
\subsection{Group actions}
Let $G$ be an algebraic group over $k$. 
\begin{defn}
	A \textbf{group action} of $G$ on a scheme $X$ over $k$ is the data of a morphism
	\[\mu:G\times X\to X\]
	such that the usual axioms hold. That is,
	\begin{center}
		\begin{tikzcd}
			G\times G\times X\ar[r,"m\times\id"]\ar[d,"\id\times \mu"] & G\times X\ar[d,"\mu"]\\
			G\times X\ar[r,"\mu"] & X
		\end{tikzcd}
		\begin{tikzcd}
			\Spec k\times x\ar[r,"e\times\id"]\ar[rd,"\sim"] & G\times X\ar[d,"\mu"]\\
			& X
		\end{tikzcd}
	\end{center}
\end{defn}
\begin{rmk}
	Apparently it was an exercise already to show that this is equivalent to an action of $h_G$ on $h_X$.
\end{rmk}
\begin{rmk}
	The map $(g,x)\mapsto(g,gx)$ is an automorphism of $G\times X$, so if $p_2:G\times X\to X$ is projection,
	\begin{center}
		\begin{tikzcd}
			G\times X\ar[rr,"\sim"] \ar[dr,"\mu"] & & G\times X\ar[dl,"p_2"]\\
			& X &
		\end{tikzcd}
	\end{center}
	commutes.
\end{rmk}
\begin{defn}
	Let $X$ and $Y$ be schemes over $k$ with a $G$ action. Then a \textbf{$G$-equivariant morphism} $f:X\to Y$ 
	is one such that for all $g\in G$, the following commutes:
	\begin{center}
		\begin{tikzcd}
			G\times X\ar[r,"\id\times f"]\ar[d,"\mu_X"] &G\times Y \ar[d,"\mu_Y"]\\
			X\ar[r,"f"] & Y
		\end{tikzcd}
	\end{center}
\end{defn}
\subsubsection{Some examples}
\begin{itemize}
	\item $G$ actions on itself by multiplication or conjugation.
	\item $\Gm$ acts on $\A1$. Geometrically, we are just looking at $k^\ast$ acting on $k$ by scaling.
	Algebraically, we want a map $\mu\Gm\times\A1\to\A1$ given by the map of algebras:
	\[k[x]\xrightarrow{\mu^\ast}k[t]_t\otimes k[x]\quad\text{via}\quad x\mapsto tx\]
	Functorially, if $S$ is a scheme over $k$, then $\Gm(S)=\Gamma(S,\O_S)^\ast$ which acts on $\A1(S)=\Gamma(S,\O_S)$,
	again by scaling.
	\item You can consider $\GL_n$ action on $\A n$ by multiplicaiton or on $\A{n\times n}$ via multiplication or conjugation.
\end{itemize}
\subsubsection{Orbits and Stabilizers}
Let $G$ be an algebraic group over $k$ action on a scheme $X$ over $k$. Let $x\in X(k)$. Then we have a map
\[\mu_x:G\times \Spec k\xrightarrow{\id\times x}G\times X\xrightarrow{\mu}X\]
where 
\[g\mapsto (g,x)\mapsto gx.\]
\begin{defn}
	The \textbf{orbit of $x$} is $Gx=\mu_x(G)\subseteq |X|$ set-theoretically. The \textbf{stabilizer of $x$ in $G$} 
	is $G_x=\mu_x^{-1}(x)\subseteq G$. 
\end{defn}
\begin{rmk}
	$G_x$ is always a closed algebraic subgroup of $G$.
\end{rmk}
\begin{prop}
	$\mu_x(G)$ is open in its closure in $|X|$.
\end{prop}
Recall first the following:
\begin{thm}[Chevalley's Theorem (different?)]
	If $f:X\to Y$ is a map of schemes of finite type over $k$, then $f(X)\subseteq Y$ is contstructible (i.e. 
	is a disjoint union of finitely many locally closed subsets).

	\noindent\textit{Recall that locally closed means closed in an open subspace.}
\end{thm}
\begin{cor}
	\textit{Maybe a definition:} The orbit $Gx\subseteq \operatorname{im}(\mu_x)\subseteq X$. If 
	$G$ is reduced, then $Gx$ is reduced.
\end{cor}

\subsubsection{Applications}
Say that $\ch k=p$. Then $\mu_p$ acts on $\Gm$ by multiplication.

\brk 

$\Gm$ acts on $\A1$ with two orbits: $\Gm\cdot 1=\{x\ne 0\}$ and $\Gm\cdot 0=\{0\}$. The stabilizers are 
$G_1=1$ and $G_0\cong\Gm$.

\brk

Consider $G$ acting on $\A2$ via $t(x,y)=(tx,t^{-1}y)$. Then the orbits are hyperbolas! There is also a notion of closed 
orbits that I didn't quite catch. Also apparently the orbit-stabilizer statement is easy to see in geometry via 
a fiber bundle $G$ over $Gx$ where the fiber over $x$ is $G_x$.

\begin{prop}
	If $\phi:G\to H$ is a homomorphism of algebraic groups, then $\phi(G)\subseteq |H|$ is closed.
\end{prop}
\begin{rmk}
	The proof included reducing first to $k=\bar k$. The trick here is to consider the group action induced by $\phi$
	and then consider the map $\mu_{e_H}$ of this action. Then $\mu_{e_H}(G)=\phi(G)$ and one can prove that this is closed.
\end{rmk}

In particular, we have that \textbf{subgroups of an algebraic group are always closed.} Note that this stands in stark contrast 
to Lie theory where you get non-closed subgroups.

\section{October 9, 2019}
Recall that last time we were considering actions of algebraic groups on schemes of finite type over $k$. We discussed the orbit and stabilizer 
of an element $x\in X$ and showed that $G\cdot x$ is open in its closure. We also saw that $G_x$ is a closed subgroup.

We also say that $\phi(H)$ (as a set) is always closed! None of these facts are true for Lie groups or relative group schemes 
(the base scheme is not $\Spec k$ for $k$ a field).

\subsection{Cartier Duality}
Let $G\to\Spec k$ be a \textbf{finite} group scheme (so $G=\Spec A$ and $A$ is a finite dimensional Hopf algebra).
Some examples of finite group schemes are:
\begin{itemize}
	\item $G$ a finite group. Then $G=\sqcup_{g\in G}\Spec k=\Spec\left(\prod_{g\in G}k\right)$
	\item $\mu_n=\Spec k[t]/(t^n-1)$
	\item $\ch k=p$ and $\alpha_p=\Spec k[t]/t^p$
\end{itemize}
\begin{rmk}
	Recall all the maps and diagrams that $A$ has as a Hopf algebra.
\end{rmk}
A question one may ask: what if we apply the idea of dualizing $(-)^\vee=\Hom_\Alg(-,k)$ to $A$?
Do we get another Hopf algebra?

The short and sweet of it is yes! But notice that we are coming from the commutative world, so we expect 
$A$ to be commutative. But in general, $A$ is not cocommutative (in fact, it is if and only if $G$ itself was commutative as a group).

Thus $A^\vee$ is indeed a (cocommutative) Hopf algebra, and when $G$ is commutative, $A^\vee$ is as well. So
\begin{prop}
	If $G=\Spec A$ is a commutative group scheme, then the Cartier dual $G^D=\Spec A^\vee$ is a commutative group scheme as well.
\end{prop}
\begin{rmk}
	The above observations gives us an anti-autoequivalence of the category of commutative affine group schemes. Furthermore $(G^D)^D=G$.
\end{rmk}
\begin{ex}
	Consider $\mu_n=\Spec A-\Spec \oplus_0^{n-1}k\cdot t^i$. So then if we let $\{e_i\}$ be the basis for $A^\vee$ dual 
	to $\{t_i\}$. we can compute comultiplication 
	\[e_i\mapsto \sum_{j=0}^{n-1}e_j\otimes e_{i-j}\]
	and multipication
	\[e_i\otimes e_j\mapsto \delta_{ij}e_i\]

	Then it can be shown that $G^D\cong\bbZ/n\bbZ$.
\end{ex}

Now given $G$, an algebraic group over $k$, define 
\[\underline\Hom(G,\Gm):\mathbf{Sch}/k\to\Set\]
which takes 
\[T\mapsto \Hom_{\mathbf{AlgGrp}}(G_T,{\Gm}_T).\]
\begin{thm}\label{thm:cartier}
	If $G$ is a commutative finite group scheme over $k$, then 
	\[G^D\cong\uHom(G,\Gm)\]
\end{thm}
Let $H=\Spec B\to \Spec R$ be a group scheme. Then 
\[H_{\mathbf{GrpSch}/R}(H,{\Gm}_R)\subseteq\operatorname{Mor}_{\mathbf{Sch}/R}(H,{\Gm}_R)\]
But the left hand side is equivalent to the grouplike elements of $B$ and the right hand side is equivalent
to $\Hom_{\Alg_R}(R[t]_t,B)$.

This leads to a proof of thm.~\ref{thm:cartier}:
\begin{prf}
	Let $G=\Spec A$ and $G^D=\Spec A^\vee$. First look at the $k$-points:
	\begin{align*}
		G^D(k)&= \Mor_{\Sch/k}(\Spec k, G^D)\\
		&=\Hom_{\Alg_k}(A^\vee,k)=\{f\in A|m^\ast(f)=f\otimes f\}\hookrightarrow \Hom_{k}A^\vee,k)\\
		&=\Hom(G,\Gm)\\
		&=\uHom(G,\Gm)(k)
	\end{align*}

	If we then look at $R$ points for a general $R$, most things just change over, but we see 
	\[\{f\in A\otimes R|m_R^\ast(f)=R\otimes R\}=\Hom_{\Alg_k}(A^\vee,R)=\Hom(G_R,\Gm)\]
	and the rest follows.
\end{prf}
A question one may ask: what is $\Hom_\AlgGrp(\Gm,\Gm)$? It ends up it is $\bbZ$. You can send $t\mapsto t^n$ for all $n\in\bbZ$.
But then $\uHom(\Gm,\Gm)$ is $\bbZ$ as a group scheme over $k$, which is not quasicompact. There was more but I am le tired.

\section{October 11, 2019}
Today we're doing problems and stuff. Forgot about that.

\subsection{Casey's Presentation}
Let $G$ and $H$ be objects in a category $\calC$ with finite products. Let $h:H\to G$ be a group homomorphism. That is,

Then we get similar diagrams for the identity and inverse maps (they are respected by $h$). 
Then there is a bunch of diagram work. It's too hard to do a diagram without knowing the shape ahead of time.

Oh hey I used Adam's site! 
\begin{center}
	\begin{tikzcd}
		\ast \arrow[rd, "{(e,e,e)}"]               &                                                                                                                                                  &                                                                                        &                                                                        &                                                                           \\
												   & H\times H\times H \arrow[r, "(i_g\circ h)\times h\times h"] \arrow[d, "1\times m_H"']                                                            & G\times G\times G \arrow[d, "1\times m_g"]                                             &                                                                        &                                                                           \\
												   & H\times H \arrow[r, "(i_G\circ h)\times h"]                                                                                                      & G\times G \arrow[rd, "m_G"]                                                            &                                                                        &                                                                           \\
												   &                                                                                                                                                  &                                                                                        & G                                                                      &                                                                           \\
	\end{tikzcd}
\end{center}
\begin{center}
	\begin{tikzcd}												   & \ast \arrow[r, "e"] \arrow[rd, "{(e,e,e)}"] \arrow[rdddd, "he"', bend right=49]                                                                   & H \arrow[rd, "{(h,h)}"]                                                                &                                                                        &                                                                           \\
												   & &H\times H\times H \arrow[d, "(i_g\circ h)\times h\times h"']                                                                                                                                                                         & G\times G \arrow[rd, "\ast\times 1"] \arrow[ld, "{(\ast,1)\times 1}"',swap] &                                                                           \\
												   &                                                                                                                                                  & G\times G\times G \arrow[d, "1\times m"'] \arrow[rd, "m\times 1"]                      &                                                                        & \ast\times G \arrow[ld, "e\times 1"'] \arrow[lldd, "\pi_1", bend left=49] \\
												   &                                                                                                                                                  & G\times G \arrow[d, "m"']                                                              & G\times G \arrow[ld, "m"]                                              &                                                                           \\
												   &                                                                                                                                                  & G                                                                                      &                                                                        &                                                                           \\
	\end{tikzcd}
\end{center}
\begin{center}
	\begin{tikzcd}
												   & \ast \arrow[rd, "he"', dashed,swap] \arrow[r, "\ast", bend left] \arrow[d, "{(e,e,1)}"'] \arrow[ld, "{(e,e,e)}"'] \arrow[dd, "{(e,e)}", bend left=60] & \ast \arrow[r, "e"]                                                                    & G                                                                      &                                                                           \\
		H\times H\times H \arrow[rd, "1\times m"'] & H\times H\times\ast \arrow[d, "1\times \pi_2"'] \arrow[l, "1\times 1\times e"']                                                                  & G \arrow[u, "\ast"'] \arrow[d, "{(1,1)}"'] \arrow[rd, "{(i,1)}", bend left] &                                                                        &                                                                           \\
												   & H\times H \arrow[r, "h\times h"] \arrow[rr, "(ih)\times h", bend right]                                                                          & G\times G \arrow[r, "i\times 1"]                                                       & G\times G \arrow[uu, "m"]                                              &                                                                          
		\end{tikzcd}
\end{center}
\begin{rmk}
	The idea above is we want to show the first diagram commutes. That is captured in the paths of the second diagram which commutes by 
	the axioms of a group object. The third diagram shows a similar commutativity for the unit $e$.
\end{rmk}

\section{October 14th, 2019}
Let $G$ be a finite group. Recall the definiton of a \textbf{representation} (a linear action of $G$ on a vector space $V/k$).
This is the same data as a group homomorphism to $\GL(V)$.

\subsection{Representations of Algebraic Groups}

Now what if $G$ is an algebraic group over $k$? Now we have some exra structure of $G$ as a variety.
\begin{defn}
	A (finite dimensional) \textbf{representation} of an algebraic group $G/k$ is a homomorphism $\rho:G\to \GL(V)$ of algebraic groups.
\end{defn}
\begin{rmk}
	Notice that when $V$ is infinite-dimensional, $\GL(V)$ is no longer of finite type, so we have to let $\rho$ be a 
	morphism of group schemes.
\end{rmk}
We have the standard representation of $\GL_n$ acting on $k^n$ in the natural way. We also have the regular representation 
$G$ action on $\Gamma(G,\O_G)$. When $G=\Gm$, we get over $\bbC$ an action of $\Gm$ on $\GL(V)$ in the usual way (scaling by $\bbC^\ast$).

Observe that 
\[\rho:G\to \GL(V)=\Spec(\operatorname{Sym}^\ast(V\otimes V^\vee))_{\det}\]
corresponds to a ring morphism 
\[\Sym^\ast(V\otimes V^\vee)_{\det}\to \Gamma(G,\O_G)\eqdef \Gamma(G)\]
which corresponds to a map 
\[V\otimes V^\vee\to \Gamma(G)\]
and then tensoring with $V$, this gives us a map
\[V\xrightarrow{\sigma}\Gamma(G)\otimes V\]

So any group action gives us a \textbf{coaction} of $\Gamma(G)$ on $V$.
\begin{defn}
	A representation of $G$ is a $k$-vector space $V$ along with a coaction 
	\[\sigma:V\to \Gamma(G)\otimes V\]
	satisfying the usual dual diagrams to actions.
\end{defn}

\begin{rmk}
	As a matter of notation, recall that if $G=\Spec A$, then $A$ is a Hopf algebra. So we call $V$ an $A$-comodule.
\end{rmk}

\subsection{Reps of diagonalizable group schemes}
Let $k$ be a field (or even a ring!) and let $A$ be a finitely-generated abelian group. Define $D(A)$ to be 
\[D(A)=\oplus_{a\in A}k\cdot t^a\eqdef \Spec R\]
Then we get a multiplication 
\[R\otimes R\to R\qquad t^a\otimes t^b\mapsto t^{a+b}\]
and comultiplication
\[R\to R\otimes R\qquad t^a\to t^a\otimes t^a\]
and counit $\varepsilon$ sending $t^a\to 1$ (all $t^a$ are primitive).
\begin{prop}
	$R$ is a Hopf algebra. In particular, $D(A)\to \Spec k$ is a linear algebraic group.
\end{prop}

As an example, consider $A=\bbZ$. Then $R\cong k[t]_t$. Thus $D(A)=\Gm$. 

If instead $A=\bbZ/n$, then $R\cong l[t]/(t^n-1)$, so $D(A)\cong\mu_n$.

Finally when $A\cong \bbZ^r\oplus\bbZ/n_1\oplus\cdots\oplus \bbZ/n_k$, then 
\[D(A)=\Gm^r\times\mu_{n_1}\times\cdots\times \mu_{n_k}\]

\begin{defn}
	An algebraic group over $k$ is \textbf{diagonalizable} if $G\cong D(A)$ for some $A$.
\end{defn}
Recall the definiton of irreduciblility.
\begin{thm}
	Let $A$ be af initely generated abelian group and $G=D(A)$. Then 
	\begin{itemize}
		\item Every irreducible representation of $G$ is one-dimensional and isomorphic to $I_a$, 
		corresponding to $k\to \Gamma(G)\otimes k$ where $1\mapsto t^a\otimes 1$ for some $a\in A$.
		\item Every representation decomposes as a direct sum of irreducibles.
	\end{itemize}
\end{thm}
\begin{prf}
	Let $\sigma:V\to \Gamma(G)\otimes V$ be a representation of a diagonalizable group. For $a\in A$, define 
	\[V_a\eqdef\{v\in V|\sigma(v)=t^a\otimes v\}\subseteq V\]
	Now the claim is that $V_a\cap V_b=0$ if $a\ne b$ and furthermore $\sum V_a=V$. The first isn't too hard to see.

	The second follows by considering $v\in V$ and looking at the image of it under $\sigma$. That is, 
	\[\sigma(v)=\sum_1^Nt^{\alpha_i}\otimes v_i\]
	where $\alpha_i\in A$ and $v_i\in V$. Then a very simple argument shows $v=\sum v_i$ (using linearity).
	Then it remains to show that $v_i\in V_{a_i}$, but this will make things work. (use the other axiom of a coaction).
\end{prf}
\begin{rmk}
	When $A=\bbZ$, $G=D(\bbZ)=\Gm$, which tells us that representations of $\Gm$ are in bijection with $\bbZ$-gradings 
	of $V\cong\oplus_{n\in\bbZ}V_n$!
\end{rmk}
\begin{defn}
	A linear algebraic group $G\to\Spec k$ is called \textbf{linear reductive} if every representation decomposes as a direct sum of irreducibles.
\end{defn}
\begin{prob}
	Show the above is equivalent to the statements
	\begin{itemize}
		\item for each $G$-representations $W\subseteq V$, there exists $W'\subseteq V$ subrepresentations such that $V\cong W\oplus W'$.
		\item $0\to W\to V\to W'\to 0$ is exact.
	\end{itemize} 
\end{prob}
\begin{rmk}
	Notice that this says that $D(A)$ is linear reducible. In particular, $\Gm$ and $|mu_n$ are in \textbf{any characteristic}. This runs counter to Maschke in 
	finite groups.
\end{rmk}

Consider $\bbZ/p$ in $\ch p$. We get an action $\bbZ/p$ on $k^2$ via 
\[1\cdot(x,y)=(x+y,y).\]
But notice that $k\stackrel{y=0}{\hookrightarrow}k^2$ is a subrepresentation, but has no complement! Thus this group is not linearly reductive!

\brk

As another example, consider
\[\Ga\cong\left\{\begin{pmatrix}
	1 & \alpha \\ 0 & 1
\end{pmatrix}\right\}
\subset\GL_2(k)\]
where $\Ga$ acts on $k^2$ by $\alpha(x,y)=(a+\alpha y,y)$. Then it can be easily seen not to be a linear representation.

\section{October 16th, 2019}
Last time we talked about representations! Woot.

Notice that if $G$ is linear (i.e. affine), then the multiplication map induces comultiplication 
\[\Gamma(G)\to \Gamma(G)\otimes\Gamma(G)\]
so $\Gamma(G)$ is the \textbf{regular} representation with coaction given by multiplication.

We also saw some equivalent contitions similar to Machke for linear reductive groups. Finally we say some examples and diagonalizable groups.

\subsection{New Stuff}
Given a $G$-representation $V$, let $V^G$ be 
\[\{v\in V|\sigma(v)=1\otimes v\}=\operatorname{Eq}\{V\stackrel{\xrightarrow{1\otimes -}}{\xrightarrow{\sigma}}\Gamma(G)\otimes V\}=\Hom^G(k,V)\subseteq V.\]
\begin{rmk}
	I need to figure out the \TeX for equalizers/parallel maps.
\end{rmk}
\begin{ex}
	Given the representation $\Gm$ action on $V=\oplus V_d$, $V^G=V_0$.
\end{ex}
\begin{prob}
	If $G(k)$ is dense in $G$, then $V^G=V^{G(k)}$.
\end{prob}
\begin{prop}
	A linear algebraic group $G$ over $k$ is linearly reductive if and only if the functor from $G$-representations to 
	$k$-vector spaces given by $V\mapsto V^G$ is exact.
\end{prop}
\begin{prf}
	If $V\cong W\oplus W'\twoheadrightarrow W$ is a $G$ representation, then 
	\[W^G\oplus(W')^G=V^G\twoheadrightarrow W^G\]
	is also surjective.

	Suppose that we have a short exact sequence 
	\[0\to W'\to V\to W\to 0\]
	and that this functor is exact. Then we want to show we get a section $\sigma:W\to V$. To do this, consider the functor 
	$\Hom^G(W,-)=\Hom^G(k,W^\vee\otimes -)=(W^\vee\otimes -)^G$, so by the assumption this is exact and we can lift the identity on $W$ 
	to a map in $\Hom^G(W,V)$, giving us our section.
\end{prf}
\begin{prop}
	Let $G$ be a linear algebraic group over $k$ and $V$ a $G$-representation. Let $W\subseteq V$ be a finite dimensional $k$-subspace (not 
	necessarily $G$-invariant). Then there exists $W\subseteq W'\subseteq V$ such that $W'$ is a finite dimensional representation of $G$.
\end{prop}
\begin{prf}
	We can assume that $W=\langle w\rangle$ for $w\in V$. Apply $\sigma:V\to \Gamma(G)\otimes V$. Then if $\{t_i\}$ is a 
	basis for $\Gamma(G)$, we get 
	\[w\mapsto\sum t_i\otimes w_i.\]
	Then we claim that $w\in\langle w_i\rangle$ and $\langle w_i\rangle\subseteq V$ is a subrepresentation.

	For the first, consider the diagrams:
	\begin{center}
		\begin{tikzcd}
			k\otimes V & \Gamma(G)\otimes V\ar[l,swap,"e^\ast\otimes\id"]\\
			& V\ar[ul,"\sim"]\ar[u,"\sigma",swap]
		\end{tikzcd}
		\begin{tikzcd}
			\sum e^\ast(t_i) w_i & \sum t_i\otimes w_i\ar[l]\\
			& w\ar[u]
		\end{tikzcd}
	\end{center}
	so $w$ is in the span of the $w_i$.

	For the second claim, we need to show that 
	\[\sigma(w_i)\in\Gamma(G)\otimes\langle w_i\rangle.\]
	To see this consider the diagram 
	\begin{center}
		\begin{tikzcd}
			\Gamma(G)\otimes\Gamma(G)\otimes V & \Gamma(G)\otimes V\ar[l,"m^\ast\otimes\id"]\\
			\Gamma(G)\otimes V\ar[u,"\id\otimes\sigma"] & V\ar[l,"\sigma"]\ar[u,"\sigma"]
		\end{tikzcd}
	\end{center}
	And tracing through $w\in V$, we get that 
	\[\sum t_i\otimes\sigma(w_i)=\sum_{i,j,k}\alpha_{i,j,k}t_i\otimes t_j'\otimes w_k\]
	and so by looking at coefficients of $t_i\otimes \Gamma(G)\otimes V=\sigma(w_i)$ (look closer here), we see it is 
	\[\sum_{j,k}\alpha_{i,j,k}t_j'\otimes w-k\in\Gamma(G)\otimes\langle w_i\rangle.\]
\end{prf}
\begin{cor}
	If $V$ is a $G$ representation, then 
	\[W=\bigcup_{W\subset V}W\]
	where the union is over all finite dimensional subgroups.
\end{cor}
\begin{cor}
		If $G$ is a linear algebraic group (affine finite type over $k$), then for some $n$ $G\subseteq \GL_n$ is a closed subgroup.
		In other workds, there exists a faithful representation $V$ of $G$.
\end{cor}

Now consider the regular representation $\Gamma(G)\to \Gamma(G)\otimes \Gamma(G)$. Notice that $\Gamma(G)$ is 
a $k$-algebra of finite type. Choose generators $g_1,\dots,g_n$ for $\Gamma(G)$. Take a subrepresentation of $\Gamma(G)$
spanned by $\langle h_1,\dots,h_N\rangle$ containing the span of the $g_i$.

We have a map $G\to \GL(W)$ where $W=\langle h_i\rangle$ and we aske whether the induced map 
\[\Sym^\ast(W\otimes W^\vee)_{\det}\]
is surjective.

Let's say that $h_i\mapsto \sum_j\gamma_{i,j}\otimes h_j$ under $\sigma$. Then using this map and the natural pairing between $W$ and $W^\vee$, we get a map 
\[\Sym^\ast(W\otimes W^\vee)\supset W\otimes W^\vee\to\Gamma(G)\otimes W\otimes W^\vee\to \Gamma(G)\]
where we send 
\[h_i\otimes h_j^\ast\mapsto \gamma_{i,j}\]
So using the counit identiy we can write 
\[h_i=\sum_j e^\ast(\gamma_{i,j})h_j\]
but we really want to write $h_i$ as a linear combination of the $\gamma_{i,j}$ (since we have shown they all lie in the image of this map).
We don't know how to finish up.

\section{October 18th, 2019}
Last time we say that any linear algebraic group $G$ embeds into $\GL_n$. The argument was basically that you look at the global functions 
$\Gamma(G)$ and doing cool stuff. Right at the end Taffy and Tuomas figured out that we just needed to use the other ``side'' of the counit diagram.

\subsection{An example}
Consider 
\[\operatorname{PGL}_2=(\operatorname{Proj} k[a,b,c,d])_{ad-bc]}=\Spec(k[a,b,c,d]_{\det})_0\]
Then consider the representation spanned by 
\[\left\langle\frac{a^2}{\det},\frac{ab}{\det},\dots,\frac{d^2}{\det}\right\rangle\]
which has dimension 10 in $\Gamma(\operatorname{PGL}_2)$.

Thus we have a representation $\operatorname{PGL}_2\to\GL_{10}$. We can compute the matrix representing a matrix (whose determinant
can be assumed to be 1 since we are modding out by scalars).
\begin{prob}
	Do this! In Sage or something.
\end{prob}
\subsection{Special Linear Groups}
Lets discuss $\SL_2$. 
\begin{thm}
	If $\ch k=0$, then 
	\begin{itemize}
		\item $\SL_2$ is linearly reductive.
		\item Every irreducible representation of $\SL_2$ is isomorphic to $\Sym^d k^2$ for some $d$.
	\end{itemize}
	where $k^2$ is the standard representation of $\SL_2$.
\end{thm}
\begin{prf}
	\textit{Sketch:} Recall that in the proof of Maschke one takes a surjection $V\twoheadrightarrow W$ 
	of $G$ representations and we want to show it has a section. We pick a section $s$ in terms of vector spaces and then ``average''
	it:
	\[\tilde s:W\to V\qquad w\mapsto\frac{1}{|G|}\sum_{g\in G}g\cdot s(g^{-1}w)\]
	which is our section.

	Then using the Harr measure on the group, we get from the inclusion of (compact) $\operatorname{SU}_2$ in $\SL_2$ with quotient $\bbC$
	and we can construct the section via 
	\[w\mapsto \int_Gg\cdot s(g^{-1}w)\,\mathrm{d}g\]
	and we get $T_e\SL_2=(T_e\operatorname{SU}_2\otimes_\bbR\bbC)$ and then there is a bit more Lie theory needed to show this makes full sense over $\bbC$.
\end{prf}

Now consider $\Sym^d(k^2)^\vee$, the degree $d$ polynomials on two variables. Then we get an action of $\SL_2$ via 
\[g\cdot f(x,y)=f(g^{-1}\binom{x}{y})-f(dx-by,-cx+ay)\]
Then, as many arguments in linear algebraic groups, we can reduce to a so-called \textit{maximal torus} 
of matrices $(\begin{smallmatrix}a & 0\\0 & a^{-1}\end{smallmatrix})\cong\Gm$. Then we can use techniques on the lie algebra 
$T_e\SL_2=\sl_2$.

Whoa coool. The short exact sequence 
\[1\to \SL_2\to \GL_n\xrightarrow{\det}\Gm\to 1\]
gives us that any representation of $\SL_2$ lifts via $-\otimes \det^i$ to a representation of $\GL_2$.
Sean asked whether this actually gets all the representations or just the polynomial ones. I feel like I should know the 
answer to this. Jarod seems to think this is all of them. I think \href{https://math.stackexchange.com/questions/694001/non-polynomial-representations-of-gl-n}{this stack post}
says something about that.

\textit{I got caught up in thinking and googling and missed an example.}

Let $\ch k=p$ and let $\alpha\in k^\times$. Then define $\alpha\cdot f=\alpha^pf$. This gives us a map 
\[\Sym^d k^n\to \Sym^{dp}k^n\]
that is additive taking $p^{th}$ powers.

So then $\Sym^N k^n$ is \textbf{not simple} if $p|N$.

\section{October 21st, 2019}
Recall that we had that any linear algebraic group over a field $k$ injects into $\GL_n$ as a closed subgroup for some $n$.

An open question is as follows: if $G\to\Spec k[\varepsilon]/\varepsilon^2$ is flat affine group scheme of finite type
Then is $G\subseteq\GL_{n,k[\varepsilon]}$ for some $n?$ This question was asked (as far as Jarod knows) by Brian Conrad on Stack Overflow 
and is still open.

Our goal is to answer the following: if $\phi:H\to H$ then $K=\ker\varphi=H\times_G\Spec k\subseteq H$ is a closed subgroup. What about its image $H/K$?

\subsection{Torsors}
Today we are going to be talking about $G$-torsors.
\begin{defn}
	If $G$ is a group, a \textbf{torsor under $G$} is a set $P$ with a free and transitive group action.
\end{defn}
\begin{rmk}
	So then by fixing a point $p\in P$, we get $G\xrightarrow{\sim}P$ by sending $g\mapsto pg$. In this way, it is like thinking about 
	a group without the identity.
\end{rmk}
An example is by taking a Galois extension $K(\alpha)/K$ with minimal polynomial $f$ of $\alpha$. Then $G=\operatorname{Gal}(K(\alpha)/K)$ acts on 
$\{x|f(x)=0\}$, which is a $G$-torsor.
\begin{defn}
	A \textbf{$G$-torsor over a set $S$} is a set $P$ with a free right $G$-action such that $P\to S$ is $G$-invariant and $S\cong P/G$.
\end{defn}
\begin{rmk}
	Notice that a torsor under $G$ is a specialization of this definition by requiring that $S=\{\ast\}$, the singleton set.
\end{rmk}
\begin{ex}
	Let $H\subseteq G$. THen $H$ acting on $H\to H\setminus G$ (left cosets $gH$ of $H$) is an $H$-torsor.
\end{ex}
\subsection{Flatness}
\begin{defn}
	A map of rings $A\to B$ is \textbf{flat} if $-\otimes_A B$ is exact.
\end{defn}
\begin{rmk}
	Equivalently: for all $p\in \Spec A$, $A_p\to B_p$ is flat. Also: for all $q\in\Spec B$. $A_{\phi^{-1}(q)}\to B_q$ is flat.
\end{rmk}
\begin{defn}
	$A\to B$ is \textbf{faithfully flat} if and only if $-\otimes_AB$ is \textbf{faithfully exact} (exactness and its converse).
\end{defn}
\begin{rmk}
	Other equivalence to faithful flatness: $A\to B$ is flat and $\Spec B\twoheadrightarrow\Spec A$; or $A\to B$ 
	is flat and for any $A$-module $M$, $M=0\leftrightarrow m\otimes_AB=0$.
\end{rmk}
\begin{rmk}
	If $\Spec B\twoheadrightarrow\Spec A$ is faithfully flat and finite presented, then $\Spec A$ has the quotient topology.
\end{rmk}

\begin{prop}
	Let $S=\Spec A$ be Noetherian. Let $G\to S$ be an affine group scheme of flat and finite type over $S$. Let $P\to S$ 
	be a scheme over $S$ with a right $G$-action $P\times_SG\xrightarrow{\sigma}P$. Then the following are equivalent:
	\begin{enumerate}
		\item $P\to S$ is affine, (faithfully?)\footnote{We tried to prove this in class and it seemed not to be true if we don't say this}
		flat, finite type and $(\sigma,\pi_P):(p,g)\mapsto (pg,p)$ is an isomorphism.
		\item There exists a faithfully flat $S'$ such that 
		\begin{center}
			\begin{tikzcd}
				P_{S'}\ar[r]\ar[d] & P\ar[d]\\
				S'\ar[r,"\sigma"] &S
			\end{tikzcd}
		\end{center}
		And $P_{S'}\cong G_{S'}$ as $G_{S'}$-modules.
	\end{enumerate}
\end{prop}
\begin{rmk}
	Note that the above says exactly that $P$ is a $G$-torsor. Another name that has been mentioned and Jarod seems to 
	like is \textbf{principal $G$ bundle.}
\end{rmk}

\subsection{Descent}
Along the way of proving the above propositon, we use the idea of descent.
\begin{lem}
	Consider $X'=\Spec B'\stackrel{\text{f.flat, f.type}}{\twoheadrightarrow}X=\Spec B\to Y=\Spec A$
	where $X',X,$ and $Y$ are Noetherian. Then 
	\begin{enumerate}
		\item $X'\to Y$ is flat implies that $X\to Y$ is flat.
		\item $X'\to Y$ is faithfully flat implies that $X\to Y$ is faithfully flat.
		\item $X'\to Y$ is finite type implies that $X\to Y$ is.
	\end{enumerate}
\end{lem}
\begin{rmk}
	The idea for the first two is just looking at the functors using that $B\to B'$ is faithfully flat. For the third, if $B=\cup_\lambda B_\lambda$,
	then $A\to B_\lambda$ is finitely generated. Then tensoring over $B$ with $B'$ gets us $B'=\cup_\lambda B_\lambda\otimes_B B'$.

	But since $A$ is finitely generated over $B_\lambda$< eventually $B_\lambda\otimes_BB'=B'$. Then 
	consider
	\[0\to B_\lambda\to B\to B/B_\lambda.\]
	After tensoring with (faithfully flat!) $B'$ over $B$, since for some $\lambda$
	\[0\to B_\lambda\otimes_B B'\xrightarrow{\sim}B'\to B/B_\lambda\otimes_BB'\to 0\]
	is exact, forcing the rightmost term to be zero. But by faithfulness this implies $B/B_\lambda=0$ and we are done.
\end{rmk}
\begin{prop}
	Consider
	\begin{center}
		\begin{tikzcd}
			X'\ar[r]\ar[d] & X\ar[d]\\
			S'=\Spec A'\ar[r,"\text{f.type, f.flat}",two heads,swap] & S=\Spec A
		\end{tikzcd}
	\end{center}
	which is a Cartesian square. Then 
	\begin{enumerate}
		\item $X'\to S'$ is an isomorphism iff $X\to S$ is.
		\item $X'\to S'$ is affine iff $X\to S$ is.
	\end{enumerate}
\end{prop}

\section{October 23rd, 2019}
Recall the following definition/proposition:
\begin{defn}
	Let $S=\Spec R$ be Noetherian. Let $G\to S$ be an affine group scheme that is flat and of finite type. Let $G$ be a scheme over $S$ with a right $G$-action.

	Then the following are equiavlent:
	\begin{itemize}
		\item $P\to S$ is a \textbf{$G$-torsor}
		\item $P\to S$ is faithfully flat and of finite type and $P\times_SG\xrightarrow{(\sigma,\pi_1}P\times_S P$ is an isomorphism.
		\item There exists $S'\twoheadrightarrow S$ faithfully flat such that 
		\begin{center}
			\begin{tikzcd}
				G\times_S S'\cong P\times_S S'\ar[r]\ar[d] & P\ar[d]\\
				S'\ar[r] & S
			\end{tikzcd}
		\end{center}
		commutes (where the isomorphism shown is $G\times_SS'$-equivariant.)
	\end{itemize}
\end{defn}

\subsection{Some Examples}
We have the trival torsor $P=G\to S$. It is a propositoon that $P\to S$ is trivial iff there exists a section $s:S\to P$.

Let $L/K$ be a finite Galois extension. Then we get $\operatorname{Gal}(L/K)$ acting on $P=\Spec L\to \Spec K$ is a $G=\operatorname{Gal}(L/K)$-torsor.
Then in the diagram in the definition above, $\Spec L$ plays the part of $S'$. Then we get that $L\otimes L\cong L[x]/f\cong \prod_{g\in G}L$ and 
\[G\times_{\Spec K}\Spec L\cong \Spec(L\otimes L)\cong \sqcup_{g\in G}\Spec L.\]

\brk

Now let $X$ be a scheme and let $\Gm$ act on a line bundle $L\to X$ with section $o:X\to L$. Then $(L\setminus o(X))\to X$ is a $\Gm$ torsor.
It is a result, although we don't have the machinery yet, that 
\begin{prop}
	There is a bijection between line bundles on $X$ and $\Gm$-torsors.
\end{prop}
\begin{rmk}
	We can do womething similar with any vector bundle over $X$: if $V\to X$ is one, then $V_x$ over $x\in X$ is a vector space. 
	We just send $V$ to $\operatorname{Frame}(V)$, which over any $x\in X$ we have the set of orderd bases of $V_x$. This gives us a $\GL_n$-torsor.
\end{rmk}

\brk

If we have a subgroup (say of an algebraic group) $H\subseteq G$, recall that we wanted to show the existence of an $H$-torsor $G\to G/H$.

We begin by talking about \textit{abstract groups}. Assume we have an exact sequence 
\[1\to K\to G\xrightarrow{\pi}Q.\]
Then $G\times K\xrightarrow{\sim}G\times_QG$ via the map $(g,k)\mapsto(g, gk)$. The proof isn't too bad.

So now consider a geometric group. If we have the same exact sequence of \textit{algebraic groups over $k$}, we get $K=G\times_Q\Spec k$(??)
and then evaluating at any scheme $S$, we get 
\[1\to K(S)\to G(S)\to Q(S)\]
and consider the map $G(S)\times K(S)\to G(S)\times_{Q(S)}G(S)$ as above. Then by Yoneda we get an isomorphism $G\times K\cong G\times_Q G$ of schemes.

\begin{cor}
	If $\pi:G\to Q$ is a faithfully flat map of linear algebraic groups over $k$, then $G\to Q$ is a torsor under $K=\ker\pi$
\end{cor}
Let's do even better!
\begin{cor}
	Let $\pi:G\to Q$ be dominant (i.e. the image is dense in $Q$) and furthermore that $Q$ is reduced. Then $G\to Q$ is faithfully flat and in particular $G\to Q$ is a $K$-torsor.
\end{cor}

\begin{prf}
	(Of the second corollary): We use the idea of ``generic flatness''. That is there exists a $U\subseteq Q$ such that $\pi^{-1}(U)\to U$ is flat. Then we can translate this by 
	the $G$-action (after passing to the algebraic closure of $k$ so that the points are dense) and then flat descent gives us the result we want. :)
\end{prf}

\begin{thm}
	Let $G$ be alinearly algebraic group over $k$. Let $X$ be a scheme over $k$ of finite type.
	Then
	\[\{G\text{-bundles on $X$}\}\cong H_{fl}^1(X,G)\]
	where $H_{fl}$ is flat cohomology.
\end{thm}
The idea here is clear when $G=\Gm$: you get connections between line bundles on $X$ (i.e. the Picard group of $X$) and $\Gm$-torsors and similarly between the line bundles and $H_{Zar}^1(X,\O_X^\times)$, 
using the (usual) Zariski sheaf cohomology.

\section{October 25th, 2019}
We are going to deviate slightly from the result promised last time, because we want to talk first a bit about 
\subsection{Descent}
Recall that we had two results before about descent involving faithfully flat morphisms of finite type (where the schemes are Noetherian).

\begin{ex}Let $X=\Spec B$ and let $U_i$ be a Zariski-open cover. Then the map 
\[\sqcup U_i\to X\]
is faithfully flat and of finite type.
\end{ex}
So then one idea is that if we have a nice Zariski cover of $X$, flatness, faithful flatness, and finite type are all ``local on the source''
in that you just have to check the property locally on $X$.

\brk

\begin{prop}
	Let $A\to B$ be a faithfully flat ring homomorphism. Then 
	\begin{center}
		\begin{tikzcd}
			A\ar[r] & B\ar[r,shift left=0.75ex, "p_1"]\ar[r,shift right=0.75ex,"p_2",swap] & B\otimes _AB
		\end{tikzcd}
	\end{center}
	where $p_1(b)=b\otimes 1$ and $p_2(b)=1\otimes b$ is an exact sequence. More generally, 
	\begin{center}
		\begin{tikzcd}
			M\ar[r] & M\otimes_A B\ar[r,shift left=0.75ex,"\id\otimes p_1"]\ar[r,shift right=0.75ex,swap,"\id\otimes p_2"] & M\otimes_A B\otimes_A B
		\end{tikzcd}
	\end{center}
	is exact.
\end{prop}

There is some geometry that lost me a bit. Sorry. :(
\begin{prop}
	If $A\to B$ is faithfully flat, then the map $\rmod A\to\{(N,\alpha)|N\in\lmod B, \alpha:p_1^\ast N\cong p_2^\ast N, P(\alpha)\}$ 
	sending
	\[M\mapsto(M_B=M\otimes_AB,\alpha_{can})\]
	is 
	an equivalence of categories where $P(\alpha)$ means that $\alpha$ satisfies a cocyle condition. 
\end{prop}
\begin{rmk}
	The canonical isomorphism above is 
	\[\alpha_{can}:M_B\otimes_{B,p_1}(B\otimes_AB)\to M_B\otimes_{B,p_2}(B\otimes_A B)\]
\end{rmk}

\section{November 4th, 2019}
Today we are going to do a bit of review as well as do a little more work with $G$-torsors.

\subsection{Review}
Recall the following setup: $S=\Spec A$ is Noetherian. Then let $G\to S$ be an affine group scheme of finite type over $S$. Let $P\to S$ 
be a scheme with a right $G$-action:
\[P\times_S G\xrightarrow{\sigma}P.\]
\begin{defn}\label{defn:torsor}
	Then $P\to S$ is a \textbf{$G$-torsor} if either of the following hold:
	\begin{itemize}
		\item $P\to S$ is affine, surjective, of finite type, and furthermore transitive (that is $P\times_S G\xrightarrow{(\sigma,\pi_1)}P\times_SP$ is an isomoporphism).
		\item There exists a faithfully flat $S'\to S$ with $P\times_S S'\cong G\times_S S'$ withere this isomorphism is $G\times_S S'$-equivariant.
	\end{itemize}
\end{defn}
\begin{rmk}
	If $G$ is a linear algebraic gorup over $k$ and $S$ is defined over $k$, then we say $P\to S$ is a $G$-torsor if it is a torsor under $G\times_k S$.
\end{rmk}

Now a \textbf{trivial $G$-torsor} over $S$ is $S\times G\to S$ if and only if $P\to S$ has a section. 

Some examples:
\begin{itemize}
	\item The $\Gm$ torsors over $S$ are in bijection with the line bundles over $S$.
	\item The $\GL_n$ torsors over $S$ are in bijection with the vector bundles over $S$.
\end{itemize}
Since line and vector bundles are tivialized in the Zariski topology, we see that the second condition in definition~\ref{defn:torsor} can be replaced with: 
There exists an open cover $\{S_i\}$ such that $P|_{S_i}\cong S_i\times G$.

\brk

We proved some stuff with descent. Look at it.
\begin{cor}
There exists an equivalence between $\Alg_A$ and the category of $B$ algebras $C$ along with isomorphisms $\alpha:p_1^\ast C\to p_2^\ast C$ along with the cocyle condition.
\end{cor}
\begin{rmk}
	To prove this, we descend a $(B,\alpha)$ to an $A$-module and show that the multiplication on $B$ descends nicely to $A$.
\end{rmk}
\begin{cor}
	We can also descend $G$-torsors. That is, we have an equivalence between $G$-torsors $P\to S$ and $G$-torsors $P'\to S'$ with isomorphisms and the cocycle condition.
\end{cor}
This last result takes quite a bit of doing although we have all the machinery we need. This is the 
one we're going to be using.

\subsection{New Stuff}
Let $X$ be a scheme. Then $\operatorname{Pic}(X)$ is the group of line bundles, or equivalently $\Gm$-torsors on $X$.
\begin{thm}
	$\operatorname{Pic}(X)=H^1(X,\O_X^\times)=H^1(X,\Gm)$.
\end{thm}
Notice that $\O_X^\times$ is a sheaf assigning to each $U$ $\O_X(U)^\times$. On Zariski opens, this is the same as $\Gm(U)$.

To prove the above result, we need to discuss 
\subsection{\texorpdfstring{\v C}{C}ech Cohomology}
Assume $X$ is separated. Let $\calU=\{U_i\}$ be an affine covering where $U_i\cap U_j$ is also affine. Then considering the complex 
\[\sqcup U_i\cap U_j\cap U_k\to \sqcup U_i\cap U_j\to \sqcup U_i\to X\]
where we have several a parallel maps for each $n-1$-tuple of intersections in the previous term. Then we can applie $\O_X^\times$:
\[\prod \O_X(U_i)^\times\to \prod\O_X(U_i\cap U_j)^\times\to\cdots\]
where, for instance, the first map above sends a product of maps $(s_i)$ to $(s_i|_{U_i\cap U_j}-s_j|_{U_i\cap U_j})$.

\begin{defn}
	$\hat H_\calU^i(X,\O_X^\times)$ is the $i^{th}$ cohomoogy of the above complex.
\end{defn}
Then it can be shown that $\hat H_\calU^i(X,\O_X^\times)$ is independent of cover if the $U_i$ are affine.

So what is $H^1(X,\O_X^\times)$? It is the $s_{i,j}in\O_X^\times(U_i\cap U_j)$ modulo the cocycle condition.

\subsection{A new result}
\begin{thm}
	$H^1(X,G)$ is identified with $G$-torsors.
\end{thm}
What is $H^1(X,G)$? We are going to try to recover \v Cech cohomology. Let $X$ be quasi-compact and let $\cup_1^n U_i\to X$ be 
faithfully flat. The seperatedness gets us the intersections are affine.

Then we can play the same game, but this time applying $G$: (notice that $U_i\cap U_j=U_i\times_X U_j$)
\[\prod G(U_i)\to \prod G(U_i\times_X U_j)\to\cdots\]
where (notice now $G$ may be nonabelian!) we map 
\[(s_i)\mapsto (s_i|_{U_i\times_X U_j}\cdot s_j|_{U_i\times_X U_j}^{-1})_{i,j}\]
and 
\[(s_{i,j})\mapsto (s_{i,j}\cdot s_{j,k}\cdot s_{i,k}^{-1})\]
where each $s$ is restricted to $U_i\times_XU_j\times_X U_k$.

Then we can define $\hat H_\calU^1(X,G)$ to be the first cohomology of the above chain. Then 
\begin{defn}
	The \textbf{flat cohomology} is 
	\[H^1_\text{flat}(X,G)=\operatorname{colim}_\calU H^1_\calU(X,G).\]
\end{defn}

The overall result here is 
\begin{thm}
	The $G$-torsors on $X$ in bijection with the elements of $H^1_\text{flat}(X,G)$.
\end{thm}

\section{November 6th, 2019}
Today we are going to prove some more things about representation theory. Down the pipe somewhere we will hope to talk about Tanakka duality, 
which gives us that Morita equivalence (is this still what it's called?) of two algebraic groups gives us the groups are isomorphic.

Given an algebraic group $G/k$, we could consider representations, which were either a group scheme morphism $G\to \GL(V)$ or else a coaction $V\xrightarrow{\sigma} \Gamma(G)\otimes_k V$.
Then we said that $G$ was linear reductive if and only if every representation is completely reducible,

\begin{ex}
	If $G=\Gm$, then every irreducible representation is one-dimensional: $\Gm\to\Gm$ sends $t\mapsto t^\alpha$.
\end{ex}
A natural question one may ask: what can we say in general?
\subsection{The regular representation}
The regular representation is $\Gamma(G)$ with coaction given by comultiplicaton.
\begin{prop}
	And finite dimensional representation $V$ of $G$ embeds $V\subseteq \Gamma(G)^{\oplus n}$.
\end{prop}
\begin{prf}
	The coaction gives us a map 
	\[V\xrightarrow{\sigma}\Gamma(G)\otimes V=\Gamma(G)\otimes_k V_{\text{vs}}\cong \Gamma(G)^{\oplus\dim V}\]
	where $V_{vs}$ is the underlying vector space of $V$. Then the claim is that the overall map is injective and that it is a map of $G$-reps.

	The latter property is almost tautological by the property of a coaction. Injectivity follows from the fact that 
	\[(e^\ast\otimes\id)\circ\sigma\]
	is an isomorphism $V\to k\otimes V$ (again by one of the coaction axioms), so $\sigma$ is injective.
\end{prf}
\begin{prop}
	If $V$ is a finite dimensional faithful representation, then every other finite-dimensional representation can be obtained from $V$ by direct sums, tensors, duals, subrepresentations, and quotients.
\end{prop}
\begin{rmk}
	More precisely, $W\subseteq (V\oplus V^\vee)^{\otimes n}$.
\end{rmk}
\begin{prf}
	$G\subseteq\GL(V)$ is a closed subgroup. Notice that since we have an injective map into $\GL(V)$, $G$ is already a linear algebraic group.
\end{prf}
\begin{thm}[Peter-Weyl]
	If $G$ is linear reductive, then 
	\[\Gamma(G)=\oplus_{\text{irr $V$}}(V\otimes V^\vee)\]
\end{thm}
\begin{rmk}
	Notice that this is as two-sided representations, but if we only want to look at (say) left representations, we give $V^\vee$ the trivial 
	representation structure and we get the more familiar 
	\[\Gamma(G)=\oplus_{\text{irr $V$}} V^{\oplus\dim V}\]
	and then the formula from finite groups:
	\[|G|=\sum_{\text{irr $V$}} (\dim V)^2\]
\end{rmk}
As an example, consider $\Gm=\Spec k[t]_t$ and $k[t]_t=\oplus_{n\in\bbZ}k\langle t^n\rangle$. As an exercise, thing about what happens for $\SL_2$.

\subsection{Stabilizers of subspaces}
If $G$ is an algebraic group over $k$ and $V$ is a $G$ representation, let $W\subseteq V$ be a subspace. We want a subgroup $G_W$ which plays a similar role as the stabilizer in group theory.

Consider $G$ as a functor from $\Alg_k\subseteq \Sch/k\to\Grp$ taking $T\mapsto\Hom(T,G)$. Then for a $k$-algebra $R,$ $G(R)$
acts on $V_R=V\otimes R$.
\begin{prop}
	The functor $\Alg_k\to\Grp$ sending 
	\[R\mapsto \{g\in G(R)|gW_R=W_R\}\]
	is representable by a subgroup of $G$, which we will denote $G_W$.
\end{prop}
\begin{rmk}
	The idea here is to fix a basis $e_i$ of $W$ and complete them to a basis of $V$ by appending $f_i$. Then take the coaction $V\to\Gamma(G)\otimes V$ and consider the image 
	\[f_k\mapsto \sum_{i\in I}a_{ki}\otimes e_i+\sum_{j\in J}a_{kj}\otimes f_j\]
	suppose we ahve $g\in G(R)=\Hom(\Spec R, G)$. This gives us a map in $\Hom(\Gamma(G),R)$. Then using the action on $V\otimes R$:
	\[g\cdot(f_k\otimes 1)=\sum g(a_{ki})\otimes e_i+\sum g(a_{kj})\otimes f_j\in V_R\]
	and we see that this is actually in $W_R$ exactly when $g(a_{kj})=0$ for all $j$.

	But then this means that $g$ lands in $V(a_{kj})$, the vanishing of the functions $a_{kj}\otimes 1\in\Gamma(G)_R$, so we get that 
	the closed subscheme $V(a_{kj})$ represents the functor $G_W$.
\end{rmk}

For the converse, let $G$ be an algebraic group over $k$ and let $H\subseteq G$ be a subgroup. Then there exists a finite dimensional representation $V$ of $G$ and 
a $L\subseteq V$ a one-dimensional subspace such that $H=G_L$.

As an example, look at $G=\SL_2$ acting on $k^2$. Let $L=\langle 1,0\rangle$. We can see easily that $L$ is preserved by $g\in G$ if and only if the lower-left coordinate of $g$ is zero.
Now if we take $\Gm=\operatorname{diag}(t,t^{-1})\subseteq\SL_2$, consider the representation $\langle xy\rangle\subseteq \operatorname{Sym}^2k^2$. This isn't quite it but maybe you can work it out!

\section{November 8th, 2019}
We started by talking elections. Go Andrew Lewis. You can do it.

Recall that last time we had a result that said that we had an analog of the stabilizer $G_W\subseteq G$ 
representing the functor 
\[R\mapsto \{g\in G(R)|gW_R=W_R\}.\]
The idea was to take a basis for $W$ and complete it to one of $V$ and then to find a group that fixes $W$. We found 
that $G_W=V(\{a_{kj}\})$.
\begin{thm}[Chevalley]
	If $H\subseteq G$ is a subgroup of an algebraic group over $k$, then there exists a $G$ representation $V$ and a line $L\subseteq V$ such that $H=G_L$.
\end{thm}
\begin{prf}
	Let $\pi:\Gamma(G)\twoheadrightarrow\Gamma(H)$ be the map induced by $H\hookrightarrow G$. Let $q_1,\dots,q_n$ be
	generators of $\ker\pi=I\subseteq\Gamma(G)$, the regular representation. Take a finite dimensional representation $V$ containing $I$ in $\Gamma(G)$
	and pick a basis $e_1,\dots,e_s$ of $W=V\cap I$. Extend this basis to one of $V$ with the additional vectors denoted $f_1,\dots,f_t$.

	Now the image of the coaction gives us 
	\[e_i\mapsto\sum_k a_{ik}\otimes e_k+\sum_k b_{jk}\otimes f_k\]
	and let $I'=(b_{jk})$. We claim that $I=I'$. If this is true, then $H=G_W$ and $W\subseteq V$, so we set
	\[L=\bigwedge^{\dim W}W\subseteq \bigwedge^{\dim W}V\]
	and now $H= G_L$.

	To see the claim, consider that since $e_i\in I$, we get $\sum_ka_{ik}\otimes e_i\in \Gamma(G)\otimes I$.
	But then the comultiplication on $\Gamma(G)$ gives us that 
	\[m^\ast(I)\subseteq I\otimes \Gamma(G)+\Gamma(G)\otimes I\subseteq \Gamma(G)\otimes\Gamma(G)\]
	so we get that 
	\[\sum b_{ik}\otimes f_k\in I\otimes \Gamma(G)+\Gamma(G)\otimes I\]
	but since $f_{ij}\notin I$, this forces $b_{ik}\in I$. Thus $I'\subseteq I$.

	For the other containment I got behind.
\end{prf}
\subsection{Quotients}
Let $H\subseteq G$ be normal. Now the goal is to construct $G/H$ as an algebraic group. As a preview, we are going to use the last theorem 
to get a representation $V$ of $G$ such that $H=G_L$ such that $G\to \GL(V)$ descends to a representation $G\to \GL(V^H)$. Then $G/H$ will be the image of this map in $\GL(V^H)$.

\begin{defn}
	A subgroup $H\subseteq G$ of an algebraic group over $k$ is \textbf{normal} if for all $k$-schemes $S$, $H(S)\le G(S)$ is anormal subgroup.
\end{defn}
\begin{rmk}
	As an exercise one can show that if $G(k)\subseteq G$, then $H\subseteq G$ is normal if and only if for all $g\in G(k)$, $gHg^{-1}\subseteq H$.
\end{rmk}
\begin{lem}
	If $\pi:H\to G$ is a homomorphism of algebraic groups, then $\ker\pi\subseteq H$ is normal.
\end{lem}
To see the above, consider the kernel is the pullback:
\begin{center}
	\begin{tikzcd}
		S\ar[dr,dashed]\ar[drr,bend left]\ar[ddr,bend right] & &\\
		& \ker\pi\ar[r]\ar[d] & \Spec k\ar[d,"e_G"]\\
		& H\ar[r,"\pi"] & G
	\end{tikzcd}
\end{center}
And then the kernel $\ker \pi(S)=K(S)$ is normal in $H$. Think about this one some more.
\begin{defn}
	A homomorphism of algebraic groups $G\to Q$ is a \textbf{quotient map} if $G\to Q$ is faithfully flat.
\end{defn}
\begin{prop}
	A quotient map of linear algebraic groups $\pi:G\to Q$ satisfies the universal property: for all $f:G\to H$,
	\begin{center}
		\begin{tikzcd}
			K=\ker\pi\ar[r]\ar[rd,"0"] & G\ar[r,"\pi"]\ar[d,"f"] & Q\ar[dl,dashed,"!"]\\
			& H &
		\end{tikzcd}
	\end{center}
\end{prop}

\section{November 13, 2019}
The big theorem here:
\begin{thm}
	Let $G$ be an algebraic group over $k$. Let $H\subseteq G$ be an algebraic subgroup (over $k$). Then the
	quotient $G/H$ exists as a quasi-projective scheme over $k$.

	Moreover if $H\subseteq G$ is normal, then $G/H$ is also an algebraic group.
\end{thm}

Today we are going to prove/see the case when $G$ is a linear algebraic group and smooth (reduced) and that $H\subseteq G$ is normal and $k=\bar k$.
The closedness of $k$ and the smoothness can be dropped at the expense of a couple extra lectures, probably.

\begin{defn}
	A map $G\xrightarrow{\pi} Q$ of algebraic groups over $k$ is a \textbf{quotient} if $\pi$ is faithfully flat.
\end{defn}

Using descent, we showed 
\begin{prop}
	If $G\xrightarrow{\pi} Q$ is a quotient, then  if we have another map $G\xrightarrow{\phi} P$ and furthermore that $\ker\pi\subseteq \ker\phi$,
	then we get a (unique) factorization $Q\to P$.
\end{prop}

We also have
\begin{prop}
	Any map of linear algebrai groups $G\xrightarrow{\phi} H$ factors as $G\to Q\hookrightarrow H$ where $G\to Q$ is the quotient map.
\end{prop}
\begin{prf}
	In the case that $G$ is reduced, we get a surjection onto $\Im\phi^\ast$ of $\Gamma(H)$, and $Q=\Spec\Im(\phi^\ast)$.
	Then it just takes showing that the surjection $\Gamma(H)\twoheadrightarrow \Im(\phi^\ast)$ induces a closed immersion of $Q$ in $H$.
	
	We also need a lemma that I missed but it can be found both in Milne and Waterhouse. This is where the heavy lifting is done.
\end{prf}
\section{November 15th, 2019}
Today we are going to be talking about properties of algebraic groups as well as properties of lements of $g\in G$. Today, we will be focusing on 
$g\in \GL(V)$, so essentially talking linear algebra.

\begin{defn}
	Let $V$ be a finite dimensional vector space over any field $k$. Let $g\in \GL(V)(k)$. Then $g$ is
	\begin{itemize}
		\item \textbf{diagonalizable} if there exists a basis of eigenvectors.
		\item \textbf{semisimple} if there exsits an extension $k'/k$ such that $g\otimes\id\in\GL(V\otimes_kk')$ is diagonalizable.
		\item \textbf{unipotent} if $g-\id$ is nilpotent.
		\item \textbf{triagonalizable} if there exists a basis so that $g$ is upper triangular.
	\end{itemize}
\end{defn}

Let $P_g\in k[T]$ be the characteristic polynomial for $g$. We say the \textit{eigenvalues of $g$ are in $k$} if $P_g$ splits over $k$. Thus if $g$ is diagonalizable, 
the eigenvalues are in $k$ and this is exactly equivalent to $g$ being triagonalizable. Furthermore $g$ is unipotent if and only if 
all of its eigenvalues are 1 or 0 (in $k$).
\begin{prop}
	If the eigenvalues $\lambda_i$ are in $k$, then 
	\[V\cong \oplus\ker(g-\lambda_i\id)^{a_i})\]]
	which gives us our Jordan decomposition in terms of generalized eigenspaces.
\end{prop}

\subsection{Jordan Decomposition}
\begin{thm}
	Let $k$ be perfect and let $g\in\GL(V)$ with $V$ finite dimensional over $k$. Then there exist unique $g_s$ and $g_u$ in $\GL(V)$ such that 
	\begin{itemize}
		\item $g=g_sg_u=g_ug_s$
		\item $g_s$ is semisimple and $g_u$ is unipotent.
	\end{itemize}
	Moreover, $g_s$ and $g_u$ are polynomials in $g$.
\end{thm}
\begin{prf}
	Assume first that $P_g(T)$ splits over $k$. Thus we can decompose 
	\[V\cong V_{\lambda_i}\]
	where $V_{\lambda_i}=\ker((g-\lambda_i\id)^{a_i})$.
	This gives us a Jordan decomposition of $g$ as a matrix with eigenvalues on the diagonal and is upper triangular. Take $g_s$ to be $\diag(\lambda_1,\cdots,\lambda_1,\dots,\lambda_s)$, 
	the diagonal of the matrix in this form. This is invertible (since $g\in\GL(V)$) so we set $g_u=g_s^{-1}g$. It is clear from this that these matrices commute and that they are 
	semisimple and unipotent, respectively.

	To get that these are polynomials in $g$, you go through an argument I missed. The uniqueness comes from looking at 
	\[g=g_sg_u=h_sh_u\]
	and taking 
	\[h_s^{-1}g_s=h_ug_u^{-1}\]
	and since everything commutes (check this), you get that these two elements are both semisimple and unipotent. But then they are diagonalizable with all 
	eigenvalues 1, so they are the identity and we are done.

	For the more general case, choose a finite extension $k'/k$ such that $P_g$ splits. Since $k$ is perfect, this is seperable. So we can assume it is Galois and 
	set $G=\Gal(k'/k)$. So $g\otimes \id\in\GL(V\otimes_k k')$ factors uniquely as $g\otimes \id=g_ug_s$ with $g_u,g_s\in\GL(V)(k')$.

	But then using the action of $G$ on $\GL(V)(k')$, by uniqueness of this decomposition $\sigma g_s=g_s$ and $\sigma g_u=g_u$ for all $\sigma\in G$ so 
	the elements are in fact in the field fixed by $G$ (i.e. $k$). Similarly $G$ acts on $Q(T)\in k'[T]$ but must fix the polynomials describing the factors as polynomials in $g$.
\end{prf}

As an example, consider a field $k$ of characteristic 2 and pick $\alpha\in k/k^2$ (obviously $k$ is not perfect). Then the matrix $g=(\begin{smallmatrix}
	0& 1\\\alpha,0
\end{smallmatrix})$ has polynomial $T^2-\alpha$, so with a single repeated eigenvalue $\sqrt a\notin k$.

\subsection{A broader context}
So then if $\iota:G\hookrightarrow\GL(V)$ is inclusion (faithful representation) we can decompose $\iota(g)=g_ug_s$ using the above results. Some properties of this phenomenon:
\begin{prop}
	If $L:V\to W$ is a linear transformation and $g\in\GL(V)$ and $h\in\GL(W)$ such that $L\circ g=h\circ L$,
	then $L\circ g_\ast=h_\ast\circ L$ for $\ast=u,s$.
\end{prop}
\begin{prop}
	\begin{align*}
		(g\oplus h)_\ast&=g_\ast\oplus h_\ast\\
		(g\otimes h)_\ast&=g_\ast\otimes h_\ast
	\end{align*}
	for $\ast=s,u$.
\end{prop}
\begin{prop}
	If $\rho:\GL(V)\to \GL(W)$ is a representation, then $\rho(g)_\ast=\rho(g_\ast)$.
\end{prop}

\begin{defn}
	For $g\in G(k)$, $g$ is \textbf{semisimple (resp. unipotent)} if for all representations $\rho:G\to \GL(V)$,
	$\rho(g)$ is semisimple (resp. unipotent).
\end{defn}

\section{November 18th, 2019}
\subsection{Generalizing the Jordan decomposition}
The further generalization of the Jordan decomposition we saw last time is
\begin{thm}
	If $k$ is perfect and $G$ is a linear algebraic group over $k$, then for any $g\in G(k)$, there exist unique $g_s,g_u\in G(k)$ such that 
	\begin{itemize}
		\item $g=g_sg_u=g_ug_s$
		\item $g_s$ is semisimple and $g_u$ is unipotent.
		\item $g_s$ and $g_u$ are polynomials in $g$.
	\end{itemize}
\end{thm}
Last time we proved this for $G=\GL(V)$. Then we saw a collection of propositions that gave us that semisimplicity and unipotence act nicely 
with respect to direct sum, tensor, and representations. This allowed us to define what semisimple and unipotent objects are in $G(k)$. This gave us 
a result which I think I botched last time:
\begin{prop}
	If $G$ is alinear algebraic group and $V$ is a faithful finite dimentional representation ($G\hookrightarrow \GL(V)$ is a closed embedding), then eery representation 
	of $G$ is obtained from $V$ using sums, tensors, subrepresentations, and quotients.
\end{prop}
\begin{rmk}
	An aside: what does it mean for a subspace $W\subset V$ to be fixed by $g\in G$ ($V\in \lmod G$)? If $V$ is finite-dimensional, then one can just take the image of $G$ in $\GL(V)$
	and ask whether $gW\subseteq W$.

	More generally since $g$ is a $k$-point, there is a maximal ideal $\frakm_g\lhd \Gamma(G)$ such that 
	\[\Gamma(G)/\frakm_g\cong k\] 
	with quotient map $q$. Then this gives us a map $\Gamma(G)\otimes V\xrightarrow{q\otimes\id} k\otimes V\cong V$ which, when composed with the coaction $V\to\Gamma(G)\otimes V$,
	gives us a map $m_g:V\to V$. Then we can say that $W$ is fixed by $g$ if $m_g(W)\subseteq W$.
\end{rmk}
\begin{prf}[in general]
	Since $G$ is linear algebraic, we can embed $G\hookrightarrow\GL_n$ for some $n$. This gives us a way to decompose $g=g_sg_u$, but the question is 
	whether $g_s$ and $g_u$ actually lie in (the image of) $G$. Thus for every representation $\rho:G\to \GL(W)$, we get a pair of elements $\rho(g)_s$ and $\rho(g)_u$. Due to the properties we saw before, 
	these are nice.

	The idea here is that one representation we have at our disposal is the regular representation. In general this is large, but it recovers our group $G$ in some way and this will be key to our discussion.
	Let $V=\Gamma(\GL_n)$, the regular representation of $\GL_n$. Then we get maps 
	\[G\hookrightarrow\GL_n\to\GL(V)\]
	and we have an ideal $I\subseteq \Gamma(GL_n)$ where $I$ is the ideal defining $G$. Then the claim is that $g\in G(k)$ stabilizes $I$. This is in Waterhouse chapter 9.
\end{prf}

\subsection{Group properties}
This enables us to make more classifications of groups:
\begin{prop}
	We say that $G$ is \textbf{diagonalizable} if there is a faithful representation $G\hookrightarrow\GL_n$ that factors through the 
	diagonal matrices in $\GL_n$.
\end{prop}
\begin{rmk}
	This is equivalent to the earlier definition of there being a closed embedding of $G\hookrightarrow\Gm^n$.
\end{rmk}
\begin{defn}
	We say that $G$ is of \textbf{multiplicative type} if $G\times_k\bar k$ is diagonalizable.
\end{defn}

\begin{prop}
	If $G$ is a commutative linear algebraic group over $K$ and all elements $g\in G(k')$ for $k\to k'$ a field extension are semisimple, then $G$ is of multiplicative type.
\end{prop}
\begin{prf}
	We can assume that $k=\bar k$ and we know that all $g\in G(k)$ are diagonalizable. Then we use the fact that commuting diagonalizable 
	matrices are simultaneously diagonalizable. The result follows.
\end{prf}
\begin{rmk}
	The proof of the linear algebra fact above comes from taking your commuting diagonalizable elements $g$ and $h$ and showing that each 
	$\lambda$ eigenspace of $g$ is fixed by $h$ and vice versa, giving us the same number and dimension of eigenspaces.
\end{rmk}

\end{document}