\documentclass[12pt]{article}

\usepackage{setspace}

\usepackage{amsmath, amsfonts, amssymb, graphicx, color, fancyhdr, lipsum, scalerel, stackengine, mathrsfs, tikz-cd, mdframed, enumitem, framed, adjustbox, bm, upgreek, x	color}
\usepackage[framed,thmmarks]{ntheorem}
\usepackage[mathscr]{euscript}

%set up theorem/definition/etc envs
%Problems will be created using their own counter and style
\theoreminframepreskip{0pt}
\theoreminframepostskip{0pt}
\newframedtheorem{prob}{Problem}[part]
\renewcommand\theprob{\arabic{part}.\arabic{prob}}

%solution template
\theoremstyle{nonumberbreak}
\theoremindent0.5cm
\theorembodyfont{\upshape}
\theoremseparator{:}
\theoremsymbol{\ensuremath\spadesuit}
\newtheorem{sol}{Solution}

%Theorems, Lemmas, and Corollaries
\theoremstyle{changebreak}
\theoremseparator{}
\theoremsymbol{}
\theoremindent0.5cm
\theoremheaderfont{\color{blue}\bfseries} 

\newtheorem{thm}{Theorem}[subsection]
\theoremheaderfont{\bfseries}
\newtheorem{lem}[thm]{Lemma}
\newtheorem{cor}[thm]{Corollary}

%Create a new env that references a theorem and creates a 'primed' version
%Note this can be used recursively to get double, triple, etc primes
\newenvironment{thm-prime}[1]
  {\renewcommand{\thethm}{\ref{#1}$'$}%
   \addtocounter{thm}{-1}%
   \begin{thm}}
  {\end{thm}}

\setlength\fboxsep{15pt}

%Shade definitions
\theoremindent0cm
\theoremheaderfont{\normalfont\bfseries} 
\def\theoremframecommand{\colorbox[rgb]{.7,.9,1}}
\newshadedtheorem{defn}[thm]{Definition}

%Man, that's really good! Let's use the same thing for definitons.
\newenvironment{def-prime}[1]
  {\renewcommand{\thethm}{\ref{#1}$'$}%
   \addtocounter{thm}{-1}%
   \begin{def}}
  {\end{def}}

%proofs
\theoremstyle{nonumberplain}
\theoremindent0.5cm
\theoremheaderfont{\sc}
\theoremseparator{:}
\theoremsymbol{\ensuremath\spadesuit}
\newtheorem{prf}{Proof}

%remarks
\theoremstyle{change}
\theoremindent0.5cm
\theoremheaderfont{\sc}
\theoremseparator{:}
\theoremsymbol{}
\newtheorem{rmk}[thm]{Remark}

%waterhouse problem
\newenvironment{wprob}[1]{\begin{prob}{\normalfont\bfseries (Waterhouse #1) }\itshape}{\end{prob}}

%Replacement for the old geometry package
\usepackage{fullpage}

%Put page breaks before each part
\let\oldpart\part%
\renewcommand{\part}{\clearpage\oldpart}%

%Center each figure by default
\makeatletter
\g@addto@macro\@floatboxreset{\centering}
\makeatother

%header stuff
\setlength{\headsep}{24pt}  % space between header and text
\pagestyle{fancy}     % set pagestyle for document
\lhead{Notes on Affine Group Schemes and Their Representations} % put text in header (left side)
\rhead{Nico Courts} % put text in header (right side)
\cfoot{\itshape p. \thepage}
\setlength{\headheight}{15pt}
\allowdisplaybreaks

%Set of Integers
\newcommand*{\Z}{
\mathbb{Z}
}
%Set of Natural Numbers
\newcommand*{\N}{
\mathbb{N}
}
%Set of Real Numbers
\newcommand*{\R}{
\mathbb{R}
}
%Set of Complex Numbers
\newcommand*{\C}{
\mathbb{C}
}
%Field
\newcommand*{\F}{
\mathbb{F}
}
%Rationals
\newcommand*{\Q}{
\mathbb{Q}
}

%Section break
\newcommand*{\brk}{
\rule{2in}{.1pt}
}

\DeclareMathOperator{\Aut}{Aut}

%raise that Chi!
\DeclareRobustCommand{\Chi}{{\mathpalette\irchi\relax}}
\newcommand{\irchi}[2]{\raisebox{\depth}{$#1\chi$}} 

%Image
\DeclareMathOperator{\im}{Im}

%Hom
\DeclareMathOperator{\Hom}{Hom}

%Ext
\DeclareMathOperator{\Ext}{Ext}

%Coker
\DeclareMathOperator{\coker}{coker}

%characteristic
\DeclareMathOperator{\ch}{char}

%restriction
\DeclareMathOperator{\Res}{Res_H}

%socle
\DeclareMathOperator{\Soc}{Soc}

%rank
\DeclareMathOperator{\rank}{rank}

%induction
\DeclareMathOperator{\Ind}{Ind_H^G}

%identity map
\DeclareMathOperator{\id}{id}

%schemes
\DeclareMathOperator{\Ga}{\ensuremath\mathbf{G_a}}
\DeclareMathOperator{\Gm}{\ensuremath\mathbf{G_m}}
\newcommand*{\GL}[1]{\ensuremath\mathbf{GL_#1}}
\newcommand*{\SL}[1]{\ensuremath\mathbf{SL_#1}}
\newcommand*{\mun}[1]{\ensuremath\mathbf{\bm\upmu_#1}}
\newcommand*{\alp}[1]{\ensuremath\mathbf{\bm\upalpha_#1}}

%fix tilde
\let\tilde\relax
\newcommand*{\tilde}[1]{\widetilde{#1}}

% Enumerate will automatically use letters (e.g. part a,b,c,...)
\setenumerate[0]{label=(\alph*)}

\begin{document}
%make the title page
\title{Notes and Problems from My Research\vspace{-1ex}}
\author{Nico Courts}
\date{}
\maketitle

%begin problems

\part{Autumn 2018}
\section{Problems}
\begin{prob}
	Assume that $k$ is a field and let $K=k(t)$ (notice $K$ is a transcendental extension). Prove that $\Hom_k(K,k)\not\cong K$.
\end{prob}

\begin{sol}
	This is basically just a cardinality argument. I don't think it's particularly worth doing at this juncture.
\end{sol}

\begin{prob}
	Let $G$ be a finite group scheme (actually we need only assume that $G$ is a Frobenius algebra so that a module
	is injective if and only if it is projective). Prove that unless $M$ is projective, its projective dimension is 
	infinite. Conclude that $H^n(G,M)=0$ for $n>N$ implies that $M$ is projective.
\end{prob}
\begin{sol}
	Assume $M$ itself is not projective so that its minimal projective resolution is nontrivial and furthermore
	that it is finite. That is, let $P_i$ be projective modules such that
	\[0\to P_n\xrightarrow{f_n} P_{n-1}\to\cdots \xrightarrow{f_{1}}P_0\xrightarrow{f_0} M\to 0\]
	is a minimal length projective resolution of $M$ (notice here that $n\ge 1$). 
	
	Next consider the short exact sequence
	\[0\to P_n\xrightarrow{f_n}P_{n-1}\to \coker f_n\to 0\]
	since $P_n$ is projective (and thus injective!) this sequence splits and therefore $P_{n-1}\cong P_n\oplus\coker f_n.$
	But then consider the sequence
	\[0\to P_n\xrightarrow{g} P_{n-2}\to\cdots\xrightarrow{f_0}M\to 0\]
	where above we are using $P_{n-1}\supseteq P_n\cong f_n(P_n)$  and that
	$g=f_{n-1}|_{f_n(P_n)}$. This map is injective since $\ker f_{n-1}=\coker f_n$, which is disjoint
	from $f_n(P_n)\cong P_n$. Exactness everywhere else is evident since the maps are not effectively changed.

	But then the existence of this sequence contradicts the minimality of the original sequence, so
	no finite sequence can exist.

	The last statement (as discussed with Julia) is actually false. 
\end{sol}

\begin{prob}
	Establish the five-term exact sequence for spectral sequences.
\end{prob}
\begin{sol}
	I plan to return to this problem in the future. I have other priorities at the moment,
	but I will eventually return to cohomology and spectral sequences and this will be a good
	exercise at that point.
\end{sol}

\begin{wprob}{1.1}
	\begin{enumerate}
		\item If $R$ and $S$ are two $k$ algebras and $F$ is a representable functor, show $F(R\times S)\cong F(R)\times F(S)$.
		\item Show there is no representable functor $R$ such that every $F(R)$ has exactly two elements.
		\item Let $F$ be the functor represented by $k\times k$. Show that $F(R)$ has two elements exactly when $R$ has no idempotents besides 0 and 1.
	\end{enumerate}
\end{wprob}

\begin{sol}
	\subsubsection*{(a)}
	Let $A$ be the $k$-algebra representing $F$. Thus $F(R)$ is naturally isomorphic to
	$\Hom_k(A,R)$ and $F(S)\simeq\Hom(A,S)$. Then define the map $\Phi:\Hom(A,R\times S)\to\Hom(A,R)\times\Hom(A,S)$
	via
	\[\Phi(\varphi)=(\pi_R\circ\varphi,\pi_S\circ\varphi)\]
	where $\pi_X$ is the canonical projection onto $X$.

	This is surjective since (by the universal property of products) any pair of maps
	$\varphi_R:A\to R$ and $\varphi_S:A\to S$ factors through the product $R\times S$
	and furthermore it does so \textit{uniquely}, giving us injectivity. Thus this map
	(which is clearly a homomorphism since $\pi_X$ is) is a bijection.
	\subsubsection*{(b)}
	By the last problem this is impossible since if $|F(k)|=2$ then 
	\[|F(k\times k)|=|F(k)\times F(k)|=4.\]
	\subsubsection*{(c)}
	Let $F$ be such a functor. Consider any $\varphi\in\Hom(k\times k, R)\simeq F(R)$.
	Assume first that $F(R)\cong \Z/2$ and let $r$ be an idempotent in $R$. 
\end{sol}

\begin{wprob}{1.2}
	Let $E$ be a functor represented by $A$ and let $F$ be any functor. Show that the natural
	maps $\eta:E\to F$ correspond to elements in $F(A)$.
\end{wprob}
\begin{sol}
	Consider the map $\Phi$ from natural maps $E\to F$ to elements in $F(A)$ defined by
	(again leveraging the representability of $E$)
	\[\eta\mapsto \eta(\operatorname{id}_A)\in F(A).\]
	Conversely, consider the map $\Psi$ from $F(A)$ to the natural maps $E\to F$ via
	\[x\mapsto \xi_x\]
	where $\xi_x$ where for any $Y$ and $y\in E(Y)\cong \Hom(A,Y)$ we define the $Y^{th}$ component of $\xi_x$
	as 
	\[\xi_x(y)=F(y)(x)\in F(Y)\]
	where (for clarity while I get a grasp here) $F(y):F(A)\to F(Y)$.

	Since we are only looking for a bijection, we only need that these maps are inverses.
	Consider that for all $Y$ and $y\in E(Y)$,
	\begin{align*}
		\Psi\circ\Phi(\eta)(y)&=\Psi\left(\eta(\operatorname{id}_A)\right)(y)\\
		&= \xi_{\eta(\operatorname{id}_A)}(y)\\
		&= F(y)\circ \eta (\operatorname{id}_A)\\
		&= \eta\circ E(y) (\operatorname{id}_A)\\
		&= \eta(y\circ\operatorname{id}_A)=\eta(y)
	\end{align*}
	where above we used the naturality of $\eta$ along with the fact that $E(y)$ is just
	precomposition with $y$. Thus $\Psi\circ\Phi(\eta)=\eta.$

	But then for any $x\in F(A)$,
	\begin{align*}
		\Phi\circ\Psi(x)&=\Phi\circ\xi_x\\
		&=\xi_x(\operatorname{id}_A)\\
		&=F(\operatorname{id}_A)(x)\\
		&=\operatorname{id}_{F(A)}(x)=x
	\end{align*}
	completing the proof.
\end{sol}

\begin{wprob}{1.3}
	Let $E$ be a functor represented by $A$, and let $F$ be any functor. Let $\Psi:F\to E$
	be a natural map with surjective component maps. Show there is a natural map $\Phi:E\to F$
	with $\Psi\circ\Phi=\operatorname{id}_E$.
\end{wprob}
\begin{sol}
	Since in particular $\Psi_A$ is surjective, there is an $x\in F(A)$ such that $\Psi(x)=\operatorname{id}_A$.
	Then using the map from the last problem, let $\Phi=\xi_x$. Then we can compute for any $R$ and $g\in E(R)$
	\begin{align*}
		\Psi\circ\Phi (g) &= \Psi\circ F(g) (x)\\
		&=E(g)\circ \Psi(x)\\
		&=E(g)(\operatorname{id}_A)\\
		&=g\circ \operatorname{id}_A=g
	\end{align*}
	since $g:A\to R$, so $E(g):E(A)\to E(R)$, which is just composition with $g$.
\end{sol}

\begin{wprob}{1.5}
	Write out $\Delta,\varepsilon,$ and $S$ for the Hopf algebras representing $\SL{2}, \mun{n},$ and $\alp{p}$.
\end{wprob}
\begin{sol}
	\subsubsection*{$\SL{2}$:}
	Notice $SL_2$ is represented by $A=k[X_1,X_2,X_3,X_4]/(X_1X_4-X_3X_2-1)$ so take two elements
	$f,g\in\Hom(A,R)$ where $f(X_i)=a_i\in R$ and $g(X_i)=b_i\in R$ and notice that we want
	\[(f,g)\Delta=h\]
	where since
	\[\begin{pmatrix}
		a_1 & a_2\\ a_3 & a_4
	\end{pmatrix}\begin{pmatrix}
		b_1 & b_2\\ b_3 & b_4
	\end{pmatrix}=\begin{pmatrix}
		a_1b_1+a_2b_3 & a_1b_2+a_2b_4\\ a_3b_1+a_4b_3 & a_3b_2+a_4b_4
	\end{pmatrix}=\begin{pmatrix}
		c_1 & c_2\\ c_3 & c_4
	\end{pmatrix}\]
	we want to have that $h(X_i)=c_i.$

	So then if $\Delta:A\to A\otimes A$ is defined as follows:
	\begin{align*}
		X_1&\mapsto X_1\otimes X_1+X_2\otimes X_3\\
		X_2&\mapsto X_1\otimes X_2 +X_2\otimes X_4\\
		X_3&\mapsto X_3\otimes X_1+X_4\otimes X_3\\
		X_4&\mapsto X_3\otimes X_2+X_4\otimes X_4
	\end{align*}
	Where one can compute
	\begin{align*}
		\id\otimes\Delta\circ \Delta (X_1)&= (\id\otimes\Delta)(X_1\otimes X_1+X_2\otimes X_3)\\
		&= X_1\otimes X_1\otimes X_1 + X_1\otimes X_2\otimes X_3 + X_2\otimes X_3\otimes X_1 + X_2\otimes X_4\otimes X_3\\
		&= (\Delta\otimes\id)(X_1\otimes X_1+X_2\otimes X_3)=\Delta\otimes\id\circ \Delta (X_1)
	\end{align*}
	and similar equality holds for the other $X_i$, so this is $\Delta$. 

	Using that we want $\varepsilon\otimes\id\circ \Delta(X_i)=1\otimes X_i$, we see that the map
	$\varepsilon$ sending $X_1$ and $X_4$ to 1 and $X_2$ and $X_3$ to zero is the map we want.

	Notice that as a sanity check we get that 
	\[\begin{pmatrix}
		\varepsilon(X_1) & \varepsilon(X_2)\\
		\varepsilon(X_3) & \varepsilon(X_4)
	\end{pmatrix}=\begin{pmatrix}
		1 & 0\\
		0 & 1
	\end{pmatrix}=I_2\]

	Finally using that $(S,\id)\circ\Delta(X_i)=\varepsilon(X_i)$ and the fact that in $A$,
	$\det=X_1X_4-X_3X_2=1$, we can define $S$ such that
	\[\begin{pmatrix}
		S(X_1) & S(X_2)\\
		S(X_3) & S(X_4)
	\end{pmatrix}=\begin{pmatrix}
		X_4 & -X_2\\
		-X_3 & X_1
	\end{pmatrix}=\begin{pmatrix}
		X_1 & X_2\\
		X_3 & X_4
	\end{pmatrix}^{-1}\]
	and one can verify that this satisfies the relation above.
	\subsubsection*{$\mun{n}:$}
	For this scheme, $A=k[X]/(X^n-1)$ is the representing algebra. If $f,g\in\Hom(A,k)$ with
	$f(X)=r$ and $g(X)=s$, then we want $(f,g)\Delta(X)=\sum f(X_{(1)})g(X_{(2)})=m(r,s)=rs$ where 
	$\Delta(X)=\sum X_{(1)}\otimes X_{(2)}$. An obvious choice is the diagonal map.

	Then choosing $\varepsilon(X)=1$ satisfies the diagrams (as before in $G_m$) and using our
	intuition, $S(X)=X^5$ which also works.
	\subsubsection*{$\alp{p}$:}
	This time we are working with $A=\Z/p[X]/(X^p)$. This time (since the group is additive) we want
	$\Delta(X)=X\otimes 1+1\otimes X$, which we can see works with associativity immediately.

	Following suit with the other additive group scheme $G$, setting $\varepsilon(X)=0$ and $S(X)=-X$
	we can quickly check these still satisfy the given axioms.
\end{sol}

\begin{wprob}{1.6}
	In $A=k[X_{11},\dots,X_{nn},1/\operatorname{det}]$ representing $GL_n$, show that $\Delta(X_{ij})=\sum X_{ik}\otimes X_{kj}$.
	What is $\varepsilon(X_{ij})$?
\end{wprob}
\begin{sol}
	Due to the uniqueness of $\Delta,\varepsilon,$ and $S$, we need only find maps satisfying 
	the diagrams. I claim that $\varepsilon(X_{ij})=\delta_{ij}$. In this case, notice
	\[(\varepsilon\otimes\operatorname{id})\circ\Delta(X_{ij})=\varepsilon\otimes\operatorname{id}\left( \sum X_{ik}\otimes X_{kj} \right)=\sum \delta_{ik}\otimes X_{kj}=1\otimes X_{ij}\]
	exactly as we want.

	For associativity, notice
	\[(\Delta\otimes \operatorname{id})\circ \Delta (X_{ij})=\Delta\otimes\operatorname{id}\left( \sum_k X_{ik}\otimes X_{kj} \right)=\sum_k\left(\sum_l X_{il}\otimes X_{lk}\right)\otimes X_{kj}\]
	and then the associativity of $\Delta$ follows simply from the associativity of the tensor product.

	For the last axiom, we compute $S$ such that $(S,\operatorname{id})\circ\Delta=\iota\circ\varepsilon$ where $\iota:K\to A$ is the map sending $k\mapsto k\cdot 1_A$.
	That is, we define $S:A\to A$ so that
	\[\sum_k S(X_{ik})X_{kj}=\delta_{ij}.\]

	We want to leverage the fact that for a fixed $i$ and $j$, the determinant is
	\begin{align*}
		\operatorname{det} &= \sum_{\sigma\in S_n}\operatorname{sgn}(\sigma)\prod_l X_{\sigma(l)l}\\
		&=\sum_\sigma \operatorname{sgn}(\sigma)X_{\sigma{j}j}\prod_{l\ne j} X_{\sigma(l)l}\\
		&=\sum_i X_{ij}\left(\sum_{\sigma(j)=i}\operatorname{sgn}(\sigma)\prod_{l\ne j}X_{\sigma(l)l}\right)
	\end{align*}
	and so we want that 
	\[S(X_{ik})=\frac{1}{\operatorname{det}}\sum_{\sigma(i)=k}\operatorname{sgn}(\sigma)\prod_{l\ne k}X_{\sigma(l)l}\]
	so that when $i=j$,
	\[\sum_k S(X_{ik})X_{kj}=\frac{1}{\operatorname{det}}\sum_k X_{kj}\sum_{\sigma(j)=k}\operatorname{sgn}(\sigma)\prod_{l\ne k}X_{\sigma(l)l}=1=\delta_{ij}\]
	whenever $i\ne j$, however, this equation is the determinant of the matrix where we have replaced
	the $j^{th}$ column with a copy of the $i^{th}$ column. This is linearly dependent, so 
	\[\frac{1}{\operatorname{det}}\sum_k S(X_{ik})S_{kj}=0=\delta_{ij}.\]
	Thus these are precisely the maps we desire.
\end{sol}

\begin{wprob}{1.10}
	Prove the following Hopf algebra facts by interpreting them as statements about group functors:
	\begin{enumerate}
		\item $S\circ S=\id$
		\item $\Delta\circ S=(\operatorname{twist})\circ(S\otimes S)\Delta$
		\item $\varepsilon\circ S=\varepsilon$
		\item The map $A\otimes A\to A\otimes A$ sending $a\otimes b$ to $(a\otimes 1)\Delta(b)$
		is an algebra isomorphism.
	\end{enumerate}
\end{wprob}
\begin{sol}
	\subsubsection*{(a)}
	Dualizing, we get
	\begin{figure}[h]
		\begin{tikzcd}[column sep=small]
			A\ar[rr,equal]\ar[rd,"S",swap] & & A\\
			& A\ar[ru,"S",swap] &
		\end{tikzcd}
		\begin{tikzcd}[column sep=small]
			A\ar[rr,equal] & & A\ar[ld,"inv"]\\
			& A\ar[lu,"inv"] &
		\end{tikzcd}
	\end{figure}

	\noindent so this statement is equivalent to the group (scheme) fact that $(g^{-1})^{-1}=g$.
	\subsubsection*{(b)}
	Using a similar duality argument, this is equivalent to saying 
	\[\operatorname{inv}\circ m = m\circ(\operatorname{inv}\times\operatorname{inv})\circ(\operatorname{twist})\]
	but if we consider arbitrary elements $g,h\in G(R)$, this means
	\[(gh)^{-1}=m\circ(\operatorname{inv}\times\operatorname{inv})(h,g)=m(h^{-1},g^{-1})=h^{-1}g^{-1}\]
	which is clearly true.
	\subsubsection*{(c)}
	This one is equivalent to $(\operatorname{inv})\circ i=i$ where if $g\in G(R)$, 
	\[(\operatorname{inv})\circ i(g)=(\operatorname{inv})(e)=e=i(g)\]
	or in other words $e^{-1}=e.$
	\subsubsection*{(d)}
\end{sol}

\part{Winter 2019}
\section{Preparation for the Quarter}
This is my first official quarter as Julia's student! My plan for now is to continue working on Waterhouse as well
as learn about algebraic geometry (alongside my usual classes, of course).
To give a sense of direction, Julia recommended that I take a look at the following regularity theorem:
\begin{thm}[Smoothness Theorem]\label{smooth}
	Let $G$ be an algebraic affine group scheme over a field $k$. Then $k[G]\otimes \bar k$ is reduced if and
	only if $\dim G = \rank \Omega_{k[G]}$.
\end{thm}
\section{Affine Group Schemes (Waterhouse and Jantzen)	}
I have quite a bit of information to process before this, so I will get started!
\begin{defn}[Closed Embedding]
	If $G$ and $H$ are affine group schemes represented by $A$ and $B$, respectively and
	if $\psi:H\to G$ is a homomorphism of affine group schemes (locally a group homomorphism)
	then if the corresponding algebra map $A\to B$ is surjective, then $\psi$ is called
	a \textbf{closed embedding.}
\end{defn}

As its name suggests, this means that $\psi$ is an isomorphism onto a \textbf{closed subgroup}
$H'$ of $G$. This is, in fact, a definition of this property. One can also think about 
it in the following ways: a group scheme $H$ is closed in $G$ if
\begin{itemize}
	\item $H$ is defined by the relations imposed by $G$ plus some additional ones.
	\item $H=V(I)$ for some ideal $I\subset k[G]$. 
\end{itemize}
Thinking back to our algebraic geometry, these are not too hard to see as equivalent. 
For instance, there is a closed embedding of $\mun n$ in $\Gm$ (simply adding in $x^n-1$)
and of $\SL{n}$ in $\GL{n}$ (adding $\det = 1$).

\subsection{Hopf Ideals}
One problem with the above characterization that one cannot choose $I$ arbitrarily and end
up with a group scheme. This is equivalent to arbitrarily adding relations to a group, which is
not always guaranteed to work out well (think adding $\det = 2$ to $\GL n$).

Actually, we can exactly categorize the closed embeddings of subgroups in $G$ by considering certain
ideals of the algebra $A$ which represents it. 
\begin{defn}
	Let $A$ be an algebra and $I\lhd A$. Then if 
	\begin{itemize}
		\item $\Delta(I)$ goes to zero under the map $A\otimes A\to A/I\otimes A/I$,
		\item $S(I)\subseteq I$
		\item $\varepsilon(I)=0$
	\end{itemize}
	then $I$ is called a \textbf{Hopf Ideal of $A$.}
\end{defn}

\begin{rmk}
	That these ideals exactly characterize closed subgroups follows since any group (scheme) represented
	by $A'$ has the property that $\Delta'(A')\subseteq A'\otimes A'$. Since the new comultiplication map
	$\Delta'$ is derived from the old one $\Delta$, we have the diagram
	\begin{figure}[h]
		\begin{tikzcd}
			A \ar[r,"\Delta"]\ar[d] & A\otimes A\ar[d]\\
			A/I\ar[r,"\Delta'"] & A/I\otimes A/I
		\end{tikzcd}
	\end{figure}
	
	\noindent from which we see that $\Delta(I)$ must go to zero in the map on the right.
	Similarly, we want that $S$ and $\varepsilon$ satisfy similar diagrams.

	Note that an example of such an ideal is $I=\ker(\varepsilon)$.
\end{rmk}

\begin{defn}
	A \textbf{character} is a scheme morphism $G\to \Gm$.
\end{defn}

Let $\Phi$ be a character of $G$ and notice that the corresponding hopf algebra map 
$\varphi:k[X,1/X]\to A$ is defined by $\varphi(X)=b$, which
is automatically invertible in $A$. Furthermore since for any $a,b\in G(R)$ we have
\[\Phi\circ m(a,b)=\Phi(ab)=\Phi(a)\Phi(b)=m\circ(\Phi,\Phi)(a,b)\]
we get the diagram
\begin{figure}[h]
	\begin{tikzcd}
		G\times G\ar[r,"\Phi\times\Phi"]\ar[d,"m"] & \Gm\times \Gm\ar[d,"m"]\\
		G\ar[r,"\Phi"] & \Gm
	\end{tikzcd}
\end{figure}

\noindent and then by dualizing we get that
\newpage
\begin{figure}[h]
	\begin{tikzcd}
		A\otimes A & k[X,X^{-1}]\otimes k[X,X^{-1}]\ar[l,"\varphi\otimes\varphi",swap]\\
		A\ar[u,"\Delta"] & k[X,X^{-1}]\ar[u,swap,"\Delta"]\ar[l,"\varphi"]
	\end{tikzcd}
\end{figure}

\noindent where in both diagrams we are abusing notation by using the same symbols for 
(co)multiplication in the different schemes. So we conclude that 
\[\Delta(b)=\Delta\circ\varphi(X)=(\varphi\otimes\varphi)\circ \Delta(X)=(\varphi\otimes\varphi)(X\otimes X)=b\otimes b.\]

Then we can easily compute that $\varepsilon(b)=1$ and $S(b)=b^{-1}$. This is precisely:
\begin{defn}
	Let $A$ be a Hopf algebra and $a\in A$ such that
	\begin{itemize}
		\item $a$ is invertible
		\item $\Delta(a)=a\otimes a$
		\item $\varepsilon(a)=1$
		\item $S(a)=a^{-1}$
	\end{itemize}
	then $a$ is called a \textbf{group-like} element of $A$.
\end{defn}
\begin{rmk}
	As we saw above, every character of an affine group scheme $G$ corresponds to a 
	group-like element of its representing algebra.
\end{rmk}
\begin{rmk}
	Using a parallel construction with any morphism $\Phi:G\to\Ga$, we get an element $b\in A$
	such that $\Delta(b)=b\otimes 1+1\otimes b$, $\varepsilon(b)=0$ and $S(b)=-b$. These elements
	are called \textbf{primitive}
\end{rmk}

\begin{defn}
	A group scheme $G$ represented by $A$ that consists entirely of group-like elements is
	called \textbf{diagonalizable.}
\end{defn}
\begin{rmk}\label{rmk-diag}
	Notice that an alternative definition is to begin with an abelian group $M$ and to 
	define $S,\Delta,$ and $\varepsilon$ such that each element is group-like. The resulting
	Hopf algebra represents a diagonalizable group scheme.
\end{rmk}

Consider the group algebra $k[M]$ on the group in Remark \ref{rmk-diag}. If this algebra
is finitely generated, we get the following:
\begin{thm}
	Let $G$ be diagonalizable and represented by $A$ and assume $A$ is finitely generated
	as a $k$-algebra. Then $G$ is a finite product of copies of $\Gm$ and $\mun n$.
\end{thm}
\begin{prf}
	Let $x_1,\dots, x_n$ be generators for $k[M]=A$. Each $x_i$ can be written as a (finite!)
	$k$-linear combination of elements in $M$, so we can instead use the finitely many $m_i$
	which generate these generators, and this gives us a $k$-algebra basis for $k[M]$. Call this
	new generating set $U$. 

	Let $M'$ be the abelian group generated by $U$ and notice that $k[M']$ is a subalgebra of
	$k[M]$ containing $U$ so $k[M']=k[M]$ and therefore $M'=M$. This establishes that $M$ is 
	a finitely generated abelian group, so we can split up the algebra into a tensor product
	of $k[\Z]$ and $k[\Z/n\Z]$.

	When $M=\Z$, $k[M]\cong k[X,X^{-1}]$, so $G= \Gm$\footnote{Notice that here Waterhouse also
	uses that the basis elements $e_i$ are group-like and so $\Delta(e_i)=e_i\otimes e_i$. I am not
	quite sure why this is necessary.} Similarly if $M=\Z/n$ then $G\cong \mun{n}$
\end{prf}

Looking at my progress over the last several days, I am thinking that I am getting
too much in the weeds writing out all the lemmas and proofs. Perhaps it is better to read more quickly
and only take note of theorems as I need/use them.

\subsection{Cartier Duals}
The gist here is that if $G$ is represented by $A$, then we can define the dual 
$A^D=\Hom(A,k)$ and then $A^D$ represents a new scheme $G^D$ called the \textbf{Cartier Dual.}

We do some work to show that, in fact, elements of $G^D$ are in correspondence with the 
group-like elements of $A$, or equivalently the character group of $G$, $X_G$.

It is easy enough to evaluate $G^D(k)\cong\Hom(A^D,k)$, but luckily one shows that
``dualizing'' commutes with tensor products (and thus with extension of scalars) and $\Hom$. Thus
if $G_R$ is the scheme represented by $A\otimes_k R$, then using that $\Hom_k(A^D,R)\cong\Hom_R(A^D\otimes R,R)$,
\[G^D(R)=(G^D)_R(R)\]
which is represented by $(A\otimes R)^D=A^D\otimes R$, so finally 
\[G^D(R)=(G_R)^D(R)=\{\text{group-like elements in }A\otimes_k R\}\cong \Hom(G_R,(\Gm)_R)\]

\section{Algebraic Geometry}

\section{Problems}
\subsection{Week 1}

\subsection{Waterhouse}
\begin{wprob}{2.1}
	\begin{enumerate}
		\item Show that there are no nontrivial homomorphisms from $\Gm$ to $\Ga$.
		\item If $k$ is reduced, show that there are no nontrivial homomorphisms from $\Ga$ to $\Gm$.
		\item For each nonzero $b\in k$ with $b^2=0$, find a nontrivial homomorphism $\Ga\to\Gm$.
	\end{enumerate}
\end{wprob}
Hmm, there must be a small bug in my macro but I can't find it. I get an error about \texttt{missing `\textbackslash item'}
in the definition below. It seems to only occur when I use \texttt{\textbackslash subsubsection*} and the like.

\begin{sol}
	\subsubsection*{(a)}
	Let $\Phi:\Gm\to\Ga$ and let $\varphi:k[X]\to k[X,X^{-1}]$ be the corresponding Hopf
	algebra map. But then we have that 
	\[\varphi\otimes\varphi \circ \Delta_a(X)=\Delta_m\circ\varphi(X)\]
	and so using that $\Delta_m(a)=a\otimes a$ and $\Delta_a(b)=b\otimes1+1\otimes b$, we get
	\[\varphi(X)\otimes\varphi(1)+\varphi(1)\otimes\varphi(X)=\varphi(X)\otimes\varphi(X)\]
	
\end{sol}

Consider the representations of A over a field $k$ (or more generally $A$ modules. W can make this a tensor catecory)
Recall that in this category we have enough projectives and (with finitely generated hopf algebras) we get that all projectives are injective.

Then we end up defining $\Ext_A^*(k,k)=H^*_A(A,k)$, the cohomology ring of $A$ over $k$.

The conjecture here is that when $A$ is finite dimensional that $H^*(A,k)$ is finitely generated as a $k$-algebra.

This is known in some special cases:
\begin{itemize}
	\item Finite groups
	\item Finite group schemes (in positive characteristic) due to Friedlander and Suslin.
	\item In characteristic zero we have
	\begin{itemize}
		\item Quantum groups
		\item Hopf algebras that come from Nichols algebras
	\end{itemize}
\end{itemize}

Papers to look at: Ginzburg and Kumar '94 Something like ``Cohomology of small quantum groups''. There is also a '10 paper on Julia's website with Sarah Witherspoon and two others on Nichols algebras.

Sarah Witherspoon has a book she is writing on Hochschild cohomology that could be very good. It's on her website (Texas A\& M). Check out chapter 9 and appendix A.

\end{document}
