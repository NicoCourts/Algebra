\documentclass[12pt]{article}

%%%%%%%%%%%%%%%%%%%%%%%%%%%%%%%%%%%%%%%%%%%%%%%%%%%%%%%%%%%%%%%%%%%%%%%%%%%%%%%%%
%%% Package Includes
\usepackage{setspace}
\usepackage{amsmath, amsfonts, amssymb, graphicx, color, fancyhdr, scalerel, mathrsfs, tikz-cd, mdframed, enumitem, framed, adjustbox, bm, upgreek, xcolor, hyperref}
\usepackage[style=alphabetic]{biblatex}
\usepackage[framed,thmmarks]{ntheorem}
\usepackage[mathscr]{euscript}

%%%%%%%%%%%%%%%%%%%%%%%%%%%%%%%%%%%%%%%%%%%%%%%%%%%%%%%%%%%%%%%%%%%%%%%%%%%%%%%%%
%%%% Theorems, definitions, problems, solutions, etc
%Problems will be created using their own counter and style
\theoreminframepreskip{0pt}
\theoreminframepostskip{0pt}
\newframedtheorem{prob}{Problem}[part]
\renewcommand\theprob{\arabic{part}.\arabic{prob}}

%solution
\theoremstyle{nonumberbreak}
\theoremindent0.5cm
\theorembodyfont{\upshape}
\theoremseparator{:}
\theoremsymbol{\ensuremath\spadesuit}
\newtheorem{sol}{Solution}

%Theorems, Lemmas, Corollaries, and Conjectures
\theoremstyle{changebreak}
\theoremseparator{}
\theoremsymbol{}
\theoremindent0.5cm
\theoremheaderfont{\color{blue}\bfseries} 
\newtheorem{thm}{Theorem}[subsection]
\theoremheaderfont{\bfseries}
\newtheorem{lem}[thm]{Lemma}
\newtheorem{cor}[thm]{Corollary}
\newtheorem{conj}[thm]{Conjecture}

%Create a new env that references a theorem and creates a 'primed' version
%Note this can be used recursively to get double, triple, etc primes
\newenvironment{thm-prime}[1]
  {\renewcommand{\thethm}{\ref{#1}$'$}%
   \addtocounter{thm}{-1}%
   \begin{thm}}
  {\end{thm}}

%Man, that's really good! Let's use the same thing for definitions.
\newenvironment{def-prime}[1]
  {\renewcommand{\thethm}{\ref{#1}$'$}%
   \addtocounter{thm}{-1}%
   \begin{def}}
  {\end{def}}

%Shade definitions
\theoremindent0cm
\theoremheaderfont{\normalfont\bfseries} 
\def\theoremframecommand{\colorbox[rgb]{.7,.9,1}}
\newshadedtheorem{defn}[thm]{Definition}

%proofs
\theoremstyle{nonumberplain}
\theoremindent0.5cm
\theoremheaderfont{\sc}
\theoremseparator{:}
\theoremsymbol{\ensuremath\spadesuit}
\newtheorem{prf}{Proof}

%remarks
\theoremstyle{change}
\theoremindent0.5cm
\theoremheaderfont{\sc}
\theoremseparator{:}
\theoremsymbol{}
\newtheorem{rmk}[thm]{Remark}

%questions
\theoremheaderfont{\sc}
\newframedtheorem{qst}{Question}[subsection]
%\newtheorem{qst}[thm]{Question}

%waterhouse problem
\newenvironment{wprob}[1]{\begin{prob}{\normalfont\bfseries (Waterhouse #1) }\itshape}{\end{prob}}

%%%%%%%%%%%%%%%%%%%%%%%%%%%%%%%%%%%%%%%%%%%%%%%%%%%%%%%%%%%%%%%%%%%%%%%%%%%%%%%%%
%%% MATH DEFINITIONS
%Blackboard Sets
\newcommand*{\bbZ}{\mathbb{Z}}
\newcommand*{\bbN}{\mathbb{N}}
\newcommand*{\bbR}{\mathbb{R}}
\newcommand*{\bbC}{\mathbb{C}}
\newcommand*{\bbF}{\mathbb{F}}
\newcommand*{\bbQ}{\mathbb{Q}}

%Homological Algebra homological 
\DeclareMathOperator{\im}{Im}
\DeclareMathOperator{\Hom}{Hom}
\DeclareMathOperator{\Ext}{Ext}
\DeclareMathOperator{\HH}{HH}
\DeclareMathOperator{\Gr}{Gr}

%Representations
\DeclareMathOperator{\Res}{Res_H}
\DeclareMathOperator{\Ind}{Ind_H^G}
\DeclareMathOperator{\Soc}{Soc}

%schemes
\newcommand*{\bbG}{\ensuremath\mathbb{G}}
\newcommand*{\Ga}{\ensuremath\mathbf{G_a}}
\newcommand*{\Gm}{\ensuremath\mathbf{G_m}}
\newcommand*{\GL}[1]{\ensuremath\mathbf{GL_#1}}
\newcommand*{\SL}[1]{\ensuremath\mathbf{SL_#1}}
\newcommand*{\mun}[1]{\ensuremath\mathbf{\bm\upmu_#1}}
\newcommand*{\alp}[1]{\ensuremath\mathbf{\bm\upalpha_#1}}

%Lie Algebras
\newcommand*{\g}{\mathfrak{g}}
\newcommand*{\n}{\mathfrak{n}}
\newcommand*{\h}{\mathfrak{h}}
\let\sl\relax
\newcommand*{\sl}{\mathfrak{sl}}
\newcommand*{\gl}{\mathfrak{gl}}
\DeclareMathOperator{\ad}{ad}
\DeclareMathOperator{\Int}{Int}
\DeclareMathOperator{\Der}{Der}

%categories
\newcommand*{\Mod}{\text{\textbf{Mod}}}
\newcommand*{\C}{\mathcal{C}}
\newcommand*{\T}{\mathcal{T}}
\let\O\relax
\newcommand*{\O}{\mathcal{O}}
\newcommand*{\Ass}{\text{\textbf{Ass}}}
\newcommand*{\Vect}{\text{\textbf{Vect}}}
\newcommand*{\Grp}{\text{\textbf{Grp}}}
\newcommand*{\Ab}{\text{\textbf{Ab}}}
\newcommand*{\Ring}{\text{\textbf{Ring}}}
\newcommand*{\Lie}{\text{\textbf{Lie}}}

%general commands
\DeclareMathOperator{\ch}{char}
\DeclareMathOperator{\Aut}{Aut}
\DeclareMathOperator{\rank}{rank}
\DeclareMathOperator{\id}{id}
\DeclareMathOperator{\coker}{coker}
\DeclareRobustCommand{\sqbinom}{\genfrac[]{0pt}{}}

%%%%%%%%%%%%%%%%%%%%%%%%%%%%%%%%%%%%%%%%%%%%%%%%%%%%%%%%%%%%%%%%%%%%%%%%%%%%%%%%%
%%% Hacks, tweaks, and general helpful stuff
%fix tilde
\let\tilde\relax
\newcommand*{\tilde}[1]{\widetilde{#1}}

% Enumerate will automatically use letters (e.g. part a,b,c,...)
\setenumerate[0]{label=(\alph*)}

%Why is chi set so low?
\DeclareRobustCommand{\Chi}{{\mathpalette\irchi\relax}}
\newcommand{\irchi}[2]{\raisebox{\depth}{$#1\chi$}} 

%Center each figure by default
\makeatletter
\g@addto@macro\@floatboxreset{\centering}
\makeatother

\setlength\fboxsep{15pt}

%Section break
\newcommand*{\brk}{
\rule{2in}{.1pt}
}

%Replacement for the old geometry package
\usepackage{fullpage}

%Put page breaks before each part
\let\oldpart\part%
\renewcommand{\part}{\clearpage\oldpart}%

%Get q-binomial coefficients
\newcommand{\qbinom}{\genfrac{[}{]}{0pt}{}}

%Set the bibliography file
\bibliography{sources}

%%%%%%%%%%%%%%%%%%%%%%%%%%%%%%%%%%%%%%%%%%%%%%%%%%%%%%%%%%%%%%%%%%%%%%%%%%%%%%%%%
%%% Header Stuff
\setlength{\headsep}{24pt}  % space between header and text
\pagestyle{fancy}     % set pagestyle for document
\lhead{Notes on Affine Group Schemes and Their Representations} % put text in header (left side)
\rhead{Nico Courts} % put text in header (right side)
\cfoot{\itshape p. \thepage}
\setlength{\headheight}{15pt}
\allowdisplaybreaks

%%%%%%%%%%%%%%%%%%%%%%%%%%%%%%%%%%%%%%%%%%%%%%%%%%%%%%%%%%%%%%%%%%%%%%%%%%%%%%%%%
%%%%%%%%%%%%%%%%%%%%%%%%%%%%%%%%%%%%%%%%%%%%%%%%%%%%%%%%%%%%%%%%%%%%%%%%%%%%%%%%%

\begin{document}
%make the title page
\title{Notes and Problems from My Research\vspace{-1ex}}
\author{Nico Courts}
\date{}
\maketitle

\part{Autumn 2018}
\section{Problems}
\begin{prob}
	Assume that $k$ is a field and let $K=k(t)$ (notice $K$ is a transcendental extension). Prove that $\Hom_k(K,k)\not\cong K$.
\end{prob}

\begin{sol}
	This is basically just a cardinality argument. I don't think it's particularly worth doing at this juncture.
\end{sol}

\begin{prob}
	Let $G$ be a finite group scheme (actually we need only assume that $G$ is a Frobenius algebra so that a module
	is injective if and only if it is projective). Prove that unless $M$ is projective, its projective dimension is 
	infinite. Conclude that $H^n(G,M)=0$ for $n>N$ implies that $M$ is projective.
\end{prob}
\begin{sol}
	Assume $M$ itself is not projective so that its minimal projective resolution is nontrivial and furthermore
	that it is finite. That is, let $P_i$ be projective modules such that
	\[0\to P_n\xrightarrow{f_n} P_{n-1}\to\cdots \xrightarrow{f_{1}}P_0\xrightarrow{f_0} M\to 0\]
	is a minimal length projective resolution of $M$ (notice here that $n\ge 1$). 
	
	Next consider the short exact sequence
	\[0\to P_n\xrightarrow{f_n}P_{n-1}\to \coker f_n\to 0\]
	since $P_n$ is projective (and thus injective!) this sequence splits and therefore $P_{n-1}\cong P_n\oplus\coker f_n.$
	But then consider the sequence
	\[0\to P_n\xrightarrow{g} P_{n-2}\to\cdots\xrightarrow{f_0}M\to 0\]
	where above we are using $P_{n-1}\supseteq P_n\cong f_n(P_n)$  and that
	$g=f_{n-1}|_{f_n(P_n)}$. This map is injective since $\ker f_{n-1}=\coker f_n$, which is disjoint
	from $f_n(P_n)\cong P_n$. Exactness everywhere else is evident since the maps are not effectively changed.

	But then the existence of this sequence contradicts the minimality of the original sequence, so
	no finite sequence can exist.

	The last statement (as discussed with Julia) is actually false. 
\end{sol}

\begin{prob}
	Establish the five-term exact sequence for spectral sequences.
\end{prob}
\begin{sol}
	I plan to return to this problem in the future. I have other priorities at the moment,
	but I will eventually return to cohomology and spectral sequences and this will be a good
	exercise at that point.
\end{sol}

\begin{wprob}{1.1}
	\begin{enumerate}
		\item If $R$ and $S$ are two $k$ algebras and $F$ is a representable functor, show $F(R\times S)\cong F(R)\times F(S)$.
		\item Show there is no representable functor $R$ such that every $F(R)$ has exactly two elements.
		\item Let $F$ be the functor represented by $k\times k$. Show that $F(R)$ has two elements exactly when $R$ has no idempotents besides 0 and 1.
	\end{enumerate}
\end{wprob}

\begin{sol}
	\subsubsection*{(a)}
	Let $A$ be the $k$-algebra representing $F$. Thus $F(R)$ is naturally isomorphic to
	$\Hom_k(A,R)$ and $F(S)\simeq\Hom(A,S)$. Then define the map $\Phi:\Hom(A,R\times S)\to\Hom(A,R)\times\Hom(A,S)$
	via
	\[\Phi(\varphi)=(\pi_R\circ\varphi,\pi_S\circ\varphi)\]
	where $\pi_X$ is the canonical projection onto $X$.

	This is surjective since (by the universal property of products) any pair of maps
	$\varphi_R:A\to R$ and $\varphi_S:A\to S$ factors through the product $R\times S$
	and furthermore it does so \textit{uniquely}, giving us injectivity. Thus this map
	(which is clearly a homomorphism since $\pi_X$ is) is a bijection.
	\subsubsection*{(b)}
	By the last problem this is impossible since if $|F(k)|=2$ then 
	\[|F(k\times k)|=|F(k)\times F(k)|=4.\]
	\subsubsection*{(c)}
	Let $F$ be such a functor. Consider any $\varphi\in\Hom(k\times k, R)\simeq F(R)$.
	Assume first that $F(R)\cong \bbZ/2$ and let $r$ be an idempotent in $R$. 
\end{sol}

\begin{wprob}{1.2}
	Let $E$ be a functor represented by $A$ and let $F$ be any functor. Show that the natural
	maps $\eta:E\to F$ correspond to elements in $F(A)$.
\end{wprob}
\begin{sol}
	Consider the map $\Phi$ from natural maps $E\to F$ to elements in $F(A)$ defined by
	(again leveraging the representability of $E$)
	\[\eta\mapsto \eta(\operatorname{id}_A)\in F(A).\]
	Conversely, consider the map $\Psi$ from $F(A)$ to the natural maps $E\to F$ via
	\[x\mapsto \xi_x\]
	where $\xi_x$ where for any $Y$ and $y\in E(Y)\cong \Hom(A,Y)$ we define the $Y^{th}$ component of $\xi_x$
	as 
	\[\xi_x(y)=F(y)(x)\in F(Y)\]
	where (for clarity while I get a grasp here) $F(y):F(A)\to F(Y)$.

	Since we are only looking for a bijection, we only need that these maps are inverses.
	Consider that for all $Y$ and $y\in E(Y)$,
	\begin{align*}
		\Psi\circ\Phi(\eta)(y)&=\Psi\left(\eta(\operatorname{id}_A)\right)(y)\\
		&= \xi_{\eta(\operatorname{id}_A)}(y)\\
		&= F(y)\circ \eta (\operatorname{id}_A)\\
		&= \eta\circ E(y) (\operatorname{id}_A)\\
		&= \eta(y\circ\operatorname{id}_A)=\eta(y)
	\end{align*}
	where above we used the naturality of $\eta$ along with the fact that $E(y)$ is just
	precomposition with $y$. Thus $\Psi\circ\Phi(\eta)=\eta.$

	But then for any $x\in F(A)$,
	\begin{align*}
		\Phi\circ\Psi(x)&=\Phi\circ\xi_x\\
		&=\xi_x(\operatorname{id}_A)\\
		&=F(\operatorname{id}_A)(x)\\
		&=\operatorname{id}_{F(A)}(x)=x
	\end{align*}
	completing the proof.
\end{sol}

\begin{wprob}{1.3}
	Let $E$ be a functor represented by $A$, and let $F$ be any functor. Let $\Psi:F\to E$
	be a natural map with surjective component maps. Show there is a natural map $\Phi:E\to F$
	with $\Psi\circ\Phi=\operatorname{id}_E$.
\end{wprob}
\begin{sol}
	Since in particular $\Psi_A$ is surjective, there is an $x\in F(A)$ such that $\Psi(x)=\operatorname{id}_A$.
	Then using the map from the last problem, let $\Phi=\xi_x$. Then we can compute for any $R$ and $g\in E(R)$
	\begin{align*}
		\Psi\circ\Phi (g) &= \Psi\circ F(g) (x)\\
		&=E(g)\circ \Psi(x)\\
		&=E(g)(\operatorname{id}_A)\\
		&=g\circ \operatorname{id}_A=g
	\end{align*}
	since $g:A\to R$, so $E(g):E(A)\to E(R)$, which is just composition with $g$.
\end{sol}

\begin{wprob}{1.5}
	Write out $\Delta,\varepsilon,$ and $S$ for the Hopf algebras representing $\SL{2}, \mun{n},$ and $\alp{p}$.
\end{wprob}
\begin{sol}
	\subsubsection*{$\SL{2}$:}
	Notice $SL_2$ is represented by $A=k[X_1,X_2,X_3,X_4]/(X_1X_4-X_3X_2-1)$ so take two elements
	$f,g\in\Hom(A,R)$ where $f(X_i)=a_i\in R$ and $g(X_i)=b_i\in R$ and notice that we want
	\[(f,g)\Delta=h\]
	where since
	\[\begin{pmatrix}
		a_1 & a_2\\ a_3 & a_4
	\end{pmatrix}\begin{pmatrix}
		b_1 & b_2\\ b_3 & b_4
	\end{pmatrix}=\begin{pmatrix}
		a_1b_1+a_2b_3 & a_1b_2+a_2b_4\\ a_3b_1+a_4b_3 & a_3b_2+a_4b_4
	\end{pmatrix}=\begin{pmatrix}
		c_1 & c_2\\ c_3 & c_4
	\end{pmatrix}\]
	we want to have that $h(X_i)=c_i.$

	So then if $\Delta:A\to A\otimes A$ is defined as follows:
	\begin{align*}
		X_1&\mapsto X_1\otimes X_1+X_2\otimes X_3\\
		X_2&\mapsto X_1\otimes X_2 +X_2\otimes X_4\\
		X_3&\mapsto X_3\otimes X_1+X_4\otimes X_3\\
		X_4&\mapsto X_3\otimes X_2+X_4\otimes X_4
	\end{align*}
	Where one can compute
	\begin{align*}
		\id\otimes\Delta\circ \Delta (X_1)&= (\id\otimes\Delta)(X_1\otimes X_1+X_2\otimes X_3)\\
		&= X_1\otimes X_1\otimes X_1 + X_1\otimes X_2\otimes X_3 + X_2\otimes X_3\otimes X_1 + X_2\otimes X_4\otimes X_3\\
		&= (\Delta\otimes\id)(X_1\otimes X_1+X_2\otimes X_3)=\Delta\otimes\id\circ \Delta (X_1)
	\end{align*}
	and similar equality holds for the other $X_i$, so this is $\Delta$. 

	Using that we want $\varepsilon\otimes\id\circ \Delta(X_i)=1\otimes X_i$, we see that the map
	$\varepsilon$ sending $X_1$ and $X_4$ to 1 and $X_2$ and $X_3$ to zero is the map we want.

	Notice that as a sanity check we get that 
	\[\begin{pmatrix}
		\varepsilon(X_1) & \varepsilon(X_2)\\
		\varepsilon(X_3) & \varepsilon(X_4)
	\end{pmatrix}=\begin{pmatrix}
		1 & 0\\
		0 & 1
	\end{pmatrix}=I_2\]

	Finally using that $(S,\id)\circ\Delta(X_i)=\varepsilon(X_i)$ and the fact that in $A$,
	$\det=X_1X_4-X_3X_2=1$, we can define $S$ such that
	\[\begin{pmatrix}
		S(X_1) & S(X_2)\\
		S(X_3) & S(X_4)
	\end{pmatrix}=\begin{pmatrix}
		X_4 & -X_2\\
		-X_3 & X_1
	\end{pmatrix}=\begin{pmatrix}
		X_1 & X_2\\
		X_3 & X_4
	\end{pmatrix}^{-1}\]
	and one can verify that this satisfies the relation above.
	\subsubsection*{$\mun{n}:$}
	For this scheme, $A=k[X]/(X^n-1)$ is the representing algebra. If $f,g\in\Hom(A,k)$ with
	$f(X)=r$ and $g(X)=s$, then we want $(f,g)\Delta(X)=\sum f(X_{(1)})g(X_{(2)})=m(r,s)=rs$ where 
	$\Delta(X)=\sum X_{(1)}\otimes X_{(2)}$. An obvious choice is the diagonal map.

	Then choosing $\varepsilon(X)=1$ satisfies the diagrams (as before in $G_m$) and using our
	intuition, $S(X)=X^5$ which also works.
	\subsubsection*{$\alp{p}$:}
	This time we are working with $A=\bbZ/p[X]/(X^p)$. This time (since the group is additive) we want
	$\Delta(X)=X\otimes 1+1\otimes X$, which we can see works with associativity immediately.

	Following suit with the other additive group scheme $G$, setting $\varepsilon(X)=0$ and $S(X)=-X$
	we can quickly check these still satisfy the given axioms.
\end{sol}

\begin{wprob}{1.6}
	In $A=k[X_{11},\dots,X_{nn},1/\operatorname{det}]$ representing $GL_n$, show that $\Delta(X_{ij})=\sum X_{ik}\otimes X_{kj}$.
	What is $\varepsilon(X_{ij})$?
\end{wprob}
\begin{sol}
	Due to the uniqueness of $\Delta,\varepsilon,$ and $S$, we need only find maps satisfying 
	the diagrams. I claim that $\varepsilon(X_{ij})=\delta_{ij}$. In this case, notice
	\[(\varepsilon\otimes\operatorname{id})\circ\Delta(X_{ij})=\varepsilon\otimes\operatorname{id}\left( \sum X_{ik}\otimes X_{kj} \right)=\sum \delta_{ik}\otimes X_{kj}=1\otimes X_{ij}\]
	exactly as we want.

	For associativity, notice
	\[(\Delta\otimes \operatorname{id})\circ \Delta (X_{ij})=\Delta\otimes\operatorname{id}\left( \sum_k X_{ik}\otimes X_{kj} \right)=\sum_k\left(\sum_l X_{il}\otimes X_{lk}\right)\otimes X_{kj}\]
	and then the associativity of $\Delta$ follows simply from the associativity of the tensor product.

	For the last axiom, we compute $S$ such that $(S,\operatorname{id})\circ\Delta=\iota\circ\varepsilon$ where $\iota:K\to A$ is the map sending $k\mapsto k\cdot 1_A$.
	That is, we define $S:A\to A$ so that
	\[\sum_k S(X_{ik})X_{kj}=\delta_{ij}.\]

	We want to leverage the fact that for a fixed $i$ and $j$, the determinant is
	\begin{align*}
		\operatorname{det} &= \sum_{\sigma\in S_n}\operatorname{sgn}(\sigma)\prod_l X_{\sigma(l)l}\\
		&=\sum_\sigma \operatorname{sgn}(\sigma)X_{\sigma{j}j}\prod_{l\ne j} X_{\sigma(l)l}\\
		&=\sum_i X_{ij}\left(\sum_{\sigma(j)=i}\operatorname{sgn}(\sigma)\prod_{l\ne j}X_{\sigma(l)l}\right)
	\end{align*}
	and so we want that 
	\[S(X_{ik})=\frac{1}{\operatorname{det}}\sum_{\sigma(i)=k}\operatorname{sgn}(\sigma)\prod_{l\ne k}X_{\sigma(l)l}\]
	so that when $i=j$,
	\[\sum_k S(X_{ik})X_{kj}=\frac{1}{\operatorname{det}}\sum_k X_{kj}\sum_{\sigma(j)=k}\operatorname{sgn}(\sigma)\prod_{l\ne k}X_{\sigma(l)l}=1=\delta_{ij}\]
	whenever $i\ne j$, however, this equation is the determinant of the matrix where we have replaced
	the $j^{th}$ column with a copy of the $i^{th}$ column. This is linearly dependent, so 
	\[\frac{1}{\operatorname{det}}\sum_k S(X_{ik})S_{kj}=0=\delta_{ij}.\]
	Thus these are precisely the maps we desire.
\end{sol}

\begin{wprob}{1.10}
	Prove the following Hopf algebra facts by interpreting them as statements about group functors:
	\begin{enumerate}
		\item $S\circ S=\id$
		\item $\Delta\circ S=(\operatorname{twist})\circ(S\otimes S)\Delta$
		\item $\varepsilon\circ S=\varepsilon$
		\item The map $A\otimes A\to A\otimes A$ sending $a\otimes b$ to $(a\otimes 1)\Delta(b)$
		is an algebra isomorphism.
	\end{enumerate}
\end{wprob}
\begin{sol}
	\subsubsection*{(a)}
	Dualizing, we get
	\begin{figure}[h]
		\begin{tikzcd}[column sep=small]
			A\ar[rr,equal]\ar[rd,"S",swap] & & A\\
			& A\ar[ru,"S",swap] &
		\end{tikzcd}
		\begin{tikzcd}[column sep=small]
			A\ar[rr,equal] & & A\ar[ld,"inv"]\\
			& A\ar[lu,"inv"] &
		\end{tikzcd}
	\end{figure}

	\noindent so this statement is equivalent to the group (scheme) fact that $(g^{-1})^{-1}=g$.
	\subsubsection*{(b)}
	Using a similar duality argument, this is equivalent to saying 
	\[\operatorname{inv}\circ m = m\circ(\operatorname{inv}\times\operatorname{inv})\circ(\operatorname{twist})\]
	but if we consider arbitrary elements $g,h\in G(R)$, this means
	\[(gh)^{-1}=m\circ(\operatorname{inv}\times\operatorname{inv})(h,g)=m(h^{-1},g^{-1})=h^{-1}g^{-1}\]
	which is clearly true.
	\subsubsection*{(c)}
	This one is equivalent to $(\operatorname{inv})\circ i=i$ where if $g\in G(R)$, 
	\[(\operatorname{inv})\circ i(g)=(\operatorname{inv})(e)=e=i(g)\]
	or in other words $e^{-1}=e.$
	\subsubsection*{(d)}
\end{sol}

%%%%%%%%%%%%%%%%%%%%%%%%%%%%%%%%%%%%%%%%%%%%%%%%%%%%%%%%%%%%%%%%%
%%%%%%%%%%%%%%%%%%%%%%%%%%%%%%%%%%%%%%%%%%%%%%%%%%%%%%%%%%%%%%%%%
%%%%%%%%%%%%%%%%%%%%%%%%%%%%%%%%%%%%%%%%%%%%%%%%%%%%%%%%%%%%%%%%%

\part{Winter 2019}
\section{Preparation for the Quarter}
This is my first official quarter as Julia's student! My plan for now is to continue working on Waterhouse as well
as learn about algebraic geometry (alongside my usual classes, of course).
To give a sense of direction, Julia recommended that I take a look at the following regularity theorem:
\begin{thm}[Smoothness Theorem]\label{smooth}
	Let $G$ be an algebraic affine group scheme over a field $k$. Then $k[G]\otimes \bar k$ is reduced if and
	only if $\dim G = \rank \Omega_{k[G]}$.
\end{thm}

\section{Overarching Ideas and Notes}
\subsection{Week 2}
Consider the representations of A over a field $k$ (or more generally $A$ modules, which we can make into a tensor category).
Recall that in this category we have enough projectives and (with finitely generated Hopf algebras) we get that all projectives are injective.

The object of study here:
\begin{defn}
	The \textbf{Hochschild cohomology} of a Hopf algebra $A$ over the field $k$ is 
	\[\Ext_A^*(k,k)=H^*_A(A,k)\].
\end{defn}

The conjecture here is
\begin{conj}
	when $A$ is a finite dimensional Hopf algebra, $H^*(A,k)$ (the Hochschild cohomology) 
	is finitely generated as a $k$-algebra.
\end{conj}
This is known in some special cases:
\begin{itemize}
	\item Finite groups
	\item Finite group schemes (in positive characteristic) due to Friedlander and Suslin.
	\item In characteristic zero we have
	\begin{itemize}
		\item Quantum groups
		\item Hopf algebras that come from Nichols algebras
	\end{itemize}
\end{itemize}

Papers to look at:
\begin{itemize}
	\item Ginzburg and Kumar '94 ``Cohomology of Quantum Groups at Roots of Unity'' \cite{ginzburg-kumar}
	\item Mastnak, Pevtsova, Shauenburg, and Witherspoon '09 ``Cohomology of Finite Dimensional Pointed Hopf Algebras''\cite{julia10-pointed}
	\item Witherspoon '17 ``Varieties for Modules of Finite Dimensional Hopf Algebras''\cite{witherspoon-expository}
\end{itemize}. 

Sarah Witherspoon has a book she is writing on Hochschild cohomology that could be very good. It's on her website (Texas A\& M). Check out chapter 9 and appendix A.

\subsection{Week 3}
We primarily discussed support varieties in the context of Sarah Witherspoon's \textit{Varieties for Modules of Finite Dimensional Hopf Algebras.}\cite{witherspoon-expository}

\subsubsection{Questions}
\begin{enumerate}
	\item On page 5, in the definition of the support variety of a Hopf module $M$ (see Definition~\ref{def-support-var}),
	the construction seems to depend rather heavily on the choice (and existence!) of a graded subalgebra $H\le \HH^*(A)$
	satisfying (fg1) and (fg2). But the notation seems to imply there is no dependence there. What's going on?
	\item On page 9, why is $\mathcal{Z}(kG)=\HH^0(kG)$ a local ring (for $G$ a finite group)?
	\item Directly after the last question, why would this ring being local imply that, on the variety level,
	we can use $H^{ev}(G,k)$ instead of $H^{ev}(G,k)\cdot \HH^0(kG)$?
\end{enumerate}

\subsubsection{Answers/Hints}
\begin{enumerate}
	\item Actually the definition of support variety, when properly generalized to triangulated tensor categories,
	doesn't rely on the choice of this algebra. But to see this, we need to dive deeper into that subject. Julia gave 
	me a book by Dave Benson, Srikanth Iyengar, and Henning Krause on local cohomology and support. It was written
	as a compendium from a course at Oberwolfach that they gave. It is available on the Arxiv.
	\item Julia suggests that this is a ``standard'' result that one might find in Webb's book (or another book where the author has taken care to be thorough).
	I couldn't locate it there immediately, but I found a page online that gave me the maximal ideal, so I am going to try to prove it as lemma~\ref{lem-local}.
	\item The take-away here was that when computing the varieties the nilpotent elements are irrelevant. Since 
	the only maximal ideal in $\HH^0(kG)$ is composed of nilpotents, it doesn't contribute to the support variety.
\end{enumerate}

\begin{defn}[Support Variety]\label{def-support-var}
	Let $A$ be a finite dimensional algebra and assume $_AA$ is injective (notice that finite dimensional Hopf algebras are
	Frobenius whence self-injective). Assume further that there is a graded subalgebra $H\le \HH^*(A)$ that satisfies the following 
	finiteness conditions:
	\begin{itemize}
		\item \textbf{(fg1):} $H$ is finitely generated, commutative, and $H^0:=H\cap \HH^0(A)=\HH^0(A)$;
		\item \textbf{(fg2):} for all finite dimensional $A$-modules $M$, $\Ext^*_A(M,M)$ is finitely generated as an $H$-module.
	\end{itemize}

	Then if $I_A(M)$ is the annihilator of $\Ext_A^*(M,M)$ in $H$, the \textbf{support variety} is
	\[V_A(M)=\operatorname{MaxSpec(H/I_A(M))}\]
\end{defn}
\begin{rmk}
	In taking with Julia about this definition she mentioned that the use of the max spectrum instead of the entire
	spectrum is a bit watered-down and, while simpler, perhaps hearkens back to the beginnings of AG where that is what one considered.
\end{rmk}

\begin{lem}\label{lem-local}
	If $G$ is a finite $p$-group and $k$ a field of characteristic $p$, $\mathcal{Z}(kG)$ is local.
\end{lem}
\begin{prf}
	I claim that the maximal ideal is
	\[I=\left\{\sum_{g\in G}a_gg:\sum_ga_g=0\right\}\]
	To see this, let $A=\mathcal{Z}(kG)$, and notice that $A=\ker\varphi$ where $\varphi(\sum a_g g)=\sum a_i$.
	Since $\varphi(kG)=k$, this gives us that $I$ is a maximal ideal in $A$.

	Notice that if we prove that $I$ consists entirely of nilpotents, we will be done. This is because
	the nilradical of $A$ is the intersection of its prime ideals. But if $I$ is contained in the nilradical, 
	it must, in particular, be contained in every maximal ideal of $A$. Thus it must be the only maximal ideal.

	To see this, we proceed by induction on $n$ where $|G|=p^n$. When $n=0$ this 
	is obvious and when $n=1$, this means $G=\bbZ/p\bbZ=\langle\alpha\rangle$. In this case each element is nilpotent of 
	degree $p$ due to the fact that (since in this case our ring is commutative):
	\[\left(\sum_1^p a_i\alpha^i\right)^p=\sum_1^p a_i^p\alpha^{pi}=\sum_1^p a_i1_G=0.\]

	Now assume that $I$ consists of nilpotents for all $n\le k$ and let $|G|=p^{k+1}$. Let $Z\le G$ be the
	(necessarily nontrivial) center of $G$. Extend the canonical quotient map $q:G\to G/Z$ to a map 
	between group rings $\tilde q:kG\to k(G/Z)$. Let $I_Z$ be the augmentation ideal of $k(Z)$. Then
	$\tilde q$ is surjective with kernel $I_Z\cdot kG$. That this ideal is contained within the kernel is clear
	since if $y$ is a $kG$-linear sum of elements of the form $\sum_Z a_zz$ where $\sum_Z a_z=0$, then
	the image is just the sum of zeros.

	To see that every element of the kernel is of this form, let $y=\sum k_i g_i$ and notice
	\[\tilde q\left(\sum k_ig_i\right)=\sum k_i\tilde q(g_i)=\sum_j\left(\sum_{g_i-g_j\in Z}k_i\right)g_j\]
	and this is zero precisely when all the sums are. But then the sum of the coefficients of $y$ restricted to any coset of $Z$
	are zero. 

	Thus in particular $kG/(I_Z)\cong k(G/Z)$.
	
	Then look at the augmentation ideal $I_{G/Z}\lhd k(G/Z)$ which, by the induction hypothesis, consists of nilpotents.
	Pull this back through $\tilde q$ to the ideal $I=(I_{G/Z})+ (I_Z)$. Since some power of $p$ kills all elements of the two 
	summands, this implies that some power of $p$ kills each element in $I$, completing the proof.
\end{prf}

\subsection{Week 4}
Recall that $\mathcal{U}:\Lie\to \Ass$ is a functor that assigns to each 
Lie algebra the universal enveloping algebra: the tensor algebra over $\g$ modulo the relations
$x\otimes y-y\otimes x-[x,y]$.

Then we have the PBW theorem:
\begin{thm}[Poincar\'e-Birkhoff-Witt]
	If $\g$ is a finite dimensional Lie algebra, with basis $x_1,\dots,x_n$, then $\mathcal{U}(\g)$ has a basis
	in the form $\prod x_i^{a_i}$ for $a_i\in\bbZ_{\ge 0}$.
\end{thm}
\begin{rmk}
	A fancier way of saying this is to say that $\mathcal{U}(\g)$ admits a filtration such 
	that $\operatorname{gr}\mathcal{U}(\g)\cong S^*(\g)$.
\end{rmk}

Furthermore, there is a more general class of algebras called \textbf{PBW algebras} where you get commutation
modulo lower-order terms. This idea is also what forms the basis of the proof for the above remark.

The idea here, then, is to take the center and see how it acts on the universal enveloping algebra.

\subsubsection{The BGG category $\mathcal{O}$}
The representations of finite dimensional Lie algebras are in correspondence with the dominant integral weights. This
will be worth looking into, perhaps.

The general question we want to answer is what he structure of finite dimensional representations of $\g$. In the case
when $\g=\sl_2$, we have the representation $V=ke_1\oplus ke_2$. 

Here I need to work out some of the notation, but $k[x,y]$ is a representation and $S^d(V)=k_d[x,y]$ is as well. (F\&H lecture 11 for more details there).

So we end up realizing that we can generate any such Lie algebra by taking a single highest-weight vector and 
applying $\mathcal{U}(\mathfrak{n}^-)$ to get the whole thing. This help motivate:
\begin{defn}[BGG Category $\mathcal{O}$]
	$\mathcal{O}$ is the full subcategory of $\mathcal{U}(\g)$ modules such that
	\begin{itemize}
		\item $M$ is finitely generated over $\mathcal{U}(\g)$ -- this is necessary since we want free modules
		and $\mathcal{U}(\g)$ itself to be representations under study and these are excluded from finite ($k$-) dimensional algebras.
		\item $M$ is $\mathfrak{h}$ semisimple.
		\item $M$ is $\n$-locally finite -- for all $v\in M$, $\mathcal{U}(\n)v$ is finite dimensional.
	\end{itemize}
\end{defn}

Ideas for things to understand/cover for next week:
\begin{itemize}
	\item First things first, understand the representations of $\sl(2,k)$.
	\item Then in Humphreys(I), read/do section \S 6.21 to understand the universal enveloping algebra.
	\item Read Humphreys(II) chapter 1 as much as possible.
\end{itemize}

\subsection{Week 5}
We spent most of the time deriving the fact that any simple $\sl_2$ module is isomorphic to one of the ``standard''
highest weight modules.

Our assignment for this upcoming week:
Write down the Serre relations. Then do some computations and read section 21 in Humphreys. Read those lemmas and theorems and start to describe decomposition of general modules of Lie algebras.

\subsection{Week 6}
We skipped Lie algebras this week because of snow (and then Ale couldn't make our meeting). So instead
Julia just met up to talk about my reading.

My primary question (regarding the Ginzburg and Kumar paper): why are we factoring out the toral elements in quantum groups? 
The answer here is that these are called the ``small quantum groups''. So certainly having a finite-dimensional algebra makes certain things
more tractable, although when computing cohomology that isn't always the case. Sometimes when you get smaller 
objects you end up getting more nontrivial elements in the cohomology. There's a clear analog to topology here.

So then maybe a refinement of the question is this: why is it necessary at all to have this larger object serving as the ambient 
space for the object we truly want to study? Here the answer more-or-less boils down to the fact that it end up playing a central
role in computations involving these objects.

Having the big algebra $U$ over $u$ (modulo something central) gives you that thePBW basis for $U$ descends to 
$u$. In general (e.g. Nichols algebras) you fail to have a PBW basis for the cover, but somehow (for a technical reason that 
will become apparent as we read more) you still can use similar techniques to do the computations (the more general theory arises in the 2010 paper).

Julia is going to sent me a paper that is a survey of finite dimensional algebras that will give some sense for how the small quantum groups fit in.

\subsubsection{This week's assignment}
Continue reading through GK until we get to the spectral sequence nonsense. Then this points to Jantzen (AG). Try to recreate the
argument for constructing the first page of the spectral sequence, but for the GZ case (note, you'll use projectives instead of injectives)
After that, go to Weibel and try to read the section of spectral sequences over filtered complexes \S 5.4 (what we are usually interested in).
Don't worry too much about issues of convergence -- we are mostly interested in the ``first quadrant'' case.

David Anick is the name of an algebraic topologist who started off created (free) resolutions that we will end up looking at. 
His paper \cite{anick-resolution} will probably be useful at some point.

\subsection{Week 8}
The last paper on Julia's page is a preprint with her and Dave Benson. There is a spectral sequence computation that might be illustrative.
There is also an older paper...

One of the things Julia is interested in at the moment is a super group scheme. These arise by having a $\bbZ_2$ grading on your Hopf algebra.
For instance we can look at exterior algebra $\Lambda(x)$, which represents the super group scheme $G_a^-$. 

The first non-elementary example is $G_a^-\times G_a^-$ with a $\bbZ_p$ action on it that ``folds'' the second factor
onto the first. To write this out explicitly, we consider the group algebra $\Lambda(u,v)$ where $uv=-vu$ and the coordinate algebra of $\bbZ_p$
is $k[t]/t^p$ where $t\cdot u=0$ and $t\cdot v=\alpha u$ for some $\alpha$. This is a $4p$ dimensional algebra.

Then we want to compute coholomology. We know $H^{**}(k\bbZ/p,k)\cong k[x]\oplus\Lambda(y)$ for $|x|=2,0$.
Note there is a notion of dimension that is an ordered pair $(a,b)$ of the graded degree and cohomology degree.
THis give us new elements of degree $(1,1)$

Look up Koszul duality! It is in the representations and local cohomology book! 

\subsection{Week 10}


%%%%%%%%%%%%%%%%%%%%%%%%%%%%%%%%%%%%%%%%%%%%%%%%%%%%%%%%%%%%%%%%%
%%%%%%%%%%%%%%%%%%%%%%%%%%%%%%%%%%%%%%%%%%%%%%%%%%%%%%%%%%%%%%%%%
%%%%%%%%%%%%%%%%%%%%%%%%%%%%%%%%%%%%%%%%%%%%%%%%%%%%%%%%%%%%%%%%%

\newpage
\section{Lie Algebras}
We're planning on digging into the category $\O$, the category of (reasonably finite) $\mathcal{U}(\g)$ modules.
To begin, however, we are going to make sure we have our fundamentals vis a vis (semi) simple representations of Lie algebras.

\subsection{$\sl_2$}
In the case when $L=\sl_2$, everything is relatively nice. We read (and discussed) the fact that any simple, finite-dimensional
$\sl_2$-module is isomorphic to $V(\lambda)$ for some integer $\lambda$ (highest weight).

The action of $L$ on these modules are precisely as one would hope: $x$ and $y$ (the off-diagonal generators) act by raising 
and lowering the weight of a vector by 2 and the central generator fixes each eigenspace.

\subsection{The Serre Relations}
Let $L$ be a semisimple Lie algebra and $\h$ a Cartan subalgebra with corresponding roots $\Phi$ with simple root system $\Delta$.
Recall that 
\[\langle\alpha_i,\alpha_j\rangle=\frac{2(\alpha_i,\alpha_j)}{(\alpha_j,\alpha_j)}=\alpha_i(h_{\alpha_j})\]
Fix generators $x_i$ and $y_i$ of $L_{\alpha_i}$ and $L_{-\alpha_i}$, respectively, so that $[x_i,y_i]=h_i=h_{\alpha_i}$
Recall that the $x_i$ and $y_i$ generate $L$ (as a Lie algebra).

Then in particular we have the relations:
\begin{align*}
	[h_i,h_j]&=0\tag{S1}\\
	[x_i,y_j]&=\delta_{ij}h_i\tag{S2}\\
	[h_i,x_j]=\langle\alpha_j,\alpha_i\rangle x_j\text{ and }& [h_i,y_j]=-\langle\alpha_j,\alpha_i\rangle y_j\tag{S3}
\end{align*}
\begin{defn}[Serre Relations]
	With the notation above, the relations S1-3 above are called the \textbf{Serre relations}.
\end{defn}	

These relations end up being crucial in making the connection between root systems and semisimple Lie algebras concrete:
taking the abstract free Lie algebra on generators satisfying these properties, we get a finite dimensional semisimple Lie algebra
with roots corresponding exactly to the $\alpha_i$. This also gets us uniqueness of such a root system.

\subsection{The more general case.}
This primarily borrows from \S 21 in \cite{humphreys1} but also relies on some results from earlier that I will cite as 
necessary. 

Throughout this discussion, we rely on the theory developed in sections 15 and 16 regarding Cartan and Borel subalgebras:
\begin{defn}[Cartan Subalgebra (CSA)]
	If $L$ is a Lie algebra, then $C\subseteq L$ is called a \textbf{Cartan subalgebra} if $C$ is
	\begin{enumerate}
		\item Nilpotent; and
		\item $N_L(C)=C$.
	\end{enumerate}
\end{defn}
\begin{defn}[Borel Subalgebra]
	If $L$ is a Lie algebra, $B\subseteq L$ is called a \textbf{Borel subalgebra} if $B$ is a
	maximal solvable subalgebra.
\end{defn}
\begin{rmk}
	Since every nilpotent Lie algebra is solvable, each CSA lies in some Borel subalgebra $B$ of $L$.
	then by showing that any two Borel subalgebras are conjugate under $\mathscr{E}(L)$ (the subalgebra of $\Int L$ generated
	by the strongly $\ad$-nilpotent elements of $L$), it suffices to 
	show that two CSAs are $\mathscr{E}(L)$-conjugate in any \textit{semisimple} Lie algebra.
\end{rmk}

In all that follows, we've fixed a CSA $\h$ of $L$ and root system $\Phi$ with simple root system $\Delta=\alpha_1,\dots,\alpha_l$
of positive roots with $\mathcal{W}$ the Weyl group.

\subsection{Representations}
Since $\h$ is semisimple (since $L$ is), it acts diagonally on any finite dimensional $L$-module. Thus we can always
do the same trick of decomposing $V$ into eigenspaces $V=\sqcup_{\lambda\in \h^*}V_\lambda$ that we do for $\sl_2$.
Slightly more generally (when $V$ is infinite-dimensional), we can consider the sum of weight spaces $V_\lambda$ of $V$, 
which is necessarily  direct\footnote{Think: vectors can't have multiple eigenvalues.}, so there always
exists a direct sum of simple modules contained in $V$. When $\dim V<\infty$, this is equal to $V$ itself.

\subsubsection{Standard Cyclic Modules}
Here


%%%%%%%%%%%%%%%%%%%%%%%%%%%%%%%%%%%%%%%%%%%%%%%%%%%%%%%%%%%%%%%%%
%%%%%%%%%%%%%%%%%%%%%%%%%%%%%%%%%%%%%%%%%%%%%%%%%%%%%%%%%%%%%%%%%
%%%%%%%%%%%%%%%%%%%%%%%%%%%%%%%%%%%%%%%%%%%%%%%%%%%%%%%%%%%%%%%%%

\newpage
\section{Quantum Groups}\label{sec-quantum}
I am mostly working out of Jantzen's \textit{Lectures on Quantum Groups} (\cite{janzten-quantum}) with the goal of reading through Ginzburg \& Kumar's
(apparently influential!) paper \textit{Cohomology of Quantum Groups at Roots of Unity} (\cite{ginzburg-kumar}).

\subsection{What they are}
The core idea to keep in mind is that quantum groups (or quantum enveloping algebras) are ``deformations'' (or perhaps $q$-analogues) of
regular enveloping algebras. Well, at least that is the kind we're mostly working with here. A general definition eludes me (and Jantzen).

Mostly we will be interested in $U=\mathcal{U}_q(\sl_2)$. In this setting, we get the definition
\begin{defn}[Quantum Enveloping Algebra of $\sl_2$]
	$\mathscr{U}_q(\sl_2)$ is the associative algebra $k\langle E,F,K,K^{-1}\rangle$ under the following relations:
	\begin{align*}
		KK^{-1}&=K^{-1}K=1\tag{R1}\\
		KEK^{-1} &= q^2 E\tag{R2}\\
		KFK^{-1} &= q^{-2}F\tag{R3}\\
		EF-FE &= \frac{K-K^{-1}}{q-q^{-1}}\tag{R4}
	\end{align*}
	where $q$ is some nonzero element of $k$ such that $q^2\ne 1$.
\end{defn}

Some notation that may be helpful in the future:
\begin{itemize}
	\item For $a\in\bbZ$, define $[a]=\frac{v^a-v^{-a}}{v-v^{-1}}\in\bbQ(v)$ -- the $v$ analogue of $a$. Equivalently, (and perhaps suggestively):
	\[[a]=v^{a-1}+v^{a-3}+\cdots+v^{-a+3}+v^{-a+1}.\]
	\item the $v$ analogue of the factorial is
	\[[a]^!=[a][a-1]\cdots[1]\]
	\item Then you get the binomial coefficients:
	\[\qbinom{a}{n}=\frac{[a]^!}{[n]^![a-n]^!}\]
	\item \[[K;a]=\frac{Kq^a-K^{-1}q^{-a}}{q-q^{-1}}\]
\end{itemize}

\subsection{Getting a basis}
The first part of the first chapter focuses on getting a PBW-style basis for $U$, which comes in the form $F^s,K^n,E^r$ for $r,s\in\bbN$, $n\in\bbZ$.
This is mostly standard but I thought this proof had a neat idea in it:

\begin{prf}
	Let $A=k[X,Y,Z]$ and define $f,e,h\in \Hom_k(A,A)$ such that 
	\begin{align*}
		f(Y^sZ^nX^r)&=Y^{s+1}Z^nX^r,\\
		e(Y^sZ^nX^r)&=q^{-2n}Y^sZ^nX^{r+1}+[s]Y^{s-1}[Z;1-s]Z^n X^r,\\
		h(Y^sZ^nX^r)&=q^{-2s}Y^sZ^{n+1}X^r
	\end{align*}
	One can check that these vector space homomorphisms exactly mirror multiplication of these monomials
	by $F, E,$ and $H$, respectively, so we get a map $\varphi:U\to\Hom_k(A,A)$. Then take any linear combination of 
	$F^sK^nE^r$, mapped though to $A$ via $\varphi$ and then evaluation at 1, gives us a linear combination of the $Y^sZ^nX^r$,
	so it must be trivial. Sick.
\end{prf}

\brk

After that, the book focuses on $U_0=k\langle Z,Z^{-1}\rangle$ as well as $U^+=k\langle E\rangle$ and $U^-=k\langle F\rangle$,
the latter of two which are isomorphic to $k[x]$ by the linear independence above.

\brk

Here's another cool bit: By studying products of monomials (or just defining degree outright) we can impose a graded structure on $U$:
let $\deg(E)=1$, $\deg(F)=-1$ and $\deg(K)=\deg(K^{-1})=0$ and notice (R1-4) are homogeneous according to this degree.
Thus the quotient by this homogeneous ideal gives a graded structure to $U$.

Furthermore if $u\in U$ is homogeneous of degree $i$, 
\[KuK^{-1}=q^{2i}u\]
so when the powers of $q$ are distinct ($|q|\ne 1$), the graded pieces are precisely the 
 of $K$ under the adjoint action.

\subsection{Representations of $U$}

%%%%%%%%%%%%%%%%%%%%%%%%%%%%%%%%%%%%%%%%%%%%%%%%%%%%%%%%%%%%%%%%%
%%%%%%%%%%%%%%%%%%%%%%%%%%%%%%%%%%%%%%%%%%%%%%%%%%%%%%%%%%%%%%%%%
%%%%%%%%%%%%%%%%%%%%%%%%%%%%%%%%%%%%%%%%%%%%%%%%%%%%%%%%%%%%%%%%%

\newpage
\section{Reading Ginzburg \& Kumar}
This section contains my notes and thoughts about \cite{ginzburg-kumar}. As a high-level summary, this paper establishes
methods for computing (Hochschild) cohomology of the small quantum groups. It begins by establishing how they are constructed
as quotients or subalgebras of quantized universal enveloping algebras (see section~\ref{sec-quantum}), and then proceeds to compute
the cohomology using a filtration (this is vital!) and spectral sequences.

This paper is important because it ends up that this general formula is the same as is applied later in \cite{julia10-pointed} to prove
a much more general result. This also extends a method originally created by \cite{FP} for computing the cohomology of Lie algebras and
algebraic groups.

\subsection{The Main Result}
In this paper they prove the following (see \ref{sec-defs} for notation):
\begin{thm}[GK '93]
	If $k=\bbQ(\xi)$ for a primitive root of unity of odd order (not 1) and $u_\xi$ the restricted enveloping algebra of some $\g$. Then
	\[H^{odd}(u_\xi,k)=0\]
	and there is a natural graded algebra isomorphism
	\[H^{ev}(u_\xi,k)\cong k^\bullet[\mathcal N].\]
\end{thm}

\subsection{The Objects in Question}
We begin by considering the objects in question and how they are constructed. One thing we haven't seen (yet!) in Lie algebras
but comes up in the structure theory is the 
\begin{defn}[Kostant $\bbZ$-form]\label{def-kostant}
	Let $U(\g)$ be the universal enveloping algebra of $\g$, spanned (over $k$) by the $x_\alpha,y\alpha,$ and $h_\alpha$ for all $\alpha\in\Phi$. 
	Then the \textbf{Kostant $\bbZ$-form} is the $\bbZ$-span of the elements
	\[X^{(r)}_\alpha=\frac{x_\alpha^r}{r!},\quad Y^{(r)}_\alpha=\frac{y_\alpha^r}{r!},\quad H^{(r)}_\alpha=\frac{h_\alpha^r}{r!},\quad \]
	the so-called \textbf{divided powers} of $x_\alpha, y_\alpha,$ and $h_\alpha$. This ($\bbZ$-)algebra is denoted $U_\bbZ$.
\end{defn}

Before we use it, we had better write down the definiton (via \cite{janzten-quantum}) of a
\begin{defn}[Quantized Enveloping Algebra]
	Fix some odd root of unity $q$ in the field\footnote{Here Jantzen also requires that $\ch k\ne2$ and that $\ch k\ne3$ if $\Phi$ has an irreducible component of type $G_2$. These can be treated, but not in quite the same manner.} $k$. 
	Let $\g$ be a Lie algebra over $k$ and let $U_q(\g)$ denote the $k$ algebra with generators $E_\alpha,F_\alpha, K_\alpha,$ and $K_\alpha^{-1}$
	for all $\alpha$ in a basis $\Pi$ of $\Phi$, subject to the relations for all $\alpha,\beta\in\Pi$
	\begin{align*}
		K_\alpha K_\alpha^{-1}=&1=K^{-1}_\alpha K_\alpha\\
		K_\alpha K_\beta &= K_\beta K_\alpha\\
		K_\alpha E_\beta K_\alpha^{-1} &= q^{(\alpha,\beta)}E_\beta\\
		K_\alpha F_\beta K_\alpha^{-1} &= q^{-(\alpha,\beta)}F_\beta\\
		E_\alpha F_\beta - F_\beta E_\alpha &= \delta_{\alpha\beta}\frac{K_\alpha-K_\alpha^{-1}}{q_\alpha-q_\alpha^{-1}}\\
		\sum_{s=0}^{1-a_{\alpha\beta}}(-1)^s\sqbinom{1-a_{\alpha\beta}}{s}_{q_\alpha} &E_\alpha^{1-a_{\alpha\beta}-s}E_\beta E_\alpha^s = 0\\
		\sum_{s=0}^{1-a_{\alpha\beta}}(-1)^s\sqbinom{1-a_{\alpha\beta}}{s}_{q_\alpha} &F_\alpha^{1-a_{\alpha\beta}-s}F_\beta F_\alpha^s = 0
	\end{align*}
	where $q_\alpha=q^{d_\alpha}=q^{(\alpha,\alpha)/2}$ and $a_{\alpha,\beta}=\langle\alpha,\beta\rangle=2(\alpha,\beta)/(\alpha,\alpha)$.
\end{defn}

\begin{rmk}
	In \cite{janzten-quantum}, he mentions that he will often work in the algebra that includes all the above relations
	but \textbf{omitting the last two}. Apparently this will be done for computational simplicity and then studying $U_q(\g)$ 
	itself will amount to taking a quotient. This may not be relevant to me, but it's good to keep in mind for this reference.
\end{rmk}

\brk

For the case of quantum groups, we are extending all of our notions to quantized enveloping algebras in the case when our field $k$ is the rational function field
$\bbQ(v)$.
\begin{defn}[$q$-analog of the Kostant $\bbZ$-form]
	Let $U=U_q(\g)$ be the (Hopf!) algebra described above. Then we define the $q$-analog of the Kostant $\bbZ$-form
	to be the $\bbZ[v,v^{-1}]$-span of the $q$-divided powers the $E_\alpha,F_\alpha$ and $K_\alpha^{\pm 1}$, which are the same as
	the divided powers in definiton~\ref{def-kostant} except the factorial is replaced by the $q$-analog thereof.
\end{defn}

\subsubsection{Notation and Related Algebras}\label{sec-defs}
But this isn't what we're studying yet! (This are the so-called ``big'' quantum groups: what Drinfeld studies in \cite{drinfeld}). 
We are interested in taking particular quotients and subalgebras. Here we define the notation used throughout and how they are related to $U=U_q(\g)$.

We will be interested primarily in the case when $v=\xi$, that is, when $k$ is the cyclotomic field over $\bbQ$ found by adjoining
an \textbf{odd} primitive $l^{th}$ root of unity ($l$ not 1). For the rest of this paper, fix such a $k$. Then define
\[U_\varepsilon :=(k\otimes_{\bbZ[v,v^{-1}]}U_\bbZ)/(K_\alpha^l-1)=:U_k/(K_\alpha^l-1)\]
where we defined $U_k$ by extension of scalars from $U_\bbZ$ and we are using here the fact that any $\bbZ[\xi]$-module 
becomes a $\bbZ[v,v^{-1}]$-module by letting $v$ act by $\xi$.

\brk

Another construction is the so called \textbf{restricted enveloping algebra} $u_\xi$ of $U_\xi$ generated by the 
degree one generators $F_\alpha, E_\alpha$, and $K^{\pm 1}_\alpha$. This is a finite-dimensional algebra of dimension $l^{\dim \g}$.
This is actually the object that we are studying here. It amounts to restricting coefficients in $U_\xi$ to polynomials in $\xi$ with integer coefficients.

\brk

Define $\mathcal N$ to be the \textbf{nilpotent cone} of $\g$, consisting of all $\ad$-nilpotent elements in $\g$. This admits a 
natural graded structure over the nonnegative integers by recognizing it as the ring of regular functions on a cone.

\subsubsection{Another formulation}

There is a parallel construction of $u_\xi$ that highlights a ``duality'' property of these objects. Set $\mathscr U_\bbQ$ to be the 
$\bbQ[v,v^{-1}]$-span of $E_\alpha,F_\alpha,K^{\pm 1}_\alpha$ (notice we are not using divided powers here, but allowing coefficients in $\bbQ$) 
and extend scalars again to $\mathscr U_k$. Again let $\mathscr U_\xi$ be the quotient
where we identify $K_\alpha^l=1$. Define $\mathscr L$ to be the subalgebra of $\mathscr U_\xi$ generated by $E_\alpha^l$ and $F_\alpha^l$.

Then it is a result of Lusztig that $\mathscr L$ and $u_\xi$ are normal (algebras) in $\mathscr U_\xi$ and $U_\xi$, respectively,
and that 
\[\mathscr U_\xi//\mathscr L\cong u_\xi\qquad\text{and}\qquad U_\xi//u_\xi\cong U(\g)\]
where
\begin{defn}
	Given an augmented algebra $B$ with augmentation ideal $B_+$, a subalgebra $A\subseteq B$ is called \textbf{normal} in $B$ if 
	\[B\cdot (A\cap B_+)=:B\cdot A_+=A_+\cdot B.\]
\end{defn}

(My) intuition for augmented algebras come from group algebras $kG$, where you have the augmentation map $\epsilon(\sum_G c_gg)=\sum_G c_g$.
Here the augmentation ideal is $\ker\varphi$. In fact, any map $\epsilon:A\to k$ you have induces such a structure. Usually in the case of Hopf algebras
we pick the counit (as we do in group algebras).

\begin{defn}
	If $A$ is normal in $B$, then we write 
	\[B//A:=B/(B\cdot A_+)=B/(A_+\cdot B).\]
\end{defn}

The duality we mentioned earlier comes from the fact that $\mathscr L$ is a commutative Hopf algebra that is ``dual'' to (the cocommutative) $U(\g)$.

\subsubsection{Triangular Decompositions}
We use superscripts $+,-,$ and $0$ to denote the subalgebras generated, respectively, by the $E$'s, $F$'s, and $K$'s\footnote{with the exception of $U_0$ which has another set of generators due to the divided powers}.

Then for $U_\xi$, $\mathscr U_\xi$, and $u_\xi$ we get the standard triangular decompositions; e.g.
\[u_\xi = u_\xi^+\oplus u_\xi^0\oplus u_\xi^-.\]

For each, define the \textbf{Borel part} to be (e.g.) $\mathscr B_\xi=\langle K_\alpha, E_\alpha\rangle$ (i.e. the nonnegative part).

\subsection{The Argument}
\subsubsection{Filtrations}
The core idea (according to Julia) is the existence of certain filtrations such that the associated graded algebras have nice structure.

In this paper, we are using an earlier result in \cite{de-concini} which tells us that $\mathscr B_\xi$ admits such a filtration:
(Recall that we are using roots $\alpha$ in some basis $\Pi$ of $\Phi$)
\begin{thm}[DK '90]
	The algebra $\mathscr B_\xi$ has a multiplicative filtration such that $\Gr \mathscr B_\xi$ is generated by the homogeneous elements $\{E_\alpha,K_\beta\}_{\alpha\in \Delta_+,\beta\in \Pi}$
	subject to the relations
	\[K_iK_j=K_jK_i\quad K_i^l=1\quad K_iE_\alpha=\xi^{\langle \alpha,\alpha_i\rangle} E_\alpha K_i\]
	and
	\[E_\alpha E_\beta=\xi^{\langle\alpha,\beta\rangle}E_\beta E_\alpha\]
	if $\alpha\succ\beta$.
\end{thm}
Then we can use this filtration to get filtrations on all quotients and subalgebras, most importantly
$\mathscr B_\xi,\mathscr U_\xi^+, b_\xi,$ and $u_\xi^+$.
\begin{qst}
	What is the filtration here? 
\end{qst}
I ended up finding the book in which \cite{de-concini} is published (in the library) and the answer is that you look at 
monomials 
\[M_{k,r;u}=F^kuE^r\]
where $F^k=\prod_\alpha F_\alpha^{k_{\alpha}}$ and similar for $E^r$ and $u\in\mathcal U^0$.

Then you can define the \textbf{height:}
\[\operatorname{ht}(M_{k,r;u})=\prod_\alpha (k_\alpha+r_\alpha)\operatorname{ht}\alpha\in\bbZ_+\]
(What is $\operatorname{ht}(\alpha)?$) and then finally define \textbf{degree:}
\[d(M_{k,r;u})=(k_N,k_{N-1},\dots,k_1,r_1,\dots,r_N,\operatorname{ht}(M_{k,r;u}))\in\bbZ_+^{2N+1}\]
which is totally ordered via the lexicographic ordering and therefore gives us a $\bbZ^{2N+1}$ grading.

\brk

Then we proceed to prove that $H^\bullet(\Gr\mathcal U_\xi^+)\cong\Lambda_\xi$, which follows from some arguments that can be found 
in \cite{priddy}.
\begin{qst}
	What role do augmented algebras play in this theory? I have noted on several occasions that augmentations
	have been used to define normal subalgebras. It also popped up in the discussion of Koszul algebras in \cite{priddy}.
\end{qst}

The idea here is to give $k$ the structure of an $A$-module. This is non-trivial for the folloiwng reason:
consider $A=k[x_i]$ and localize at a non-maximal prime $p$. Then the residue field will have positive transcendence degree
over $k$, and since $p$ will now be the (only!) maximal ideal, this will be the smallest field to which there is such a map.

\begin{qst}
	In the computation of cohomology for $\mathcal L^+$, why do the terms in the Koszul complex
	look like $\mathcal L^+\otimes_k \Lambda^i(\mathfrak N)$? Is this exploiting somehow that 
	all Hopf modules are trivial (e.g. $\mathcal L^+$) determines the entire action by 
\end{qst}

Let's take a regular sequence $(x_1,\dots,x_n)$. Then $R\xrightarrow{x_1} R$ is $K(x_1)$, the 
Koszul complex for this regular subsequence. If we have two elements, then $K(x_1)\otimes K(x_2)$
is the complex $R\to R\oplus R\to R$. Corollary 4.5.5 in Weibel says that whenever the sequence is regular, if you 
construct the entire Koszul complex to get a resolution of the quotient $R/(x_i)$.

Note that this essentially boils down to the computation that $\Lambda^iS^n\cong \Lambda^i(k^n\otimes_k S)=(\lambda^i k^n)\otimes S$.

%%%%%%%%%%%%%%%%%%%%%%%%%%%%%%%%%%%%%%%%%%%%%%%%%%%%%%%%%%%%%%%%%
%%%%%%%%%%%%%%%%%%%%%%%%%%%%%%%%%%%%%%%%%%%%%%%%%%%%%%%%%%%%%%%%%
%%%%%%%%%%%%%%%%%%%%%%%%%%%%%%%%%%%%%%%%%%%%%%%%%%%%%%%%%%%%%%%%%


\newpage
\section{Julia \& Dave Benson's preprint}
As an aside, in response to me saying I needed to practice doing computations with spectral sequences,
Julia recommended I read her current work \cite{julia-dave} with Dave Benson wherein there is a supposedly
straightforward spectral computation in support of their larger work \cite{detecting-nilpotence}.

I have already spent some time reading this paper, but the core objects in question are \textbf{supergroup schemes}
which have coordinate Hopf algebras algebras with a $\bbZ/2$ grading that are furthermore (graded) cocommutative (just 
as group schemes correspond to finite dimensional cocommutative Hopf algebras). The fundamental example is $\bbG_a^-$,
which has as its coordinate ring $k[x]/(x^2)$, or equivalently the exterior algebra on a single element $x$,
where $x$ is taken to be a primitive element.

Then if we let $A$ be the coordinate algebra of $\bbG_a^-\times\bbG_a^-$ and $H$ that of $\bbG_{a(r)}\times(\bbZ/p)^s$ where $\bbG_{a(r)}$ is the
$r^{th}$ Frobenius kernel of $\bbG_a$, we can define the \textbf{smash product} or in this case, since 
$A$ is graded cocommutative, the \textbf{semidirect product} $A\#H=A\rtimes H$ where we define the action
to be (if $\tau:H\otimes A\to A$ is the action of $H$ on $A$)
\[(a\otimes h)\cdot(a'\otimes h')=\sum (-1)^{|h_1||b|}a\tau(h_1\otimes b)\otimes h_2g.\]

So we use that $A\cong \Lambda(k[u,v])$, the exterior algebra on two (primitive) generators $u$ and $v$
and that, if $t_i$ is the generator for the coordinate algebra of the $i^{th}$ copy of $\bbZ/p$, and if 
we let $g_i=t_i-1$, these along with the $s_1,\dots s_r$ generating the Frobenius kernel, generate $H$.

One of the first propositions (3.1) of the paper computes the cohomology of $\bbG_a^-$ with 
coefficients in $k$ (actually, refers to another paper) and computes how the ($\bbZ/2$-graded)
Steenrod operations act on it. 

\begin{defn}[Steenrod Operations]
	If $p$ is odd and $A$ is a $\bbZ$-graded cocommutative Hopf algebra, then there are natural maps/operations
	\[\mathcal P^i:H^{s,t}(A,k)\to H^{s+(2i-t)(p-1),pt}(A,k)\]
	\[\beta \mathcal P^i:H^{s,t}(A,k)\to H^{s+1+(2i-t)(p-1),pt}(A,k)\]
	satisfying the properties
	\begin{enumerate}
		\item $\mathcal P^i=0$ whenever $2i<t$ or $2i>s+t$,
		\item $\beta\mathcal P^i=0$ whenever $2i<t$ or $2i\ge s+t$.
		\item $\mathcal P^i(x)=x^p$ when $2i=s+t$
		\item $\mathcal P^j(xy)=\sum_i\mathcal P^i(x)\mathcal P^{j-i}(y)$
		\item $\beta\mathcal P^j(xy)=\sum_i(\beta\mathcal P^i(x)\mathcal P^{j-i}(y)+\mathcal P^i(x)\beta\mathcal P^{j-i}(y))$
		\item $\mathcal P$ and $\beta\mathcal P$ satisfy the Adem relations.
		\item $\mathcal P^0(\lambda u)=\lambda^p\mathcal P^0(u)$ for all $\lambda \in k$ ($\ch k= p$)
	\end{enumerate}
\end{defn}

In the course of proving things here, the authors adapt these operations to the $\bbZ/2$ grading
by reindexing, calling these new operations $\mathscr P_i$ and $\beta\mathscr P^i$ (they use \texttt{mathscr} but I've
reassigned mine to a font I like better. :P)

Okay, time is running out, so I am pulling together some questions:
\begin{qst}
	Is the cohomology you're using here group cohomology? It seems to me the Lyndon-Hochschild-Serre
	applies to group cohomology.

	If so, I will need to focus some time on getting familiar with it.
\end{qst}

Yes it is group cohomology, or more specifically group \textit{scheme} cohomology. Here we 
are computing $\Ext_G^{i,\varepsilon}(k,k)$.

We begin with our group algebra (here $\Lambda(v_1,v_2)\#k\bbG_{a(1)}$) and take 
a projective resolutions and computing $H^*(\Hom_A(p^\bullet,k))$.

Recall the syzygys, $\Omega^i k$ that occur in a projective resolution of $k$. Then we can compute the cohomology
via the stable module category:
\[\underline{\Hom}_G(\Omega^ik,k)=\Ext_G^{i,\varepsilon}(k,k)\]

There is a result by Jeremy Rickard (rep theory) and by Buchweitz (more generally) that proves

\begin{qst}
	Where is $H^{*,*}(\bbG_a^-,k)$ computed? The papers claim it is $S^{*,*}(V^\sharp)\cong k[\zeta,\eta]$
	where $\zeta$ and $\eta$ are the dual generators corresponding to the generators $u$ and $v$ of $V$,
	the augmentation ideal of $k(\bbG_a^-\times \bbG_a^-)$.
\end{qst}

Again, this is an example of Koszul duality -- use that $k\bbG_a^-=\Lambda(v)$ where $v$ is primitive in degree 1.
But then $k(\bbG_a^-\times\bbG_a^-)\cong\Lambda(v_1,v_2)$. Then applying Koszul duality you get the 
symmetric algebra. Also you need the K\"unneth formula, where as algebras
\[H^*(A\otimes B,k)\cong H^*(A,k)\otimes H^*(A,k).\]

Evans has a book on The groups of cohomology. Here there are computations and resolutions of (cyclic)
groups. There is also a discussion of spectral sequences which is relatively terse (although it defers
somewhat to other sources).

\begin{qst}
	What is the importance of suspensions and what role do they play in this theory? They 
	have come up at several points and I am realizing they are fairly pervasive. 

	In particular, I understand the topological definition, but I don't see why they correspond
	to what we want. I am looking at the formula from \cite{detecting-nilpotence}:
	\[H^{1,1}(G,k)=\Hom_{sGr/k}(G^0,\mathbb{G}_a^-)^\pi\]
	where $\pi$ is the group of connected components of $G$.
\end{qst}
	
\begin{qst}
	In $G=(\bbG_a^-\times\bbG_a^-)\rtimes G_{a(1)}$, where the action is said to be ``nontrivial'',
	why does this seem to imply that the action on one of the copies of $\bbG_a^-$ is trivial?

	The fact that there is a LHS spectral sequence 
	\[H^{*,*}(\bbG_a^-\times \bbG_{a(1)},H^{*,*}(\bbG_a^-,k))\Rightarrow H^{*,*}(G,k)\]
	seems to imply that there is a short exact sequence 
	\[1\to \bbG_a^-\to G\to \bbG_a^-\times \bbG_{a(1)}\to 1\]
	but since the product is direct on the right, doesn't this imply that the action of $\bbG_{a(1)}$ on 
	$\bbG_a^-$ is trivial?
\end{qst}

Path 
%%%%%%%%%%%%%%%%%%%%%%%%%%%%%%%%%%%%%%%%%%%%%%%%%%%%%%%%%%%%%%%%%
%%%%%%%%%%%%%%%%%%%%%%%%%%%%%%%%%%%%%%%%%%%%%%%%%%%%%%%%%%%%%%%%%
%%%%%%%%%%%%%%%%%%%%%%%%%%%%%%%%%%%%%%%%%%%%%%%%%%%%%%%%%%%%%%%%%


\newpage
\section{Affine Group Schemes (Waterhouse and Jantzen)}
I have quite a bit of information to process before this, so I will get started!
\begin{defn}[Closed Embedding]
	If $G$ and $H$ are affine group schemes represented by $A$ and $B$, respectively and
	if $\psi:H\to G$ is a homomorphism of affine group schemes (locally a group homomorphism)
	then if the corresponding algebra map $A\to B$ is surjective, then $\psi$ is called
	a \textbf{closed embedding.}
\end{defn}

As its name suggests, this means that $\psi$ is an isomorphism onto a \textbf{closed subgroup}
$H'$ of $G$. This is, in fact, a definition of this property. One can also think about 
it in the following ways: a group scheme $H$ is closed in $G$ if
\begin{itemize}
	\item $H$ is defined by the relations imposed by $G$ plus some additional ones.
	\item $H=V(I)$ for some ideal $I\subset k[G]$. 
\end{itemize}
Thinking back to our algebraic geometry, these are not too hard to see as equivalent. 
For instance, there is a closed embedding of $\mun n$ in $\Gm$ (simply adding in $x^n-1$)
and of $\SL{n}$ in $\GL{n}$ (adding $\det = 1$).

\subsection{Hopf Ideals}
One problem with the above characterization that one cannot choose $I$ arbitrarily and end
up with a group scheme. This is equivalent to arbitrarily adding relations to a group, which is
not always guaranteed to work out well (think adding $\det = 2$ to $\GL n$).

Actually, we can exactly categorize the closed embeddings of subgroups in $G$ by considering certain
ideals of the algebra $A$ which represents it. 
\begin{defn}
	Let $A$ be an algebra and $I\lhd A$. Then if 
	\begin{itemize}
		\item $\Delta(I)$ goes to zero under the map $A\otimes A\to A/I\otimes A/I$,
		\item $S(I)\subseteq I$
		\item $\varepsilon(I)=0$
	\end{itemize}
	then $I$ is called a \textbf{Hopf Ideal of $A$.}
\end{defn}

\begin{rmk}
	That these ideals exactly characterize closed subgroups follows since any group (scheme) represented
	by $A'$ has the property that $\Delta'(A')\subseteq A'\otimes A'$. Since the new comultiplication map
	$\Delta'$ is derived from the old one $\Delta$, we have the diagram
	\begin{figure}[h]
		\begin{tikzcd}
			A \ar[r,"\Delta"]\ar[d] & A\otimes A\ar[d]\\
			A/I\ar[r,"\Delta'"] & A/I\otimes A/I
		\end{tikzcd}
	\end{figure}
	
	\noindent from which we see that $\Delta(I)$ must go to zero in the map on the right.
	Similarly, we want that $S$ and $\varepsilon$ satisfy similar diagrams.

	Note that an example of such an ideal is $I=\ker(\varepsilon)$.
\end{rmk}

\begin{defn}
	A \textbf{character} is a scheme morphism $G\to \Gm$.
\end{defn}

Let $\Phi$ be a character of $G$ and notice that the corresponding hopf algebra map 
$\varphi:k[X,1/X]\to A$ is defined by $\varphi(X)=b$, which
is automatically invertible in $A$. Furthermore since for any $a,b\in G(R)$ we have
\[\Phi\circ m(a,b)=\Phi(ab)=\Phi(a)\Phi(b)=m\circ(\Phi,\Phi)(a,b)\]
we get the diagram
\begin{figure}[h]
	\begin{tikzcd}
		G\times G\ar[r,"\Phi\times\Phi"]\ar[d,"m"] & \Gm\times \Gm\ar[d,"m"]\\
		G\ar[r,"\Phi"] & \Gm
	\end{tikzcd}
\end{figure}

\noindent and then by dualizing we get that
\newpage
\begin{figure}[h]
	\begin{tikzcd}
		A\otimes A & k[X,X^{-1}]\otimes k[X,X^{-1}]\ar[l,"\varphi\otimes\varphi",swap]\\
		A\ar[u,"\Delta"] & k[X,X^{-1}]\ar[u,swap,"\Delta"]\ar[l,"\varphi"]
	\end{tikzcd}
\end{figure}

\noindent where in both diagrams we are abusing notation by using the same symbols for 
(co)multiplication in the different schemes. So we conclude that 
\[\Delta(b)=\Delta\circ\varphi(X)=(\varphi\otimes\varphi)\circ \Delta(X)=(\varphi\otimes\varphi)(X\otimes X)=b\otimes b.\]

Then we can easily compute that $\varepsilon(b)=1$ and $S(b)=b^{-1}$. This is precisely:
\begin{defn}
	Let $A$ be a Hopf algebra and $a\in A$ such that
	\begin{itemize}
		\item $a$ is invertible
		\item $\Delta(a)=a\otimes a$
		\item $\varepsilon(a)=1$
		\item $S(a)=a^{-1}$
	\end{itemize}
	then $a$ is called a \textbf{group-like} element of $A$.
\end{defn}
\begin{rmk}
	As we saw above, every character of an affine group scheme $G$ corresponds to a 
	group-like element of its representing algebra.
\end{rmk}
\begin{rmk}
	Using a parallel construction with any morphism $\Phi:G\to\Ga$, we get an element $b\in A$
	such that $\Delta(b)=b\otimes 1+1\otimes b$, $\varepsilon(b)=0$ and $S(b)=-b$. These elements
	are called \textbf{primitive}
\end{rmk}

\begin{defn}
	A group scheme $G$ represented by $A$ that consists entirely of group-like elements is
	called \textbf{diagonalizable.}
\end{defn}
\begin{rmk}\label{rmk-diag}
	Notice that an alternative definition is to begin with an Abelian group $M$ and to 
	define $S,\Delta,$ and $\varepsilon$ such that each element is group-like. The resulting
	Hopf algebra represents a diagonalizable group scheme.
\end{rmk}

Consider the group algebra $k[M]$ on the group in Remark \ref{rmk-diag}. If this algebra
is finitely generated, we get the following:
\begin{thm}
	Let $G$ be diagonalizable and represented by $A$ and assume $A$ is finitely generated
	as a $k$-algebra. Then $G$ is a finite product of copies of $\Gm$ and $\mun n$.
\end{thm}
\begin{prf}
	Let $x_1,\dots, x_n$ be generators for $k[M]=A$. Each $x_i$ can be written as a (finite!)
	$k$-linear combination of elements in $M$, so we can instead use the finitely many $m_i$
	which generate these generators, and this gives us a $k$-algebra basis for $k[M]$. Call this
	new generating set $U$. 

	Let $M'$ be the abelian group generated by $U$ and notice that $k[M']$ is a subalgebra of
	$k[M]$ containing $U$ so $k[M']=k[M]$ and therefore $M'=M$. This establishes that $M$ is 
	a finitely generated abelian group, so we can split up the algebra into a tensor product
	of $k[\bbZ]$ and $k[\bbZ/n\bbZ]$.

	When $M=\bbZ$, $k[M]\cong k[X,X^{-1}]$, so $G= \Gm$\footnote{Notice that here Waterhouse also
	uses that the basis elements $e_i$ are group-like and so $\Delta(e_i)=e_i\otimes e_i$. I am not
	quite sure why this is necessary.} Similarly if $M=\bbZ/n$ then $G\cong \mun{n}$
\end{prf}

Looking at my progress over the last several days, I am thinking that I am getting
too much in the weeds writing out all the lemmas and proofs. Perhaps it is better to read more quickly
and only take note of theorems as I need/use them.

\subsection{Cartier Duals}
The gist here is that if $G$ is represented by $A$, then we can define the dual 
$A^D=\Hom(A,k)$ and then $A^D$ represents a new scheme $G^D$ called the \textbf{Cartier Dual.}

We do some work to show that, in fact, elements of $G^D$ are in correspondence with the 
group-like elements of $A$, or equivalently the character group of $G$, $X_G$.

It is easy enough to evaluate $G^D(k)\cong\Hom(A^D,k)$, but luckily one shows that
``dualizing'' commutes with tensor products (and thus with extension of scalars) and $\Hom$. Thus
if $G_R$ is the scheme represented by $A\otimes_k R$, then using that $\Hom_k(A^D,R)\cong\Hom_R(A^D\otimes R,R)$,
\[G^D(R)=(G^D)_R(R)\]
which is represented by $(A\otimes R)^D=A^D\otimes R$, so finally 
\[G^D(R)=(G_R)^D(R)=\{\text{group-like elements in }A\otimes_k R\}\cong \Hom(G_R,(\Gm)_R)\]


%%%%%%%%%%%%%%%%%%%%%%%%%%%%%%%%%%%%%%%%%%%%%%%%%%%%%%%%%%%%%%%%%
%%%%%%%%%%%%%%%%%%%%%%%%%%%%%%%%%%%%%%%%%%%%%%%%%%%%%%%%%%%%%%%%%
%%%%%%%%%%%%%%%%%%%%%%%%%%%%%%%%%%%%%%%%%%%%%%%%%%%%%%%%%%%%%%%%%

\newpage
\section{Algebraic Geometry}


\newpage
\section{Problems}

\subsection{Waterhouse}
\begin{wprob}{2.1}
	\begin{enumerate}
		\item Show that there are no nontrivial homomorphisms from $\Gm$ to $\Ga$.
		\item If $k$ is reduced, show that there are no nontrivial homomorphisms from $\Ga$ to $\Gm$.
		\item For each nonzero $b\in k$ with $b^2=0$, find a nontrivial homomorphism $\Ga\to\Gm$.
	\end{enumerate}
\end{wprob}
Hmm, there must be a small bug in my macro but I can't find it. I get an error about \texttt{missing `\textbackslash item'}
in the definition below. It seems to only occur when I use \texttt{\textbackslash subsubsection*} and the like.

\begin{sol}
	\subsubsection*{(a)}
	Let $\Phi:\Gm\to\Ga$ and let $\varphi:k[X]\to k[X,X^{-1}]$ be the corresponding Hopf
	algebra map. But then we have that 
	\[\varphi\otimes\varphi \circ \Delta_a(X)=\Delta_m\circ\varphi(X)\]
	and so using that $\Delta_m(a)=a\otimes a$ and $\Delta_a(b)=b\otimes1+1\otimes b$, we get
	\[\varphi(X)\otimes\varphi(1)+\varphi(1)\otimes\varphi(X)=\varphi(X)\otimes\varphi(X)\]
\end{sol}

\section{To be filed:}


%%%%%%%%%%%%%%%%%%%%%%%%%%%%%%%%%%%%%%
%%%%%%%%%%  Bibliography %%%%%%%%%%%%%
%%%%%%%%%%%%%%%%%%%%%%%%%%%%%%%%%%%%%%
\medskip

\printbibliography

\end{document}
