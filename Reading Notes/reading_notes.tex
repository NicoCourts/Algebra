\documentclass[12pt]{article}

\usepackage{setspace}

\usepackage{xcolor, amsmath, amsfonts, amssymb, graphicx, color, fancyhdr, lipsum, scalerel, stackengine, mathrsfs, tikz-cd, mdframed, enumitem, framed, adjustbox}
\usepackage[framed,thmmarks]{ntheorem}
\usepackage[mathscr]{euscript}

%set up theorem/definition/etc envs
%Problems will be created using their own counter and style
\theoreminframepreskip{0pt}
\theoreminframepostskip{0pt}
\newframedtheorem{prob}{Problem}[section]

%solution template
\theoremstyle{nonumberbreak}
\theoremindent0.5cm
\theorembodyfont{\upshape}
\theoremseparator{:}
\theoremsymbol{\ensuremath\spadesuit}
\newtheorem{sol}{Solution}

%Theorems, Lemmas, and Corollaries
\theoremstyle{changebreak}
\theoremseparator{}
\theoremsymbol{}
\theoremindent0.5cm
\theoremheaderfont{\bfseries} 
\newtheorem{thm}{Theorem}[subsection]
\newtheorem{lem}[thm]{Lemma}
\newtheorem{cor}[thm]{Corollary}

%Create a new env that references a theorem and creates a 'primed' version
%Note this can be used recursively to get double, triple, etc primes
\newenvironment{thm-prime}[1]
  {\renewcommand{\thethm}{\ref{#1}$'$}%
   \addtocounter{thm}{-1}%
   \begin{thm}}
  {\end{thm}}

\setlength\fboxsep{15pt}

%Shade definitions
\theoremindent0cm
\theoremheaderfont{\normalfont\bfseries} 
\def\theoremframecommand{\colorbox[rgb]{.9,.9,1}}
\newshadedtheorem{defn}[thm]{Definition}

%Man, that's really good! Let's use the same thing for definitons.
\newenvironment{def-prime}[1]
  {\renewcommand{\thethm}{\ref{#1}$'$}%
   \addtocounter{thm}{-1}%
   \begin{def}}
  {\end{def}}

%proofs
\theoremstyle{nonumberbreak}
\theoremindent1cm
\theoremheaderfont{\sc}
\theoremseparator{}
\theoremsymbol{\ensuremath\spadesuit}
\newtheorem{prf}{Proof}

\usepackage{fullpage}
%set margins
%\usepackage[
%top    = 1.0in,
%bottom = 1.0in,
%left   = 1.0in,
%right  = 1.0in]{geometry}

%header stuff
\setlength{\headsep}{24pt}  % space between header and text
\pagestyle{fancy}     % set pagestyle for document
\lhead{Notes on Affine Group Schemes and Their Representations} % put text in header (left side)
\rhead{Nico Courts} % put text in header (right side)
\cfoot{\itshape p. \thepage}
\setlength{\headheight}{15pt}
\allowdisplaybreaks

%Set of Integers
\newcommand*{\Z}{
\mathbb{Z}
}
%Set of Natural Numbers
\newcommand*{\N}{
\mathbb{N}
}
%Set of Real Numbers
\newcommand*{\R}{
\mathbb{R}
}
%Set of Complex Numbers
\newcommand*{\C}{
\mathbb{C}
}
%Field
\newcommand*{\F}{
\mathbb{F}
}
%Rationals
\newcommand*{\Q}{
\mathbb{Q}
}

%Section break
\newcommand*{\brk}{
\rule{2in}{.1pt}
}

\DeclareMathOperator{\Aut}{Aut}

%raise that Chi!
\DeclareRobustCommand{\Chi}{{\mathpalette\irchi\relax}}
\newcommand{\irchi}[2]{\raisebox{\depth}{$#1\chi$}} 

%Image
\DeclareMathOperator{\im}{Im}

%Hom
\DeclareMathOperator{\Hom}{Hom}

%Ext
\DeclareMathOperator{\Ext}{Ext}

%Coker
\DeclareMathOperator{\coker}{coker}

%characteristic
\DeclareMathOperator{\ch}{char}

%restriction
\DeclareMathOperator{\Res}{Res_H}

%socle
\DeclareMathOperator{\Soc}{Soc}

%rank
\DeclareMathOperator{\rank}{rank}

%induction
\DeclareMathOperator{\Ind}{Ind_H^G}

%fix tilde
\let\tilde\relax
\newcommand*{\tilde}[1]{\widetilde{#1}}

% Enumerate will automatically use letters (e.g. part a,b,c,...)
\setenumerate[0]{label=(\alph*)}

\begin{document}
%make the title page
\title{Notes and Problems from My Research\vspace{-1ex}}
\author{Nico Courts}
\date{}
\maketitle

%begin problems

\section{Autumn 2018}
\subsection{Problems}
\begin{prob}
	Assume that $k$ is a field and let $K=k(t)$ (notice $K$ is a transcendental extension). Prove that $\Hom_k(K,k)\not\cong K$.
\end{prob}

\begin{sol}
	This is basically just a cardinality argument. I don't think it's particularly worth doing at this juncture.
\end{sol}

\begin{prob}
	Let $G$ be a finite group scheme (actually we need only assume that $G$ is a Frobenius algebra so that a module
	is injective if and only if it is projective). Prove that unless $M$ is projective, its projective dimension is 
	infinite. Conclude that $H^n(G,M)=0$ for $n>N$ implies that $M$ is projective.
\end{prob}
\begin{sol}
	Assume $M$ itself is not projective so that its minimal projective resolution is nontrivial and furthermore
	that it is finite. That is, let $P_i$ be projective modules such that
	\[0\to P_n\xrightarrow{f_n} P_{n-1}\to\cdots \xrightarrow{f_{1}}P_0\xrightarrow{f_0} M\to 0\]
	is a minimal length projective resolution of $M$ (notice here that $n\ge 1$). 
	
	Next consider the short exact sequence
	\[0\to P_n\xrightarrow{f_n}P_{n-1}\to \coker f_n\to 0\]
	since $P_n$ is projective (and thus injective!) this sequence splits and therefore $P_{n-1}\cong P_n\oplus\coker f_n.$
	But then consider the sequence
	\[0\to P_n\xrightarrow{g} P_{n-2}\to\cdots\xrightarrow{f_0}M\to 0\]
	where above we are using $P_{n-1}\supseteq P_n\cong f_n(P_n)$  and that
	$g=f_{n-1}|_{f_n(P_n)}$. This map is injective since $\ker f_{n-1}=\coker f_n$, which is disjoint
	from $f_n(P_n)\cong P_n$. Exactness everywhere else is evident since the maps are not effectively changed.

	But then the existence of this sequence contradicts the minimality of the original sequence, so
	no finite sequence can exist.
\end{sol}

\begin{prob}
	Establish the five-term exact sequence for spectral sequences.
\end{prob}
\begin{sol}
	I plan to return to this problem in the future. I have other priorities at the moment,
	but I will eventually return to cohomology and spectral sequences and this will be a good
	exercise at that point.
\end{sol}

\section{Winter 2019}
This is my first official quarter as Julia's student!! 
\subsection{Preparation, Waterhouse, and G\"ortz \& Weddhorn}
To give a sense of direction, Julia recommended that I take a look at the following regularity theorem:
\begin{thm}[Smoothness Theorem]\label{smooth}
	Let $G$ be an algebraic affine group scheme over a field $k$. Then $k[G]\otimes \bar k$ is reduced if and
	only if $\dim G = \rank \Omega_{k[G]}$.
\end{thm}
I have quite a bit of information to process before this, so I will get started!
\begin{defn}[Closed Embedding]
	If $G$ and $H$ are affine group schemes represented by $A$ and $B$, respectively and
	if $\psi:H\to G$ is a homomorphism of affine group schemes (locally a group homomorphism)
	then if the corresponding algebra map $A\to B$ is surjective, then $\psi$ is called
	a \textbf{closed embedding.}
\end{defn}

As its name suggests, this means that $\psi$ is an isomorphism onto a \textbf{closed subgroup}
$H'$ of $G$. This is, in fact, a definition of this property. One can also think about 
it in the following ways: a group scheme $H$ is closed in $G$ if
\begin{itemize}
	\item $H$ is defined by the relations imposed by $G$ plus some additional ones.
	\item $H=V(I)$ for some ideal $I\subset k[G]$. 
\end{itemize}
Thinking back to our algebraic geometry, these are not too hard to see as equivalent. 
For instance, there is a closed embedding of $\mu_n$ in $G_m$ (simply adding in $x^n-1$)
and of $SL_n$ in $GL_n$ (adding $\det = 1$).

\subsubsection{Hopf Ideals}
One problem with the above characterization that one cannot choose $I$ arbitrarily and end
up with a group scheme. This is equivalent to arbitrarily adding relations to a group, which is
not always guaranteed to work out well (think adding $\det = 2$ to $GL_n$).

Actually, we can exactly categorize the closed embeddings of subgroups in $G$ by considering certain
ideals of the algebra $A$ which represents it. 
\begin{defn}
	Let $A$ be an algebra and $I\lhd A$. Then if 
	\begin{itemize}
		\item $\Delta(I)$ goes to zero under the map $A\otimes A\to A/I\otimes A/I$,
		\item $S(I)\subseteq I$
		\item $\epsilon(I)=0$
	\end{itemize}
	then $I$ is called a \textbf{Hopf Ideal of $A$.}
\end{defn}

\end{document}
