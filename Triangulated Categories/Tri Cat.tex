\documentclass[12pt]{article}

\usepackage{setspace}

\usepackage{amsmath, amsfonts, amssymb, graphicx, color, fancyhdr, lipsum, scalerel, stackengine, mathrsfs, tikz-cd, mdframed, enumitem, framed, adjustbox, bm, upgreek, x	color}
\usepackage[framed,thmmarks]{ntheorem}

%Replacement for the old geometry package
\usepackage{fullpage}

%Input my definitions
%set up theorem/definition/etc envs
%Problems will be created using their own counter and style
\theoremstyle{break}
\theoreminframepreskip{0pt}
\theoreminframepostskip{0pt}
\newframedtheorem{prob}{Problem}[section]

%solution template
\theoremstyle{nonumberbreak}
\theoremindent0.5cm
\theorembodyfont{\upshape}
\theoremseparator{:}
\theoremsymbol{\ensuremath\spadesuit}
\newtheorem{sol}{Solution}

%Theorems
\definecolor{thmcol}{RGB}{120,100,50}
\theoremstyle{changebreak}
\theoremseparator{}
\theoremsymbol{}
\theoremindent0.5cm
\theoremheaderfont{\color{thmcol}\bfseries} 
\newtheorem{thm}{Theorem}[subsection]

%Lemmas and Corollaries
\theoremheaderfont{\bfseries}
\newtheorem{lem}[thm]{Lemma}
\newtheorem{cor}[thm]{Corollary}
\newtheorem{prop}[thm]{Proposition}

%Create a new env that references a theorem and creates a 'primed' version
%Note this can be used recursively to get double, triple, etc primes
\newenvironment{thm-prime}[1]
  {\renewcommand{\thethm}{\ref{#1}$'$}%
   \addtocounter{thm}{-1}%
   \begin{thm}}
  {\end{thm}}

\setlength\fboxsep{15pt}

%Example
\theoremstyle{break}
\def\theoremframecommand{\colorbox[rgb]{0.9,0.9,0.9}}
\newshadedtheorem{ex}{Example}[section]

%Man, that's really good! Let's use the same thing for definitons.
\newenvironment{def-prime}[1]
  {\renewcommand{\thethm}{\ref{#1}$'$}%
   \addtocounter{thm}{-1}%
   \begin{def}}
  {\end{def}}

%proofs
\theoremstyle{nonumberbreak}
\theoremindent0.5cm
\theoremheaderfont{\sc}
\theoremseparator{}
\theoremsymbol{\ensuremath\spadesuit}
\newtheorem{prf}{Proof}

\theoremstyle{nonumberplain}
\theoremseparator{:}
\theoremsymbol{}
\newtheorem{conj}{Conjecture}

%remarks
\theoremstyle{change}
\theoremindent0.5cm
\theoremheaderfont{\sc}
\theoremseparator{:}
\theoremsymbol{}
\newtheorem{rmk}[thm]{Remark}

%Put page breaks before each part
\let\oldpart\part%
\renewcommand{\part}{\clearpage\oldpart}%

% Blackboard letters
\newcommand*{\bbA}{\mathbb{A}}
\newcommand*{\bbB}{\mathbb{B}}
\newcommand*{\bbC}{\mathbb{C}}
\newcommand*{\bbD}{\mathbb{D}}
\newcommand*{\bbE}{\mathbb{E}}
\newcommand*{\bbF}{\mathbb{F}}
\newcommand*{\bbG}{\mathbb{G}}
\newcommand*{\bbH}{\mathbb{H}}
\newcommand*{\bbI}{\mathbb{I}}
\newcommand*{\bbJ}{\mathbb{J}}
\newcommand*{\bbK}{\mathbb{K}}
\newcommand*{\bbL}{\mathbb{L}}
\newcommand*{\bbM}{\mathbb{M}}
\newcommand*{\bbN}{\mathbb{N}}
\newcommand*{\bbO}{\mathbb{O}}
\newcommand*{\bbP}{\mathbb{P}}
\newcommand*{\bbQ}{\mathbb{Q}}
\newcommand*{\bbR}{\mathbb{R}}
\newcommand*{\bbS}{\mathbb{S}}
\newcommand*{\bbT}{\mathbb{T}}
\newcommand*{\bbU}{\mathbb{U}}
\newcommand*{\bbV}{\mathbb{V}}
\newcommand*{\bbW}{\mathbb{W}}
\newcommand*{\bbX}{\mathbb{X}}
\newcommand*{\bbY}{\mathbb{Y}}
\newcommand*{\bbZ}{\mathbb{Z}}
%Fraktur letters
\newcommand*{\frakA}{\mathfrak{A}}
\newcommand*{\frakB}{\mathfrak{B}}
\newcommand*{\frakC}{\mathfrak{C}}
\newcommand*{\frakD}{\mathfrak{D}}
\newcommand*{\frakE}{\mathfrak{E}}
\newcommand*{\frakF}{\mathfrak{F}}
\newcommand*{\frakG}{\mathfrak{G}}
\newcommand*{\frakH}{\mathfrak{H}}
\newcommand*{\frakI}{\mathfrak{I}}
\newcommand*{\frakJ}{\mathfrak{J}}
\newcommand*{\frakK}{\mathfrak{K}}
\newcommand*{\frakL}{\mathfrak{L}}
\newcommand*{\frakM}{\mathfrak{M}}
\newcommand*{\frakN}{\mathfrak{N}}
\newcommand*{\frakO}{\mathfrak{O}}
\newcommand*{\frakP}{\mathfrak{P}}
\newcommand*{\frakQ}{\mathfrak{Q}}
\newcommand*{\frakR}{\mathfrak{R}}
\newcommand*{\frakS}{\mathfrak{S}}
\newcommand*{\frakT}{\mathfrak{T}}
\newcommand*{\frakU}{\mathfrak{U}}
\newcommand*{\frakV}{\mathfrak{V}}
\newcommand*{\frakW}{\mathfrak{W}}
\newcommand*{\frakX}{\mathfrak{X}}
\newcommand*{\frakY}{\mathfrak{Y}}
\newcommand*{\frakZ}{\mathfrak{Z}}
\newcommand*{\fraka}{\mathfrak{a}}
\newcommand*{\frakb}{\mathfrak{b}}
\newcommand*{\frakc}{\mathfrak{c}}
\newcommand*{\frakd}{\mathfrak{d}}
\newcommand*{\frake}{\mathfrak{e}}
\newcommand*{\frakf}{\mathfrak{f}}
\newcommand*{\frakg}{\mathfrak{g}}
\newcommand*{\frakh}{\mathfrak{h}}
\newcommand*{\fraki}{\mathfrak{i}}
\newcommand*{\frakj}{\mathfrak{j}}
\newcommand*{\frakk}{\mathfrak{k}}
\newcommand*{\frakl}{\mathfrak{l}}
\newcommand*{\frakm}{\mathfrak{m}}
\newcommand*{\frakn}{\mathfrak{n}}
\newcommand*{\frako}{\mathfrak{o}}
\newcommand*{\frakp}{\mathfrak{p}}
\newcommand*{\frakq}{\mathfrak{q}}
\newcommand*{\frakr}{\mathfrak{r}}
\newcommand*{\fraks}{\mathfrak{s}}
\newcommand*{\frakt}{\mathfrak{t}}
\newcommand*{\fraku}{\mathfrak{u}}
\newcommand*{\frakv}{\mathfrak{v}}
\newcommand*{\frakw}{\mathfrak{w}}
\newcommand*{\frakx}{\mathfrak{x}}
\newcommand*{\fraky}{\mathfrak{y}}
\newcommand*{\frakz}{\mathfrak{z}}
% Caligraphic letters
\newcommand*{\calA}{\mathcal{A}}
\newcommand*{\calB}{\mathcal{B}}
\newcommand*{\calC}{\mathcal{C}}
\newcommand*{\calD}{\mathcal{D}}
\newcommand*{\calE}{\mathcal{E}}
\newcommand*{\calF}{\mathcal{F}}
\newcommand*{\calG}{\mathcal{G}}
\newcommand*{\calH}{\mathcal{H}}
\newcommand*{\calI}{\mathcal{I}}
\newcommand*{\calJ}{\mathcal{J}}
\newcommand*{\calK}{\mathcal{K}}
\newcommand*{\calL}{\mathcal{L}}
\newcommand*{\calM}{\mathcal{M}}
\newcommand*{\calN}{\mathcal{N}}
\newcommand*{\calO}{\mathcal{O}}
\newcommand*{\calP}{\mathcal{P}}
\newcommand*{\calQ}{\mathcal{Q}}
\newcommand*{\calR}{\mathcal{R}}
\newcommand*{\calS}{\mathcal{S}}
\newcommand*{\calT}{\mathcal{T}}
\newcommand*{\calU}{\mathcal{U}}
\newcommand*{\calV}{\mathcal{V}}
\newcommand*{\calW}{\mathcal{W}}
\newcommand*{\calX}{\mathcal{X}}
\newcommand*{\calY}{\mathcal{Y}}
\newcommand*{\calZ}{\mathcal{Z}}
% Script Letters
\newcommand*{\scrA}{\mathscr{A}}
\newcommand*{\scrB}{\mathscr{B}}
\newcommand*{\scrC}{\mathscr{C}}
\newcommand*{\scrD}{\mathscr{D}}
\newcommand*{\scrE}{\mathscr{E}}
\newcommand*{\scrF}{\mathscr{F}}
\newcommand*{\scrG}{\mathscr{G}}
\newcommand*{\scrH}{\mathscr{H}}
\newcommand*{\scrI}{\mathscr{I}}
\newcommand*{\scrJ}{\mathscr{J}}
\newcommand*{\scrK}{\mathscr{K}}
\newcommand*{\scrL}{\mathscr{L}}
\newcommand*{\scrM}{\mathscr{M}}
\newcommand*{\scrN}{\mathscr{N}}
\newcommand*{\scrO}{\mathscr{O}}
\newcommand*{\scrP}{\mathscr{P}}
\newcommand*{\scrQ}{\mathscr{Q}}
\newcommand*{\scrR}{\mathscr{R}}
\newcommand*{\scrS}{\mathscr{S}}
\newcommand*{\scrT}{\mathscr{T}}
\newcommand*{\scrU}{\mathscr{U}}
\newcommand*{\scrV}{\mathscr{V}}
\newcommand*{\scrW}{\mathscr{W}}
\newcommand*{\scrX}{\mathscr{X}}
\newcommand*{\scrY}{\mathscr{Y}}
\newcommand*{\scrZ}{\mathscr{Z}}

%Section break
\newcommand*{\brk}{
\rule{2in}{.1pt}
}

%General purpose stuff
\DeclareMathOperator{\Aut}{Aut}
\DeclareMathOperator{\ch}{char}
\DeclareMathOperator{\rank}{rank}
\DeclareMathOperator{\End}{End}
\let\Im\relax
\DeclareMathOperator{\Im}{Im}

%Category Theory
\DeclareMathOperator{\Hom}{Hom}
\let\hom\relax
\DeclareMathOperator{\hom}{hom}
\DeclareMathOperator{\id}{id}
\DeclareMathOperator{\coker}{coker}
\DeclareMathOperator{\colim}{colim}
\DeclareMathOperator{\invlim}{\lim_{\leftarrow}}
\DeclareMathOperator{\dirlim}{\lim_{\rightarrow}}

%Commutative Algebra
\DeclareMathOperator{\gldim}{gldim}
\DeclareMathOperator{\projdim}{projdim}
\DeclareMathOperator{\injdim}{injdim}
\DeclareMathOperator{\findim}{findim}
\DeclareMathOperator{\flatdim}{flatdim}
\DeclareMathOperator{\depth}{depth}

%Common Categories
%\newcommand*{\modR}{\mathbf{mod}\text{-}R}
%\newcommand*{\Rmod}{R\text{-}\mathbf{mod}}
\newcommand{\rmod}[1]{\mathbf{mod}\text{-}#1}
\newcommand{\lmod}[1]{#1\text{-}\mathbf{mod}}
\DeclareMathOperator{\Vectk}{\mathbf{Vect}_k}
\DeclareMathOperator{\Ch}{\mathbf{Ch}}
\newcommand*{\Ab}{\mathbf{Ab}}
\newcommand*{\Grp}{\mathbf{Grp}}
\newcommand*{\Alg}{\mathbf{Alg}_k}
\newcommand*{\Ring}{\mathbf{Ring}}
\newcommand*{\K}{\mathbf{K}}
\newcommand*{\D}{\mathbf{D}}
\newcommand*{\Db}{\mathbf{D}^b}
\newcommand*{\Dpos}{\mathbf{D}^+}
\newcommand*{\Dneg}{\mathbf{D}^-}
\newcommand*{\Dbperf}{\mathbf{D}^b_{\text{perf}}}
\newcommand*{\Dsing}{\mathbf{D}_{sing}}
\newcommand{\CRing}{\mathbf{CRing}}
\DeclareMathOperator{\stmod}{\mathbf{stmod}}
\DeclareMathOperator{\StMod}{\mathbf{StMod}}
\DeclareMathOperator{\sHom}{\underline{Hom}}

%Homological algebra
\DeclareMathOperator{\cone}{cone}
\DeclareMathOperator{\HH}{HH}
\DeclareMathOperator{\Der}{Der}
\DeclareMathOperator{\Ext}{Ext}
\DeclareMathOperator{\Tor}{Tor}

%Lie algebras
\DeclareMathOperator{\ad}{ad}
\newcommand*{\gl}{\mathfrak{gl}}
\let\sl\relax
\newcommand*{\sl}{\mathfrak{sl}}
\let\sp\relax
\newcommand*{\sp}{\mathfrak{sp}}
\newcommand*{\so}{\mathfrak{so}}

% Hacks and Tweaks
% Enumerate will automatically use letters (e.g. part a,b,c,...)
\setenumerate[0]{label=(\alph*)}
% Always use wide tildes
\let\tilde\relax
\newcommand*{\tilde}[1]{\widetilde{#1}}
%raise that Chi!
\DeclareRobustCommand{\Chi}{{\mathpalette\irchi\relax}}
\newcommand{\irchi}[2]{\raisebox{\depth}{$#1\chi$}} 



%header stuff
\setlength{\headsep}{24pt}  % space between header and text
\pagestyle{fancy}     % set pagestyle for document
\lhead{Notes on Hopf Algebras} % put text in header (left side)
\rhead{Nico Courts} % put text in header (right side)
\cfoot{\itshape p. \thepage}
\setlength{\headheight}{15pt}
\allowdisplaybreaks

% Document-Specific Macros
\newcommand*{\dgVectk}{\mathbf{dgVect}_k}
\newcommand*{\dgmodA}{\mathbf{dgmod}\text{-}A}
\newcommand*{\modA}{\mathbf{mod}\text{-}A}
\newcommand*{\Amod}{A\text{-}\mathbf{mod}}
\DeclareMathOperator{\depth}{depth}

\begin{document}
%make the title page
\title{Cohomology Theories, Triangulated Categories, and Applications\vspace{-1ex}}
\author{A course by: Prof. James Zhang\\
Notes by: Nico Courts}
\date{Spring 2019}
\maketitle

\renewcommand{\abstractname}{Introduction}
\begin{abstract}
	These are the notes I took in class during the Winter 2019 topics course
	\textit{Math 583H - Cohomology Theories, Triangulated Categories, and Applications} at University of Washington, Seattle. 
	
	The course description follows:

	\brk

	This covers some (co)homology theories and structures of triangulated categories recently used in noncommutative algebra and 
	noncommutative algebraic geometry. In addition, we will discuss some latest developments and applications in representation 
	theory and noncommutative algebraic geometry. Some basics (see first two topics) can be found in the book \textit{An Introduction to Homological Algebra}
	by Charles Weibel. Here is a list of possible topics:

\begin{itemize}
	\item Hochschild cohomology and group cohomology.
	\item Homotopy categories and derived categories.
	\item Dualizing complexes over noncommutative algebras and applications.
	\item Artin-Schelter regular algebras and noncommutative projective spaces.
	\item Semiorthogonal decomposition of derived categories.
\end{itemize}
\end{abstract}

\section{April 1, 2019}

The topic for this week includes dg-algebras and -categories. There is no great 
textbook for these subjects (for the next topics you can use Weibel), but there is 
a paper by B. Keller ``Deriving dg-categories'' which is pretty abstract. We might 
need a little time to understand it, however.

\subsection{What we are working with}
Let $k$ be a field throughout the course. As a matter of notation, we let $\dgVectk$
denote the category of differential graded vector spaces over $k$. Today we will discuss two ways to
view this, as a \textit{category} and as a \textit{dg-category.}

\subsection{As a category}
Recall that $\Vectk$ is the category with $k$-vector spaces with $k$-linear maps. Now $\dgVectk$ consists of
objects that are (cochain) complexes of $k$-modules $(M^i)_{i\in\bbZ}$ with differential
$\partial_M=(\partial_M^i)_{i\in\bbZ}$ where each 
\[\partial_M^i:M^i\to M^{i+1}\]
is a differential in the usual sense in homological algebra.

To describe the morphisms in this category: Let $M,N$ be two objects in $\dgVectk$. Then 
a (degree zero) morphism $f:M\to N$ are chain morphisms (the components commute with the differentials).

\begin{ex}
	$M^i=kx^i\oplus ky^i$ and $\partial_M^i(x^i)=0$, $\partial_M^i(y^i)=x^{i+1}$. This clearly
	works with everything above, so $M\in\dgVectk$.

	Now let $N^i=kx^i$ and $\partial_N^i(x^i)=0$. Then there is a morphism $f:M\to N$ sending $y^i\mapsto x^i$ and $x^i\mapsto 0$.
\end{ex}

\begin{prob}
	Check that $\dgVectk$ is an abelian category.
\end{prob}

\begin{defn}
	If $M,N\in \dgVectk$ and $f:M\to N$ such that $f^i:M^i\to N^{i+d}$ for all $i$, 
	we say $\deg f=|f|=d$.
\end{defn}

\subsection{As a symmetric monoidal category}
The first thing to do is to define the tensor product. Let $M,N\in \dgVectk$. Then we define
$M\otimes N$ to be
\[(M\otimes N)^i=\oplus_{s+t=i} M^s\otimes N^t\]
with differential (if $m\in M^s$ and $n\in N^t$)
\[\partial_{M\otimes N}^i(m\otimes n)=\partial_M^s(m)\otimes n+(-1)^sm\otimes \partial_N^t(n)\]
where we will often use the more universal notation $(-1)^s=(-1)^{|m|}$.

\begin{rmk}
	Notice that $|\partial|=1$, so the above sign can be more suggestively written as $(-1)^{|m||\partial|}$
	which is a nod to the Koszul sign rule below.

	To be clear, the Koszul sign rule states, if $f$ and $g$ are graded (that is, chain) maps, then
	\[(f\otimes g)(m\otimes n)=(-1)^{|m||g|}f(m)\otimes g(n)\]
	and so the map we are considering above is $\id_M\otimes \partial_N+\partial_M\otimes \id_N$
\end{rmk}
\begin{prob}
	Check that $\partial_{M\otimes N}$ is a differential.
\end{prob}

We want this monoidal category to be \textit{symmetric}, however. So to see this, define the twist functor
\[\tau:M\otimes N\to N\otimes M\]
such that $\tau(m\otimes n)=(-1)^{|m||n|}n\otimes m$ (Koszul sign rule).
\begin{lem}
	\begin{enumerate}
		\item $\tau\in\Hom_{\dgVectk}(M\otimes N,N\otimes M)$. 
		\item $\tau\circ\tau|_{M\otimes N}=\id_{M\otimes N}$.
	\end{enumerate}
\end{lem}

Then since we have the twist functor $\tau$ satisfying the above properties, we get that $\dgVectk$ is 
symmetric monoidal. Notice that the unit object for the monoidal aspect is the complex
\[\cdots\to 0\to k\to 0\to\cdots\]

\subsection{Differential Graded Algebras}
\begin{defn}
	A differential graded algebra $A$ is an algebra object in $\dgVectk$.
\end{defn}
\begin{defn}
	A differential graded algebra $A$ is a $\bbZ$-graded algebra $A=\oplus_{i\in \bbZ}A_i$ with differential
	$\partial_A^i:A^i\to A^{i+1}$ satisfying:
	\begin{enumerate}
		\item $1\in A_0$
		\item $A_iA_j\in A_{i+j}$
		\item $\partial_A(1_A)=0$
		\item $\partial_A(ab)=\partial_A(a)b+a\partial_A(b)$
	\end{enumerate}
\end{defn}

\section{April 3, 2019}
At the end of the last lecture, we wondered why we want to do homological algebra (and thus why
are we interested in dg-structures)?

\subsection{Some History}
Suppose we have a class $\mathcal C$ of objects. Then around 1900, Poincar\'e noted we can construct 
a class of complexes consisting of these objects. Then one can compute (co)homologies and their dimensions (Betti numbers).
Then the alternating sum of the Betty numbers gives us the Euler characteristic.

In topology, around 1925, people did this with topological spaces and singular (co)homology. Later, in the 1930's, 
de Rham introduced de Rham complexes for manifolds and continued similarly. Also around this time, \v Cech developed 
\v Cech complexes for schemes for \v Cech cohomology. In 1945, Hochschild developed Hochschild complexes.
In 1948, Chevalley-Eilenberg did Lie algebras and Harrison did commutative algebras.

\subsection{Commutative DGAs}
Last time we saw the definition of a differential graded algebra. 
\begin{defn}
	A DGA $A$ is called \textbf{commutative} if 
	\[ab=(-1)^{|a||b|}ba\]
	for homogeneous elements $a,b\in A$.
\end{defn}

\subsection{Returning to $\dgVectk$}
Recall that the objects in this category are complexes $M^\bullet$ with differential $\partial^i:M^i\to M^{i+1}$.
\begin{defn}
	For $M,N\in \dgVectk$, define
	\[\Hom_{\dgVectk}(M,N)=(\Hom_{\dgVectk}(M,N)^i,\partial^i)\]
	where 
	\[\Hom_{\dgVectk}(M,N)^i:= \prod_{s\in\bbZ}\Hom(M^s,N^{s+i})\]
	and for each $i$, $\partial^i_{\Hom}:\Hom_{\dgVectk}(M,N)^i\to \Hom_{\dgVectk}(M,N)^{i+1}$, where
	\[\partial^i_{\Hom}((f^s)_{s\in\bbZ})=(\partial_N^{s+i}\circ f^s-(-1)^{|f||partial|}f^{s+1}\circ\partial_M^s)_{s\in\bbZ}\]
\end{defn}
\begin{prob}
	Show that $\partial_{\Hom}$ is a differential. Hint: use that $\partial_M$ and $\partial_M$ are.
\end{prob}
\begin{rmk}
	The composition in $\dgVectk$ is defined as: if $f\in\Hom_{\dgVectk}(M,N)^i$ and $g\in\Hom_{\dgVectk}(K,M)^j$,
	then $f\circ g\in\Hom_{\dgVectk}(K,N)^{i+j}$ where $(f\circ g)=(f^{s+j}\circ g^s)_{s\in\bbZ}$.


\end{rmk}
\begin{lem}
	\begin{enumerate}
		\item $1_M\in\Hom_{\dgVectk}(M,M)^0.$
		\item Composition has the following properties:
		\begin{enumerate}
			\item $\circ$ is associative.
			\item $\Hom_{\dgVectk}(N,P)\otimes \Hom_{\dgVectk}(M,N)\xrightarrow{\circ} \Hom_{\dgVectk}(M,P)$
			is a morphism of complexes. 
		\end{enumerate}
	\end{enumerate}
\end{lem}

\subsection{dg Categories}
\begin{defn}
	A \textbf{dg Category} is a category $\mathcal C$ enriched over $\dgVectk$.
\end{defn}
Okay, that is slightly cheating although it is quite nice. 
\begin{defn}
	Let $\mathcal C$ be a $k$-linear category. We say $\mathcal C$ is a \textbf{dg category} if:
	\begin{enumerate}
%		\item For all objects in $M,N\in\mathcal C$, $\Hom_{\mathcal C}(M,N)$ is a complex $(\Hom_{\mathcal C}(M,N)^i,\partial^i_\Hom )_{i\in\bbZ}$
		\item $1_M\in\Hom_{\dgVectk}(M,M)^0$
		\item The other stuff we talked about in the above lemma.
	\end{enumerate}
\end{defn}
\begin{rmk}
	So a natural question to arise is whether a specific category admits an ``enhancement'', or enrichment 
	over $\dgVectk$. If so, we can do homological algebra!

	Next lecture will be more concrete. :)
\end{rmk}

\section{April 5, 2019}
Today we will just focus on examples of dg categories! No definitions today. :)

\subsection{Examples of dg categories}
\begin{ex}
	$\dgVectk$ is a dg category. We also use this notation to refer to the category without the dg structure.

	Some notation: when we write 
	\[\hom(M,N)=\{\text{maps of degree zero}\}\]
	we are taking about the regular category . Then we call the same category where the maps 
	have arbitrary degree as the dg enchancement.
\end{ex}
\begin{ex}
	A dg algebra $A$ is a dg category with one object. It has a single object $\ast$ and $\Hom_A(\ast,\ast)=A$.
\end{ex}
\begin{rmk}
	This gives us a third definition for a dg category: a dg algebra with several objects. This parallels the fact
	that an algebra over $k$ can be considered a $k$-linear category with a single object and a $k$-algebra with several
	objects is a $k$-linear category. This is a bit funny.
\end{rmk}
In this last example, the differential on $A$ is $\Hom(\ast,\ast)$ and composition respects grading 
and the differential. Furthermore, composition is a homomophism of degree zero ($\deg(a\circ b)=\deg a+\deg b$)
and finally it is a morphism of complexes of degree zero.
\begin{ex} 
	Let $R$ be a algebra, then the category of complexes of right $R$-modules $\Ch=\Ch(\mathbf{mod}\text{-}R)$
	is a dg category whose objects are complexes of right $R$-modules and whose morphisms are themselves complexes.

	That is, $\Hom_{\Ch}(M,N)^i=\prod_{s\in\bbZ}\Hom_R(M^s,N^{s+i})$ with differential
	\[d_{\Hom}(f)=d_N\circ f-(-1)^{|f|}f\circ d_M.\]
\end{ex}
\begin{rmk}
	Usually when one studies $\Ch$, one only considers the regular (non dg) category structure. Actually 
	this is the case in books like Weibel and Rotman.

	When one is looking at representation theory, on considers $R$-modules and so it is natural to consider
	$\mathbf{D}(\modR)$, the derived category where objects are complexes. Then $\Ch(R)$
	has complexes for both objects and morphisms.

	Recovering $R$ from the modules is Morita theory. Getting $\mathbf{mod}\text{-}R$ from its derived category
	is derived Morita theory. 
\end{rmk}

\subsection{He lied about no definitions}
\begin{defn}
	Let $A=(A,m,u,\partial)$ be a dg algebra. $M$ is called a \textbf{dg right $A$-module} if 
	\begin{enumerate}
		\item $M$ is a complex with differential $\partial_M$. 
		\item There is a right action on $M$ given by $m_M:M\otimes A\to M$ such that $m_M$ is a morphism
		of complexes of degree zero (i.e. $\partial_M(ma)=\partial_M(m)a+(-1)^{|m|}m\partial_A(a)$). Furthermore
		$m_M$ satisfies the usual diagrams for a right action.
	\end{enumerate}
\end{defn}
\begin{rmk}
	We write $\dgmodA$ to denote the right $A$-modules. As usual, we can define it as a dg category
	or regular category.

	Let $M$ and $N$ be two dg right $A$-modules. A morphism of degree zero from $M$ to $N$ maps
	$M^i$ to $N^i$ for all $i$ and commutes with the differential and is $A$-linear.

	When scaling up to the dg structure, we get $\Hom_{\dgmodA}(M,N)$ consists of complexes where each 
	component consists of maps of the appropriate degree that are all still (right) $A$-linear.

	Notice that in the general case we don't get commutativity with differentials.
\end{rmk}
\begin{rmk}
	Notice that these are nonempty since $\id_M\in\Hom(M,M)^0$ and if $I$ is any dg-ideal  of $A$
	(graded ideal closed under $\partial$), then the quotient map is in $\Hom(A,A/I)^0$.
\end{rmk}
The differential on $\Hom(M,N)$ is the usual thing with the sign trick.
\begin{prob}
	Prove that $\partial_{\Hom}\circ \partial_{\Hom}=0$ and that $\dgmodA$ is a dg category.
\end{prob}

Note that if we let $A=(\cdots\to 0\to R\to 0\to \cdots)$ then $\dgmodA$ is the same as $\Ch(\modR)$.

\section{April 8, 2019}
Today we will mostly be playing with morphisms. We expect it to be quite easy. On Wednesday we will play with the objects.

\subsection{Morphisms in a dg category}
Notice that if $A$ is a dg algebra (or just an algebra) -- recall that this is a (dg) Category with one object.
Then we can define $B\subseteq A$ subalgebras, $I\subseteq A$ ideals, and quotients $A/I$ as usual.

It is important to notice that properties can change from $A\to B$ and $A\to A/I$. The set of units may change, for instance.

Today we will let $\mathcal C$ be a dg category and then analyze 6 (dg) categories and ideals associated to $\mathcal C$. Along
the way, it is good to keep $\mathbf{dgAlg}$ and $\dgmodA$ in mind as examples.

Recall we had one definition of a dg category $\mathcal C$ as an Algebra with several objects. Then we 
are really studying $\mathcal B,\mathcal I\subseteq\mathcal C$ and $\mathcal C/\mathcal I$. Here we are 
assuming that the objects in all related categories are the same.
\begin{prob}
	Write down the definitions of subcategory, ideal, and quotient that work with these definitions.
\end{prob}

\subsection{Homological Categories?!}
Let $\mathcal C$ be a dg category. Then we will define analogs of the homological algebra objects 
we look for usually.
\begin{defn}
	Define $Z(\mathcal C)$ to be the category with the same objects as $\mathcal C$ with 
	\[Z\Hom_{\mathcal C}(M,N)=\Hom_{Z(\mathcal C)}(M,N)=\{f\in \Hom_{\mathcal C}(M,N)|\partial_{\Hom_{\mathcal C}}(f)=0\}\]
\end{defn}
\begin{lem}
	$Z(\mathcal C)$ is a dg subcategory of $\mathcal C$ (with zero differentials).
\end{lem}
\begin{prf}
	It is clear $Z(\mathcal C)\subseteq \mathcal C$. Recall that an axiom of a dg category is that $\partial 1_M=0.$
	Thus $1_M\in \Hom_{Z(\mathcal C)}(M,M). $

	Let $f\in\Hom(M,N)$ and $g\in\Hom(N,P)$. Then 
	\[\partial(g\circ f)=\partial (g)\circ f+(-1)^{|g|}g\circ \partial(f)=0\]
	and we are done. :)
\end{prf}
\begin{defn}
	$B(\mathcal C)$ again consists of the same objects as $\mathcal C$ but now 
	\[B\Hom_{\mathcal C}(M,N)=\Hom_{B(\mathcal C)}(M,N)=\{f\in \Hom(M,N)|f=\partial_{\Hom}s f'\}\]
\end{defn}
\begin{lem}
	$B(\mathcal C)$ is an ideal of $Z(\mathcal C)$.
\end{lem}
\begin{rmk}
	To show that $B(\mathcal C)$ is an ideal in $Z(\mathcal C)$, we need to check that for all $f\in Z(\mathcal C)$ and $g\in B(\mathcal C)$
	that (if $f$ and $g$ can be composed) that $f\circ g$ and/or $g\circ f$ are back in $B(\mathcal C)$.
\end{rmk}
\begin{prf}
	Let $f\in \Hom_Z(M,N)$ and $g\in\Hom_B(L,K)$. Then by definition $g=\partial g'$. If $f\circ g$ 
	is defined, then 
	\begin{align*}
		(-1)^{|f|}f\circ g &= f\circ \partial(g')\\
		&= \partial(f)\circ g +(-1)^{|f|}f\circ\partial(g')\\
		&=\partial(f\circ g')\in B(\mathcal C)
	\end{align*}
	reversing the composition, if allowable, can be checked similarly.
\end{prf}
\begin{defn}
	$H(\mathcal C)$ is defined to have the objects of $\mathcal C$ and
	\[H\Hom(M,N)=\Hom_{H(\mathcal C)}(M,N)=\Hom_Z(M,N)/\Hom_B(M,N).\]
\end{defn}
James uses the book \textit{Rational Homotopy Theory} when he is talking about dgas and dg categories.
He says it's a good reference for this part of the class.

\begin{defn}
	Suppose $\mathcal C$ is a dg category with zero differential. Then define
	$\mathcal C^0$ to be the category with the same objects and 
	\[\Hom_{\mathcal C^0}(M,N)=\Hom_{\mathcal C}(M,N)^0\]
\end{defn}
\begin{lem}
	$\mathcal C^0$ is a category.
\end{lem}
\begin{prf}
	``It's kinda obvious, right?''
\end{prf}

\subsection{Homotopy categories}
We can define in the more concrete case the category $\mathbf{K}(\modR):=H^0(\Ch(\modR))$.

\section{April 10, 2019}
Today is going to be a three-star day. We are going to be doing the next natural thing: ``playing with objects \textit{and} morphisms.''

\subsection{Playing with Objects}
Last time we defined the categories we were talking about without changing the underlying objects of the category.
We need to define:
\begin{itemize}
	\item Shift or suspension
	\item Mapping cone
	\begin{itemize}
		\item To do this we will need some motivation.
	\end{itemize}
\end{itemize}
Recall that last time we discussed the process of creating $H^0(\dgmodA)$. Here the $\hom$ sets 
are just vector spaces (instead of dg vector spaces) so we lose some structure. Then the shift and the
cone operations will help recover (some) of the lost information.

\begin{rmk}
	From last time: let $A$ be a (dg) algebra. Then we can define a subalgebra $B$ of $A$
	and ideals $I\lhd A$ and quotients $A/I$. We did everything in categories on Monday: subcategories, ideals, and quotients.

	One important process in algebras is localization. The analog also exists in category theory: we can invert 
	a collection of morphisms. This is more complicated (as expected) than in the case of algebras, but analogous.
\end{rmk}
\begin{rmk}
	$Z^0(\dgmodA)$ is an Abelian category. $H^0(\dgmodA)$ is usually not Abelian. So we can't really
	talk about subobjects and quotients, etc, in the homotopy category although we can talk about them in the 
	cycle category and remember the correspondence.
\end{rmk}
\begin{rmk}
	Consider $\Ch(\modR)$. We can define three different functors:
	\begin{itemize}
		\item $P:\modR\to H^0(\Ch(\modR))$, which takes the projective resolution of a module.
		\item $i:\modR\to H^0(\Ch(\modR))$, which embeds $M$ as a complex $\cdots\to 0\to M\to 0\to \cdots$.
		\item $I:\modR\to H^0(\Ch(\modR))$, which assigns to each module an injective resolution.
	\end{itemize}
	The first and third functors are called the \textbf{projective} and \textbf{injective deriving functors,} respectively.

	We have a natural transformations $P\to i\to I$ in $H^0(\Ch(\modR))$, but when we pass to $D(\modR)$,
	these become natural isomorphisms.
\end{rmk}

\subsection{Shift/Suspension}
\begin{defn}
	Let $\mathcal C=\dgmodA$. Let $M$ be a right dg $A$ module (complex). Then the \textbf{shift}
	of $M$, denoted by $\Sigma M$, is another right dg $A$-module such that:
	\begin{itemize}
		\item $\Sigma M=M$ as a right $A$-module.
		\item $(\Sigma M)^i=M^{i+1}$ for all $i$.
		\item $\partial_{\Sigma M}=\partial_M$
	\end{itemize}
\end{defn}
\begin{rmk}
	Elements in $(\Sigma M)^i=M^{i+1}$ are denoted by $sm$, where $m\in M^{i+1}$. Then 
	\[\partial_{\Sigma M}(sm)=(-1)s(\partial_M(m))\in (\Sigma M)^{i+1}=M^{i+2}.\]
\end{rmk}
\begin{rmk}
	Actually, $\Sigma$ is a functor! Just using that $|\partial|=1$ got us the sign, but we 
	can generalize completely using $\sigma f=(-1)^|f|f.$
\end{rmk}
\begin{ex}
	Suppose $A=(\cdots\to 0\to R\to 0\to \cdots)$ with $R$ in the zeroth position. Then $\Sigma M$ is the same complex
	with $R$ shifted to the $(-1)^{st}$ position.
\end{ex}
\begin{lem}
	$\Sigma$ is invertible and the inverse is denoted $\Sigma^{-1}$ in $\dgmodA$ and all cycle, cocycle, and homology categories.
\end{lem}
\begin{rmk}
	It is a bit troubling talking about functors on $B^0(\mathcal C)$, which is an ideal category, but not 
	a category proper. Basically the idea here is that we still get that it preserves morphisms, but identities may 
	not exist and therefore can't be preserved.
\end{rmk}
\begin{lem}
	\[\Hom_{\dgmodA}(M,\Sigma N)=\Sigma|_{\dgVectk} \Hom_{\dgmodA}(M,N)\] 
	Hint: $\Hom(M,\Sigma N)^i=(\Sigma\Hom(M,N))^i$.
\end{lem}
\subsection{Mapping Cone}
As a construction, we first consider it as a map of sets
\[\cone:\Hom_{Z^0(\dgmodA)}(M,N)\to \dgmodA.\]
\begin{defn}
	If $f:M\to N$ is a cycle in $\dgmodA$, then we define \textbf{the cone of $f$}, denoted $\cone(f)$,
	to be a right dg $A$ module such that
	\begin{itemize}
		\item $(\cone(f))^i=(\Sigma M\oplus N)^i=M^{i+1}\oplus N^i$
		\item $(sm+n)\cdot a=s(m\cdot a)+n\cdot a$
		\item $\partial_{\cone(f)}(sm+n)=-s\partial_M^{i+1}(m)+[f(m)+\partial_N^i(n)]$
	\end{itemize}
\end{defn}
\begin{lem}
	$\partial^{i+1}\circ\partial^i=0$.
\end{lem}
\begin{prf}
	\[\partial(sm+n)=-s\partial_M(m)+[f(m)+\partial_N(n)]\]
	so applying the differential again:
	\[\partial^2(sm+n)=s\partial_M^2(m)+-s\partial\]
	I am missing something here but I believe it is clear.
\end{prf}
\begin{thm}
	$\cone:\operatorname{Mor}(Z^0(\dgmodA))\to \dgmodA$ descends to a map to $H^0(\dgmodA).$
\end{thm}
Here is some homework or just some facts if you are feeling lazy:
\begin{itemize}
	\item $0\to N\to \cone(f)\to \Sigma M\to 0$ is short exact (in $\dgmodA$).
	\item If $K\in \dgmodA$, then 
	\[\Hom_{\dgmodA}(K,\cone(f))\cong\cone_{\dgVectk}(\Hom_{\dgmodA}(K,f))\]
\end{itemize}

\section{April 12, 2019}
Next week's topic will be Hochschild cohomology, which you can find in Weibel (chapter 9). Today we are going to finish
the lecture we began on Wednesday. We will also review projective resolutions.

\subsection{Mapping Cone (again)}
Recall that if we have $f:M\to N$ in $Z^0(\dgmodA)$, we can construct $\cone(f)\in\dgmodA$ where
\[\cone(f)^i=(\Sigma M\oplus N)^i\]
and
\[\partial_{\cone(f)}=\partial_{\Sigma M}\oplus(f+\partial_N)\]
and more specifically
\[\partial_{\cone(f)}(sm+n)=-s\partial_M^{i+1}(m)+(f(m)+\partial_N^i(n)).\]

\begin{rmk}
	Where does this formula come from? Consider the diagram

	\begin{center}
		\begin{tikzcd}
			M^{i-1}\ar[r,"\partial_M^{i-1}"]\ar[d,"f^{i-1}"] & M^i\ar[r,"\partial^i_M"]\ar[d,"f^i"]\ar[dl] & M^{i+1}\ar[d,"f^{i+1}"]\ar[dl]\\
			N^{i-1}\ar[r,"\partial_N"] & N^i\ar[r,"\partial N"] & N^{i+1}
		\end{tikzcd}
	\end{center}

	\noindent where the diagonal arrow indicate ``pushing down'' a copy of $M^i$ to direct sum with $N^{i-1}$.
	But then using the maps you have in the diagrams, you get the maps $\partial_M^{i+1},f^{i+1}$ and $\partial_N^i$
	between $M^{i+1}\oplus N^i$ and $M^{i+1}\oplus N^{i+1}$, and $\partial_{\cone(f)}$ is just their sum!

	I need to think more to see whether this fact is actually meaningful or just a mnemonic.
\end{rmk}
\begin{ex}
	Consider a projective resolution $P_M$ of $M$ and the map $f:P_0\to M$. Then considering that $M$ is 
	a complex with zeros everywhere except the zero position and consider the chain map $f$ induced by $f_0$
	(and zeros elsewhere) and then we can compute
	\[\cone(f)=\cdots \to P^{-1}\to P^0\to M\to 0. \]
\end{ex}
\subsection{Facts about $\cone(f)$}
\begin{rmk}
	There exists a short exact sequence in $\dgmodA$
	\[0\to N\to\cone(f)\to\Sigma M\to 0\]
	where $n\mapsto 0+n$ under the first man and $sm+n\mapsto sm$ in the second.

	This is called the \textbf{mapping cone sequence.}
\end{rmk}
\begin{rmk}
	Let $K$ be another dg right $A$ module. Then
	\[Z^0(\dgVectk)\ni\Hom_{\dgmodA}(K,f):\Hom_{\dgmodA}(K,M)\to\Hom_{\dgmodA}K,N)\]
	and furthermore
	\[\cone(\Hom_{\dgmodA}(K,f))\cong\Hom_{\dgmodA}(K,\cone(f))\]
	and
	\[\cone(\Hom_{\dgmodA}(\Sigma f,K))\cong\Hom_{\dgmodA}(\cone(f),K)\]
\end{rmk}
\begin{rmk}
	Let $K$ be a left dg $A$ module. Then the map
	\[f\otimes_A \id_K:M\otimes_AK\to N\otimes_AK\]
	is such that
	\[\cone(f\otimes \id_K)=\cone(f)\otimes K.\]

	We can also switch ``left'' and ``right.''
\end{rmk}
\subsection{A ``problem''}
The map
\[\cone:\operatorname{Mor}(Z^0(\dgmodA))\to \dgmodA\]
is \textbf{not} functorial! 

However, suppose that $f,g:M\to N$ and $[f]=[g]\in H^0(\dgmodA)$. Whenever this is true, there exists
a map $\Phi:\cone(f)\to \cone(g)$ that fits into the diagram
\begin{center}
\begin{tikzcd}
	0\ar[r] & N\ar[d,"\cong"]\ar[r]& \cone(f)\ar[r]\ar[d,dashed,"\Phi"] & \Sigma M\ar[d,"\cong"]\ar[r] & 0\\
	0 \ar[r] & N\ar[r] & \cone(g)\ar[r] & \Sigma M\ar[r] & 0
\end{tikzcd}
\end{center}

\noindent which is clearly an isomorphism via the five lemma. In fact, such a map exists 
if and only if these two maps are homotopic!

\subsection{Classical Homological Algebra}
Recall that we can always construct projective resolutions of a (right) $R$-module by taking
a free module that maps onto $M$ and then iterate. Thus such a resolution always exists.

Let $f:M\to N$ be a morphism of right $R$-modules. Then there exits $F$, the lift of $f$,
between any projective resolutions $P_M$ and $P_N$. Notice that the resolutions \textit{don't}
include the module themselves. Furthermore any two lifts of $f$ differ only by a boundary in $\Ch(R)$,
and thus are equal in the homotopy category.

From this we can conclude that any $M$ in $H^0(\Ch(R))$, there exists a unique projective resolution,
up to (a unique) isomorphism. In this way, we can see that ``taking a projective resolution'' defines a 
functor from $\modR$ to $H^0(\Ch(R))$.

For any two $R$ modules $M$ and $N$, 
\[\Ext_R^i(M,N):=H^i(\Hom_{\Ch(R)}(P_M,P_N)\left(=H^i(\Hom_{\Ch(R)}(P_M,N)\right).\]

If $M=N$, then $\Hom_{\Ch(R)}(P_M,P_M)$ is a dg algebra. As a consequence,
\[\Ext_R^*(M,M)=\oplus_i H^i(\Hom(P_M,P_M))\]
is a graded algebra.

\begin{defn}
	Then we can define \textbf{the Hochschild cohomology} as
	\[\oplus_{i\in \bbZ}\Ext_{R\otimes R^{op}}^i(R,R)\]
\end{defn}
\begin{rmk}
	A nickname of Hochschild cohomology is ``the derived functor of the center.'' 
\end{rmk}

\section{April 15, 2019}
Today we will focus on understanding $\HH^0(A)$, the zeroth Hochschild cohomology. Recall that 
Sarah Witherspoon has a preprint (I have it) of a book she is going to release on the subject.

Define the Hochschild cohomology as we saw last week. Here james spent some extra time defining $A^{op}$ 
as well as going over why it suffices to define it over $A^e$.

\subsection{Hochschild Cohomology}
For clarity:
\begin{defn}[Hochschild Cohomology]
	For every $i\in\bbZ_{\ge 0}$, the $i^{th}$ Hochschild cohomology of $A$ is defined to be
	\[\HH^i(A)=\Ext_{A^e}^i(A,A)=H^i(\Hom_{\Ch(A^e)}(P_A,P_A))\]
	where $A$ is any associative algebra.
\end{defn}
\begin{rmk}
	In particular, $\HH^0(A)=\Ext_{A^e}^0(A,A)=\Hom_{A^e}(A,A)$.
\end{rmk}
\begin{rmk}
	``\textbf{Slogan 1}'': Hochschild cohomology is the derived functor of $\Hom_{A^e}(A,A)$.
\end{rmk}
\begin{rmk}
	In fact, we can (and will!) define \textit{Hochschild cohomology of an algebra with coefficents
	in an $A^e$ module $M$} as $\HH^i(M)=\Ext_{A^e}^i(A,M).$
\end{rmk}
\begin{lem}
	$\Hom_{A^e}(A,A)=Z(A)$.
\end{lem}
\begin{prf}
	Given any $f\in \Hom(A,A)$, set $c=f(1)$. We claim that $c\in Z(A)$. To see this, notice that for 
	every $a\in A$, $ac=af(1)=f(a\cdot 1)=f(1\cdot a)=f(1)a=ca.$ 

	Thus we can define a map $f\mapsto f(1)$. To define the inverse, just sent $c\in Z(A)$ to 
	$f$ mapping $a\mapsto ca(=ac)$. Then a standard verification shows what we want.
\end{prf}
\begin{rmk}
	``\textbf{Slogan 2}'': Hochschild cohomology is the derived functor of ``the center.''
\end{rmk}
This leads us to a really ``wild'' conjecture:
\begin{conj}
	$\HH^*(A)$ is a commutative algebra.
\end{conj}
\begin{rmk}
	We really have only proved this for the zeroth piece, which is why this is so wild. In fact, it is false.
\end{rmk}
Instead,
\begin{thm}
	$\HH^*(A)$ is \textit{graded} commutative.
\end{thm}
\begin{rmk}
	We do not have the machinery we need to prove this yet. For today we will prove a weaker result.
\end{rmk}
\begin{thm}
	$\HH^*(A)$ is a graded algebra.
\end{thm}
\begin{prf}
	$\HH^*(A)=\Ext_{A^e}^{\ge 0}(A,A)$ by definition. This can be computed as
	\[H^*(\Hom_{\Ch(A^e)}(P_A,P_A))=H^*(B)\]
	where $B$ is a dg algebra. But then $H*(B)$ is a graded algebra.
\end{prf}

\subsection{Some more categories}
Let $\calC$ be an abelian category and let $\mathbf{Rex}(\calC)$ be the category of functors $\calC\to \calC$
that are right exact and commute with $\oplus.$
\begin{rmk}
	Notice that whenever $\operatorname{Fun}(\calC,\calC)$ can be a replacement, but it isn't always abelian.
\end{rmk}
\begin{thm}[Watts]
	If $\calC=R\text{-}\mathbf{mod}$, then $\mathbf{Rex}(\calC)=A^e\text{-}\mathbf{mod}$.
\end{thm}
\begin{rmk}
	More generally, any right exact functor from $A$ modules to $B$ modules that commutes with direct sums
	is isomorphic to $M\otimes_A -$ for some $(B,A)$-bimodule.
\end{rmk}
\begin{lem}
	Let $\calC$ be $A$-$\mathbf{mod}$. Then $\id_\calC=A\otimes_A-$.
\end{lem}
\begin{defn}
	Let $\calC$ be an abelian category. For every $i\in\bbZ$, the $i^{th}$ Hochschild cohomology
	of $\calC$ is defined to be
	\[\HH^i(\calC)=\Ext^i_{\mathbf{Rex}(\calC)}(\id_\calC,\id_\calC).\]
\end{defn}
\begin{rmk}
	If $\calC$ is $A$-$\mathbf{mod}$, then this agrees with $\HH^i(A)$. We have a map $\Phi:\mathbf{Rex}(\calC)\to A^e$-$\mathbf{mod}$
	sending $\id_\calC$ to $A$, and when $i=0$,
	\[\HH^0(\calC)=\Hom_{\mathbf{Rex}}(\id_\calC,\id_\calC)=\operatorname{Mor}_\calC(\id_\calC,\id_\calC)\]
	where $\operatorname{Mor}$ denotes the natural transformations. This forms a commutative ring where the multiplication
	is composition and the commutativity follows from the naturality squares.
\end{rmk}
\begin{defn}
	With $\calC$ as above, $\operatorname{Mor}_\calC(\id_\calC,\id_\calC)$ is called the \textbf{center of $\calC$}. This definition
	arises from the analogy with algebras.
\end{defn}
\begin{rmk}
	Hochschild cohomology is the derived functor of $\Hom_{\mathbf{Rex}}(\id_\calC,\id_\calC)$.
\end{rmk}
Next time we will consider $\HH^1$!

\section{April 17, 2019}
The lectures this week will cover three apparently distinct topics: the center, derivations, and deformations. The 
underlying math that connects these is the Hochschild complex, which we won't define until next week.

Some homework (for those who haven't seen it):
\begin{ex}
	Find the definition of a Lie algebra and work out some examples.
\end{ex}
\subsection{Derivations and $\HH^1(A)$}
Recall that $\HH^i(A)=\Ext_{A^e}^i(A,A)$.
\begin{defn}
	Let $A$ be an algebra. A $k$-linear map $f:A\to A$ is called a \textbf{derivation} if 
	\[f(1)=0\]
	and for all $a,b\in A$,
	\[f(ab)=f(a)b+af(b).\]

	The set of all derivations of $A$ is denoted $\Der(A)$.
\end{defn}
\begin{defn}
	Let $x\in A$. Define $I_x:A\to A$ by $I_x(a)=xa-ax$. One can easily check that $I_x\in\Der(A)$.
	This derivation is very special: it is called an \textbf{inner derivation.} Any derivation which is 
	not inner is \textbf{outer.}
\end{defn}
\begin{ex}
	$A=k[x]$ with derivation $\frac{d}{dx}$. Furthermore $\Der(A)=\{f(x)\frac{d}{dx}|f(x)\in A\}$. The only
	inner derivation is zero since $A$ is commutative.
\end{ex}
\begin{rmk}
	Notice that $\Der(A)$ is a Lie algebra with $[f,g]=fg-gf\in\Der(A)$. Furthermore $\text{Inn}(A)$ is a 
	Lie ideal, and $\Der(A)/\text{Inn}(A)$ is a quotient Lie algebra.
\end{rmk}
And finally we get to another definition of $\HH^1$:
\begin{defn}
	The first Hochschild cohomology group of $A$ is defined to be
	\[\HH^1(A)=\Der(A)/\text{Inn}(A)\]
	which we sometimes write as $\text{Out}(A)$, the outer derivations of $A$.
\end{defn}
\begin{rmk}
	(\textbf{Slogan 4}): $\oplus\HH^{i+1}(A)$ is the derived functor of $\text{Out}(A)$.
\end{rmk}
\begin{thm}
	$\oplus\HH^{i+1}(A)$ is a graded Lie algebra.
\end{thm}
\begin{prf}
	Nope.
\end{prf}
\begin{rmk}
	Notice that if $\HH^1(A)=0$, then every derivation is inner. For example, when 
	$A=M_n(k)$, every derivation of $A$ is inner. To see this, notice that $A\otimes A^{op}=M_n(k)\otimes M_n(k)=M_{n^2}(k)$,
	which has global dimension zero. So $\Ext_{A^e}^1(A,A)=0$.
\end{rmk}

\subsection{Using Categories}
Recall that $\HH^i(A)=\Ext_{\mathbf{Rex(\calC)}}^i(\id_{\calC},\id_\calC)$. It ends up that if we pick two 
algebras $A$ and $B$ which are Morita equivalent (that is, their module categories are equivalent) and let $\calC=A\text{-}\mathbf{mod}$ or $B\text{-}\mathbf{mod}$, this construction
yields the same thing! So $\HH^i(A)=\HH^i(B)$. This is not at all obvious on the level of complexes, 
so it is a vote in favor of abstract nonsense.

\section{April 19, 2019}
Today we are talking about \textbf{infinitesimal deformations of algebras} and how they will relate to $\HH^2(A)$.

It ends up that ``\textit{every deformation (moduli) problem is related to some cohomology.}'' We will use the letter 
``i'' to stand for ``infinitesimal.'' Recall that we have usually been considering algebras $A$ over a field $k$, but
we also want to consider algebras over a ring $R=k[t]/(t^2)\cong k\oplus kt$
\subsection{Infinitesimal Deformations and $\HH^2$}
\begin{defn}
	Let $(A,m)$ be a $k$ algebra. An \textbf{infinitesimal deformation} of $A$ is a $k[t]/t^2$ algebra
	structure $m_t=\ast_t$ (this is the multiplication) on $A\otimes k[t]/t^2$ (note this is free over $R$, using the $k$-basis of $A$).

	The multiplication on the deformation is defined as 
	\[a\ast_t b=ab+\mu(a,b)t\in A\oplus At\cong A\otimes R\]
	for some $\mu:A\otimes A\to A$. This only defines the multiplication for elements in $A$, but this extends by linearity to $A\oplus At:$
	\[(a+a't)\ast_t(b+b't)=a\ast_t b+(a'\ast_t b)t+(a\ast_t b')t+0\]
	since in $R$, $t^2=0$.
\end{defn}
\begin{rmk}
	Notice that it is actually part of the definition of an infinitesimal deformation is \textbf{associative.}
	I didn't see this on the board, but rather than having this family of algebras freely parameterized by $\mu$, 
	we only take those $\mu$ such that the deformation is associative.
\end{rmk}
\begin{ex}
	The \textbf{trivial infinitesimal deformation of $A$} is determined by $\mu=0$.
\end{ex}
\begin{ex}
	Let $A=k[x,y]$ (works for arbitrary variables) where we define, for all $f,g\in A$
	\[f\ast_t g=fg+\left(\frac{\partial f}{\partial x}\frac{\partial g}{\partial y}-\frac{\partial f}{\partial y}\frac{\partial g}{\partial x}\right)t.\]
	It ends up that this algebra is associative! You can show it. :) This version of Nico doesn't feel like typing it.
\end{ex}
\begin{lem}[Condition 2-C]\label{lem-2-C}
	$m_t$ is associative if and only if for all $a,b\in A$,
	\[a\mu(b,c)-\mu(ab,c)+\mu(a,bc)-\mu(a,b)c=0.\]
\end{lem}
\begin{lem}
	An $i$-deformation is completely determined by $\mu:A\otimes A\to A$ satisfying the 2-C condition in lemma~\ref{lem-2-C}.
\end{lem}
\begin{prf}
	Suppose $a,b,c\in A$ and that $(a\ast b)\ast c=a\ast(b\ast c)$. Then computing using the definitions gets us
	\[(ab)c+\mu(ab,c)t+\mu(a,b)ct=a(bc)-\mu(a,bc)t+a\mu(b,c)t.\]

	To go in the opposite direction, you basically use this identity to recover the equality for associativity.
\end{prf}

\begin{cor}
	The collection of infinitesimal deformations is in one-to-one correspondence with the collection of $k$-linear maps $\mu A\otimes A\to A$ 
	satisfying 2-C.
\end{cor}
\begin{cor}
	The collection of infinitesimal deformations is a $k$ vector space.
\end{cor}
\begin{rmk}
	Just formally define each infinitesimal deformation to be a vector corresponding to the $\mu$ which defines it.
	The fact that the set of such $\mu$ is a $k$ vector space means the structure can be ``pushed over'' to the other collection.
\end{rmk}

\begin{defn}
	Let $m_t^1$ and $m_t^2$ be two infinitesimal deformations of $A$. We say they are \textbf{equivalent} if 
	there is an $R=k[t]/t^2$-algebra isormorphism
	\[f_t:(A\otimes R,m_t^1)\to (A\otimes R, m_t^2)\]
	such that $f(a)=a+\zeta(a)t$ for all $a\in A$. In this case we write $m_t^1\sim m_t^2$ or $\mu^1\sim \mu^2$.
\end{defn}
\begin{defn}
	$m_t$ is called \textbf{quasi-trivial} if $m_t$ is equivalent to the trivial infinitesimal deformation.
\end{defn}

\begin{lem}[Condition 2-B]
	Let $m_t^1$ and $m_t^2$ be two infinitesimal deformations of $A$. Then $m_t^1\sim m_t^2$ if and only if 
	there exists $\zeta:A\to A$ such that $\mu^1(a,b)-\mu^2(a,b)=a\zeta(b)-\zeta(ab)+\zeta(a)b$ for all $a,b\in A$.
	This is called condition 2-B.
\end{lem}
\begin{lem}
	Every equivalence between $m_t^1$ and $m_t^2$ is completely determined by $\zeta$ in 2-B.
\end{lem}
\begin{defn}
	$\mu:A\otimes A\to A$ is called a \textbf{2-coboundary} if there exists $\zeta:A\to A$
	such that
	\[\mu(a,b)=a\zeta(b)-\zeta(ab)+\zeta(a)b\]
	(that is if $\mu$ is quasi-trivial).
\end{defn}
\begin{defn}
	$\mu$ is called a \textbf{2-cocycle} if it satisfies 2-C.
\end{defn}
\begin{rmk}
	This gives us a vector subspace of coboundaries (quasi-trivial infinitesimal deformations) of the
	space of all infinitesimal deformations, so we can do cohomology!
\end{rmk}
\begin{defn}
	The \textbf{second Hochschild cohomology of $A$} is
	\[\HH^2(A)=\frac{\text{{infinitesimal deformations}}}{\text{{quasi-trivial i-defs}}}=\frac{\text{{2-cocycles}}}{\text{{2-coboundaries}}}\]
\end{defn}

\section{April 22, 2019}
Today we will focus on connecting the notions that we have seen in the past several lectures (center/derivations/infinitesimal deformations) 
via the Hochschild complex. On Wednesday we are going to prove that the cohomology associated with the Hochschild complex are what we have computed so far.

Recall the definitions and constructions of $\HH^i(A)$ for $i=0,1,2$. Furthermore we asserted that $\HH^*(A)$
is a graded commutative ring. Furthermore we showed that $\HH^1(A)$ is a Lie algebra and that $\HH^{*+1}(A)$ is a 
graded Lie algebra.

A natural question one may ask is if there is a ``multiplication'' operation on $\HH^2$ that we can apply, analogous to the commutative ring/Lie algebra
in the first two examples. In that case, can we extend this structure to $\HH^{*+2}(A)$?
\begin{rmk}
	Weibel has a different way to talk about $\HH^2$. In fact, James seems to imply there are several different ways to think about it.
\end{rmk}

\subsection{The Hochschild Complex}
\begin{defn}
	Let $A$ be an algebra over $k$. Define the \textbf{Hochschild complex} $C(A)$ of $A$ to be a cochain complex 
	such that 
	\[C(A)^n=\Hom_k(A^{\otimes n},A)\]
	for $n\ge 0$ and $0$ for $n<0$. Note that $A^{\otimes 0}=k$.

	The differential is defined to be a map $\partial^n:C(A)^n\to C(A)^{n+1}$ where for every $f\in C(A)^n$,
	\begin{align*}
		\partial^n f(a_1\otimes \cdots\otimes a_{n+1})&=a_1f(a_2\otimes a_3\otimes\cdots\otimes a_{n+1})\\
		&\qquad +\sum_1^n(-1)^if(a_1\otimes\cdots\otimes a_ia_{i+1}\otimes\cdots \otimes a_n)\\
		&\qquad + (-1)^{n+1}f(a_1\otimes\cdots\otimes a_n)a_{n+1}
	\end{align*}
\end{defn}
\begin{rmk}
	The idea here is the first and last terms actually arise from the fact that $\Hom_k(A^{\otimes n},A)$ has a left- and right 
	$A$-module structure while the interior terms arise more from the chain complex.
\end{rmk}
\begin{lem}
	$C(A)$ is a cochain complex.
\end{lem}
\begin{rmk}
	We omit the proof for the sake of time. James assures us it is true. :P
\end{rmk}

\subsection{Double-checking our work}
Let's compute! 
\begin{defn}
	Define the $i$-cocycles and $i$-coboundaries as we do with any chain complex.
\end{defn}
Then we can compute $H^0(C(A))=Z^0(C(A))/B^0(C(A))=Z^0(C(A))$. But $C(A)^0=\Hom_k(k,A)\cong A$, 
so we can determine when $\partial(f)=0$:
\[\partial(f)(a)=af(1)-f(1)a=0\]
for all $a$, so this means that $f(1)$ is in the center of $A$, so $Z^0(C(A))=H^0(C(A))\cong\cal Z(A)$.

\brk

What about $\HH^1$? Now $Z^1$ is the set of $f\in \Hom(A,A)$, such that
\[\partial(f)(a\otimes b)=af(b)-f(ab)+f(a)b=0\]
and so rewriting this, we get $f(ab)=af(b)+f(a)b$, so $f$ is a derivation!

To compute the coboundaries, say $f=\partial(g)\in C(A)^1$. Then 
\[f(a)=\partial(g)(a)=ag(1)-g(1)a=-I_{g(1)}(a)\]
so $f$ is an inner derivation of $A$. So we recover $\HH^1$.

\brk

Now $Z^2$ is the set of $f\in C(A)^2$ such that $\partial(f)=0$ -- so
\[\partial(f)(a\otimes b\otimes c)=af(b\otimes c)-f(ab\otimes c)+f(a\otimes bc)-f(a\otimes b)c=0\]
which, via the universal property of the tensor product gets us a $\mu$ satisfying condition 2-C, giving us
an infinitesimal derivation.

Similarly unfurling definitions gets us that $B^2$ are the 2-coboundaries.

\section{April 24, 2019}
Today we are going to talk about the equivalence (in all degrees) of $H^i(C(A))$ and $\Ext_A^i(A,A)=\HH^i(A)$.
On Friday, we well begin working on triangulated and derived categories and will spend about two weeks there. 
For a reference, you can read Weibel, but the lecture will likely not follow his exposition for the most part.

\subsection{The bar resolution}
\begin{defn}
Let $A$ be an algebra over $k$. The \textbf{bar resolution} of $A$ is a chain complex
($\deg\partial=-1$) which is defined as follows:
\[B(A)=B(A;A)=\left(\cdots A^{\otimes 4}\to A^{\otimes 3}\to A^{\otimes 2}\to A\to 0\to \cdots\right)\]
where the tensor power in the $i^{th}$ term is $i+2$, whenever that makes sense.

The differential is zero eventually, and otherwise is defined to be $\partial_N:A^{\otimes n+2}\to A^{\otimes n+1}$ via
\[\partial(a_0\otimes a_1\otimes \cdots\otimes a_{n+1})=\sum_{i=0}^n(-1)^ia_0\otimes\cdots\otimes a_ia_{i+1}\otimes\cdots\otimes a_{n+1}\]
\end{defn}
\begin{rmk}
	We could actually define this as a cochain, but it is pretty standard to use a chain instead.
\end{rmk}
\begin{rmk}
	The name ``bar resolution'' comes from the fact that the notation for these tensor was (traditionally)
	\[a_0\otimes\cdots\otimes a_{n+1}\sim[a_0|a_1|\cdots|a_{n+1}].\]
\end{rmk}
\begin{ex}
	Some computations: $\partial_0(a_0\otimes a_1)=a_0a_1$ and
	\[\partial(a_0\otimes a_1\otimes a_2)=a_0a_1\otimes a_2-a_0\otimes a_1a_2\]
	so the idea is we are ``generalizing multiplication'' to higher tensor powers.
\end{ex}
\begin{lem}
	$B(A)$ is a complex.
\end{lem}
\begin{lem}
	Consider each $A^{\otimes n+2}$ as an $(A,A)$-bimodule using the natural multiplication on each 
	factor. Then $\partial_n$ is an $(A,A)$-bimodule morphism.
\end{lem}
\begin{cor}
	This gives us that $B(A)$ is a complex of $(A,A)$-bimodules.
\end{cor}
\begin{prf}
	For any $x\in A$, we need to show that 
	\[\partial_n(x\cdots(a_0\otimes\cdots\otimes a_{n+1}))=x\cdots\partial_n(a_0\otimes\cdots\otimes a_{n+1})\]
	but this is pretty much immediate once you define it the right way:

	Basically, define the left action of $A$ on $A^{\otimes n}$ by multiplication (in $A$ on the leftmost factor)
	and the right action by multiplication on the rightmost factor. There is your structure and it is easy enough 
	to confirm is compatible with your differential.
\end{prf}
\begin{thm}
	There exist maps (note: not bimodule maps) $s_n:B(A)_n\to B(A)_{n+1}$ such that for all $n$,
	\[\id_{B(A)_n}=s_{n-1}\circ\partial_n+\partial_{n+1}\circ s_n\]
\end{thm}
\begin{prf}
	It is easy enough to confirm that the maps
	\[s_n(a_0\otimes\cdots\otimes a_{n+1})=1\otimes a_0\otimes \cdots\otimes a_{n+1}\]
	work.
\end{prf}
\begin{lem}
	$B(A)$ is exact.
\end{lem}
\begin{rmk}
	We've just shown that $\id_{B(A)}\sim 0$, so we're basically done. For a reminder: say $f\in \ker\partial_n$. 
	But then 
	\begin{align*}
		f&=\id_{B(A)_n}(f)\\
		&= s_{n-1}\partial_n(f)+\partial_{n+1}s_n(f)\\
		&= 0 + \partial_{n+1}(s_n(f))\in\operatorname{Im}\partial_{n+1}.
	\end{align*}
\end{rmk}
\begin{thm}
	$B(A)_{n\ge 0}$ is a free/projective resolution of the $A^e$ module $A$.
\end{thm}
\begin{prf}
	We have already seen that the $\delta$ are $A^e$ module maps and that this is an exact sequence.
	It remains to see that each $A^{\otimes n+2}$ is free. But
	\[A^{\otimes n+2}=A\otimes A^{\otimes n}\otimes A\cong A\otimes A\otimes A^{\otimes n}\cong A^e\otimes A^{\otimes n}\]
	and so this is free over the $k$-basis for the $A^{\otimes n}$ term.
\end{prf}

\begin{cor}
	$\Ext_{A^e}^i(A,A)=H^i(\Hom_{\Ch(A^e\text{-}\mathbf{mod}}(B(A)_{n\ge 0},(B(A)_{n\ge 0})))=H^i(\Hom_{\Ch}(B(A)_{n\ge 0},A))$
\end{cor}
\begin{thm}
	$\Hom_{\Ch}(B(A)_{n\ge 0},A)\cong C(A)$
\end{thm}
\begin{prf}
	The tricky part here is defining $\partial_{\Hom_{\Ch}(B(A),A)}:f\mapsto f\circ\partial_{B(A)}$ and then doing the computation. But recall that
	$\partial_{B(A)}$ and $\partial_{C(A)}$ are almost the same except the first and last terms. Well, use the
	$(A,A)$-bimodule structure to pull out $a_1$ and $a_{n+1}$ and you get what you want!
\end{prf}
\begin{cor}
	$H^i(C(A))\cong\Ext_{A^e}^i(A,A)$
\end{cor}

\subsection{The more general problem}
\begin{lem}
	Let $M$ be a (left) $A$ module. Then $B(A)_{n\ge 0}\otimes_A M$ is a free resolution of $M$.
\end{lem}
\begin{rmk}
	We could have just as easily have used right module structure by swapping the tensor.
\end{rmk}
\begin{rmk}
	This gives us a very nice thing -- we can get a free resolution of any module! But unfortunately this 
	is quite a terrible resolution to use if you want to do any computations.
\end{rmk}

\section{April 26, 2019}
Finally we are talking about the thing I named these notes after: triangulated categories!
I was late, but I don't believe that James has done the axioms yet.
\begin{rmk}
	It is an open question (Julia also mentioned this) about whether the octahedral axiom is really required for a triangulated category.
\end{rmk}
\subsection{Triangulated Categories}
Some motivation (from Grothendieck): Let $X$ be a projective scheme of dimension $d$ and dualizing 
sheaf $\omega_X$. Let $M$ be a coherent sheaf on $X$. Then Serre duality gets us that
\[H_X^i(M)^*\cong H^{d-i}_X(\omega_x\otimes M)\]
for all $i$ (as long as $X$ is Cohen-Macaulay).

But when $X$ is NOT CM, what can we do? Well there is a complex $\Omega_X$ in the bounded derived category of coherent sheaves over $X$
such that (in what is now called Grothendieck duality)
\[H_X^i(M)^*\cong H^{d-i}_X(\Omega_X\otimes M).\]

A triangulated (or derived) category is the place where such an $\Omega_X$ lives.
\begin{defn}
	A \textbf{triangle} in $\cal C$ is a diagram in $\cal C$:
	\[X\to Y\to Z\to\Sigma X.\]
\end{defn}
\begin{defn}
	A \textbf{morphism of triangles} is a commutative diagram
	\begin{center}
		\begin{tikzcd}
			X\ar[r,"u"]\ar[d,"f"]& Y\ar[r,"v"]\ar[d,"g"] & Z\ar[r,"w"]\ar[d,"h"] & \Sigma X\ar[d,"\Sigma f"]\\
			X'\ar[r] & Y'\ar[r] & Z'\ar[r] & \Sigma X'
		\end{tikzcd}
	\end{center}
\end{defn}
\begin{lem}
	The collection of all triangles in $\cal C$ with triangle morphisms form a $k$-linear additive category denoted $\Delta(\cal C)$
\end{lem}
\begin{rmk}
	By forgetting all the information except the $X\to Y$ (resp. just $X$) we can define a forgetful
	functor $\Delta(\cal C)\to\mathbf{Mor}(\cal C)$ (resp. $\Delta(\cal C)\to\cal C$). Our goal is to define a map in 
	the reverse direction (an adjoint?).
\end{rmk}
\begin{defn}
	The \textbf{rotation} of a triangle $X\xrightarrow{u} Y\to Z\to \Sigma X$ is $Y\to Z\to \Sigma X\xrightarrow{-\Sigma u}\Sigma Y.$
	We may be using $\mathscr R(\Delta)$ for the rotation of $\Delta$.
\end{defn}

\begin{defn}
	Let $\cal C$ be a $k$-linear additive category with a \textbf{suspension} or \textbf{shift} functor (auto-equivalence, actually)
	\[\Sigma:\cal C\to\cal C.\]

	Then $({\cal C}, \Sigma ,D)$ is \textbf{pre-triangulated} category where $D$ is a full, nonempty subcategory $D\subseteq\Delta(\cal C)$
	with shift functor $\Sigma$ of $D$ and we have the following axioms:
	\begin{itemize}
		\item (\textbf{TR0}) $X\xrightarrow{\id} X\to 0$ is in $D$ for each $X\in \cal C$ and furthermore $D$ is closed under both shifts and triangle isomorphisms.
		\item (\textbf{TR1}) [\textit{Mapping Cone Axiom}] For any $f:X\to Y$ in $\cal C$, there is a triangle
		\[X\xrightarrow{f} Y\to Z\to\Sigma X\]
		\item (\textbf{TR2}) [\textit{Rotation Axiom}] If $F\in D$, then $\mathscr R(F),\mathscr R^{-1}(F)\in D$
		\item (\textbf{TR3}) [\textit{Morphism Axiom}] Given two triangles 
		\begin{center}
			\begin{tikzcd}
				X\ar[r]\ar[d,"f"] & Y\ar[r]\ar[d,"g"] & Z\ar[r] & \Sigma X\ar[d,"\Sigma f"]\\
				X'\ar[r] & Y'\ar[r] & Z'\ar[r] & \Sigma X'
			\end{tikzcd}
		\end{center}
		with maps $f$ and $g$, there exists an $h:Z\to Z'$. I am unclear about whether the resulting diagram should commute.
	\end{itemize}
\end{defn}
\begin{defn}
	A triple $(\cal C,\Sigma, D)$ is a \textbf{triangulated category} if it is pre-triangulated and, in addition,
	satisfies the axiom \textbf{TR4} (Verdier/octahedral axiom):

	Suppose we are given three triangles: $X\xrightarrow{u}Y\to Z'\to\Sigma X$, $Y\xrightarrow{v}Z\to X'\to\Sigma Y$
	and $X\xrightarrow{v\circ u} Z\to Y'\to\Sigma X$. Then there is a triangle $Z'\to Y'\to X'\to\Sigma Z'$ such that
	\begin{center}
		\begin{tikzcd}
			X\ar[dr,"u"]\ar[rr,"v\circ u",bend left] && Z\ar[rr,bend left]\ar[dr] && X'\ar[rr,dashed,bend left]\ar[dr] && \Sigma Z'\\
			& Y\ar[ur,"v"]\ar[dr] && Y'\ar[ur,dashed]\ar[dr] && \Sigma Y\ar[ur] &\\
			&& Z'\ar[rr,bend right]\ar[ur,dashed] && \Sigma X\ar[ur] &&
		\end{tikzcd}
	\end{center}
\end{defn}
\begin{rmk}
	Notice that the map constructing a triangle for any morphism is not well-defined but it exists. Furthermore
	it is far from being functorial. This is one of the qualms people have with the definition of these categories.
	Grothendieck attempted to fix this in an unpublished paper on ``derivators'' but it is significantly more complicated.
\end{rmk}
\begin{rmk}
	Topologists generally drop the $k$-linearity in $\cal C$. James is unsure if it actually ever comes up.
\end{rmk}
\begin{rmk}
	The model category you should be thinking about is $H^0(\Ch(\modR))$.
\end{rmk}

\section{April 29, 2019}
Recall (notice?) that, given maps $f:X\to X'$ and $g:Y\to Y'$, using the morphism axiom,
we can define a morphism of triangles by just this data.

We have a couple other diagrams equivalent to the octahedral diagram (with slightly different information):
\begin{center}
	\begin{tikzcd}
		X\ar[r,"u"]\ar[d,"\sim"] & Y\ar[r]\ar[d,"v"] & Z'\ar[r]\ar[d,dashed] &\Sigma X\ar[d,"\sim"]\\
		X\ar[r,"v\circ u"] & Z\ar[r]\ar[d] & Y'\ar[r]\ar[d,dashed] &\Sigma X\ar[d,"\Sigma u"]\\
		& X'\ar[r,"\sim"] \ar[d] & X\ar[r]\ar[d,dashed] & \Sigma Y\\
		&\Sigma Y\ar[r] & \Sigma Z' &
	\end{tikzcd}
\end{center}

\begin{center}
	\begin{tikzcd}
		& & X\ar[dl,"u"]\ar[dd,"v\circ u"] &\\
		& Y\ar[dr,"v"]\ar[ddl] & &\\
		& & Z\ar[d]\ar[dr] &\\
		Z'\ar[rr,dashed] & & Y'\ar[r,dashed] & X'
	\end{tikzcd}
\end{center}

\begin{ex}
	Let ${\cal C}=K(\modR)=H^0(\Ch(\modR))$. This is a triangulated category. It is easy to check that $\cal C$ is $k$-linear additive.

	$\Sigma$ is given by shifting of complexes and is an auto-equivalence. $D$ is the full subcategory containing all triangles that are
	isomorphic to $\cone(f)$ where 
	\[X\xrightarrow{f} Y\to \cone(f)\to \Sigma X\]
	is an object such that $\cone(f)^i=(\Sigma X\oplus Y)^i$ and 
	\[\partial_{\cone}:(sm)+(n)\mapsto (-s\partial (m))+(f(m)+\partial(m)).\]
	Notice that there is some collision of notation: $\cone(f)$ can refer either to the complex or the triangle above.

	Then we need to find a chain homotopy to establish that $X\to X\to \cone(\id_X)\to\Sigma X$ is isomorphic to $X\to X\to 0\to\Sigma X$ 
	to show that $D$ satisfies (TR0).

	To prove (TR1), let $u:X\to Y$ be any morphism. Then we have the triangle
	\[X\xrightarrow{u} Y\to\cone(u)\to\Sigma(X)\]
	which gives us the axiom.

	For (TR2): start with a triangle $X\to Y\to Z\to \Sigma X=F$ we want to show that $F\in D$ if and only if $\mathcal{R}(F)\in D$.
	Since all our triangles are isomorphic to the mapping cone, we can just assume that $F$ is the mapping cone of $X\xrightarrow{u}Y$.
	To show $\mathcal{R}(F)$ is in $D$, we want to show that 
	\[(Y\xrightarrow{i}\cone(u)\to\Sigma X\to \Sigma Y)\cong (Y\xrightarrow{i}\cone(u)\to\cone(i)\to \Sigma Y)\]
	The idea here is that you notice that $\cone(\id_Y)=\Sigma Y\oplus Y$ is nullhomotopic. Then you get your copy of $\Sigma X$
	since $\cone(i)=\Sigma Y\oplus(\Sigma X\oplus Y)$ (with the appropriate differential).
	
	For (TR3), if we have $a:X\to X'$ and $b:Y\to Y'$ and $u:X\to Y$ and $u':X'\to Y'$, then we can define the 
	map $\Sigma a+b:\cone(u)\to\cone(u')$ by 
	\[\Sigma a+b(sm+n)=sa(m)+b(n).\]
	This map is natural and makes the diagram commute so we're golden.

	(TR4) is a bit of a pain. But given the triangles 
	\[X\xrightarrow{u}Y\to Z'\to \Sigma X,\quad X\xrightarrow{v\circ u}Z\to Y'\to \Sigma X,\quad Y\xrightarrow{v}Z\to X'\to\Sigma Y\]
	we can again assume that these are mapping cones. That this works is slightly less trivial because of the dimensionality of the 
	diagrams necessary, but can be done. 

	So here we assume $Z'=\cone(u)$, $Y'=\cone(v\circ u)$ and $X'=\cone(v)$. This gives us the diagram
	\begin{center}
		\begin{tikzcd}
			x\ar[r,"u"]\ar[d,"\sim"] & Y\ar[r]\ar[d,"v"] & \cone(u)\ar[r]\ar[d,dashed,"\id_{\Sigma X}\oplus v"] &\Sigma X\ar[d,"\sim"]\\
			X\ar[r,"v\circ u"] & Z\ar[r]\ar[d] & \cone(v\circ u)\ar[r]\ar[d,dashed,"\Sigma u+\id_Z"] &\Sigma X\ar[d,"\Sigma u"]\\
			& \cone(v)\ar[r,"\sim"] \ar[d] & \cone(v)\ar[r]\ar[d,dashed,"\pi_{\Sigma Y}"] & \Sigma Y\\
			&\Sigma Y\ar[r] & \Sigma Z'=\Sigma^2X\oplus \Sigma Y &
		\end{tikzcd}
	\end{center}
\end{ex}
\begin{prob}
	Fill in the details of the above proof for (TR2) (which is surprisingly similar to the ``fake'' proof above.)
\end{prob}

\begin{defn}
	We define $\Ch^b(\modR)$, $\Ch^+(\modR)$ and $\Ch^-(\modR)$ to be the subcategories of $\Ch(\modR)$ consisting 
	of the (totally) bounded, bounded below, and bounded above chains.
\end{defn}

\begin{prop}
	$H^0(\Ch^x(\modR))=K^x(\modR)$ are triangulated categories for $x\in\{b,+,0\}$. 
\end{prop}
\begin{rmk}
	People refer to these as 
	subcategories. Even though this isn't strictly true, it most works out.
\end{rmk}
\begin{rmk}
	It ends up that you can just pop different properties on your modules (e.g. finite dimensional, projective injective, flat)
	are all preserved by your shift functor and other operations, so these are also examples. ``It works with any 
	property you can think of.'' --James
\end{rmk}
\subsection{Properties of Triangulated Categories}
\begin{prop}
	IF $X\xrightarrow{u}Y\xrightarrow{v} Z\xrightarrow{w}\Sigma X\in D$ then $v\circ u=0$.
\end{prop}
\begin{prf}
	Consider the diagram
	\begin{center}
		\begin{tikzcd}
			X\ar[r,"\id_X"]\ar[d,"\id_X"] & X\ar[r]\ar[d,"u"] & 0\ar[d,dashed,"0"]\ar[r] & \Sigma X\ar[d,"\Sigma(\id_X)"]\\
			X\ar[r,"u"] & Y\ar[r,"v"] & Z\ar[r] &\Sigma X
		\end{tikzcd}
	\end{center}
	where both the above triangles are in $D$, so we can apply (TR3) to get the existence of the dashed map above, 
	and then by commutativity $v\circ u=0.$
\end{prf}
\begin{prop}
	If $(\cal C,\Sigma, D)$ is a triangulated category, then so is $(\cal{C}^{op},\Sigma^{-1},D^{op})$.
\end{prop}

\section{May 4, 2019}
Missed this day!

\section{May 6, 2019}
Today we will discuss 3 kinds of localizations. We will construct the derived category $D(\modR)$ via localization.

\subsection{Localization}
\begin{defn}
	Let $\cal C$ be a category. Let $S$ be a collection of morphisms in $\cal C$. The localization of $\cal C$ with respect
	to $S$ is a category $S^{-1}\cal C$ together with a quotient map $\pi:\cal{C}\to S^{-1}\cal C$ such that 
	\begin{enumerate}
		\item The objects in $S^{-1}\cal C$ are the same as those in $\cal C$.
		\item For any $s\in S$, $\pi(s)$ is invertible (an isomorphism).
		\item For any functor $F:\cal{C}\to\cal D$ such that $F(s)$ is invertible for all $s\in S$, $F$ factors through $S^{-1}\cal C$.
	\end{enumerate}
\end{defn}
\begin{ex}
	If $G$ is a semigroup and $\cal C$ is a one-object category whose morphisms are the elements in $G$, then if 
	we let $S=G$, then $S^{-1}\cal C$ is the category with one object whose morphisms are $\langle G\rangle$.
\end{ex}
\begin{ex}
	If $\cal C$ is the one-object category for $R$, a $k$-algebra, then we can pick a (multiplicative) subset $S$ of $R$ 
	to get $S^{-1}\cal C$ is essentially $R_S$.
\end{ex}
\begin{ex}
	$K(\modR)$ is a localization of $\Ch(\modR)$.
\end{ex}
\begin{rmk}
	Ignoring set-theoretic issues, $S^{-1}\cal C$ always exists. If we want to fix any such problems, we 
	can just work over a (locally) small category. Generally if $S$ is a set we are okay, but we will see 
	soon that we will want to include all our identity maps! So we had better have a small $\cal C$.
\end{rmk}

\subsection{Ore Localization}
\begin{defn}
	A collection $S$ of morphisms in $\cal C$ is called a \textbf{multiplicative system} 
	if
	\begin{itemize}
		\item $S$ is closed under composition and contains $\id_X$ for all $X\in\cal C$.
		\item (\textbf{Right Ore Condition}) If $t:Z\to Y$ is in $S$ and $g:X\to Y$ in $\cal C$,
		then there are maps $f$ and $s$ and a commutative diagram
		\begin{center}
			\begin{tikzcd}
				W\ar[r,"f",dashed]\ar[d,"s",dashed] & Z\ar[d,"t"]\\
				X\ar[r,"g"] & Y
			\end{tikzcd}
		\end{center}
		such that $s\in S$.
		\item (\textbf{Right Ore Condition}) $\mathcal{C}^{op}$ has the right Ore condition.
		\item (\textbf{Cancellation}) If $f,g:X\to Y$ are two morphisms, then the following are equivalent:
		\begin{itemize}
			\item $sf=sg$ for some $s:Y\to Z$ in $S$
			\item $ft=gt$ for some $t:V\to X$ in $S$
		\end{itemize}
	\end{itemize}
\end{defn}
\begin{defn}
	If $S$ is a multiplicative system, then $S^{-1}\cal C$ is called an \textbf{Ore localization} of $\cal C$.
\end{defn}
\begin{thm}
	Let $S$ be a multiplicative system in $\cal C$. Then the $\Hom$ in $S^{-1}\cal C$ can be described 
	as
	\[\Hom_{S^{-1}\cal C}(X,Y)=\{X\xleftarrow{s}Z\xrightarrow{f}Y| s\in S\}/\sim=\{(s,f)|s\in S\}/\sim\]
	where the equivalence $\sim$ is defined as follows: $(s_1,f_1)\sim(s_2,f_2)$ if there exists $(s_3,f_3)$
	and $a_1,a_2\in S$ such that the following commutes:
	\begin{center}
		\begin{tikzcd}
			& Z_1\ar[dl,"s_1"]\ar[dr,"f_1"] &\\
			X & Z_3\ar[u,"a_1"]\ar[d,"a_2"]\ar[l,"s_3"]\ar[r,"f_3"] & Y\\
			& Z_2\ar[ul,"s_2"]\ar[ur,"f_2"] &
		\end{tikzcd}
	\end{center}
	In this case, the equivalence class of $(s,f)$ is denoted $[fs^{-1}]$ or $[f/s]$.
\end{thm}
\begin{rmk}
	Dually we can characterize $\Hom$ using diagrams $X\xrightarrow{g} Z\xleftarrow{t} Y$ for $t\in S$. These are
	equivalent but we have to use the left Ore condition instead of right to do composition.
\end{rmk}

So then we get a functor $\pi:\mathcal{C}\to S^{-1}\mathcal{C}$ that sends $X\mapsto X$ and $(f:X\to Y)\mapsto (X\xleftarrow{\id_X}X\xrightarrow{f}Y)$.
Composition is just what you'd expect, using the ore conditions. Checking that composition is well-defined (modulo the equivalence relation)
seems like a bit of a chore, but maybe worth checking.

\subsection{Verdier Localization}
Let $(\mathcal{C}, \Sigma, D)$ be a triangulated category. let $\cal T$ be a full triangulated subcategory of $\cal C$.
Then $\mathcal{C}/\mathcal{T}$ is the category such that $X\simeq 0$ for all $X\in\mathcal{T}$.

Recall
\begin{prop}
	If $X\xrightarrow{f} Y\to Z\to\Sigma X$ is a triangle, $f$ is an isomorphism if and only if $Z\simeq 0$.
\end{prop}

Set $S_{\cal T}$ to be $\{f|\cone(f)\in\mathcal{T}\}$. Then 
\begin{lem}
	$\Sigma S_{\mathcal{T}}=S_{\mathcal{T}}$.
\end{lem}
\begin{lem}
	$S_{\mathcal{T}}$ is a multiplicative system.
\end{lem}
\begin{thm}
	$S_{\mathcal{T}}^{-1}\mathcal{C}$ is a triangulated category.
\end{thm}
\begin{rmk}
	James may or may not prove this on Wednesday. Some sketches: $\Sigma_{S^{-1}\cal C}$ sends $X\to\Sigma X$ and $[f/s]$ to $[(\Sigma f)/(\Sigma s)]$.

	Then $D_{S^{-1}\cal C}$ is the set of all triangles in $\Delta(S^{-1}\cal C)$ that are isomorphic to some $X\to Y\to Z\to\Sigma X$ in $D(\mathcal{C})$.
\end{rmk}

\section{May 8, 2019}
Today we will discuss examples of Verdier localization. First some motivation for it: in commutative algebra or in algebraic 
geometry, there is always a concept of localization and this (along with ``local-global principles'') gives us a nice 
way to get a handle on things.

In a more general case, we no longer have localization! So we don't have hope of chopping up things into local pieces.
So instead we can move to a triangulated category! So if we are trying to study $\modR$ for some algebra $R$,
we can instead consider $D(\modR)$ where we can localize by any subcategory we choose!

\subsection{Examples of Verdier Localization}
As a convention, let $(\mathcal{C},\Sigma,D)$ be a triangulated category and let $\mathcal{T}$ be a full triangulated subcategory
such that if $Y\in\mathcal T$ and $X\cong Y$, then $X\in\mathcal T$ ($\mathcal T$ is closed under isomorphism). Furthermore, $\mathcal T$ is 
triangulated via $(\mathcal{T}, \Sigma|_{\mathcal T}, D\cap\Delta(\mathcal{T}))$.

Now let $P$ be some property. First, we will say $P$ is a property on objects.

\begin{ex}
	Let $\mathcal C=K(\modR)$. Let $P(C)$ mean that $C$ is a (bounded/bounded above/bounded below) complex of 
	(projective/injective/flat/f.g./some mixture). Then take $\mathcal T=\{X|\exists Y: Y\cong X,P(Y)\}$. We need to check that $P$ is 
	stable under suspension and that the set of $P$-objects is closed under cone constructions.

	In this case since the cone is just a coproduct, it suffices to show that this preserved under coproducts.
\end{ex}

Next $P$ is a property on morphisms:
\begin{ex}
	Now let $X$ be a $P$ object if $\Hom_{\calC}(\Sigma^nX,-)$ commutes with coproducts for all $n\in\bbZ$. Alternatively, we could 
	let the $P$ objects be those $Y$ such that $\Hom_\calC(\Sigma^nX,Y)=0$ for all $n\in\bbZ$ for some fixed $X$.

	Then we can take the objects of $\calT$ to be the objects in $\calC$ isomorphic to a $P$ object. Once one proves that these  
	objects are stable under suspension and cone constructions, $\calC/\calT$ exists.
\end{ex}
\begin{lem}
	$\Hom_{\Ch(\modR)}(R,X)=X$.
\end{lem}
\begin{prf}
	Look at the $i^{th}$ component of each:
	\[\Hom_{\Ch(\modR)}(R,X)^i=\prod_{s\in\bbZ}\Hom_R(R^s,X^{s+i})=\Hom_R(R^0,X^i)=X^i\]
	It remains to show the differentials are the same, but it can be shown (HW).
\end{prf}
\begin{lem}
	If $X$ and $Y$ are acyclic complexes and $X\xrightarrow{f}Y\to Z\to\Sigma X$ is a triangle, then $Z$ is acyclic.
\end{lem}
\begin{rmk}
	To prove this, one applies $\Hom_\calC(\Sigma^n R,-)$ to the triangle and uses that $\Hom_\calC(\Sigma^n R,\Sigma X)=\Hom_{\calC}(\Sigma^{n-1}R,X)=0$.
\end{rmk}
\begin{ex}
	Let $\calC=K(\modR)$. Let the $P$ objects to be the $X$ such that $\Hom_\calC(\Sigma^n R,X)=0$. Now notice 
	\[\Hom_\calC(\Sigma^0R,X)=\Hom_{H^0(\Ch(\modR))}(R,X)=H^0(\Hom_{\Ch(\modR)}(R,X))=H^0(X)\]
	and (a claim) $\Hom_{\calC}(\Sigma^nR,X)=H^n(X)=0$.

	Thus $X$ has $P$ if and only if $H^n(X)=0$ for all $n$ -- that is, if $X$ is \textbf{acyclic.}
\end{ex}
\begin{defn}
	The \textbf{derived category of complexes of right $R$ modules} is 
	\[D(\modR):=K(\modR)/K_{\text{acyclic}}(\modR)\]
\end{defn}
\begin{rmk}
	There are some useful subcategories $D^b(\modR)$, the bounded derived category, as well as $D^+(\modR)$ and $D^-(\modR)$,
	where are the bounded (resp.) below and above complexes. Here one applies the property to the homotopy category before taking the quotient. 
	One also considers $D_{f.g.}^b(\modR)$ where one puts the finite generation 
	condition on $\modR$ first, then takes the quotient of the appropriate homotopy (sub) categories.
\end{rmk}

\begin{ex}
	Let $\scrD=D^b_{f.d.}(\modR)$ and let $P$ be the complexes of finitely generated projective modules (i.e. \textbf{perfect complexes}).
	Then $\calT_P:=D_{per}^b(\modR)$, a sub triangulated category of $\scrD$. Then 
	\[D_{sing}(\modR):=D^b_{f.g.}(\modR)/D_{per}^b(\modR)\]
	is called the \textbf{singular category} and corresponds to the singularities of $\operatorname{spec} R$.
\end{ex}
\begin{rmk}
	If $\gldim R <\infty$ and $R$ is Noetherian, then $D_{sing}(R)=0$.
\end{rmk}

\section{May 10, 2019}
Today will be largely random remarks. There will continue to be holes in proofs. :)

Last time, we let $R$ be an algebra and considered the derived category $D(\modR)$, which is a Verdier localization of $K(\modR)$
or (somehow equivalently) a quotient. I think he is playing a bit fast and loose here.

Note that the mapping cones (the triangles) are precisely the quasi-isomorphisms. This can be seen by applying $H^0=\Hom_K(R,-)$
to $\Sigma^{-1}\cone(f)\to X\xrightarrow{f} Y\to \cone(f)$
\begin{rmk}
	If $S$ is the set of quasi-isomorphisms in $\Ch(\modR)$, then 
	\[S^{-1}\Ch(\modR)\cong D(\modR),\] 
	but this is just (for now?) as $k$-linear categories, since $\Ch(\modR)$ is not triangulated (it's abelian!).
\end{rmk}
\begin{rmk}
	There exists a fully faithful embedding $\modR$ into $D(\modR)$ where we have $\Hom_R(M,N)=\Hom_{D(\modR)}(M,N)$.
\end{rmk}
\begin{rmk}
	$\Ext_R^i(M,N)=\Hom_{D(\modR)}(M,\Sigma^iN)$. This can be useful for computations, but recall that composition in the Verdier localization is 
	all about them roof constructions.
\end{rmk}
\begin{rmk}
	Let $P_\bullet\to N$ be a projective resolution. Then $P_\bullet$ is quasi-isomorphic to $N$, so they become (honestly) isomorphic in $D(\modR)$
	even though $P_\bullet\not\cong N$ in $K(\modR)$.
\end{rmk}
\begin{rmk}
	Suppose $X$ is a bounded-above complex of projectives. Then 
	\[\Hom_{D(\modR)}(X,Y)=\{X\to Y\}/\sim = \Hom_{K(\modR)}(X,Y)=H^0(\Hom_{\Ch(\modR)}(X,Y)).\]
	This motivates somewhat why we often want to have projective resolutions of our objects -- they become more computationally feasible 
	in this case.
\end{rmk}
\begin{rmk}
	$K^{-}(\mathbf{projmod}\text{-}R)\cong D^-(\modR)$. In other words, (recall we have a fully faithful embedding of $\modR$ into $D(\modR)$)
	there is a fully faithful embedding of $\modR$ into $K^-(\mathbf{projmod}\text{-}R)$ sending each $M$ to 
	a projective resolution of $M$.

	Dually, $D^+(\modR)$ is equivalent to $K^+(\mathbf{projmod}\text{-}R)$, so $\modR$ embeds fully faithfully into it.
\end{rmk}
\begin{rmk}
	In particular, $D^b(\mod-R)$ embeds into either $K^+(\mathbf{injmod}\text{-}R)$ or $K^-(\mathbf{injmod}\text{-}R)$.
\end{rmk}

\subsection{Derived Functors}
Consider algebras $A$ and $B$. Let $F$ be an additive $k$-linear functor from $A$ modules to $B$ modules. This easily extends 
to a functor between the chain and homotopy categories of these categories. But \textbf{there is not a clear extension of $F$ to the derived category.}

Consider the composition of maps (and embeddings):
\[D^-(\mathbf{mod}\text{-}A)\hookrightarrow K^-(\mathbf{projmod}\text{-}A)\xrightarrow{F} K^-(\mathbf{mod}\text{-}B)\xrightarrow{\nu} D^-(\mathbf{mod}\text{-}B)\]
where $\nu$ is the Verdier localization.

Then the composition of these maps is called the \textbf{left derived functor of $F$}, which is written $LF$.
\begin{ex}
	Let $F=-\otimes_R N:\modR\to\Vectk$. Then $H^i(LF(M))=\Tor_{-i}^A(M,N)$.
\end{ex}
Next week we are going to discuss dualizing complexes. From these you can prove some very nice results 
that otherwise require some complicated argument.

\section{May 13, 2019}
\subsection{Dualizing Complexes}
Today we will start with definitions and an introduction to the idea.

Grothendieck introduced dualizing complexes (over schemes) in the 1960's. Using these complexes, 
he proved what was called ``Grothendieck duality'', a generalization of Serre duality. In 1990(ish)
Yekutieli introduced them in the noncommutative case.

This is a powerful tool in noncommutative/homological algebra. Even the existence of dualizing 
complexes has many consequences:
\begin{thm}
	Take any simple factor ring of the universal enveloping algebra $\calU(\frakg)$ ($\frakg$ finite dimensional) is Auslander-Gorenstein and Cohen-MacCaulay.
\end{thm}
\begin{thm}[Bass Theorem (for Hopf algebras)]
	Let $H$ be a Noetherian Hopf algebra satisfying a polynomial identity. let $M$ be a finitely generated (left) $H$-module.
	Then $\injdim_H M$ is either infinite or equal to $\injdim_H H$.
\end{thm}
\begin{thm}[Auslander-Buchsbaum]
	Let $A$ be a local Noetherian algebra with balanced dualizing complex. Let $M$ be a finitely generated (left) $A$-module.
	Then $\projdim M$ is either infinite or is $\operatorname{depth}(A)-\operatorname{depth}(M)\le \operatorname{depth}(A)$.
\end{thm}

\subsection{Conventions}
We will be working a lot in $\Algk,\modR,$ and $\Ch$ over the next couple weeks. We should fix 
some conventions to work with.

\begin{itemize}
	\item Let $A,B,C$ be Noetherian $k$-algebras.
	\item Usually $R$ will stand for a dualizing complex (Except directly below where my notation varied).
	\item Let $M$ and $N$ be left (or right) modules over $A$.
	\item Let $X,Y,Z$ be complexes.
	\item The category of $(A,B)$-bimodules is identified with left $A\otimes B^{op}$ modules.
	\item Usually we work with left $A$-modules. To study the right $A$-modules, we study the left $A^{op}$ modules.
	\item We want to identify $\Db(\Rmod)$ with a subcategory of $\D(\modR)$, so notice
	\[\Db(\Rmod)\cong \{X\in \D(\modR)|X\simeq\text{ some object in } \Db(\Rmod)\}\]
	where the isomorphism there is on the level of triangulated categories.
\end{itemize}
\begin{prop}
	If $X\simeq Y$ for some bounded complex, then $H^n(X)=0$ for all $n>>0$. The converse is also true.
\end{prop}

This is a slightly nonstandard definition of projective dimension, but it is the one we will use in this class:
\begin{defn}
	Let $X\in\Db(\Rmod).$ Then 
	\[\projdim X=\max\{i|\Ext_R^i(X,M)\ne 0,\text{ some } M\in\Rmod\}\]
\end{defn}
The upshot here is
\begin{lem}
	$\projdim \Sigma X=\projdim X+1$ and therefore $\projdim\Sigma^n X=\projdim X+n.$
\end{lem}
\begin{lem}
	Let $d=\projdim X<\infty$. Then 
	\[X\simeq (\cdots 0\to P^{-d}\to\cdots\to P^s\to 0)\]
	with all $P^i$ projective.
\end{lem}

Then define $\injdim X$ dually. 

Recall that for bounded complexes $X$ and $Y$ of $R$-modules, 
\[R\Hom_R(X,Y):=\Hom_{\Ch(\Rmod)}(X,I_Y)=\Hom_{\Ch(\Rmod)}(P_X,Y)\in \D(\Vectk)\]
\begin{lem}
	Let $M$ be a left $R=A\otimes B^{op}$ module.
	\begin{enumerate}
		\item If $M$ is free over $R$, then it is free over both $A$ and $B$.
		\item If $M$ is projective over $R$, then it is projective over both $A$ and $B$.
		\item If $M$ is flat over $R$, then it is flat over both $A$ and $B$.
		\item If $M$ is injective over $R$, then it is injective over both $A$ and $B$.
	\end{enumerate}
\end{lem}
\begin{prf}
	(a): If $M\cong (A\otimes B^{op})^n$, then $A\otimes B^{op}\cong A^{\dim B^{op}}$, so $M\cong A^{n\dim B^{op}}$.

	(b): pretty easy.

	(c) and (d): a little more tough. Use adjoints.
\end{prf}
\begin{lem}
	Let $M$ and $N$ be $(A,B)$-bimodules. Then $\Hom_A(M,N)$ is a $(B,B)$-bimodule. Furthermore, $\Hom_{B^{op}}(M,N)$ is an $(A,A)$-bimodule.
\end{lem}
\begin{lem}
	Let $X,Y$ be complexes of $(A,B)$-bimodules. Then $\Hom_{\Ch(\Amod)}\in\Ch(B^e\text{-}\mathbf{mod})$.
	Also flip it and reverse it.
\end{lem}
\begin{lem}
	Let $X$ and $Y$ be complexes of $(A,B)$-bimodules. Then $R\Hom_A(X,Y)\in \D(B\otimes B^{op}\text{-}\mathbf{mod})$ and flip it.
\end{lem}

\subsection{Dualizing Complexes}
\begin{defn}
	Let $A$ and $B$ be two $k$-algebras and let $R$ be a complex in $\Db(A\otimes B^{op}\text{-}\mathbf{mod})$.
	Then $R$ is called a \textbf{dualizing complex over $(A,B)$} if 
	\begin{itemize}
		\item $H^i(R)\in A\otimes B^{op}\text{-}\mathbf{mod}$ is finitely generated on both sides.
		\item $\injdim_AR,\injdim_BR<\infty$.
		\item The natural maps $A\xrightarrow{\Phi_A}R\Hom_B^{op}(R,R)$ and $B\xrightarrow{\Phi_B} R\Hom_A(R,R)$
		are isomorphisms in $\D(A^e\text{-}\mathbf{mod})$ and $\D(B^e\text{-}\mathbf{mod})$, respectively.
	\end{itemize}
\end{defn}
\begin{rmk}
	The map Recall that $R\Hom_{B^{op}}(R,R)=\Hom_{\Ch(B^{op}\text{-}\mathbf{mod})}(P_R,P_R)$. Then $\Phi_A$ sends 
	$a\mapsto l_a:P_R\to P_R$.

	One can verify this more easily when $A=B=R=k$.
\end{rmk}

\section{May 17. 2019}
I missed a day but they talked about dualizing complexes some more and what they imply.

Let's do an example of the day:
\begin{ex}
	Let $A$ be a finite dimensional algebra over $k$. Then $R=A^\ast=\Hom_k(A,k)$ is finite dimensional over $k$. Then we claim that $R$
	is a dualizing complex over $(A,A)$. To see this, we must check:
	\begin{itemize}
		\item $H^i(R)=A^\ast$ for $i=0$ and is zero otherwise.
		\item $\injdim_AR=\injdim_AA^\ast=0$
		\item $R\Hom_A(R,R)=Hom_A(A^\ast,A^\ast)=\Hom_A(A,A)=A$.
	\end{itemize}
\end{ex}

\subsection{Proving the Dualizing Theorem}
\begin{defn}
	$R$ is a dualizing complex over $(A,B)$ if we have functors 
	\[\bbD=R\Hom_A(-,R):\Db_{fg}(\Amod)\to \Db_{fg}(\mathbf{mod}\text{-}B)\]
	and $\bbD^\circ=R\Hom_{B^{op}}(-,R)$ in the other direction.
\end{defn}
\begin{lem}
	Let $W$ be a complex of $(A,B)$ bimodules and let $X$ be a complex of $A$-modules and $Y$ a complex of $B^{op}$-modules. Then 
	\[\Hom_{\Ch(\Amod)}(X,\Hom_{\Ch(B\text{-}\mathbf{mod})}(Y,W)\cong\Hom_{\Ch(\mathbf{mod}\text{-}B)}(Y,\Hom_{\Ch(\Amod)}(X,W))\]
\end{lem}
\begin{thm}[Dualizing Theorem]\label{thm-dualizing}
	$(\bbD,\bbD^\circ)$ induces a contravariant equivalence between $\Db_{fg}(\Amod)$ and $\Db_{fg}(\mathbf{mod}\text{-}B)$
\end{thm}
\begin{rmk}
	We are looking to construct a natural isomorphism between the identity functor on $\Db_{fg}(\Amod)$ to $\bbD^\circ\bbD$.
\end{rmk}
\begin{lem}
	Let $W$ be a complex of $(A,B)$-bimodules and let $X$ and $Y$ be chains in $\Amod$ and $\mathbf{mod}\text{-}B$, respectively. Then 
	\[R\Hom_A(X,R\Hom_{B^{op}}(Y,W))\cong R\Hom_{B^{op}}(Y,R\Hom_A(X,W)).\]
\end{lem}
\begin{lem}
	Under the same conditions as above,
	\[\Hom_{\D(\Amod)}(X,R\Hom_{B^{op}}(Y,W))\cong\Hom_{\D(\mathbf{mod}\text{-}B)}(Y,R\Hom_A(X,W))\]
\end{lem}
\begin{cor}
	$R\Hom_{B^{op}}(-,W)$ and $R\Hom_A(-.W)$ is an adjoint pair. This gives us the map $\eta:1\to\bbD^\circ\bbD$.
\end{cor}
\begin{lem}
	Let $F$ and $G$ be two triangulated functors from $\calT_1\to\calT_2$ and $\eta:F\to G$ be a natural transformation. Then 
	if we let $S=\{X\in\calT|\eta_X:F(X)\to G(X)\text{ is an isomorphism}\}$, then $S$ is a triangulated subcategory.
\end{lem}
\begin{cor}
	It is a fact that the triangulated subcategory of $\D(\Amod)$ generated by $A$ is $\D_{perf}(\Amod)$. Thus, $\D_{perf}(]Amod)\subseteq S$.
	Thus in particular $\eta_X:X\mapsto\bbD^\circ\bbD(X)$ is an isomorphism for all $X\in\D_{perf}(\Amod)$.
\end{cor}
\begin{cor}
	$\bbD$ and $\bbD^\circ$ induce a contravariant equivalence between $\D_{perf}(\Amod)$ and $\D_{perf}(\mathbf{mod}\text{-}B).$
\end{cor}

\begin{defn}
	Let $X$ be a complex of $A$-modules. Then define $X_{[a,b]}$ to be 
	\[X^i_{[a,b]}=X^i\]
	whenever $i\in[a,b]$ and zero otherwise. Same for the differentials. This is a complex (obvious).
\end{defn}

\begin{lem}
\begin{itemize}
	\item If $X_{[b,\infty)}=0$, then there exists an ``embedding'' morphism $X_{[a,b]}\to X$.
	\item If $X_{(-\infty,a]}=0$ then there exists a ``projection'' morphism $X\to X_{[a,b]}$.
\end{itemize}
\end{lem}
\begin{lem}
	let $X$ be  a complex over $A$ such that $X=X_{[-b,b]}$. Let $Y$ and $Z$ be two complexes such that 
	\[Y_{[-c,c]}=Z_{[-c,c]}.\]
	Then 
	\[\Hom_{\Ch(\Amod)}(X,Y)_{[-(c-b),c-b]}=\Hom_{\Ch(\Amod)}(X,Z)_{[-(c-b),c-b]}\]

	This still holds if you switch the arguments in $\Hom$.
\end{lem}
\begin{cor}
	If $X$ and $Y$ ``agree locally'' as above, then $\bbD(Y)$ and $\bbD(Z)$ agree locally. This is slightly confusing since 
	we're working in the derived category here, but this means that they have isomorphic cohomology in some range.
\end{cor}

\begin{cor}
	$\bbD^\circ\bbD(Y)$ and $\bbD^\circ\bbD(Z)$ agree locally.
\end{cor}

\begin{prf}[of~\ref{thm-dualizing}]
	We need to show that $H^i(P)\cong H^i(\bbD^\circ\bbD(P))$ for all $i$. Fix $i$.

	Then call $Q=P_{[-c,c]}$ for $c>> i+2b$ and then $H^i(Q)\cong H^i(P)$ and looking at the naturality square everything is an isomoprhism and we're done.
\end{prf}

\section{May 20, 2019}
Today we will discuss homological identities.

\begin{defn}
	If $A$ is an algebra over $k$, it is called \textbf{local} if $A/J(A)\cong k$, where $J(A)$ is the Jacobson radical.
\end{defn}

\begin{thm}[Bass]
	Let $A$ be a local Noetherian algebra with balanced dualizing complex. Let $M$ be a finitely genreated left $A$-module with finite injective 
	dimensions. Then 
	\[\injdim M=\operatorname{depth}A<\infty.\]
\end{thm}
\begin{rmk}
	We haven't yet defined a balanced dualizing complex but we will. Later we will also see some theorems regarding 
	existence of such complexes.
\end{rmk}

\subsection{(Non)Example of the Day}
\begin{ex}
	There is an Artinian algebra $A$ such that for any algebra $B$, there is \textbf{no} dualizing complex over $(A,B)$. The construction 
	is quite complicated, but it works. Note that in particular $A$ cannot be finite dimensional since we saw that 
	the dual of $A$  is a dualizing complex for $A$ when it is.
\end{ex}
\begin{ex}
	There is an Artinian algebra $A$ such that
	\begin{itemize}
		\item There is no dualizing complex over $(A,A)$
		\item There is a dualizing complex over $(A,B)$ for some $B$.
	\end{itemize}
\end{ex}
\begin{ex}
	More concretely, let $A=(\begin{smallmatrix}k&k\\0 &k\end{smallmatrix})$ for some field $k$. This is not local since $A/J(A)\cong k\oplus k$.
	Furthermore, $\gldim A=1$. Now let $e_{11}$ and $e_{22}$ refer to the standard $2x2$ matrices with ones in the appropriate spots and zeros elsewhere.

	Then if $S_1=Ae_{11}$ and $S_2=\binom{0}{k}$, then we get a sequence 
	\[0\to S_1\to\binom{k}{k}\to S_2\to 0\]
	and so $\projdim S_1=0=\injdim S_2$ and $\projdim S_2=1=\injdim S_1$, giving us a counterexample when we don't have locality.
\end{ex}
\begin{ex}
	More simply, if $A=k[[t]]$, $\projdim A=0\ne1\projdim k.$
\end{ex}

\subsection{Some Review}
Let $X$ be a bounded (this can be defined more generally but James' definiton requires this) complex of left $A$-modules.
\begin{defn}
	The projective dimension is defined to be 
	\[\projdim X=\sup\{i|\Ext^i(X,M)\ne 0,\text{ some M}\in\Amod\}\]
	or equivalently
	\[\projdim X=\inf_P\sup\{i|P^{-i}\ne 0\}\]
	where $P$ ranges over all projective resolutions of $X$.
\end{defn}
\begin{defn}
	The injective dimension is defined to be 
	\[\injdim X=\sup\{i|\Ext^i(M,X)\ne 0,\text{ some M}\in\Amod\}\]
	or equivalently
	\[\injdim X=\inf_I\sup\{i|I^{-i}\ne 0\}\]
	where $I$ ranges over all injective resolutions of $X$.
\end{defn}

Now let $A$ be local and write ${_A}k=A/J(A)$.
\begin{defn}
	Let $X$ be a bounded complex of $A$ modules. Then the \textbf{depth of $X$} is
	\[\operatorname{depth}X=\inf\{i|\Ext^i(k,X)\ne 0\}\]
\end{defn}
\begin{rmk}
	Notice that, in general, $\depth X\le \injdim X$.
\end{rmk}
\begin{rmk}
	In commutative algebra, the more natural definition is in terms of lengths of regular sequences. This ends up being 
	equivalent but requires a little work.
\end{rmk}
\begin{defn}
	Let $X$ be a bounded complex. Then 
	\[\inf X=\inf\{i|H^i(X)\ne 0\}\]
	and 
	\[\sup X=\sup\{i|H^i(X)\ne 0\}\]
\end{defn}
\begin{rmk}
	So then $\depth X=\inf(R\Hom_A(k,X)).$
\end{rmk}
\begin{lem}
	Let $A$ be a Noetherian local algebra and let $X\in\Dneg_{fg}(\Amod)$. Let $P$ be a minimal free resolution of $X$.
	Then 
	\[\projdim X=\max\{i|P^{-i}\ne 0\}\]
\end{lem}
\begin{prf}
	We always have that $\projdim X$ is less than or equal to this by the definition. So it suffices to show the reverse inequality.
	That is, we want to show that $\projdim X\ge i$ if $P^{-i}\ne 0.$

	Since $P$ is minimal, the complex $\Hom_{\Ch(\Amod)}(P,k)$ has zero differential (since $\partial_P^i$ maps into $J(A)P^{i+1}$).
	But then computing:
	\[\Ext_A^i(X,k)=H^i(\Hom(P,k))=\Hom(P,k)^i=\Hom_A(P^{-i},k)\ne 0\]
	whence we can conclude that $\projdim X\ge i$.
\end{prf}

\subsection{Canonical Truncations}
\begin{defn}
	Let $X$ be a complex in $\Amod$. The \textbf{$n^{th}$ bounded above truncation of $X$ is}
	\[\calT^{\le n}X=\cdots\to X^i\to\cdots\to X^{n-1}\to\ker(\partial_X^n)\to 0\to \cdots\]
	and the \textbf{$n^{th}$ bounded below truncation} is
	\[\calT^{\ge n}X=\cdots 0\to\coker(\partial_X^{n-1})\to X^{n+1}\to X^{n+2}\to\cdots\]
\end{defn}
\begin{thm}
	Let $A$ be a Noetherian local algebra and let $X,Y\in\Db_{fg}(\Amod)$. Suppose $\projdim X<\infty$.
	Then 
	\[\sup(R\Hom_A(X,Y))=\projdim X+\sup Y.\]
\end{thm}
\begin{prf}
	Not enough time for the full thing. So we prove ``$\le$''. Let $P$ be a minimal free resolution of $X$. Then 
	$P=0\to P^{-d}\to\cdots\to P^b\to 0\to\cdots$. Then $d=\projdim X$. 

	Say $Y=0\to Y^a\to\cdots\to Y^s\to 0$, so $\sup Y=s$. But notice $R\Hom(X,Y)=\Hom_{\Ch(\Amod)}(P,Y)$
	so for $i>d+s$,
	\[\Hom(P,Y)^i=\prod_{s\in\bbZ}\Hom_A(P^s,Y^{s+i})=0\]
	so therefore $\sup \Hom(P,Y)\le d+s$.
\end{prf}
\begin{cor}
	Let $Y$ be a finitely generated module and $X$ a module with finite projective dimension. Then 
	\[\projdim X=\sup(R\Hom_A(M,N)).\]
\end{cor}

\section{May 22, 2019}
\subsection{Example of the day}
This is a combination of some results of Artin, Van den Bergh, Yehutieli, Wu, Chan, and many others.

Suppose $A$ is a Noetherian algebra satisfying the following conditions:
\begin{itemize}
	\item $A$ is either connected graded or complete local
	\item $A$ is a finite module over it center.
\end{itemize}
Then there exists a dualizing complex $R$ over $(A,A)$, such that 
\begin{itemize}
	\item $R\Hom_A(k,R)\cong k$ and $R\Hom_{A^{op}}(k,R)\cong k$. This is sometimes called the \textit{pre-balanced condition.}
\end{itemize}

\begin{ex}
	Let $q$ be a root of unity. Let $A=k_q[[x_1,\cdots,x_n]]$ where we say any $x_i$ and $x_j$ $q$-commute, for $i\ne j$. Then this satisfies the hypotheses and $R$ exists.
	There is a method of constructing it via locak cohomology that is perhaps too much for what we have left in our course. We will, however, discuss local 
	cohomology and the $\chi$-condition next time.
\end{ex}

\subsection{Back to the future (Monday)}
We're going to tweak things a little.
\begin{thm}
	Let $A$ be a Noetherian local algebra. Let $X\in\Db_{fg}(\Amod)$ with $\projdim X<\infty$.
	Then there is a $Y\in\Db(\Amod)$ (here's where we weaken slightly) such that $\Hom_A(H^s(Y),k)\ne 0$ where 
	$s=\sup(Y)$. Then 
	\[\sup(R\Hom_A(X,Y))=\projdim X+\sup Y.\]
\end{thm}
\begin{prf}
	Last time we saw ``$\le$''. Now let $d=\projdim X$ and $s=\sup Y$. It suffices to show that
	\[H^{d+s}(R\Hom_A(X,Y))=H^{d+s}(\Hom_{\Ch(\Amod)}(P,Y))\ne 0\]
	where we take $P$ to be any (in this case minimal) free resolution of $X$. But if you consider the $d+s$ morphisms from $P$ to $Y$,
	there is only one map between nonzero things! So 
	\[\Hom_{\Ch(\Amod)}(P,Y)^{d+s}=\Hom_A(P^{-d},Y^s).\]

	We want to construct an $\tilde f\in\Hom_{\Ch}(P,Y)^{d+s}$ (notice it is already in the kernel) such that $\tilde f$ isn't in the image. To see this, 
	set $N=H^s(Y)=Y^s/\Im(\partial_Y^{s-1})\ne 0$ (we can assume $Y^{s+1}=0$). Then by the hypothesis, $\Hom_A(N,k)\ne 0$,
	but this is equal to 
	\[\Hom_A(N,A/J(A))=\Hom_k(A/J(A)\otimes N, A/J(A))\ne 0.\]

	This implies that $A/J(A)\otimes N\ne 0$ or $N/J(A)N\ne 0$. Notice that
	\[N/J(A)N=Y^s/(\Im(\partial_Y)+J(A)Y^s)\ne 0.\]

	Now since $N/J(A)N$ is a direct sum of simple modules, there is a nonzero morphism $f:P^{-d}\to N/J(A)N$. Since $P^{-d}$ is 
	(in particular) projective, we can lift to a morphism $\tilde f:P^{-d}\to Y^s$. Furthermore, $\Im(\tilde f)$ is not contained within
	$\Im(\partial_Y^{s-1})+J(A)Y^s$ (since it wasn't the zero map on $N/J(A)N$).

	Now consider the chain complex $\Hom_{\Ch(\Amod)}(P,Y)^\ast$ at the $d+s-1^{st}$ term. The differential maps from 
	$\Hom_A(P^-d,Y^{s-1})\oplus\Hom_A(P^{-d+1},Y^s)\to \Hom_A(P^{-d},Y^s)$ where 
	\[\partial(\alpha+\beta)=\partial_Y\circ\alpha+\beta\circ\partial_P\]

	But then by checking maps (and using that since $P$ is minimal, $\partial_P^{-d}$ maps into $J(A)P^{-d+1}$), we can 
	show that the image of any map under this differential must map into this object, so $\tilde f$ must not lie in the image, 
	so it is a nonzero element of the homology.
\end{prf}

\subsection{Tensor Product of Complexes}
Now let $X\in\Ch(\modA)$ and $Y\in\Ch(\Amod)$. Then $X\otimes_A Y\in\Ch(\Vectk)$ is
\[(X\otimes_A Y)^i=\bigoplus_{i+j=i}X^i\otimes Y^j\]
and the differential is given by the Koszul sign rule $\partial(x\otimes y)=\partial_X(x)\otimes y+(-1)^|x|x\otimes\partial_Y(y)$

The derived functor of $-\otimes_A-$ is 
\[X\otimes_A^LY\cong P_X\otimes_A y\cong X\otimes_A P_Y\cong P_x\otimes_AP_Y\]
where, for example, $A\otimes_A^LY=A\otimes_AY=Y$ and $X\otimes_A^LA=X$.

\begin{thm}
	Let $A$ be a Noetherian local algebra. Let $X\in\Db_{fg}(\Amod)$ with $\projdim X<\infty$. Let $Y\in\Db(\modA)$
	such that $\Hom_A(k,H^i(Y))\ne 0$ where $i=\inf Y$. Then 
	\[\inf(Y\otimes_A^LX)=-\projdim X+\inf Y\]
\end{thm}

\begin{lem}
	Let $A$ be Noetherian and $X\in\Db_{fg}(\Amod)$, $Y\in\Db(A^e\text{-}\mathbf{mod})$ and $Z\in \Dbperf(\Amod)$. Then 
	\[R\Hom_A(XY\otimes_A^LZ)\cong R\Hom_A(X,Y)\otimes_A^LZ\]
\end{lem}
\begin{prf}
	Let $F(-)=R\Hom_A(X,Y\otimes_A^L-)$ and $G(-)=R\Hom_A(X,Y)\otimes_A^L-$. We want to construct 
	a natural isomorphism between them. There is a map $\eta:F\to G$ that sends 
	\[f\otimes z\mapsto \{x\mapsto f(x)\otimes z\}.\]
	Then it only remains to check it is natural.

	The next step is to show that it holds (trivially!) for $Z=A$ and then since $A$ generates $\Dbperf(\Amod)$, we get that $G\cong F$ for all 
	things in $\Dbperf(\Amod)$.
\end{prf}

\section{May 24, 2019}
Had to miss this day.

\section{May 29, 2019}

\subsection{Example of the Day}
Let $A$ be a Noetherian algebra that is a finitely generated module over its center. Furthermore, suppose that 
$\calZ(A)\cong k[x_1,\cdots,x_n]/I$. Then $A$ has a dualizing complex $R$ where $R=R\Hom_{k[x_1,\dots,x_n]}(A,k[x_1,\dots,x_n][n])$

Then $\injdim R_A=\injdim {_A}R=0$, $\Ext^i_A(S,R)=\delta_{i=0}T$ for all simple left $A$-modules $S$.

\begin{defn}
	$R$ is called balanced if 
	\begin{itemize}
		\item $\injdim {_A}R=\injdim R_A=0$
		\item $R\Hom_A(k,R)=k$ and $R\Hom_{A^{op}}(k,R)=k$
	\end{itemize}
\end{defn}

So the idea we wrote down above is very similar to the idea of a balanced dualizing complex (it is stronger).

\subsection{Bass' Theorem}
\begin{thm}[Bass]
	Let $A$ be a local Noetherian algebra with balanced dualizing complex. Let $X\in\Dbperf(\Amod)$ with finite injective dimension. Then 
	\[\injdim X=\sup X+\depth A\]
\end{thm}
\begin{cor}
	Under the same hypotheses, let $M$ be a finitely-generated left $A$-module with finite injective dimension. Then 
	\[\injdim M=\depth A\]
\end{cor}
\begin{rmk}
	Notice that this also implies that the depth of $A$ is finite.
\end{rmk}

We begin the proof of the theorem with a lemma or two:
\begin{lem}
	Suppose $Y\in\Db_{fg}(\Amod)$. Then 
	\[\projdim Y=\sup\{R\Hom_A(Y,k)\}\]
	and 
	\[\sup Y=-\inf\{R\Hom_A(Y,k)\}\]
\end{lem}
\begin{prf}
	To see this, let $P$ be a minimal projective resolution of $Y$. Write $S=\sup(U)$ and $d=\projdim Y$.
	
	The idea for completing this proof again leverages that $\partial(P^i)\subseteq J(A)P^{i+1}$ for all $i$. Then notice that 
	everything is in the kernel of $\partial^\ast$, the induced map from the functor $\Hom_A(-,k)$, which then implies that
	$\sup(\Hom_A(P,k))=d$ and that $\inf \Hom_A(P,k)=-s$ is clear. Then the result follows from using that $R\Hom_A(Y,k)=\Hom_A(P,k)$.
\end{prf}
Let's get another lemma!
\begin{lem}
	Suppose $X\in\Db_{fg}(A)$. Then 
	\[\injdim\bbD(X)\le \projdim X\]
	and 
	\[\projdim \bbD(X)\le\injdim X.\]
\end{lem}
\begin{prf}
	Recall here that $\bbD(X)=R\Hom_A(X,R)$ and $\bbD^\circ(X_A)=R\Hom_{A^{op}}(X,R).$ Then suppose that $d=\projdim X<\infty.$ Let $P$ be a minimal 
	free resolution of $X$. Write $I_R$ to denote a resolution of $R$ as an $A$-bimodules such that each term is injective on the right. Since $\injdim R=0$, $I_R^{>0}=0.$

	But now $\bbD(X)=R\Hom_A(X,R)=\Hom_A(P,I_R)$. Now note that 
	\[\Hom_A(P,I_R)^i=\oplus_{w\in\bbZ}\Hom(P^w,I_R^{i+w})=\oplus (I_r^{i+w})^{\operatorname{rank}(P^w)}\]
	so in particular, it is injective. Furthermore, $\Hom_A(P,I_k)^i=0$ whenever $i>d.$

	But then $\bbD(X)$ has an injective resolution that is bounded above at $d$.

	But then we know that 
	\[\projdim \bbD(X)=\sup\{R\Hom_{A^{op}}()\bbD(X),k)\}=\sup\{R\Hom_A(\bbD^\circ(k),\bbD^\circ\bbD(X))\}\]
	where the last equality is by the (earlier) duality theorem. But $\bbD^\circ(k)=k$ and $\bbD^\circ\bbD(X)=X$,
	and so since $\sup\{R\Hom_A(k,X)\}\le\injdim X$, we get
	\[\projdim\bbD(X)\le\injdim X\]

	I think I missed the other one?!
\end{prf}
\begin{cor}\label{cor-equalities}
	\[\projdim \bbD(X)=\injdim X\] 
	and 
	\[\injdim X=\sup\{R\Hom_A(k,X)\}\].
\end{cor}
\begin{lem}
	If $R$ is the balanced dualiznig complex of $A$,
	\[\depth A=-\sup(R)\ge 0\]
\end{lem}
\begin{prf}
	\begin{align*}
		\depth A &= \inf\{R\Hom_A(k,A)\}\\
		&= \inf\{R\Hom_{A^{op}}(\bbD(A),\bbD(k))\}\\
		&=\inf\{R\Hom_{A^{op}}(R,k)\}\\
		&= -\sup R.
	\end{align*}
\end{prf}
Next recall some identities from last week:
\begin{lem}
	Let $X\in\Dbperf(\Amod)$. Then if $\Hom(H^s(Y),k)\ne 0$ where $s=\sup Y$, then 
	\[\sup\{R\Hom(X,Y)\}=\projdim X+\sup Y.\]
\end{lem}

Finally we get around to proving the theorem:
\begin{prf}[of Bass' theorem]
	Suppose $\injdim X<\infty.$  Then by corollary~\ref{cor-equalities} above, $\projdim\bbD(X)<\infty$. Then
	\begin{align*}
		\sup X &=\sup\{R\Hom_A(A,X)\}\\
		&=\sup\{R\Hom_{A^{op}}(\bbD(X),\bbD(A))\}\\
		&=\sup\{R\Hom_{A^{op}}(\bbD(X),R)\}\\
		&=\projdim\bbD(X)+\sup R\\
		&=\injdim X-\depth A.
	\end{align*}
	And that's what we wanted to prove!
\end{prf}
\begin{rmk}
	The preceding proof was adapted by Peter Jorgensen's paper. He apparently has very clear exposition 
	and arguments where each step is carefully justified. The paper in question here is entitled something like \textit{Homological Identiies}
	but apparently all of his papers have a similar level of polish and James recommends reading them all.
\end{rmk}

\subsection{No Holes Theorem}
\begin{thm}[No Holes]
	For every $i$ between $\depth M$ and $injdim M$,
	\[\Ext_A^i(k,M)\ne 0.\]
\end{thm}

\section{May 31, 2019}
Today we begin our discussion of Artin-Schelter regular algebras.
\subsection{Some History/Comments}
In the 1980's ('86), Artin and Schelter started a project of classifying connected graded regular algebras of global dimension 3. 
This is somehow comparible to quantum or noncommutative projective planes $q\bbP^2$ or $\bbP^2_{nc}$.

There were actualy some examples of such algebras discovered befor this, however. In 1982, 
Sklyanin found examples of regular algebras of global dimension 4 in the suty of they Yang-Baxter equation and quantum groups.

In 1991, the ``AS'' project was finished by Artin, Schelter, Tate, and Ven den Bergh. All together they constitute about 
150 pages and the result was that there were 14 families. Later on these were named Artin-Shelter regular algebras.

\subsection{Examples}
\begin{ex}
	Let $d=\gldim A$. Then the following are AS-regular:
	\begin{itemize}
		\item When $d=0$, only $k$ is AS regular.
		\item When $d=1$, $k[x]$ is AS regular when $\deg x>0$.
		\item For $d=2$, if $k=\bar k$, then $k_q[x,y]:=k\langle x,y\rangle/(yx-qxy)$ and $k_J[x,y]=k\langle x,y\rangle/(yx-xy-x^2)$ are examples.
	\end{itemize}
\end{ex}
\subsection{Definitions}
Recall that $A$ is called connected graded if $A=\oplus_\bbN A_i$ with $1_A\in A_0=k$ and $A_iA_j\subseteq A_{i+j}$. Then $\bbN$ is called 
an internal (or Adams) grading, which is NOT the complex grading -- in particular the Koszul sign rule does \textbf{not} apply to this grading.
\begin{defn}
	A connected graded algebra $A$ is called \textbf{Artin-Schelter regular} (AS regular) if 
	\begin{itemize}
		\item $\gldim A=d<\infty$
		\item Let $k=A/A_{\ge 1}$ be the trivial module. Then 
	\[\Ext_A^i(k,A)=\Ext_{A^{op}}^i(k,A)=\left\{\begin{array}{lr}0, & i\ne d\\ k,& i=d\end{array}\right.\]
	\end{itemize}
\end{defn}
\begin{rmk}
	Notmally one requires that the algebra also has finite GK dimension (no idea what this is), but we will assume that our algebras are Noetherian, 
	which obviates this restriction.
\end{rmk}
\begin{prop}
	If $A$ is connected graded and commutative, then the following are equivalent:
	\begin{itemize}
		\item $A$ is AS regular.
		\item $A$ has finite global dimension
		\item $A=k[x_1,\cdots,x_n]$.
	\end{itemize}
\end{prop}
\begin{ex}
	A \textbf{non-example} is that $k\langle x_1,\cdots,x_n\rangle$ is \textit{not} AS-regular.
\end{ex}
\begin{rmk}
	A connected graded algebra is a graded version of a complete local algebra. That is, $A=\varprojlim A/A_{\ge n}$.
\end{rmk}
\begin{rmk}\label{rmk-AB}
	\textsc{(Auslander-Buchsbaum formula)}: Suppose $A$ satisfies the $\chi$ condition (that is, for all finitely generated $A$-modules $M$, $\dim\Ext_A^i(k,M)<\infty$ for all $i$)
	If $M$ has finite projective dimension then 
	\[\projdim M=\depth A-\depth M.\]
\end{rmk}

\subsection{Today's Main Result}
\begin{thm}
	Let $A$ be a connected graded Notherian algebra with finite global dimension. Then the following are equivalent:
	\begin{itemize}
		\item $A$ is AS regular
		\item $A$ satrisfies the $\chi$ condition
		\item For any $X\in\Db(\Amod_{fg})$ (can also switch sides) then $\dim\Ext_A^i(k,X)<\infty,\forall i$.
		\item $A$ has a balanced dualizing complex.
	\end{itemize}
\end{thm}
To prove this, we will first need two lemmas.
\begin{lem}
	Suppose $A$ satisfies $\chi$  and $\gldim A=d<\infty$. Then 
	\begin{itemize}
		\item $\depth A=d$
		\item $\projdim k=d$.
	\end{itemize}
\end{lem}
\begin{prf}
	Proving this uses the Auslander-Buchsbaum formula from remark~\ref{rmk-AB}. For any finitely-generated module $M$,
	\[\depth A=\projdim M=\depth M\ge \projdim M\]
	and so by definition, $\depth A\ge \gldim A=d$.

	To prove the other inequality we need to unpack the definition of depth:
	\[\depth A=\inf(R\Hom(k,A))\le \sup(R \Hom(k,A))=\projdim k\le \gldim A=d.\]
	Thus $\depth A=d$. But we also havev $\projdim k$ in there, so the second equality also holds.
\end{prf}
\begin{lem}
	Let $F:\Amod_{fd}\to \modA_{fd}$ and $G:\modA_{fg}\to\Amod_{fg}$ be inverse functors. Then $F({_A}k)=k_A$
	and $G(k_A)={_A}k.$
\end{lem}
\begin{prf}
	This is almost trivial if you think hard about where a simple trivial module has to be sent.
\end{prf}
\begin{prf}[of theorem]
	(1)$\Rightarrow$(3): By definition we know what $\Ext_A^i(k,A)$ is. Define $\calT$ to be the triangulated subcategory of 
	$\Db_{fg}(\Amod)$ of all objects $X$ such that $\dim\Ext_A^i(k,X)<\infty$ for all $i$.

	It is easy to check that this is a triangulated subcategory. By the definition of AS-regularity, $A$ is in $\calT$. And we know $\Db_{fg}(\Amod)=\Dbperf(\Amod)=\langle A\rangle$, so $\calT$ is 
	the entire category $\Db_{fg}(\Amod).$

	\brk

	(3)$\Rightarrow$(2) is left as an exercise, but we can do it by taking $X=M$.

	\brk

	(2)$\Rightarrow$(1) By the first lemma, we have $\Ext_A^i(k,A)=0$ when $i\ne d$ and is finite dimensional otherwise.
	Let $F(-)=\Ext_A^d(-,A)$ and $G(-)=\Ext^d_{A^{op}}(-,A)$. Now note that 
	\[F(M)=R\Hom_A(M,A)[-d]=R\Hom_A(M,A[d])\]
	and let $R=A[d]$. The claim is that $R$ is a dualizing complex. This implies that $F(M)=\bbD(M)$ and $G(N)=\bbD^\circ(N)$,
	and we can apply the lemma to show that $\Ext_A^d(k,A)=F(k)=k$ and $\Ext_{A^{op}}^d(k,A)=k$, so we're done.
\end{prf}	

\end{document}